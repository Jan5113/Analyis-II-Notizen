\documentclass[a4paper, 11pt]{report}


%%%%%%%%%%%%
% Packages %
%%%%%%%%%%%%

\usepackage[ngerman]{babel}
\usepackage[noheader]{packages/sleek}
\usepackage{packages/sleek-title}
\usepackage{packages/sleek-theorems}
\usepackage{packages/sleek-listings}
\usepackage{ragged2e}
\usepackage{tikz,lipsum,lmodern}
\usepackage[most]{tcolorbox}
\usepackage{quotchap}
\usepackage{biblatex}
\usepackage{graphicx} 
\usepackage{bbm}
\usepackage{enumitem}
\usepackage{pgfplots}
\usepackage{mdframed}
\usepackage{mathtools}
\usepackage[makeroom]{cancel}
\usepackage{diffcoeff}

%\tcbuselibrary{theorems}
%\usepackage{cleveref}
\addbibresource{bibliographie.bib}
\newcommand{\notimplies}{\;\not\!\!\!\implies}
\newcommand{\R}{\mathbb{R}}
\newcommand{\N}{\mathbb{N}}
\newcommand{\Q}{\mathbb{Q}}
\newcommand{\C}{\mathbb{C}}
\newcommand{\Z}{\mathbb{Z}}
\newcommand{\cA}{\mathcal{A}}
\newcommand{\cB}{\mathcal{B}}
\newcommand{\cC}{\mathcal{C}}
\newcommand{\cP}{\mathcal{P}}
\newcommand{\abk}[1]{\left\langle #1 \right \rangle}
\newcommand{\bk}[1]{\left(#1\right)} 
\newcommand{\Nrm}{\norm{\:\cdot\:}}
\newcommand{\bigmid}{\ \big|\ }
\newcommand{\set}[2]{\left\{\left.#1 \ \right|  #2 \right\}}
\newcommand{\setr}[2]{\left\{#1 \ \left|  #2 \right.\right\}}
\newcommand{\todo}[1]{TODO: \textit{#1}}
\newcommand{\limto}[1]{\lim_{#1 \to #1_0}}
\newcommand{\limtoinf}[1]{\lim_{#1 \to \infty}}
\newcommand{\limtoneginf}[1]{\lim_{#1 \to -\infty}}
\newcommand{\quer}[1]{\overline{#1}}
\newcommand{\smat}[1]{\bk{\begin{smallmatrix} #1 \end{smallmatrix}}}
\newcommand{\mat}[1]{\begin{pmatrix}#1\end{pmatrix}}

\DeclareMathOperator{\divergence}{div}
\DeclareMathOperator{\rot}{rot}
\DeclareMathOperator{\grad}{grad}



\graphicspath{ {./images/} }
%\newtheorem{defin}{Definition}[section]

\makeatletter
\newtheoremstyle{indented}
  {3pt}% space before
  {3pt}% space after
  {%\addtolength{\@totalleftmargin}{3.5em}
   \addtolength{\linewidth}{-3em}
   \parshape 1 1.5em \linewidth}% body font
  {}% indent
  {\bfseries}% header font
  {.}% punctuation
  {.5em}% after theorem header
  {}% header specification (empty for default)
\makeatother

\theoremstyle{definition}

%\newtheorem{mdexample}{Beispiel}[section]
%\newenvironment{example}%
%  {\begin{mdframed}[topline=false,  bottomline=false, rightline=false, leftline=false, innertopmargin=0,
%    innerbottommargin=0.1]\begin{mdexample}}%
%  {\end{mdexample}\end{mdframed}}

\theoremstyle{indented}
\newtheorem{example}{Beispiel}[section]


\newtheorem{exercise}[example]{Übung}

\newtheorem{tippcounter}{Tippps}[chapter]

\theoremstyle{remark}
\newtheorem*{remark}{Bemerkung}

\newtcbtheorem[use counter*=example, number within=section]{definition}{\strut Definition}
{colback=cyan!5,colframe=cyan!35!gray,fonttitle=\bfseries,breakable, parbox=false}{def}

\newtcbtheorem[use counter*=example, number within=section]{lemma}{\strut Lemma}
{colback=black!5,colframe=black!45,fonttitle=\bfseries,breakable, parbox=false}{lem}

\newtcbtheorem[use counter*=tippcounter]{tipp}{\strut Tipp}
{colback=lime!5,colframe=lime!35!gray,fonttitle=\bfseries,breakable, parbox=false}{tipp}

\newtcbtheorem[use counter*=example, number within=section]{satz}{\strut Satz}
{colback=red!5,colframe=red!35!gray,fonttitle=\bfseries,breakable, parbox=false}{satz}

\newtcbtheorem[use counter*=example, number within=section]{korollar}{\strut Korollar}
{colback=orange!5,colframe=orange!35!gray,fonttitle=\bfseries,breakable, parbox=false}{kor}


%%%%%%%%%%%%%%
% Title-page %
%%%%%%%%%%%%%%

\institute{ETH ZÜRICH}
\faculty{D-MATH}
\title{Analysis I \& II\\ \normalsize Vorlesungsnotizen ge{\TeX}t}
\subtitle{\normalsize basierend auf den Vorlesungen von Prof. Giovanni Felder HS2020/FS2021}


%\author{\\Jan \textsc{Obermeier}
%       \\Yoel \textsc{Zimmermann}
%}

\context{}
\date{\today}

%%%%%%%%%%%%%%%%
% Bibliography %
%%%%%%%%%%%%%%%%

\addbibresource{./resources/bib/references.bib}

%%%%%%%%%%
% Others %
%%%%%%%%%%

\lstdefinestyle{latex}{
    language=TeX,
    style=default,
    %%%%%
    commentstyle=\ForestGreen,
    keywordstyle=\TrueBlue,
    stringstyle=\VeronicaPurple,
    emphstyle=\TrueBlue,
    %%%%%
    emph={LaTeX, usepackage, textit, textbf, textsc}
}

\FrameTBStyle{latex}

\def\tbs{\textbackslash}

%%%%%%%%%%%%
% Document %
%%%%%%%%%%%%

\begin{document}
    \setcounter{secnumdepth}{2}
    \setcounter{tocdepth}{3}
    \maketitle
    \romantableofcontents
    %\chapter*{Vorwort}
Dieses Projekt entstand im Rahmen der Analysis I und II Vorlesungen von \textsc{Prof. Giovanni Felder} im Herbstsemester 2020. 

Da das Vorlesungsskript \cite{Vorlesungsskript} die doch sehr komplexe Materie der Vorlesungen eher trocken behandelt und die \cite{Vorlesungsnotizen} meist nur das Nötigste ohne vielen Worten beschreibt, ist das Ziel dieses Skriptes, dem Leser neben den Tatsachen auch eine Intuition zu vermitteln.

Der Inhalt bleibt dabei sehr nahe dem der Vorlesung, gelegentlich werden auch eher praxisbezogene Tipps und Tricks, welche im Rahmen der Übungslektionen vermittelt worden sind, eingeschoben. 
Unter \href{https://janiks.me/projects/eth/ana/list}{diesem Link} kann man eine Sammlung an allen Lemmata, Sätzen, etc. sowie eine Lernkartei-All finden.

Hier noch einige weitere hilfreiche Links zu Zusammenfassungen und Cheat-Sheets von anderen Analysis Vorlesungen:
\begin{itemize}
    \item \href{https://people.math.ethz.ch/~struwe/Skripten/Analysis-I-II-final-6-9-2012.pdf}{Zusammenfassung Analysis I und II}
    \item \href{https://www2.math.ethz.ch/education/bachelor/lectures/fs2014/math/analysis2.html}{Vorlesungswebseite Analysis II 2014}
    \item \href{http://citeseerx.ist.psu.edu/viewdoc/download?doi=10.1.1.621.6146&rep=rep1&type=pdf}{Vektoranalysis} (Felder)
\end{itemize}
    %\chapter{Mathematische Grundlagen}

Bevor wir zu den grossen Aussagen der Mathematik kommen können, müssen wir uns leider den aller grundlegendsten Grundlagen der Mathematik widmen. Nämlich mit der Frage, was denn überhaupt eine Aussage wahr oder falsch macht. Was macht $0=1$ zu einer falschen und $e^{i\pi} + 1 = 0$ zu einer wahren Aussage? Mit dieser Frage haben sich Mathematiker (erst) im 19. Jahrhundert beschäftigt und haben ein System an Regeln eingeführt, in welchem man für jede gut formulierte mathematische Aussage entscheiden kann, ob diese wahr oder falsch ist.

\section{Axiome, Sätze, Lemmata  und Korollare}
Die Richtigkeit einer Aussage ist vom Kontext abhängig. So ist z.B. die Aussage ''$x^2 = 2$ hat eine Lösung'' falsch, wenn man nur die rationalen Zahlen betrachtet, aber sie ist richtig, wenn man die reellen Zahlen betrachtet.

Damit der Kontext klar ist, beginnt man mit der Definition von \textbf{Axiomen}, welche ein logisches Grundgerüst bilden. Axiome sind per Definition wahr und können in dem Sinne nicht bewiesen werden. So besagt zum Beispiel das erste \textit{Peano-Axiom}, dass $1$ eine natürliche Zahl ist\footnote{Im Rahmen dieser Vorlesung werden wir $1$ als die kleinste natürliche Zahl betrachten. Je nach dem wird auch $0$ als kleinste natürliche Zahl definiert.}. Mit diesem Gerüst aus Axiomen kann man nun durch logische Umformungen eine Aussage beweisen oder widerlegen.\footnote{Man könnte sich fragen, was denn der komplette Satz an Axiomen ist; wann man ''alle'' Axiome hat. Jedoch besagt die \textit{Gödel-Unvollständigkeit}, dass es gar nicht möglich ist, einen solchen vollständigen Satz an Axiomen zu bestimmen. Don't quote me on that though...} 

Zu den heute akzeptierten Fundamenten der Mathematik gehören unter anderen die Axiome der:
\begin{itemize}
    \item Prädikatenlogik erster Stufe
    \item \textsc{Zermelo-Fraenkel} Mengenlehre (darunter das Auswahlaxiom, siehe Abschnitt \ref{cha_praedlogic})
    \item \textsc{Peano}-Axiome der natürlichen Zahlen
\end{itemize}

Damit man bei einer Beweisführung nicht immer von null beginnen muss, gibt es \textbf{Sätze}, \textbf{Lemmata} und \textbf{Korollare} (Hilfssätze), welche aus den Axiomen hergeleitet worden sind und somit ebenfalls wahre Aussagen im Rahmen der Axiome sind.

\section{Aussagenlogik}\label{cha_logic}
Nehmen wir nun an, dass die Axiome gegeben sind. In diesem Fall kann eine Aussage nur wahr ($w$) oder falsch ($f$) sein. Da interessantere Aussagen meist mehrere kleinere Aussagen verknüpft, wurden die \textbf{logischen Operatoren/Zeichen} oder \textbf{Junktoren} eingeführt, mit welchen man die Verknüpfungen formalisieren kann. Die wichtigsten Operatoren sind: Negation ($\neg$), Und ($\land$), Oder ($\lor$), ''impliziert'' ($\implies$) und ''ist äquivalent zu/genau dann, wenn'' ($\iff$). Da die Aussagen $A$ und $B$ zusammen nur vier verschiedene Werte annehmen können, kann man die Funktionswerte (also die Ergebnisse) der jeweiligen Operatoren eindeutig als Wertetabelle wie folgt darstellen/definieren:

\begin{definition}{logische Operatoren/Junktoren}{}
\centering
    \begin{tabular}{cc||c|c|c|c|c}
    $A$ & $B$ & $\neg A$ & $A \land B$ & $A \lor B$ & $A \implies B$ & $A \iff B$ \\
    \hline
    $f$ & $f$ & $w$  & $f$    & $f$   & $t$ & $t$   \\
    $f$ & $w$ & $w$  & $f$    & $w$   & $t$ & $f$   \\
    $w$ & $f$ & $f$  & $f$    & $w$   & $f$ & $f$   \\
    $w$ & $w$ & $f$  & $w$    & $w$   & $t$ & $t$  
    \end{tabular}
\end{definition}

Durch Vergleichen der Wertetabellen kann man z.B. sehen, dass $(A \land B) \lor (\neg A \land \neg B)$ genau dasselbe ist wie $A \iff B$. Man nennt zwei logische Aussagen daher äquivalent zueinander, wenn deren Wertetabellen gleich sind. Falls man wollte, könnte man die logische Äquivalenz von soeben ausdrücken als:
$$((A \land B) \lor (\neg A \land \neg B)) \equiv\footnote{In der Vorlesung wurde auf die Einführung des Äquivalenzzeichens $\equiv$ verzichtet, stattdessen wurde ''$\iff$'' verwendet. Der Unterschied liegt darin, dass ''$\iff$'' zwei Aussagen zu einer weiteren Aussage verknüpft (welche streng genommen auch wieder einen Wahrheitswert \textit{wahr} oder \textit{falsch} annehmen kann), während ''$\equiv$'' die Äquivalenz zwischen zwei Aussagen zeigt. Mach dir nicht zu viele Gedanken darüber und lese $A \equiv B$ einfach als ''$A$ macht dieselbe Aussage wie $B$''} (A \iff B)$$
\begin{remark}[ ]
Eine Spezialität von ''$\implies$'' ist, dass aus einer falschen Aussage $A$ alles folgen kann und $A \implies B$ trotzdem wahr ist. In Worten ausgedrückt, kann man aus einer falschen Behauptung alles folgern (z.B. $\pi = 3.2 \implies$ ''Quadratur des Kreises ist möglich'').
\end{remark}

\begin{example}[ ] Sei $a$ eine beliebige natürliche Zahl, dann gilt:
$$6 \mid a \implies 3 \mid a$$
In Worten: Wenn $a$ durch $6$ teilbar ist, ist diese auch durch $3$ teilbar. Beachte, dass die Rückrichtung im Allgemeinen nicht gilt (z.B. wenn $a = 9$ ist). Für das braucht es eine weitere Einschränkung, nämlich:
$$6 \mid a \iff (3 \mid a \land 2 \mid a)$$
sprich, $a$ ist durch $6$ teilbar genau dann, wenn $a$ durch $2$ und $3$ teilbar ist. (Beachte hier den Gebrauch von ''genau dann, wenn''!)
\end{example}

\begin{lemma}{Logische Äquivalenzen}{}
\begin{align*}
        A \implies B &\equiv \neg A \lor B  &  A \implies B &\equiv \neg B \implies \neg A\\
        \neg(A \land B) &\equiv \neg A \lor \neg B & \neg(A \lor B) &\equiv \neg A \land \neg B\\
        A \land (B \lor C) &\equiv (A \land B) \lor (A \land C) & A \lor (B \land C) &\equiv (A \lor B) \land (A \lor C)
\end{align*}
\end{lemma}
Die Äquivalenzen der zweiten Zeile sind die \textsc{De Morgan}'schen Gesetze. Man kann es sich wie die Distributivität des Minus-Zeichen vorstellen. Der Begriff wurde in der Vorlesung nicht eingeführt, jedoch werden wir sie in den folgenden paar Abschnitte noch einige Male antreffen, weswegen wir es trotzdem kurz erwähnt haben wollen.

\section{Prädikatenlogik}\label{cha_praedlogic}
Wie man im vorherigen Beispiel erkennt hat, will man in der Mathematik Aussagen machen können, welche sich auf alle  Elemente einer Menge\footnote{Eine Menge ist eine Kollektion an (paarweise verschiedenen) Elementen, z.B. reelle, rationale Zahlen, Vektoren oder andere Objekte} beziehen können. Konkret wollen wir mit \textbf{Quantoren} Aussagen machen können wie 
\begin{definition}{Quantoren}{}
    \begin{itemize}
    \item \textbf{Allquantor} $\forall$: ''Für alle/beliebige Elemente gilt...''
    \item \textbf{Existenzquantor} $\exists$: ''Es gibt mindestens ein Element, für welches gilt...''
    \item \textbf{Quantor der eindeutigen Existenz} $\exists!$: ''Es gibt ein eindeutiges Element, für welches gilt...''
\end{itemize}
\end{definition}

Um mit Elementen aus einer Menge hantieren zu können, möchten wir zuerst noch $\in$ (''ist Element von/in'') einführen, womit man ein beliebiges Element aus einer Menge auswählen kann\footnote{Dies folgt aus dem Auswahlaxiom der Zermelo-Fraenkel Mengenlehre: Es besagt, dass aus jeder (nicht leeren) Menge ein beliebiges Element ausgewählt werden kann.}.

Somit sind $\forall$, $\exists$ und $\exists!$ angewendet auf eine Aussage $A$ und Menge $X = \{x_1, ..., x_n\}$ äquivalent zu:

\begin{center}
    \begin{tabular}{lclcl}
        $\forall x \in X : A(x)$ &$\equiv$ &$ (\forall x)(x \in X \implies A(x))$ &$\equiv$ &$ A(x_1) \land ... \land A(x_n)$\\
        $\exists x \in X : A(x)$ &$\equiv$ &$ (\exists x)(x \in X \land A(x))$ &$\equiv$ &$ A(x_1) \lor ... \lor A(x_n)$\\
        $\exists! x \in X : A(x)$&$\equiv$ &$ (\exists! x)(x \in X \land A(x)) $&$\equiv$ &$ A(x_1) \oplus ... \oplus A(x_n)$
    \end{tabular}
\end{center}


Dabei wird die in der ersten Spalte verwendete Abkürzung für den Ausdruck der zweiten Spalte verwendet. $A(x_i)$ bezeichnet den Wahrheitswert der Aussage  $A$ für den jeweiligen Wert $x_i$ und $\oplus$ das \textbf{exklusive Oder}: $A \oplus B \equiv (A \lor B) \land (\neg A \lor \neg B)$

In Worte gefasst ist $\forall x \in X : A(x)$ nur dann wahr, wenn $A$ für alle Elemente aus $X$ wahr ist, $\exists x \in X : A(x)$ dann, wenn $A$ mindestens für ein Element aus $X$ zutrifft und $\exists! x \in X : A(x)$ dann, wenn es genau für ein einziges Element aus $X$ wahr ist.

Mit den Quantoren hat man nun ein sehr mächtiges Werkzeug, um eindeutig Aussagen über viele Elemente machen zu können. So lässt sich das obige Beispiel ganz kompakt schreiben als:

\begin{example}[Fortsetzung]
$$\forall a \in \N: 6 \mid a \implies 3 \mid a$$
Es liest sich als ''Für jede Zahl in der Menge der natürlichen Zahlen, impliziert die Teilbarkeit durch 6 die Teilbarkeit durch 3.''
\end{example}

Etwas interessanter ist die folgende, bis heute nicht bewiesene Aussage:

\begin{example}[Goldbach-Vermutung]\label{ex_goldbach}
$$\forall g \in \N : (2 \mid g \land g > 2) \implies \exists p,q \in \N: (\text{$p$ prim}) \land (\text{$q$ prim})\land (p + q = g)$$
Oder in Worten: ''Jede gerade Zahl grösser als 2 lässt sich als Summe zweier Primzahlen schreiben.'' $\exists p,q \in \N$ ist dabei eine Abkürzung für $\exists p \in \N,\ \exists q \in \N$.
\end{example}

Einige Dinge, welche noch wichtig zu wissen sind über Quantoren: 

\subsection{Reihenfolge der Quantoren}
Zwar kommutieren\footnote{lassen sich vertauschen} Allquantoren untereinander und Existenzquantoren untereinander, jedoch hat eine Vertauschung von Existenz- und Allquantoren hat einen wesentlichen Einfluss auf eine Aussage. Auch nur schon sprachlich merkt man, dass ''\textit{Es gibt ein ..., sodass für alle ...}'' viel einschränkender ist als ''\textit{Für alle ... gibt es jeweils ein ...}''. So wäre die Aussage aus Beispiel \ref{ex_goldbach} mit vertauschten Quantoren eine ganz andere: 
$$\exists p,q \in \N: \forall g \in \N : (2 \mid g \land g > 2) \implies (\text{$p$ prim}) \land (\text{$q$ prim})\land (p + q = g)$$
also ''Es gibt zwei Primzahlen, welche jede gerade Zahl grösser als 2 als Summe haben''. Diese Aussage ist offensichtlich falsch, da schnell ein Gegenbeispiel gefunden werden kann.

Man merkt sich daher besten, dass die durch die Quantoren definierten Elemente im ''Gültigkeitsbereich'' danach ''fixiert'' sind resp. nicht mehr geändert werden können. 

\subsection{Allquantor mit leeren Mengen}
Für eine leere Menge $X$ gilt für jedes Element $x$, dass $x\in X$ falsch ist. Da für die Definition des Allquantors die Implikation $\implies$ verwendet wurde, folgt aus der Bemerkung im Abschnitt \ref{cha_logic}, dass $\forall x \in X : A(x)$ wahr ist unabhängig von der Wahl von $A$.

\begin{lemma}{\textsc{De Morgan} mit Quantoren}{}
\begin{align*}
        \neg(\forall x \in X : A(x) &\equiv \exists x \in X : \neg A(x)\\
        \neg(\exists x \in X : A(x) &\equiv \forall x \in X : \neg A(x)
\end{align*}
\end{lemma}


\section{Beweismethoden}
Aus den soeben besprochenen Themen lassen sich bereits einige Beweismethoden ableiten. Das Ziel eines Beweises ist es, aus den ''Spielregeln'', welche durch die Axiome der jeweiligen Theorie gegebenen sind, weitere, tiefgründigere Aussagen zu erarbeiten. Einige Methoden basieren auf Theorie, welche wir später behandeln, dies ist entsprechend mit Referenzen gekennzeichnet.

\subsection{direkter Beweis}
Bei der direkten Beweismethode gibt es auf diesen abstrakten Level nicht viel zu erwähnen: Wir wollen lediglich zeigen, dass $A \implies B$. Hierfür verwenden wir direkt die von den Axiomen vorgegebenen Rechenregeln und zeigen, dass aus $A$ direkt $B$ folgt. Da ein direkter Beweis je nach Fragestellung sehr schwierig sein kann, findet man in den nächsten Abschnitten einige Alternativen, welche ebenfalls zu einem rigorosen Beweis einer Aussage führen können.

\subsection{Beweis einer Äquivalenz}
Wir treffen diese Art von Beweisen sehr häufig an, denn sobald man in Formulierung der Fragestellung auf den Wortlaut ''$A$ \textit{genau dann, wenn} $B$'' trifft, müssen wir die Gleichheit der Aussagen $A$ und $B$ zeigen, also:
$$A \iff B$$
In diesem Fall können wir die Äquivalenz zwischen den Aussagen
$$A \iff B \equiv (A \implies B) \land (B \implies A)$$ 
verwenden, welche soviel bedeutet wie beide Richtungen separat zu zeigen. Man könnte auch die Gleichheit direkt zeigen, jedoch ist dies meist schwieriger und fehleranfälliger. Ein ausführliches Beispiel hierfür ist der erste Teil vom Beweis \ref{prf_inverse_of_fcn}.

Dieser Ansatz lässt sich z.B. auch für die Gleichheit von Mengen (siehe Kapitel \ref{cha_settheory}) anwenden: Wenn wir $A = B$ zeigen wollen, dann können wir auch beide Inklusionen separat zeigen: $A \subseteq B \land B \subseteq A \equiv A = B$. 

\subsection{Kontraposition}
Für die Kontraposition verwenden wir die Äquivalenz der Aussagen
$$A \implies B \equiv \neg B \implies \neg A$$
und zeigen, dass $\neg B \implies \neg A$ gilt, welches je nach Fragestellung einfacher sein kann als $A \implies B$ direkt zu zeigen. Am besten zeigen wir das anhand eines kleinen Beispiels:
\begin{example}[Kontraposition]\label{ex_counterpos}
    Sei $A$ die Aussage, dass $x^2$ gerade ist und $B$, dass $x$ gerade ist, dann bedeutet $A \implies B$, dass ein gerades Quadrat nur durch Quadrieren einer geraden Zahl erhalten werden kann. Die Kontraposition davon ist $\neg B \implies \neg A$, also dass eine ungerade Zahl hoch 2 eine ungerade Zahl ergibt. Da $x$ ungerade ist, ist auch $x^2$ ungerade. Daraus folgt aus der logischen Äquivalenz direkt die Behauptung $A \implies B$.
\end{example}

\subsection{Widerspruch}
Nicht zu verwechseln mit der Kontraposition nehmen wir beim Widerspruchsbeweis $A$ und $\neg B$ an, welches zu einem Widerspruch geführt werden soll. Man kann durch Vergleichen der Wertetabllen schnell sehen, dass $A \implies B$ äquivalent ist zu $A \land \neg B \implies \bot$\footnote{$\bot$ bedeutet ''immer falsch'', also ein Widerspruch. $\top$ wäre das Symbol für ''immer richtig'', also eine Tautologie.}. Zur Veranschaulichung findet man in Beispiel \ref{ex_sqrt2_irrational} ein klassisches Beispiel für einen Widerspruchsbeweis.

\subsection{vollständige Induktion} \label{cha_induction}
Falls man eine Aussage über die Menge der natürlichen Zahlen zu zeigen hat, dann kann man sich das letzte \textit{Peano-Axiom} der natürlichen Zahlen (siehe Kapitel \ref{cha_natural_numbers}) zu Nutzen machen. Zur Demonstration der Induktion wollen wir zeigen, dass die Aussage $\sum_{i = 1}^ni^2 = \frac{n(n+1)(2n+1)}{6}$ für alle $n \in \N$ gilt.

Hierzu ist es am besten, wenn man den Beweis in drei Abschnitte gliedert:
\begin{enumerate}
    \item \textbf{Induktionsverankerung IV}: Hier müssen wir zeigen, dass die Aussage für $n = 1$ gilt. Im Beispiel ist das leicht durch Einsetzen und explizitem Ausrechnen zu zeigen: $\sum_{i = 1}^1i^2 = 1 = \frac{1(1+1)(2\cdot1+1)}{6}$ Somit ist die Aussage für $n = 1$ bereits bewiesen.
    \item \textbf{Induktionsannahme IA}: Nun wollen wir annehmen, dass die Aussage für ein beliebiges $n \in \N$ bereits gilt. Das mag erst unintuitiv klingen, jedoch denkt man hier am besten an den im ersten Schritt bewiesenen Fall. Wir behaupten also einfach, dass $A(n)$ bereits stimmt, im Fall vom Beispiel also $\sum_{i = 1}^ni^2 = \frac{n(n+1)(2n+1)}{6}$.
    \item \textbf{Induktionsschritt IS}: Nun kommen wir zum entscheidenden Schritt: Da wir bereits annehmen, dass $A(n)$ stimmt, müssen wir nur noch zeigen, dass daraus auch die Aussage $A(n+1)$ folgt. Für diesen Schritt müssen wir nun ein bisschen rechnen und können unter Verwendung der Induktionsannahme zeigen, dass:
    \begin{align*}
        \sum_{i = 1}^{n+1}i^2 &= (n+1)^2 + \sum_{i = 1}^{n}i^2 \\
                            &\stackrel{\text{IA}}{=} (n+1)^2 + \frac{n(n+1)(2n+1)}{6} \\
                            &= \frac{(n+1)((n+1)+1)(2(n+1)+1)}{6} = A(n+1)
    \end{align*}
    Aus dem letzten \textit{Peano-Axiom} folgt hiermit bereits, dass die Aussage für alle $n \in \N$ gilt und der Induktionsbeweis abgeschlossen ist.
\end{enumerate}
Ein weiteres Beispiel für die Induktion kann im Abschnitt \ref{cha_bernoulli_uneq} zur Bernoulli Ungleichung gefunden werden.

\subsection{Eindeutigkeit} Je nach dem wollen wir zeigen, dass es von einem Objekt mit bestimmten Eigenschaften nur genau eines gibt. Hierzu nimmt man zuerst an, dass es zwei solche Objekte $X_1$ und $X_2$ gibt und zeigt dann aus den gegeben Eigenschaften, dass $X_1 = X_2$ gelten muss. Ein ausführliches Beispiel hierfür ist der zweite Teil vom Beweis \ref{prf_inverse_of_fcn}.

\section{Mengenlehre} \label{cha_settheory}
Wir werden grundsätzlich die \textsc{Cantorsche} Mengenlehre besprechen, auch wenn sie gewisse Definitionslücken besitzt. Die \textsc{Zermelo-Fraenkel} Axiome der Mengenlehre beheben diese Lücken, jedoch sind diese viel komplexer, weswegen wir dieses Thema nur kurz anschneiden werden.

\subsection{\textsc{Cantor}'sche/naive Mengenlehre}
Ob in der Mathematik, der Physik oder einfach allgemein im Leben will man oft mehrere Objekt gruppieren. Z.B. haben wir für Tauben, Meisen und Adler den Überbegriff der Vögel. Oder für $2, 4, 6, 8, ...$ den Begriff der geraden Zahlen. Wir nennen diese Gruppierungen \textbf{Mengen} und definieren sie  folgendermassen:
\begin{definition}{Menge}{}
Eine Menge besteht aus beliebigen unterscheidbaren Elementen und sie ist eindeutig durch seine Elemente bestimmt.

Wir können eine Menge $X$ schreiben als $$X=\{x_1, x_2, ...\}$$ wobei $x_1$ eines der Elemente darstellt. Auch kann eine Menge $Y$ aus den Elementen definiert werden, welche eine Eigenschaft haben, für welche also eine Aussage $A$ gilt: $$Y=\{x \mid A(x)\}$$

Wobei das ''$\mid$'' nun dasjenige aus der Theorie der Mengenlehre ist und nichts mit der Teilbarkeit zu tun hat. Man kann es lesen als ''Menge aus den $x$, für welche $A(x)$ gilt''. Wir schreiben $x \in X$, falls $x$ ein Element von der Menge $X$ ist oder $x \notin X \equiv \neg (x \in X)$, falls $x$ kein Element dieser Menge ist.
\end{definition}
''Unterscheidbar'' in der Definition bezieht sich lediglich darauf, dass wie im obigen Beispiel der geraden Zahlen egal ist, ob die $2$ doppelt, dreifach, ... aufgezählt wird, sie kann entweder eine gerade Zahl sein oder nicht. Da die Menge zudem ''eindeutig durch die Elemente bestimmt'' wird, kommt es auch nicht auf die Reihenfolge der Elemente an. Darum gilt z.B. $\{1 ,2\} =\{2 ,1\} = \{1 ,2 ,1 ,1 , 2\}$

\begin{example}
Man kann die Menge aller Primzahlen wie folgt definieren: $$P = \{ p \mid \forall x \in \N, 1 < x < p: x \nmid p \}$$ Somit gilt z.B. $73 \in P$ und $42 \notin P$
\end{example}

\begin{definition}{spezielle Mengen}{}
Wir nennen die Menge ohne Elementen die \textbf{leere Menge} und schreiben
$$\{\} \qquad \text{ oder } \qquad \emptyset$$
Die \textbf{Potenzmenge} einer Menge $M$ nennen wir die Menge, deren Elemente alle möglichen Teilmengen von $M$ sind. Es ist also eine Menge von Mengen:
$$\mathcal{P}(M) = \{A \mid A \subseteq M\}$$
\end{definition}

\begin{example}[Potenzmenge]
Die Potenzmenge von $\{A,B,C\}$ ist
$$\mathcal{P}(\{A,B,C\}) = \{\emptyset, \{A\}, \{B\}, \{C\}, \{A, B\}, \{A, C\}, \{B, C\}, \{A, B, C\}\}$$
Diese Potenzmenge kann z.B. wie folgt verwendet werden: Seien $A$, $B$ und $C$ die Antwortmöglichkeiten einer Multiple-Choice Frage, für welche 0 bis 3 Antworten korrekt sein können, dann wird die korrekte Lösung eine \textit{Teilmenge} von $\{A,B,C\}$, also ein \textit{Element} aus der Potenzmenge von $\{A,B,C\}$ sein.
\end{example}

Wenn wir uns nun auf wieder zurück auf das Beispiel der Vögel beziehen wollen, dann könnte man sagen, dass die Menge der Vögel zur Menge der Tiere gehört, und diese zur Menge der Lebewesen, etc. Darum wollen wir nun das Konzept der \textbf{Teilmenge} einführen, also einer Menge, deren Elemente alle in einer anderen Menge vorhanden sind:
\begin{definition}{Teilmenge}{}
Eine Menge $A$ heisst \textbf{Teilmenge} von B, falls
$$\forall x : x \in A \implies x \in B$$
Wir schreiben
$$A \subseteq B$$
\end{definition}

\begin{example}[Teilmengen]
Aus der Definition der Teilmengen folgen z.B.:
$$\text{Meisen} \subseteq \text{Vögel} \subseteq \text{Tiere} \subseteq \text{Lebewesen}$$
$$P \subseteq \N \qquad \N \subseteq \N \qquad \N \subseteq \Z \subseteq \Q \subseteq \R \subseteq \C$$
\end{example}

Man kann die Mengen und Teilmengen sehr intuitiv in \textbf{Euler-Venn-Diagrammen} darstellen, in welchen Mengen durch Flächen und deren Elemente durch Punkte in dieser Fläche repräsentiert werden. Eine Teilmenge entspricht somit einer komplett in einer anderen enthaltenen Fläche.

\begin{definition}{Mengenkonstruktionen}{}
Für beliebige Menge $A$ und $B$ definieren wir folgende Operatoren:
\begin{itemize}
    \item \textbf{Durchschnitt}: 
        $$ A \cap B = \{x \mid x \in A \land x \in B\}$$
    \item \textbf{Vereinigung}:
        $$ A \cup B = \{x \mid x \in A \lor x \in B\}$$
    \item \textbf{symmetrische Differenz}:
        $$ A \bigtriangleup B = \{x \mid x \in A \cup B \land x \notin A \cap B\}$$
    \item \textbf{Komplement} oder \textbf{Differenz}:
        $$ A^C = B \setminus A = \{x \mid x \notin A \land x \in B\} = \{x \in B \mid x \notin A\}$$
\end{itemize}
\end{definition}
Aus der Definition für das Komplement folgt, dass $(A^C)^C = A$ gilt. Für das Komplement $A^C$ ist die Menge $B$ üblicherweise eine ''Referenz-Menge'', in welcher $A$ als Teilmenge enthalten ist, z.B. ist $B = \N$ und $A$ die Menge der Primzahlen, sodass $A^C$ die Menge aller zusammengesetzten Zahlen ist.

Im Euler-Venn-Diagramm entspricht der Durchschnitt der Fläche, welche in beiden Mengen enthalten ist und die Vereinigung der Fläche, die in mindestens einem der Mengen enthalten ist; bei der symmetrischen Differenz nur diejenige, die in genau einer Menge enthalten ist. Die Fläche der Differenz von $A$ befindet sich ausserhalb von $A$ und innerhalb von $B$. 

\begin{lemma}{Eigenschaften von $\emptyset$ und $\mathcal{P}(A)$}{}
\begin{align*}
        \{\} &= \emptyset  &  \emptyset &\subseteq A & & & & & &\\
        A \cup \emptyset &= A & A \cap \emptyset &= \emptyset &  A \setminus \emptyset &= A & \emptyset \setminus A &= \emptyset\\
        A &\in \mathcal{P}(A) & \emptyset &\in \mathcal{P}(A)& \mathcal{P}(\emptyset)&= \{\emptyset\}  & \mathcal{P}(\{\emptyset\})&=\{\emptyset, \{\emptyset\}\}
\end{align*}
Beachte hierbei genau, wann $\in$ und wann $\subseteq$ verwendet wird. 
\end{lemma}

Bis jetzt haben wir für Vereinigungen und Schnitte nur zwei Mengen betrachtet. Wenn man es sich überlegt, kann man Vereinigungen und Schnitte auch über beliebig viele Mengen definieren. Für die Formalisierung verwenden wir hierfür Kollektionen oder \textbf{Familien von Mengen}, welche per se nur Mengen sind, welche weitere Mengen als Elemente besitzen:
\begin{definition}{Beliebige Vereinigungen und Schnitte}{}
Sei $\mathcal{A}$ eine Familie von Mengen, dann definieren wir die Vereinigung respektive den Schnitt der Mengen in $\mathcal{A}$ als:
$$\bigcup_{A \in \mathcal{A}} A = \{x \mid \exists A \in \mathcal{A} : x \in A\}$$
$$\bigcap_{A \in \mathcal{A}} A = \{x \mid \forall A \in \mathcal{A} : x \in A\}$$
\end{definition}
\begin{example}[Sieb von \textsc{Eratosthenes}] Die Menge der Primzahlen $P \subseteq \N$ kann auch wie folgt definiert werden:

Sei $A_n = \{2n, 3n, 4n, 5n, ...\} \subseteq \N$, dann gilt:

$$P \cap \{1\} = \Big(\bigcup_{i = 2}^\infty A_i\Big)^C = \bigcap_{i = 2}^\infty (A_i)^C$$
Für die letzte Gleichheit wurden dabei die Gesetze von \textsc{De Morgan} verwendet:
\end{example}

\begin{definition}{\textsc{De Morgan} für Mengen}{}
Für die Teilmengen $P$ und $Q$ von $X$ gilt:
$$(P \cap Q)^C = P^C \cup Q^C$$
$$(P \cup Q)^C = P^C \cap Q^C$$
Für beliebige Schnitte einer Kollektion $\mathcal{A}$ aus Teilmengen von $X$ gilt:
$$\Big(\bigcup_{A \in \mathcal{A}} A\Big)^C = \bigcap_{A \in \mathcal{A}} A^C$$
$$\Big(\bigcap_{A \in \mathcal{A}} A\Big)^C = \bigcup_{A \in \mathcal{A}} A^C$$
\end{definition}

\subsection{Kartesisches Produkt}
Nun haben wir Mengen definiert mit dem Gedanken, dass die Elemente der Mengen jeweils spezifische Objekte sind. Dies ist jedoch nur wenig hilfreich, wenn man z.B. Zuordnungen darstellen will, wenn man also Elemente einer Menge $A$ mit denen aus einer Menge $B$ zuordnen will:

\begin{example} [ ]
Sei $A$ die Menge aller Leute, die die an der ETH sind und $B$ die Menge der Dozenten, welche Vorlesungen anbieten. Dann könnte man die Paare finden wollen, welche zeigen, wer wessen Vorlesungen besucht, wobei natürlich die Dozierenden die Vorlesungen der Kollegen besuchen können. Man kann es sich vorstellen als Tabelle mit den Leuten auf der einen und den Dozenten auf der anderen Achse, in der die Felder entsprechend abgehakt werden. Das heisst, wir wollen Paare bilden aus den Elementen von den Mengen $A$ und $B$. Dabei kommt es aber auch auf die Reihenfolge an, denn wenn Urs die Vorlesung von Gio besucht, heisst das noch lange nicht, dass Gio auch die von Urs bersucht. Hierfür kommt uns das \textbf{kartesische Produkt} zur Hilfe:
\end{example}

\begin{definition}{kartesisches Produkt}{}
Seien $A$ und $B$ Mengen, dann definieren wir das \textbf{kartesische Produkt} $A\times B$ als die Menge aller geordneten Paare:
$$A\times B = \{(a, b) \mid a \in A, b \in B\}$$
Wir nennen die Struktur der Form $(a, b)$ ein \textbf{Tupel}, welches im Gegensatz zur Menge geordnet ist, i.e. $(1,2) \neq (2,1)$. Somit ist $A\times B$ eine Menge aus Tupels mit jeweils 2 Einträgen.
\end{definition}

\begin{example}[kartesische Ebene nach \textsc{Descartes}]
Eine andere sehr intuitive Anwendung des kartesischen Produktes ist die Koordinatenebene, welche wir aus der Mittelschule kennen: Wir wollen spezielle Punkte $P$ der Form $(x,y)$, welche z.B. die Funktionsgleichung $f(x) = y$ eines Graphen erfüllen zu einer Menge (den Graphen) zusammenfassen, wobei $x$ und $y$ jeweils aus der Menge $\R$ ist. Wir schreiben die kartesische Ebene daher als
$$\R \times \R =: \R^2$$
Es wird auch intuitiv schnell klar, dass die Geordnetheit der Paare sehr wichtig ist, da z.B. der Punkt $(1,2)$ der Ebene nicht dem Punkt $(2,1)$ entspricht.
\end{example}

\begin{remark}
(Mengentheoretische Definition des kartesischen Produktes/eines Tupels) Wir haben die geordneten Paare eines kartesischen Produktes als neues Konzept eingeführt, jedoch kann man sie auch als Mengen definieren, ohne eine weitere Theorie einführen zu müssen. Es gibt verschiedene Varianten, dies zu erreichen, eine davon ist zu sagen, dass $A \times B \subseteq \mathcal{P}(\mathcal{P}(A \cup B))$ ist und ein Paar $(x, y) := \{\{x\},\{x, y\}\}$ mit $x \in A, y \in B$ ist. Man bemerkt, dass sich mit dieser Definition $(x, y)$ und $(y, x)$ auch als Mengen unterscheiden, welches auch die primäre Eigenschaft von geordneten Paaren ist.
\end{remark}

\subsection{\textsc{Russel}'s Paradox}
Zwar ist die \textsc{Cantor}'sche Mengenlehre schön und gut im ''Alltagsgebrauch'', aber das genügt den Mathematiker ja nicht. Denn die \textsc{Cantor}'sche Mengenlehre erlaubt die beliebige Konstruktion von Mengen aus Mengen, und wenn man das übertreibt, kann das zu Widersprüchen führen wie in \textsc{Russel}'s Paradox:

Wir wollen die Menge $X$ konstruieren, welche alle Mengen enthält, die sich selber nicht enthalten. So wäre die Menge $\N$ in $X$ enthalten, da deren Elemente nur einzelne natürliche Zahlen sind, aber nie die gesamte Menge $\N$ (als Element) darin enthalten ist. Anders wäre z.B. $\{\emptyset, \{\emptyset, \{\emptyset, \{...\}\}\}\}$ nicht in $X$ enthalten, da es das zweite Element von sich selber ist. Formal ausgedrückt also:
$$X = \{ A \mid \neg(A \in A) \} = \{ A \mid A \notin A \}$$
Das Paradox entsteht nun dadurch, wenn man sich fragt, ob $X$ sich selber enthält. Denn falls $X\in X$, dürfte $X$ nach der Definition nicht in sich enthalten sein. Anders herum, falls $X$ nicht in sich enthalten ist, dann müsste $X$ laut Definition in sich enthalten sein.

Das ist ein Widerspruch, welches zu mindestens einem depressiven Mathematiker und zur Begründung der heute akzeptierten \textsc{Zermelo-Fraenkel}-Axiome der Mengenlehre geführt hat. Darin werden solche Konstruktionen nicht mehr als Mengen sondern als \textit{Klassen} betrachtet, welche hierarchisch über Mengen stehen und solche Widersprüche wie \textsc{Russel}'s Paradox vermeiden.

\section{Abbildungen}
Nun wollen wir eine (eindeutige) Zuordnung von Elementen einer Menge auf die einer anderen definieren, also zum Beispiel die Menge der Studis auf die Menge der Studiengänge, die sie besuchen. Dieses Beispiel ist auch ''eindeutig'' in dem Sinne, dass jeder Student (in der Regel) nur jeweils einen Studiengang belegen kann. Eine solche Zuordnung nennen wir \textbf{Abbildung} oder \textbf{Funktion}:

\begin{definition}{Abbildung}{}
Seien $X, Y$ Mengen, dann nennen wir $f: X \to Y$ eine \textbf{Abbildung} von $X$ nach $Y$, welche jedem Element $x \in X$ das Element $f(x) \in Y$ zuordnet. Wir schreiben:
$$f: X\to Y, x\mapsto f(x)$$
Für die Definition einer Funktion müssen der \textbf{Definitionsbereich} $X$, der \textbf{Wertebereich} oder \textbf{Wertevorrat} $Y$ und die \textbf{Abbildungregel} $x \mapsto f(x)$ für jedes \textbf{Argument} $x \in X$ eindeutig definiert sein. Ansonsten ist die Abbildung nicht \textbf{wohldefiniert}. Ganz kompakt und trotzdem vollständig wäre folgende Schreibweise:
$$x \in X \mapsto f(x) \in Y$$
Wir nennen zwei Abbildungen $f: X \to Y$ und $g: X' \to Y'$ gleich, falls $X = X'$, $Y = Y'$ und $\forall x \in X: f(x) = g(x)$ gilt.

Die Elemente, auf die mindestens ein Element im Definitionsbereich abgebildet wird, bilden das \textbf{Bild} der Abbildung:
$$f(X) = \{ y \mid \exists x \in X: f(x) = y\}$$

Für eine Teilmenge $Y'$ des Wertebereichs $Y$ nennen wir die Menge
$$f^{-1}(Y') = \{x \in X \mid f(x) \in Y' \}$$
das \textbf{Urbild von Y'}
\end{definition}

\begin{remark}
Mengentheoretisch definieren wir Abbildungen von $X$ nach $Y$ als Teilmenge vom kartesischen Produkt $X \times Y$, wobei jedes $x\in X$ in genau einer Relation mit einem Element aus $Y$ ist.
$$f \subseteq X \times Y: \forall x \in X: \exists! y \in Y: x\ f\ y \iff f(x) = y$$
\end{remark}

\begin{example}[ ] \label{ex_example_function} Sei $X = \{\text{Yoel }, \text{Sebi }, \text{Jan}\}$ und $Y =  \{\text{PC-N }, \text{Physik }, \text{BWL}\}$ , dann können wir die ''Studiengangsfunktion'' $s$ wie folgt definieren:
\begin{align*}
s: X &\to Y \\
\text{Yoel} & \mapsto \text{PC-N} \\
\text{Sebi} & \mapsto \text{PC-N} \\
\text{Jan} & \mapsto \text{Physik}
\end{align*}
Das heisst, wir erhalten z.B. $s(\text{Sebi}) = \text{PC-N}$. Das Bild von $s$ ist $\{\text{PC-N }, \text{Physik}\} \subseteq Y$. Das Urbild von $\{\text{PC-N}\}$ ist $\{\text{Yoel}, \text{ Sebi}\}$ und $f^{-1}(\{\text{BWL}\}) = \emptyset$.
\end{example}

\begin{exercise}[Urbild] Sei $f: X \to Y$ eine beliebige Abbildung.
\begin{itemize}
    \item Sei $Y' \subseteq Y$, zeige, dass $f(f^{-1}(Y')) \subseteq Y'$ gilt.
    \item Sei $X' \subseteq X$, zeige, dass $f^{-1}(f(X')) \supseteq X'$ gilt.
\end{itemize}
\end{exercise}

\begin{example}[Erste Abbildungen] \label{ex_important_functions} Hier nun einige wichtige Abbildungen:
\begin{itemize}
    \item Sei $X$ eine Menge, dann ist die \textbf{Identitätsabbildung} $id_X: X \to X$ definiert durch:
    $$ x \mapsto x$$
    \item Seien $X,Y$ zwei Mengen und $y_0 \in Y$, dann ist die \textbf{konstante Abbildung} $f$ definiert durch:
    $$ x \in X \mapsto y_0 \in Y $$
    \item Sei $X$ eine Menge und $X'\subseteq X$ eine Teilmenge davon. Dann ist die \textbf{charakteristische Abbildung} $\mathbbm{1}_{X'}: X \to \{0, 1\}$ definiert durch:
    $$ x \mapsto \begin{cases}0 & x \notin X' \\ 1 & x \in X' \end{cases} $$
    \item Sei $X\times Y$ das kartesische Produkt aus zwei Mengen, dann gibt es die zwei (kanonische) \textbf{Projektionen} $\pi_X$ und $\pi_Y$:
    \begin{align*}
        \pi_X: X\times Y &\to X & \pi_Y: X\times Y &\to Y \\
        (x,y) &\mapsto x & (x,y) &\mapsto y
    \end{align*}    
    Die Projektion entspricht in der 2D-Ebene dem Ablesen der $x$- resp. $y$-Koordinate eines Punktes. D.h. für einen Punkt $P \in \R^2$ gibt $\pi_X(P)$ die $x$-Koordinate zurück. Bei höheren kartesischen Produkten schreiben wir $\pi_i$ ($i\in \N$), um die Projektion entlang des $i$-ten Eintrages zu bezeichnen.
\end{itemize}

\end{example}

\begin{example}[ ] \label{ex_functions}
Etwas interessanter sind die folgenden beiden Funktionen $f$ und $g$:
\begin{align*}
f: \R &\to \R & g: \R_{\geq 0} &\to \R \\
x&\mapsto x^2&x&\mapsto x^2
\end{align*}
In beiden Fällen erhalten wir das Quadrat des Arguments. Da die Definitionsbereiche jedoch nicht gleich sind, gilt $f \neq g$.
\end{example}

Wir erkennen schnell, dass der Definitionsbereich von $g$ eine Teilmenge des Definitionsbereichs von $f$ ist. Das wollen wir entsprechend benennen und definieren daher die 

\begin{definition}{Einschränkung}{}
Die \textbf{Einschränkung} einer Abbildung $f: X \to Y$ auf eine Teilmenge $X' \subseteq X$ ist die Abbildung
\begin{align*}
    f\big|_{X'}: X' &\to Y \\
    x & \mapsto f(x)
\end{align*}
Umgekehrt nennen wir $F$ eine \textbf{Fortsetzung} von $f$, falls $F\big|_X = f$ gilt. $f$ ist also eine Fortsetzung von $f\big|_{X'}$
\end{definition}
\begin{example}[Fortsetzung von Beispiel \ref{ex_functions}] Es gilt also $f\big|_{\R_{\geq 0}} = g$.
\end{example}

\subsection{Verknüpfung von Abbildungen}
Wenn wir zwei Abbildungen $g: X \to Y$ und $f: Y \to Z$ haben, bei denen der Wertebereich von $g$ dem Definitionsbereich von $f$ entspricht, dann kann man das Ergebnis $g(x) \in Y$ direkt der Abbildung $f$ als Argument weitergeben. Diese Aneinanderreihung von Abbildungen nennen wir 

\begin{definition}{Verknüpfung}{}
Seien  $g: X \to Y$ und $f: Y \to Z$ zwei Abbildungen, dann ist die Verknüpfung von $f$ und $g$ gegeben durch:
\begin{align*}
    f \circ g: X &\to Z\\ x &\mapsto f(g(x))
\end{align*}
\end{definition}
\begin{remark}
Man beachte die etwas unintuitive Reihenfolge der Auswertung von $f \circ g$: Es wird zuerst $g$ ausgewertet, bevor $f$ mit dem Ergebnis von $g$ ausgewertet wird. Man wertet also von rechts nach links aus. Eine Eselsbrücke bietet die englische Aussprache ''$f$ of $g$'' für $f\circ g$: Es deutet an, dass die letztere Abbildung zuerst ausgewertet wird.
\end{remark}

\begin{example}[Funktionen in $\R$]
Seien $f: \R \to \R, x \mapsto x^2$ und $g: \R \to \R, x \mapsto x + 1$, dann gilt:
\begin{align*}
    f\circ g &= f(g(x) = (g(x))^2 = (x + 1)^2\\
    g\circ f &= g(f(x) = f(x) + 1 = x^2 + 1
\end{align*}
Wir erkennen also, dass die Verknüpfung \textbf{nicht kommutativ} ist, d.h., dass $f \circ g = g \circ f$ im Allgemeinen \textit{nicht} gilt, auch wenn beide Richtungen definiert sind (wegen der nötigen Übereinstimmung von den Defintions- und Wertebereichen).
\end{example}

Zwar haben wir keine Kommutativität, jedoch können wir folgende Eigenschaften der Verknüpfung zeigen:
\begin{satz}{Assoziativität von Identität für Verknüpfungen}{}
Für die Abbildungen $h: X \to Y$, $g: Y \to Z$ und $f: Z \to A$ gilt: 
\begin{enumerate}
    \item $(f\circ g) \circ h = f \circ (h \circ g) = f \circ h \circ g$
    \item $h \circ id_X = id_Y \circ h = h$
\end{enumerate}
\end{satz}
\begin{proof} Für ein beliebiges $x \in X$ gilt:
\begin{enumerate}
    \item $(f\circ g) \circ h(x) = (f\circ g)(h(x)) = f(g(h(x))) = f((g\circ h)(x) = f \circ (g \circ h)(x)$
    \item $h \circ id_X (x) = h(id_X(x)) = h(x) = id_Y(h(x)) = id_Y \circ h (x)$
\end{enumerate}
\end{proof}

\subsection{Eigenschaften von Abbildungen}
Wie man in einigen Beispielen gemerkt hat, decken einige Abbildungen deren gesamten Wertebereich ab oder andere, welche jedes Element des Wertebereichs genau einmal verwenden. Andere erfüllen beides und wieder andere erfüllen keines dieser Kriterien. Wir wollen diese Eigenschaften benennen und definieren folgende Begriffe:

\begin{definition}{Injektivität, Surjektivität und Bijektivität}{}
Eine Abbildung $f: X \to Y$ nennen wir \textbf{injektiv}, falls für alle $x_1, x_2 \in X$ gilt:
$$f(x_1) = f(x_2) \implies x_1 = x_2$$
Eine Abbildung $f: X \to Y$ nennen wir \textbf{surjektiv}, falls für jedes $y \in Y$ gilt:
$$\exists x \in X: f(x)=y$$
welches äquivalent ist zu:
$$f(X) = Y$$
Eine Abbildung ist \textbf{bijektiv}, falls diese sowohl injektiv als auch surjektiv ist.
\end{definition}

In anderen Worten gibt es für jeden Funktionswert einer injektive Funktion ein eindeutiges Argument. Bei surjektiven Funktionen gibt es für jedes Element im Wertebereich mindestens ein Funktionsargument.

\begin{example}
Daraus folgt, dass $g: x \in \R_{\geq 0} \mapsto x^2 \in \R$ aus Beispiel \ref{ex_functions} injektiv ist, dass $\mathbbm{1}_X': X \to \{0, 1\}$ aus Beispiel \ref{ex_important_functions} für $X' \neq X$ und $X' \neq \emptyset$ surjektiv ist und dass die Identität $id_X$ surjektiv ist.
\end{example}

Wenn eine Abbildung $f$ bijektiv ist, gibt es also für jedes Element $y$ im Wertebereich ein eindeutiges Argument $x$, sodass man eine weitere Funktion definieren kann, welche $y$ eindeutig auf $x$ abbildet. Wir nennen diese Funktion die \textbf{Inverse} oder \textbf{Umkehrabbildung} von $f$:

\begin{definition}{Inverse/Umkehrabbildung}{}
Sei $f: X\to Y$ eine Abbildung, dann nennen wir $g: Y \to X$ die \textbf{Inverse} oder \textbf{Umkehrabbildung}, falls
$$f \circ g = id_Y \qquad \text{ und } \qquad g \circ f = id_X$$
das heisst
$$\forall y \in Y: f(g(y)) = y \qquad \text{ und } \qquad \forall x \in X: g(f(x)) = x$$
gelten.
\end{definition}

Die Umkehrabbildung hat (wie z.T. bereits angesprochen) einige Eigenschaften:

\begin{satz}{Umkehrabbildung und Bijektivität}{Umkehrabbildung_und_Bijektivität}
\begin{enumerate}
    \item $f: X \to Y$ hat genau dann eine Umkehrabbildung, wenn $f$ bijektiv ist.
    \item Falls $f$ bijektiv ist, dann existiert eine eindeutige Umkehrabbildung $f^{-1}$.
\end{enumerate}
\end{satz}

\begin{proof} \label{prf_inverse_of_fcn} Sei $f: X \to Y$.
\begin{enumerate}
    \item ($\Longrightarrow$) Sei $g: Y \to X$ eine Inverse von $f$. Seien $x_1, x_2 \in X$ mit $f(x_1) = f(x_2)$. Durch Anwenden von $g$ auf beiden Seiten erhalten wir aus der Definition von $g$ $x_1 = x_2$, wodurch die Injektivität von $f$ gezeigt ist. Sei nun $y \in Y$, da $f(g(y)) = y$ per Definition gilt, gibt es ein $x=g(y)\in X$, für welches $f(x) =y$ gilt. Dadurch ist $f$ auch surjektiv und somit bijektiv.\\
    ($\Longleftarrow$) Sei nun $f$ bijektiv. Da $f$ surjektiv ist, gibt es für jedes $y \in Y$ ein $x\in X$, sodass $f(x) = y$. Da $f$ injektiv ist, ist dieses $x$ auch eindeutig. Wir setzen nun $g(y) = x$ für alle $y$, wodurch wir eine Abbildung $g: Y \to X$ erhalten, für die $$\forall x \in X, \ \forall y \in Y: y = f(x) \iff x = g(y)$$ gilt. Da daraus $\forall y \in Y: f(g(y)) = y$ und $\forall x \in X: g(f(x)) = x$ folgt, ist $g$ eine Inverse von $f$.
    \item Eindeutigkeit der Inversen: Seien $g_1$ und $g_2$ zwei Inversen. Aus der Definition der Inversen folgt: $f\circ g_1 = id_Y$ und $g_2 \circ f = id_X$. Wir erhalten:
    $$g_1 = id_X \circ g_1 = (g_2 \circ f)\circ g_1 = g_2 \circ (f \circ g_1) = g_2 \circ id_Y = g_2 $$
    Somit haben wir gezeigt, dass die Inverse eindeutig ist.
\end{enumerate}
\end{proof}

Wir schreiben $f^{-1}$ für \textit{die} Umkehrabbildung von $f$, da wir nun gezeigt haben, dass die Umkehrabbildung eindeutig durch die Bedingung $$\forall x \in X, \ \forall y \in Y: y = f(x) \iff x = f^{-1}(y)$$ bestimmt wird.

\subsection{Graph einer Abbildung}
Das Konzept des Graphen einer Funktion $f: X \to Y$ kennen wir bereits aus der Mittelschule: Wir haben zwei Achsen, die $x$-Achse mit den Argumenten aus $X$ und die $y$-Achse mit den Funktionswerten $f(x) \in Y$, welche entsprechend aufgetragen sind. Doch was genau lässt sich das ausdrücken mit den Konzepten, die wir bis jetzt besprochen haben? Der \textbf{Graph} der Funktion ist doch nichts mehr als eine Teilmenge der gesamten Ebene $X \times Y$, welche die Bedingung $f(x) = y$ erfüllen. Wir definieren den Graph einer Funktion wie folgt:

\begin{definition}{Graph}{}
Sei $f: X \to Y$ eine Abbildung, dann ist der \textbf{Graph} $\Gamma$ von $f$ definiert als die Teilmenge von $X \times Y$, für die gilt:
$$\Gamma(f) = \{(x,y) \in X \times Y\mid y = f(x) \}$$
\end{definition}
Der Graph $\Gamma(f)$ ist also die Menge aller Tupels $(x, f(x))$ mit $x \in X$ und $f(x) \in Y$.
\begin{remark}
Nicht jede Teilmenge von $X\times Y$ ist ein Graph einer Abbildung $f$, z.B. die des Einheitskreises, da der Funktionswert $f(0)$ die beiden Werte 1 und -1 annehmen müsste und $f(2)$ undefiniert sein müsste, welches der Definition einer Abbildung widerspricht.
\end{remark}
Aus den Limitationen der Abbildung erkennen wir graphisch, dass eine Teilmenge von $X \times Y$ genau dann ein Graph einer Abbildung sein muss, wenn jede vertikale Linie nur genau einen Punkt mit der Teilmenge gemeinsam hat. Anders gesagt, darf die Projektion der Teilmenge auf die $x$-Achse keine Punkte aufeinander schieben (d.h. injektiv) und sie darf keine Lücken haben (d.h. surjektiv). Wir formulieren deswegen den Satz:

\begin{satz}{Projektion des Graphen}{}
Sei $A$ eine Teilmenge von $X \times Y$, dann ist $A$ genau dann ein Graph einer Abbildung $X \to Y$, wenn $\pi_X\big|_A: A \to X$ bijektiv ist.
\end{satz}

\begin{proof}
($\Longrightarrow$) Sei $A \subseteq X \times Y$ der Graph einer Abbildung $f: X \to Y$, d.h. die Elemente von $A$ sind von der Form $(x, f(x)$. Wir verwenden die Erkenntnis aus Satz \ref{satz:Umkehrabbildung_und_Bijektivität}, um mit einer Inverse von $\pi_X\vert_A$ die Bijektivität zu zeigen. Wir behaupten nun, dass $g: x \in X \mapsto (x, f(x)) \in X \times Y$ die Inverse zu $\pi_X\vert_A$ ist. Wir erhalten aus der Verknüpfung $(g \circ \pi_X\vert_A)(x, f(x)) = g(x) = (x, f(x)$ als auch für $(\pi_X\vert_A \circ g) (x) = \pi_X\vert_A(x, f(x)) = x$ die Identitätsabbildung der jeweiligen Mengen. Daraus schliessen wir nun, dass $g = (\pi_X\vert_A)^{-1}$ gilt und $\pi_X\vert_A$ somit bijektiv ist.

($\Longleftarrow$) Sei nun $\pi_X\vert_A: A \subseteq X \times Y \to X$ eine bijektive Abbildung, d.h. es gibt eine Inverse $g: X \to A$, welche die Form $x \mapsto (g_X(x), g_Y(x))$ hat. Da $\forall x \in X: (\pi_X\vert_A \circ g)(x) = x$ gelten muss, folgt,  dass $g_X(x) = x$ ist, also $g(x) = (x, g_y(x))$. Wir definieren nun die Abbildung $f(x) = g_Y(x)$ für alle $x \in X$, welches eine Abbildung von $X \to Y$ ist. Da $g$ und somit $g_Y$ surjektiv ist, lässt sich jedes Element aus $A$ durch $(x, f(x))$ schreiben. Es folgt die Behauptung, dass $\Gamma(f) = A$ gilt.
\end{proof}
Nun wissen wir, dass die Projektion $\pi_X\vert_{\Gamma(f)}$ invertierbar ist. Daraus können wir nun auch folgern, dass sich aus einem Graphen der Form $\{(x, y) \mid y = f(x)\}$ die Funktion $f$ rekonstruieren lässt:
\begin{lemma}{Konstruktion der Abbildung aus dem Graphen}{} Wir verwenden dafür die Verknüpfung  $\pi_Y \circ (\pi_X\vert_{\Gamma(f)})^{-1}: X \to X \times Y \to Y$, welche ein $x \in X$ wie folgt abbildet:
$$x \stackrel{(\pi_X\vert_{\Gamma(f)})^{-1}}{\mapsto} (x, f(x)) \stackrel{\pi_Y}{\mapsto} f(x)$$
d.h. es gilt $f = \pi_Y \circ (\pi_X\vert_{\Gamma(f)})^{-1}$
\end{lemma}
\begin{remark}
(Mengentheoretische Definition einer Abbildung) Wir haben soeben gesehen, dass sich eine Abbildung eindeutig aus dessen Graphen rekonstruieren lässt. Der Graph ist lediglich eine Teilmenge von $X \times Y$ mit der Eigenschaft, dass $\pi_X\vert_{\Gamma(f)}$ bijektiv ist und beruht daher auf den Axiomen der Mengenlehre. Daraus folgt gleich die mengentheoretische Definition der Abbildungen.
\end{remark}

Wir haben bis jetzt nur Graphen von allgemeinen Abbildungen betrachtet, nun wollen wir aber noch diejenigen von bijektiven Abbildungen besprechen. Bijektive Abbildungen haben eine Inverse und wie man graphisch bereits wissen könnte, entspricht der Graph der Inversen der Spiegelung der ursprünglichen Abbildung entlang der $(x=y)$-Diagonalen. Wir wollen dieser Spiegelung die Funktion $\sigma$ zuweisen, welche natürlicherweise Punkte aus $X \times Y$ nach $Y \times X$ abbildet: $(x,y) \mapsto (y,x)$. Wir wollen mit dem nun den folgenden Satz formulieren:

\begin{satz}{Graph der Inversen}{}
Sei $f: X \to Y$ eine bijektive Abbildung, dann ist $\Gamma(f^{-1})=\sigma(\Gamma(f))$
\end{satz}

\begin{proof}
Wir wissen, dass $\Gamma(f^{-1}) = \{ (y, x) \mid x = f^{-1}(y)\}$ gilt. Da $f$ bijektiv ist, können wir $\Gamma(f^{-1})$ unter Verwendung der Eigenschaft $$\forall x \in X, \ \forall y \in Y: y = f(x) \iff x = f^{-1}(y)$$ schreiben als:
$$\Gamma(f^{-1}) = \{ (y, x) \mid x = f^{-1}(y)\} = \{ (y, x) \mid y = f(x)\} = \sigma(\{ (x, y) \mid y = f(x)\}) = \sigma(\Gamma(f))$$
\end{proof}

\subsection{Mengen von Abbildungen}
Eine einzige Abbildung zu betrachten genügt ja keinem Mathematiker, die wollen natürlich \textit{alle} Abbildungen haben. Darum können wir auch Mengen mit Abbildungen machen. Die einfachste solche Menge bezeichnet einfach alle möglichen Abbildungen zwischen zwei Mengen:

\begin{definition}{Mengen von Abbildungen}{}
Seien $X$ und $Y$ zwei Mengen, dann bezeichnet $Y^X$ die Menge aller Abbildungen, die Elemente aus $X$ nach $Y$ abbilden.
\end{definition}
So ist also $f: X \to Y \in Y^X$. Man muss den Definitionsbereich und Wertebereich von ''oben nach unten'' lesen. Diese Notation wird erst unter Betracht der Kardinalität/Grösse dieser Mengen mehr Sinn ergeben.
\begin{example}[Einige wichtige Mengen von Abbildungen] Wir wollen einige häufig anzutreffende Beispiele geben:
\begin{itemize}
    \item $\R^X = \{\text{alle reellwertige Abbildungen } X \to \R\}$
    \item $Y^{\{1,2\}} =$ \{Menge aller Funktionen, welche durch zwei Elemente aus $Y$ gegeben sind\}\\
    Eine Abbildung $f \in Y^{\{1,2\}}$ kann eindeutig durch das Paar $(f(1), f(2)) \in Y \times Y$ definiert werden. Deswegen ist die Menge $Y^{\{1,2\}}$ auch isomorph\footnote{d.h. es gibt eine Bijektion zwischen den beiden Mengen, die die Struktur beibehält.} zu $Y^2 := Y\times Y$. Aus diesem Grund kann man auch sagen, dass jede Abbildung aus $Y^{\{1,2\}}$ einem Punkt im $Y^2$ und $f(i)$ der Projektion $\pi_i\big((f(1), f(2)\big)$ entspricht. 
     \item $Y^{\{1,2,3\}} =$ \{Funktionen, die durch drei Elemente aus $Y$ gegeben sind\}\\
    Es gilt dasselbe wie oben, nur dass diese Abbildungen $f \in Y^{\{1,2,3\}}$ durch 3 Elemente aus $Y$ eindeutig definiert sind: $(f(1), f(2), f(3))$. Man kann dies weiterführen, dabei verwenden wir $Y^n := Y^{\{1,...,n\}}$, auch bis in die Unendlichkeit, wodurch man schliesslich Folgen erhält:
    \item $Y^\N = \{\text{Folgen in } Y\}$\\
    Eine Folge $a \in Y^\N$ ist eine Abbildung $n \in \N \mapsto a(n)$, die jedem $n \in \N$ einen Wert aus $Y$ zuordnet. Da man $n$ als Index lesen kann, wird im Kontext von Folgen anstelle von $a(n)$ in der Regel $a_n$ geschrieben. Man kann wie vorher $a$ auch eindeutig als Liste/Tupel (von unendlicher Länge) darstellen: $a = (a_1, a_2, a_3, ...)$.
\end{itemize}
\end{example}

\begin{example}[kanonische bijektive Abbildungen] Wir können zudem auch kanonische, also ''natürliche'' bijektive Abbildungen $\Phi$ zwischen Mengen den $Y^m \times Y^n \to Y^{m+n}$, also $\Phi \in (Y^{m+n})^{Y^m \times Y^n}$, definieren:
$$\Phi: (x_1,...,x_m) \times (y_1, ..., y_n) \mapsto (x_1,...,x_m, y_1, ..., y_n)$$
\end{example}

\begin{example}[höhere Kreuzprodukte] Mit der Interpretation von Kreuzprodukten als Mengen von Funktionen kann man auch $Y_1 \times ... \times Y_n$ (für verschiedene $Y_i$) als Menge auffassen:
$$Y_1 \times ... \times Y_n = \{\text{Menge aller Abbildungen } f: \{1,...,n\} \to \bigcup_{i=0}^nY_i\}$$
wobei für alle $f \in Y_1 \times ... \times Y_n$ (unüblicherweise) $f(i) \in Y_i$ gilt, i.e. die Bilder aller Abbildungen dieser Menge ausgewertet bei $i$ sind enthalten in $Y_i$: $$\{f(i) \mid f \in Y_1 \times ... \times Y_n\} \subseteq Y_i$$
\end{example}

Will man nun wissen, wie viele verschiedene Funktionen es gibt, die von der Menge $X$ auf die Menge $Y$ abbilden, so kommt einem die vorher unüblich erscheinende Notation zur Hilfe:

\begin{definition}{Kardinalität von $Y^X$}{}
Seien $X$ und $Y$ endliche Mengen mit jeweils $m$ resp. $n$ Elementen. Da eine Abbildung für jedes der $m$ Elemente in $X$ $n$ verschiedene Wahlmöglichkeiten hat, gibt es $n^m$ verschiedene Abbildungen $X \to Y$, d.h. $Y^X$ hat $n^m$ Elemente. Wir schreiben:
$$\vert Y^X \vert = \vert Y \vert ^{\vert X \vert} = n^m$$
\end{definition}

\section{Relationen}
Wie drücken wir aus, dass zwei Dinge ''ähnlich'' sind oder in einen Verhältnis zueinander stehen? Wir kennen zum Beispiel das Zeichen $=$, welches die Gleichheit ausdrückt oder $\mid$, welches die Teilbarkeit ausdrückt. Der Überbegriff für dieses Konzept nennen wir \textbf{Relationen}. So steht z.B. $2$ in der Teilbarkeitsrelation $|$ mit 4, welches wir als $2\mid4$ schreiben. Wir merken, dass wir für Relationen Paare brauchen, welche in Relation sein können oder nicht. Des Weiteren bemerken wir, dass es geordnete Paare sein müssen, da die Reihenfolge z.B. bei der Teilbarkeit einen Unterschied macht: $2\mid4$ aber $4\nmid 2$. Das schreit förmlich nach der Teilmenge eines kartesischen Produktes:

\begin{definition}{Relation}{}
Wir definieren eine \textbf{Relation} $\sim$ zwischen den Mengen $X$ und $Y$ als Teilmenge von $X \times Y$:
$$\sim \ \subseteq X \times Y $$
Wir schreiben $x\sim y$, falls $(x,y) \in \ \sim$ respektive $x \nsim y$, falls $(x,y) \notin \ \sim$.
Falls $X = Y$ ist, nennen wir $\sim$ auch eine Relation auf $X$.
\end{definition}

\begin{example}[Teilbarkeit auf $\N$] So besteht die Teilbarkeitsrelation $\mid$ auf $\N$ aus Paaren $\in \N^2$, welche der Form $(a,ab)$ mit $a,b \in \N$ sind, z.B.:
$$\mid \ \supseteq \{(5 ,5), (2,4), (7,42), (1,73)\}$$
\end{example}

Wir wollen nun einige Begriffe einführen für Eigenschaften, die Relationen auf einer Menge aufweisen können:

\begin{definition}{Eigenschaften von Relationen}{}
Sei $\sim$ eine Relation auf $X$. Wir nennen $\sim$ ...

...\textbf{reflexiv}, falls für alle $x \in X: x\sim x$ gilt.

...\textbf{symmetrisch}, falls für alle $x, y \in X: x\sim y \iff y \sim x$ gilt.

...\textbf{transitiv}, falls für alle $x, y,z \in X: x\sim y \land y \sim z \implies x \sim z$ gilt.
\end{definition}

Diese Eigenschaften sind unabhängig voneinander, d.h. eine Relation kann eine von 8 Kombinationen davon aufweisen:

\begin{example}[Jede Kombination] \label{ex_all_rel_combs} Wir kennzeichnen die Reflexivität mit R, die Symmetrie mit S und die Transitivität mit T. Wir geben für jede mögliche Kombination aus diesen drei Eigenschaften ein Beispiel für $X = \{1, 2, 3\}$
\begin{align*}
    \text{--}&:  \quad \{(1, 2),(2, 2), (3, 1)\} \\
    \text{T}&:  \quad \{(1, 2),(2, 3), (1, 3)\} \\
    \text{S}&:  \quad \{(1, 4),(2, 1),(1,3),(3,1)\} \\
    \text{ST}&:  \quad \{(1,1), (1, 3),(3, 1), (3, 3)\} \\
    \text{R}&:  \quad \{(1, 1),(2, 2), (3, 3), (1,2), (3,1)\} \\
    \text{RT}&:  \quad \{(1, 1),(2, 2), (3, 3), (1,2), (2,3), (1,3)\} \\
    \text{RS}&:  \quad \{(1, 1),(2, 2), (3, 3), (1,2), (2,1), (1,3), (3, 1)\} \\
    \text{RST}&:  \quad \{(1, 1),(2, 2), (3, 3), (1,2), (2,1)\} 
\end{align*}
\end{example}

Es kann je nach dem sehr hilfreich für die Intuition sein, sich die Relationen als Tabellen und die Eigenschaften als Muster in diesen Tabellen vorzustellen.

\begin{example}[Jede Kombination als Tabelle] 
    \newcommand{\h}{$\bullet$} \newcommand{\p}{$\ \ $}
    Die selben Relationen aus vorherigem Beispiel \ref{ex_all_rel_combs} als Tabellen dargestellt. Die Zeilen entsprechen dem ersten Element und die Spalten dem zweiten. Falls das jeweilige Paar in der Relation ist, dann ist es mit einem  \h$\ $  markiert:
\begin{table}[!htbp]
    \centering
    \begin{tabular}{c|c c c}
        --   &   1   &   2   &   3 \\
        \hline
     \p 1\p &       &   \h    &       \\
        2   &       &   \h    &       \\
        3   &   \h    &       &       \\
    \end{tabular}\quad
    \begin{tabular}{c|c c c}
        T   &   1   &   2   &   3 \\
        \hline
     \p 1\p &       &   \h    &   \h    \\
        2   &       &       &    \h   \\
        3   &       &       &       \\
    \end{tabular}\quad
    \begin{tabular}{c|c c c}
        S   &   1   &   2   &   3 \\
        \hline
     \p 1\p &       &   \h    &   \h    \\
        2   &  \h     &       &     \\
        3   &   \h    &       &       \\
    \end{tabular}\quad
    \begin{tabular}{c|c c c}
        ST  &   1   &   2   &   3 \\
        \hline
     \p 1\p &   \h    &       &    \h    \\
        2   &       &       &        \\
        3   &   \h    &       &   \h    \\
    \end{tabular}
\end{table}

\begin{table}[!htbp]
    \centering
    \begin{tabular}{c|c c c}
        R   &   1   &   2   &   3 \\
        \hline
     \p 1\p &   \h    &   \h    &       \\
        2   &       &   \h    &      \\
        3   &  \h     &       &    \h   \\
    \end{tabular}\quad
    \begin{tabular}{c|c c c}
        RT  &   1   &   2   &   3 \\
        \hline
     \p 1\p &   \h  &   \h  &   \h   \\
        2   &       &   \h  &   \h   \\
        3   &       &       &   \h   \\
    \end{tabular}\quad
    \begin{tabular}{c|c c c}
        RS  &   1   &   2   &   3 \\
        \hline
     \p 1\p &   \h  &   \h  &   \h   \\
        2   &   \h  &   \h  &        \\
        3   &   \h  &       &   \h   \\
    \end{tabular}\quad
    \begin{tabular}{c|c c c}
        RST &   1   &   2   &   3 \\
        \hline
     \p 1\p &   \h  &   \h  &        \\
        2   &   \h  &   \h  &        \\
        3   &       &       &   \h   \\
    \end{tabular}
\end{table}
Einfach zu erkennen sind Reflexivität (Diagonale ist belegt) und die Symmetrie (Tabelle lässt sich entlang der Diagonalen spiegeln). Symmetrie und Transitivität führt zu ''Quadraten'' (ggf. unter Umordnung der Elemente in den Zeilen und Spalten, hier z.B. $2 \leftrightarrow 3$). Dasselbe gilt auch für Relationen mit alle drei Eigenschaften, jedoch ist die Diagonale immer Teil eines ''Quadrates'' (auch ggf. unter Umordnung der Elemente, hier nicht nötig).
\end{example}

\subsection{Äquivalenzrelation}

\begin{definition}{Äquivalenzrelation}{}
Eine \textbf{Äquivalenzrelation} ist Relation auf einer Menge $X$, welche die Reflexivität, Symmetrie als auch die Transitivität erfüllt.
\end{definition}

Wir werden sehen, dass solche Äquivalenzrelationen ihrem Namen gerecht werden und eben Äquivalenzen, also eine gewisse ''Gleichheit'' bezeichnen. 

\begin{definition}{Äquivalenzklasse und Quotientenmenge}{}
Sei $\sim$ eine Äquivalenzrelation auf der Menge $X$, dann definieren wir die \textbf{Äquivalenzklasse} von $x \in X$ durch
$$[x]_\sim = \{y \in X \mid x \sim y\}$$
Wir nennen $x$ in diesem Fall den \textbf{Repräsentanten} der Äquivalenzklasse $[x]_\sim$

Die Menge der Äquivalenzklassen von $\sim$ auf $X$ nennen wir dann die \textbf{Quotientenmenge} von $X$ modulo $\sim$ und schreiben:
$$X\big/\!\sim \ = \{[x]_\sim \mid x \in X\}$$
\end{definition}
Die Äquivalenzklasse von $x$ ist also die Menge der Elemente, welche alle in Relation zu $x$ stehen. Wir können zeigen, dass diese auch zueinander in Relation stehen müssen: Seien $y, z \in [x]_\sim$, d.h. es gelten $x\sim y$ und $x \sim z$. Da $\sim$ transitiv ist, gilt auch $y \sim x$, also folgt aus der Transitivität $y \sim z$ und aus der Symmetrie $z \sim y$. Somit stehen alle in $[x]_\sim$ enthaltenen Elemente in Relation zueinander und sind sozusagen ''gleich''.

Wir können zudem noch eine stärkere Aussage zeigen, nämlich dass für eine Äquivalenzrelation $\sim$ auf $X$ mit $x,y \in X$ folgendes gilt:
\begin{lemma}{Eigenschaften von Äquivalenzklassen}{prop_of_equivclass}
Ist $\sim$ eine Äquivalenzrelation, dann sind folgende Ausdrücke äquivalent:
\begin{align*}
   (i)\quad x &\sim y & (ii)\quad [x]_\sim &= [y]_\sim & (iii)\quad [x]_\sim \cap [y]_\sim &\neq \emptyset
\end{align*}
\end{lemma}

\begin{proof} Wir zeigen die Implikationen in eine Richtung:

$(i) \Longrightarrow (ii)$: Sei $z \in [x]_\sim$, dann gilt $z\sim x$. Aus der Behauptung folgt aus der Transitivität direkt $z \sim y$, also $z \in [y]_\sim$. Da $z$ beliebig war, folgt $[x]_\sim \subseteq [y]_\sim$. Die andere Richtung wird ähnlich gezeigt, wodurch wir $[x]_\sim = [y]_\sim$ erhalten.

$(ii) \Longrightarrow (iii)$: Da die Mengen gleich sind und nicht-leer wegen der Reflexivität. Daraus folgt direkt, dass der Schnitt nicht-leer ist.

$(iii) \Longrightarrow (i)$: Da der Schnitt nicht-leer ist, gibt es mindestens ein Element $z \in [x]_\sim$ und $z \in [y]_\sim$, d.h. es gelten $x \sim z$ und $z \sim y$. Aus der Transitivität folgt $x \sim y$.

Da wir nun die drei Implikationen im Kreis gezeigt haben, gilt die Äquivalenz zwischen den Aussagen.
\end{proof}

Man kann sich langsam ausmalen, was Äquivalenzklassen auf einer Menge $X$ mit $X$ machen: Die Menge wird aufgeteilt in disjunkte, also nicht-überlappende Teilmengen. Denn jedes Element muss wegen der Reflexivität in einer Äquivalenzklasse sein. Sobald sich zwei Mengen überlappen, folgt daraus, dass alle Elemente der beiden Mengen auch zueinander in Relation stehen, also in der selben Äquivalenzklasse sind. Anders herum könnte man dann auch aus einer Aufteilung von $X$ eine Äquivalenzrelation definieren. Bevor wir die Brücke zu diesem Konzept schlagen können, wollen wir noch kurz den Begriff der \textbf{Partition} einführen:

\begin{definition}{Partition}{}
Sei $X$ eine Menge und $\cP$ eine Familie von nicht-leeren, paarweise disjunkten Teilmengen von $X$, sodass
$$\bigsqcup_{P \in \cP}P = X$$
gilt, dann ist $\cP$ eine \textbf{Partition} von $X$.
\end{definition}

Unsere Behauptung für Äquivalenzrelationen und Partitionen ist also folgende:

\begin{satz}{Äquivalenzrelationen und Partitionen}{}
Für eine Menge $X$ entsprechen Äquivalenzrelationen und Partitionen einander wie folgt: Für eine Äquivalenzrelation $\sim$ lässt sich die Partition $\cP_\sim$ finden:
$$\cP_\sim = \{[x]_\sim \mid x \in X \}$$
und für eine Partition $\cP$ lässt sich die Äquivalenzrelation $\sim_\cP$ finden:
$$x \sim_\cP y \iff \exists P \in \cP: x \in P \land y \in P$$
Des Weiteren soll diese Konstruktionen invers zueinander sein:
$$\cP_{(\sim_\cP)} = \cP \text{ und } \sim_{(\cP_\sim)} =\ \sim$$
\end{satz}
\begin{proof} Sei $\sim$ eine Äquivalenzrelation und sei $\cP_\sim$ so definiert wie oben, dann gilt für die Vereinigung $\bigcup_{P \in \cP_\sim} P = X$, da wegen der Reflexivität für jedes $x \in X: x \in [x]_\sim \in \cP_\sim$ gilt. Die Disjunktheit aller $P \in \cP_\sim$ folgt aus der Negation von Lemma \ref{lem:prop_of_equivclass}. Somit ist $\cP_\sim$ wie behauptet eine Partition.

Sei nun $\cP$ eine Partition und $\sim_\cP$ eine Relation wie oben definiert. Wir wollen zeigen, dass $\sim_\cP$ eine Äquivalenzrelation ist. Jedes Element $x$ von $X$ ist nach der Definition einer Partition genau einer Menge von $\cP$ enthalten. Aus der Definition von $\sim_\cP$ folgt, dass somit jedes $x$ mit sich selber in Relation steht, somit ist $\sim_\cP$ reflexiv. Sei nun $x \sim_\cP y$. Aufgrund der Kommutativität von $\land$ folgt direkt $y \sim_\cP x$, also ist $\sim_\cP$ symmetrisch. Seien nun $x \sim_\cP y$ und $y \sim_\cP z$. Also gilt $x,y \in P_1$ und $y,z \in P_2$. Da $y \in P_1 \cap P_2$ ist, die Mengen von $\cP$ jedoch paarweise disjunkt sind, gilt $P_1 = P_2$ und somit auch $x,z \in P_{1,2}$, woraus $x \sim_\cP z$ und die Transitivität folgt. $\sim_\cP$ ist somit eine Äquivalenzrelation.

Um die letzten Behauptungen zu zeigen, setzen wir die Definitionen ineinander ein: Für eine Partition $\cP$ gilt $\cP_{(\sim_\cP)} = \{[x]_{\sim_\cP} \mid x \in X\} = \{\{y \in X \mid  \exists P \in \cP: x \in P \land y \in P \} \mid \ x \in X\}$. Da $\cP$ eine Partition ist, kann jedem $x \in X$ einem bestimmten $P \in \cP$ zugeordnet werden. Daraus folgt durch Umordnung der Quantoren $\{\{y \in X \mid y \in P \} \mid x \in X: \exists! P \in \cP: x \in P\}$, welches äquivalent ist zu $\{\{y \in X \mid y \in P \} \mid P \in \cP\} = \{P \mid P \in \cP\} = \cP$.

Umgekehrt sei $\sim$ eine Äquivalenzrelation, dann gilt für beliebige $x,y \in X$ : $x \sim_{(\cP_\sim)} y \iff \exists P \in \cP_\sim: x \in P \land y \in P \iff \exists P \in \{[x]_\sim \mid x \in X \}: x \in P \land y \in P $. Da $x$ wegen der Reflexivität und der Disjunktheit aus Lemma \ref{lem:prop_of_equivclass} selber genau nur in $[x]_\sim$ enthalten ist, kann die Aussage nur dann wahr sein, wenn $x \in P$ gilt, also genau dann, wenn $P = [x]_\sim$. Darum können wir auch $x \sim_{(\cP_\sim)} y \iff y \in [x]_\sim $ schreiben. Daraus folgt direkt, dass $y \sim x$ gilt und die Behauptung ist gezeigt.

\textit{Ein etwas anderer Beweis lässt sich im Skript in Abschnitt 1.62 finden.}
\end{proof}

Was lernen wir daraus? Anders wie allgemeine Relationen auf $X$, können Äquivalenzrelationen auch eindeutig als Partition $\cP$ der Menge $X$ definiert werden. Übrigens ist dann die Quotientenmenge $X/\!\sim$ einer Äquivalenzrelation, welche wir jetzt noch nicht gross erwähnt haben, sehr ähnlich zur Partition $\cP_\sim$: Beide enthalten die Mengen der Elemente, welche zueinander in Relation stehen. Man kann sich beide gut als Menge der verschiedenen Gleichheiten von $\sim$ vorstellen.

\subsection{Teilordnung} \label{cha_partial_order}
Je nach Anwendung ist es praktisch, wenn man eine Menge sortieren kann. So kann man z.B. die Menge der Münzen und Noten einer Währung nach ihrem Wert sortieren. Hierfür definieren wir eine Relation auf der zu ordnenden Menge, welche wir \textbf{Teilordnung} nennen:

\begin{definition}{Teilordnung}{}
Sei $\preceq$ eine Relation auf der Menge $X$, wir nennen $\preceq$ eine \textbf{Teilordnung}, falls sie für $x,y,z \in X$ folgende Eigenschaften aufweist:
\begin{enumerate}
    \item Reflexivität ($x \preceq x$)
    \item \textbf{Antisymmetrie} ($x \preceq y \land y \preceq x \implies x = y$)
    \item Transitivität ($x \preceq y \land y \preceq z \implies x \preceq z$)
\end{enumerate}
\end{definition}
\begin{example} \label{ex_partial_order}
Wir erkennen schnell, dass für eine Menge $X$ die Relation $\subseteq$ auf $\cP(X)$ alle Bedingungen für Teilordnungen erfüllt. Ein anderes, uns bekanntes Beispiel ist die Relation $\geq$ auf $\N$, wobei die mengentheoretisch durch $a \geq b \iff I_a \supseteq I_b \iff \{1, ..., a\} \supseteq \{1,...,b\}$ definiert ist. 
\end{example}
\begin{definition}{lineare Ordnung} Falls eine Teilordnung $\preceq$ auf $X$ zudem noch
\begin{enumerate}[resume]
    \item $\forall x,y \in X : x \preceq y \lor y \preceq x$
\end{enumerate}
erfüllt, dann nennen wir sie eine \textbf{lineare} oder \textbf{totale Ordnung}.
\end{definition}
Mit dieser Eigenschaft wird nun garantiert, dass jedes Paar aus $X$ in genau einer Richtung in Relation steht.
\begin{example}[Fortsetzung Beispiel \ref{ex_partial_order}]
Somit ist $\subseteq$ auf einer Potenzmenge keine totale Ordnung. Sei $X = \{1, 2\}$, dann ist $\cP(X) = \{\emptyset, \{1\}, \{2\}, \{1,2\}\}$. Für $\preceq$ haben wir also folgende Menge an Relationen: 
\begin{align*}
    \big\{\quad (\emptyset&, \{1\}),& (\emptyset,& \{2\}),& (\emptyset,& \{1, 2\}),& (\{1\},& \{1, 2\}),& (\{2\},& \{1, 2\})\quad \big\}
\end{align*}
Wir sehen also, dass weder $(\{1\}, \{2\})$ noch $(\{2\}, \{1\})$ in Relation stehen, welches einer linearen Ordnung widerspricht. $\geq$ auf $\N$ hingegen ist eine lineare Ordnung.
\end{example}

\section{Mächtigkeit}

\todo{Kardinalität, Gleichmächtig, ''Schmächtig'', Cantor-Schroeder-Bernstein, Cantor: $|\mathcal{P}(A)| > |A|$}

\section{Algebraische Strukturen}
Wir brauchen beim Rechnen mit Zahlen gar nicht zu überlegen, ob $a + b$ und $b + a$ dasselbe ist. Für uns sind diese und einige andere Eigenschaften schon praktisch selbstverständlich, jedoch müssen wir uns bewusst sein, dass diese Mengen bereits spezielle Strukturen hat, die diese Operationen erlauben.

In den folgenden Abschnitten wollen wir zeigen, wie \textbf{Verknüpfungen}\footnote{Verknüpfungen einer Menge $A$ sind binäre Abbildungen von $A \times A$ nach $A$.} auf die Elemente einer Menge $A$ zu wirken haben, um sie nach ihrer Struktur klassifizieren zu können. Hierfür zuerst einige Begriffe zu Eigenschaften von Verknüpfungen:

\begin{definition}{Assoziativität, Kommutativität}{}
Sei $\bullet: A^2 \to A$ eine Verknüpfung auf $A$ und $a, b, c$ beliebige Elemente aus $A$.

Wir nennen $\bullet$ \textbf{assoziativ}, falls $a \bullet (b \bullet c) = (a \bullet b) \bullet c$ gilt.

Wir nennen $\bullet$ \textbf{kommutativ}, falls $a \bullet b = b \bullet a$ gilt.
\end{definition}

\subsection{Gruppe}
\begin{definition}{Gruppe}{}
Eine \textbf{Gruppe} $(G, n, \bullet)$ ist eine Menge $G$ versehen mit dem \textbf{neutralen Element} $n$ und einer Verknüpfung $\bullet: G \times G \to G$, für die gilt:
\begin{enumerate}
    \item $\bullet$ ist assoziativ
    \item $\forall g \in G : n \bullet g = g \bullet n = g$ \hfill (\textit{Neutralelement})
    \item $\forall g \in G \ \exists g' \in G : g \bullet g' = g' \bullet g = n$ \hfill (\textit{Inverse})
\end{enumerate}
Falls  $\bullet$ zudem kommutativ ist, nennen wir $(G, \bullet, n)$ eine \textbf{abelsche Gruppe}.
\end{definition}

\begin{example}[Bijektive Funktionen als Gruppe] Sei $A$ eine Menge und $G = \{f \in A^A \mid f \text{ bijektiv}\}$, dann ist $(G, \circ, id_A)$ eine Gruppe:

Die Assoziativität der Gruppe folgt aus der Assoziativität der Verknüpfung $\circ$ von Funktionen, die Eigenschaften des Neutralelementes aus den der Identitätsabbildung. Da alle $f\in G$ auch bijektiv sind, lässt sich auch die Inverse $f^{-1}$ finden, für welche per Definition $f \circ f^{-1} = f^{-1} \circ f = id_A$ gilt. Da $\circ$ im Allgemeinen nicht kommutativ ist, ist $(G, \circ, id_A)$ keine abelsche Gruppe.

Einen Spezialfall erhalten wir, wenn $A = \{1, ..., n\}$, dann nennen wir die Menge aus bijektiven Funktionen in $A^A$ die \textbf{n-te symmetrische Gruppe} $S_n$. Deren Elemente nennen wir \textbf{Permutationen}. Dieses sind Abbildungen, die $n$ Elementen ''vertauschen''. $S_n$ ist nicht kommutativ für $n \geq 3$
\end{example}

\begin{example}[abelsche Gruppe] Wir können schnell verifizieren, dass $(\Z, +, 0)$ eine abelsche Gruppe ist. Hierfür verwenden wir $-a$ als Inverse von $a$. $(G, \cdot, 1)$ ist hingegen keine Gruppe, da z.B. $2 \in \Z$ kein Inverses in $\Z$ besitzt.
\end{example}

\begin{satz}{Eindeutigkeit des Neutralelements und der Inversen}{} Sei $(G, \bullet, n)$ eine Gruppe und $g \in G$, dann
\begin{enumerate}
    \item ist die Inverse $g^{-1}$ ist eindeutig durch $g$ gegeben.
    \item ist das Neutralelement ist eindeutig.
\end{enumerate}
\end{satz}

\begin{proof} {\ }
\begin{enumerate}
    \item Seien $g^{-1}_1, g^{-1}_2 \in G$ zwei Inversen von $g$: $g^{-1}_1 \bullet g = n =  g \bullet g^{-1}_2$. Dann erhalten wir:
    $$g^{-1}_1 = g^{-1}_1 \bullet n = g^{-1}_1 \bullet (g \bullet g^{-1}_2) = (g^{-1}_1 \bullet g) \bullet g^{-1}_2 = n \bullet g^{-1}_2 = g^{-1}_2$$
    \item Seien $n,n'$ zwei Neutralelemente: $g \bullet n = n' \bullet g = g$. Für $g = n$ erhalten wir:
    $$n' = n' \bullet n = n$$
\end{enumerate}

\end{proof}

\subsection{Ring}
Auch wenn wir die Struktur des \textbf{Rings} weniger verwenden werden, wollen wir den vollständigkeitshalber definieren:
\begin{definition}{Ring}{}
Ein \textbf{Ring} $(R, +, \cdot,0)$ ist eine Menge $R$ mit zwei Operationen $+,\cdot: R \times R \to \R$ und $0 \in R$, für die gilt:

\begin{enumerate}
    \item $(R, +, 0)$ ist eine abelsche Gruppe
    \item $\cdot$ ist assoziativ
    \item Es gilt das \textbf{Distributivgesetz}: 
    \begin{align*}
        \forall a,b,c \in R: (a + b) \cdot c &= a \cdot c + b \cdot c\\
                             c \cdot (a + b) &= c \cdot a + c \cdot b
    \end{align*}
\end{enumerate}
Falls ein Ring ein Element $1$ mit der Eigenschaft $a \cdot 1 = 1 \cdot a = a$ für alle $a \in R$ hat, dann nennen wir ihn einen \textbf{Ring mit eins}.

Falls  $\cdot$ kommutativ ist, nennen wir $(G, +, \cdot, 0)$ einen \textbf{kommutativen Ring}.
\end{definition}
\begin{example}[ ] $(\Z, +, \cdot, 0, 1)$ ist ein kommutativer Ring mit eins:
\begin{itemize}
    \item $+$ ist assoziativ, kommutativ mit Neutralelement $0$
    \item $\cdot$ ist assoziativ, kommutativ mit Neutralelement $1$
    \item Es gilt das Distributivgesetz.
    \item Für jedes $a \in \Z$ gibt es ein (eindeutiges) Element $(-a) \in \Z$, sodass $a + (-a) = 0$ gilt.
\end{itemize}
\end{example}

\subsection{Körper}
Nun kommen wir zur wichtigsten Struktur der Analysis, welche wir von den rationalen und reellen Zahlen kennen, weswegen wir nun etwas ausführlicher vorgehen werden mit den Definitionen:
\begin{definition}{Körper}{}
Ein \textbf{Körper} [field] $(K, +, \cdot,0, 1)$ ist eine Menge $R$ mit zwei Operationen $+,\cdot: K \times K \to K$ und $0, 1 \in K$, für die gilt:

\textbf{$(K, +,0)$ bildet eine kommutative Gruppe:}
\begin{enumerate}[label={K\arabic*)}]
    \item $\forall a \in K: 0+a = a + 0 = a$ \hfill (\textit{Nullelement})
    \item $\forall a \in K, \ \exists (-a) \in K: a + (-a) = (-a) + a = 0$ \hfill (\textit{additive Inverse})
    \item $+$ ist assoziativ
    \item $+$ ist kommutativ
\end{enumerate}
\textbf{$(K^\times, \cdot,1)$ bildet eine kommutative Gruppe. $K^\times := K\setminus\{0\}$}
\begin{enumerate}[resume, label={K\arabic*)}]
    \item $\forall a \in K^\times: 1 \cdot a = a \cdot 1 = a$ \hfill (\textit{Einselement})
    \item $\forall a \in K^\times \ \exists a^{-1} \in K^\times: a \cdot a^{-1} = a^{-1} \cdot a = 1$ \hfill (\textit{multiplikative Inverse})
    \item $\cdot$ ist assoziativ
    \item $\cdot$ ist kommutativ
\end{enumerate}
\textbf{Kompatibilität von $+$ und $\cdot$}
\begin{enumerate}[resume, label=K\arabic*)]
    \item $\forall a,b,c \in K: (a + b) \cdot c = a \cdot c + b \cdot c$ \hfill (\textit{Distributivgesetz})
\end{enumerate}
\end{definition}

\begin{remark}
Wir nennen einen Körper \textbf{angeordnet}, wenn Relation wie $\leq$ darauf definiert werden kann, welche die Elemente der Menge der ''Grösse'' nach ordnet. \textbf{Vollständig} nennen wir Körper, die keine ''Lücken'' haben. Somit ist $\Q$ ein angeordneter Körper, $\R$ ein vollständig angeordneter Körper und $\C$ ein vollständiger  Körper.
\end{remark}

Körper haben mit all diesen Vorgaben neben den von der Gruppen und Ringen ''geerbten'' auch noch eine weitere wichtige Eigenschaft, die

\begin{definition}{Nullteilerfreiheit}{} Sei $K$ ein Körper (mit $+$ und $\cdot$) und $a,b \in K$, dann gilt:
$$a \cdot b = 0 \iff a = 0 \lor b = 0$$
\end{definition}

\begin{proof}
Seien $x,y$ Elemente eines Körpers und gelte $x \cdot y = 0$. Sei $x \neq 0$, dann existiert $x^{-1}$. 
Es folgt $0 = x^{-1} \cdot 0 = x^{-1} \cdot (x \cdot y) = (x^{-1} \cdot x) \cdot y = y$, also $y = 0$. Es gilt also $x=0$ oder $y=0$.
\end{proof}

Diese Eigenschaft ist im Wesentlichen das, was einen Körper ausmacht: Sie besagt, dass es keine ''Nullteiler'' hat, also dass keine zwei Elemente $a,b$ in $K^\times$ gibt, welche $a \cdot b = 0$ ergeben und somit Teiler von $0$ wären, die selber nicht gleich $0$ sind. Die Nullteilerfreiheit erlaubt es uns, aus Ringen Körper zu konstruieren (siehe Konstruktion von $\Q$ aus $\Z$ in Abschnitt \ref{cha_rationals}). 

Aus den Körperaxiomen kann man dann noch folgende Rechenregeln herleiten:
\begin{satz}{Rechenregeln im Körper}{} Sei $(K, +, \cdot,0, 1)$ ein Körper und $x,y,z \in K$, dann gilt:
\begin{enumerate}[label=\alph*)]
    \item $-(-x) = x$ \hfill (\textit{Eindeutigkeit der add. Inversen})
    \item $x+z = y+z \implies x = y$\hfill (\textit{additives Kürzen})
    \item $-0 = 0$
    \item $-(x+y) = (-x) + (-y)$
    \item $x\cdot 0 = 0$
    \item $(-1)x = -x$
    \item $z \neq 0: xz = yz  \neq 0 \implies x = y$\hfill (\textit{multiplikatives Kürzen})
    \item $x \neq 0: (x^{-1})^{-1} = x$ \hfill (\textit{Eindeutigkeit der mult. Inversen})
    \item $(-x)(-y) = xy$
    \item $(-x)y = x(-y) = -(xy)$
    \item $(x + y)(x - y) = x^2 - y^2$ mit $x^2 := x\cdot x$ und $x - y := x + (-y)$
\end{enumerate}
\end{satz}
\begin{proof}\ 
\begin{enumerate}[label=\alph*)]
    \item Folgt aus den Gruppenaxiomen von $(K, +, 0)$.
    \item $x \stackrel{K1, K2}{=} ((-z) + z) + x \stackrel{K3}{=} -z + (z +x)\stackrel{b)}{=} -z + (z + y) \stackrel{K3}{=} (-z + z) + y \stackrel{K2, K1}{=} y $
    \item $0 \stackrel{K1, K2}{=} 0 + (0 + (-0)) \stackrel{K3}{=} (0 + 0) + (-0) \stackrel{K1, K1}{=} -0$
    \item $0 \stackrel{K1, K2}{=} (-x)+x+(-y)+y \stackrel{K3, K4}{=} (x+y) + (-y) + (-x) \stackrel{K2}{\implies} (-y) + (-x) = -(x+y)$
    \item $x\cdot0 \stackrel{K1}{=} x(0+0) \stackrel{K9}{=} x\cdot 0 + x \cdot 0 \stackrel{b)}{\implies} 0 = x \cdot 0$ 
    \item $0 \stackrel{e)}{=} x \cdot 0 \stackrel{K2}{=} x(1 + (-1)) \stackrel{K9, K5}{=} x + (-1)x \stackrel{K2}{\implies} -x = (-1)x$
    \item $x \stackrel{K5, K6}{=} (z^{-1} z)x\stackrel{K7, K8}{=}z^{-1}(xz)\stackrel{g)}{=}z^{-1}(yz)\stackrel{K7, K8}{=}(z^{-1}z)y\stackrel{K5, K6}{=}y$
    \item Folgt aus den Gruppenaxiomen von $(K^\times, \cdot, 1)$.
    \item $(-x)(-y) \stackrel{f)}{=}(-1)x(-1)y\stackrel{K7}{=}(-1)(-1)xy\stackrel{f)}{=}-(-1)xy\stackrel{a)}{=}xy$
    \item $(-x)y \stackrel{f)}{=}((-1)x)y\stackrel{K8}{=}(-1)(xy)\stackrel{f)}{=}-(xy) \rightsquigarrow x(-y)$
    \item $(x+y)(x-y) \stackrel{K9}{=} (x+y)x-(x+y)y \stackrel{K9, K9}{=} (x^2+yx)-(xy-y^2) \stackrel{K4, K7}{=} x^2+(xy-xy)-y^2 \stackrel{K2, K1}{=} x^2-y^2$
\end{enumerate}
\end{proof}

\begin{exercise}[Gut für das Verständnis aller Körperaxiome] Zeige die Rechenregeln für Körper. Warum hat 0 keine multiplikative Inverse?
\end{exercise}

\begin{example}[Endliche Körper] \textsc{Galois} hat gezeigt, dass $\mathbb{F}_q = \{0,...,q-1\}$ mit $+, \cdot$ für Primzahlpotenzen $q = p^n$, $p$ prim und $n\in N$, eindeutige Körper bilden (bis auf Isomorphismus).
Für $n = 1$ erhalten wir die uns bekannte Moduloarithmetik. Wir können $+$ und $\cdot$ z.B. auf $\mathbb{F}_2$ und $\mathbb{F}_5$ tabellarisch zeigen:

\begin{table}[!htbp]
    \centering
    \begin{tabular}{c|c c}
        $+$       & $\bar{0}$ & $\bar{1}$\\
        \hline
        $\bar{0}$ & $\bar{0}$ & $\bar{1}$ \\
        $\bar{1}$ & $\bar{1}$ & $\bar{0}$
    \end{tabular}
    \qquad
    \begin{tabular}{c|c c}
        $\cdot$       & $\bar{0}$ & $\bar{1}$\\
        \hline
        $\bar{0}$ & $\bar{0}$ & $\bar{0}$ \\
        $\bar{1}$ & $\bar{0}$ & $\bar{1}$
    \end{tabular}
\end{table}
\begin{table}[!htbp]
    \centering
    \begin{tabular}{c|c c c c c}
        $+$       & $\bar{0}$ & $\bar{1}$& $\bar{2}$ & $\bar{3}$ & $\bar{4}$\\
        \hline
        $\bar{0}$ & $\bar{0}$ & $\bar{1}$& $\bar{2}$ & $\bar{3}$ & $\bar{4}$\\
        $\bar{1}$ & $\bar{1}$& $\bar{2}$ & $\bar{3}$ & $\bar{4}$ & $\bar{0}$\\
        $\bar{2}$& $\bar{2}$ & $\bar{3}$ & $\bar{4}$ & $\bar{0}$ & $\bar{1}$\\
        $\bar{3}$ & $\bar{3}$ & $\bar{4}$ & $\bar{0}$ & $\bar{1}$& $\bar{2}$\\
        $\bar{4}$  & $\bar{4}$ & $\bar{0}$ & $\bar{1}$& $\bar{2}$& $\bar{3}$\\
    \end{tabular}
    \qquad
    \begin{tabular}{c|c c c c c}
        $\cdot$       & $\bar{0}$ & $\bar{1}$& $\bar{2}$ & $\bar{3}$ & $\bar{4}$\\
        \hline
        $\bar{0}$ & $\bar{0}$ & $\bar{0}$& $\bar{0}$ & $\bar{0}$ & $\bar{0}$\\
        $\bar{1}$ & $\bar{0}$& $\bar{1}$ & $\bar{2}$ & $\bar{3}$ & $\bar{4}$\\
        $\bar{2}$& $\bar{0}$ & $\bar{2}$ & $\bar{4}$ & $\bar{1}$ & $\bar{3}$\\
        $\bar{3}$ & $\bar{0}$ & $\bar{3}$ & $\bar{1}$ & $\bar{4}$& $\bar{2}$\\
        $\bar{4}$  & $\bar{0}$ & $\bar{4}$ & $\bar{3}$& $\bar{2}$& $\bar{1}$\\
    \end{tabular}
\end{table}

Man kann durch Ablesen verifizieren, dass es für jedes Element $\in \mathbb{F}_p$ ein einziges additives und für jedes Element $\in \mathbb{F}_p^\times$ ein einziges multiplikatives Inverses gibt und tatsächlich alle Körperaxiome erfüllt sind.
\end{example}

\begin{definition}{Charakteristik}{}
Wir definieren die Charakteristik eines Körpers durch
$$\text{char}(K) = \begin{cases} \min\{n \mid n \cdot 1 = 0\} & \exists n < \infty: n \cdot 1 = 0 \\ 0 & \forall n: n \cdot 1 \neq 0 \end{cases}$$
\end{definition}

\begin{remark}
Mit dieser Definition erhalten wir z.B. für $\mathbb{F}_2$ einen Körper mit $\text{char}(\mathbb{F}_2) = 2$
\end{remark}

\subsubsection{Geordneter Körper}\label{cha_ordered_field}

Je nach dem kann ein Körper weitere Eigenschaften erfüllen, nämlich dass sich seine Elemente sinnvoll ordnen lassen. Wir führen daher den Begriff der \textbf{geordneten Körper} ein:
\begin{definition}{geordneter Körper}{} Sei $(K, +, \cdot, 0, 1)$ ein Körper und $\preceq$ eine Relation auf $K$. Wir nennen $(K, +, \cdot, 0, 1, \preceq)$ einen \textbf{geordneten Körper} [ordered field], falls neben den Körperaxiomen zusätzlich folgende Eigenschaften erfüllt sind:

\textbf{$\leq$ ist eine lineare Ordnungsrelation}\footnote{siehe Abschnitt \ref{cha_partial_order}}
\begin{enumerate} [label=O\arabic*)]
    \item $\forall x \in K : x \leq x$ \hfill (\textit{Reflexivität})
    \item $\forall x,y \in K : x \leq y \land y \leq x \implies x = y$ \hfill (\textit{Antisymmetrie})
    \item $\forall x,y, z \in K : x \leq y \land y \leq z \implies x \leq z$ \hfill (\textit{Transitivität})
    \item $\forall x,y \in K : x \leq y \lor y \leq x$ \hfill (\textit{lineare Ordnung})
\end{enumerate}
\textbf{Kompatibilität von $\leq$}
\begin{enumerate} [resume, label=O\arabic*)]
    \item $\forall x,y,z \in K: x \leq y \implies x + z \leq y + z$\hfill (\textit{additive Kompatibilität})
    \item $\forall x,y \in K: 0 \leq x \land 0 \leq y \implies 0 \leq xy$\hfill (\textit{multiplikative Kompatibilität})
\end{enumerate}
Wir schreiben zudem $x\geq y \iff y \leq x$ und $x < y \iff x \leq y \land x \neq y$. Es folgt $x > y \iff y < x$.
\end{definition}

Wir haben also einen Begriff für die Ordnung auf einem Körper definieren können. Für einen geordneten Körper lassen sich nun einige Eigenschaften herleiten:

\begin{satz}{Folgerungen für geordnete Körper}{}
 Sei $(K, +, \cdot,0, 1, \leq)$ ein geordneter Körper und $w,x,y,z \in K$, dann gilt:
\begin{enumerate}[label=\alph*)]
    \item Es gilt entweder $x<y$, $x=y$ oder $x>y$ \hfill (\textit{Trichotomie})
    \item $x < y \land y \leq z \implies x < z$
    \item $x \leq y \land w \leq z \implies x + w \leq y + z$ 
    \item $x \leq y \iff 0 \leq y - x$
    \item $x^2 = x\cdot x \geq 0$ und $x^2 = 0 \implies x = 0$
    \item $x \geq 0, y \leq z \implies xy \leq xz$
    \item $x \leq 0, y \leq z \implies xy \geq xz$
    \item $0<x\leq y \implies 0 < y^{-1} \leq x^{-1}$
\end{enumerate}
\end{satz}
\begin{proof}\ 
\begin{enumerate}[label=\alph*)]
    \item O4) gilt immer. Falls $x = y \stackrel{def. <}{\implies} x \nless y \land x \ngtr y$, falls $x \neq y \stackrel{\neg O2)}{\implies} x \nleq y \lor x \nleq y$. Aus der Umkehrung der Relationen erhalten wir $x > y \lor x < y$, wovon wegen O2) nur eine Aussage gilt.
    \item $x \neq y \land x \leq y \land y \leq z \stackrel{O3)}{\implies} x \leq z$. Um die strikte Ungleichheit zu zeigen, nehme $x = z$ an: $x \neq y \land x \leq y \land y \leq x \stackrel{O3)}{\implies}  x \neq y \land x = y$ Widerspruch $\implies x < z$
    \item O5): $x + w \leq y + w$ und $w + y \leq z + y$. Aus K4) und O3) folgt $x + w \leq y + z$.
    \item $x \leq y \stackrel{O5)}{\implies} x + (-x) \leq y + (-x) \iff 0 \leq y - x$ und $0 \leq y - x \stackrel{O5)}{\implies} x \leq y$.
    \item Verwende a) (Trichotomie): Sei $x \geq 0 : x \cdot x \stackrel{O6)}{\implies} x^2 \geq 0$. Sei $x \leq 0 \stackrel{d)}{\implies} 0 > -x \stackrel{O6)}{\implies} (-x)(-x) = -(-1)x^2 = x^2 > 0$. Sei $x^2=0$, aus Nullteilerfreiheit folgt $x=0$.
    \item $y \leq z \stackrel{d)}{\implies} 0 \leq z - y \stackrel{O6)}{\implies} 0 \leq x(z-y) = xz - xy \stackrel{d)}{\implies} xy \leq xz$
    \item $x \leq 0 \stackrel{d)}{\implies} 0 \leq -x$. Aus f) mit $(-x)$ folgt: $(-x)y \leq (-x)z \iff -(xy) \leq -(xz) \stackrel{O5, O5)}{\implies} xy + xz -(xy) \leq xy + xz -(xz) \iff xy \geq xz$
    \item Es gilt: $xx^{-1} = 1 = 1^2 \stackrel{d)}{\geq} 0$. Da $1 \stackrel{K5)}{\neq} 0$ gilt $xx^{-1} > 0$. Mit O6) folgt, dass $x^{-1} > 0$, $y^{-1} > 0$ und auch $x^{-1}y^{-1} > 0$ gilt. $y^{-1}=y^{-1}x^{-1}x \stackrel{O6)}{\leq}y^{-1}x^{-1}y = x^{-1}$.
\end{enumerate}
\end{proof}

\subsubsection{vollständiger Körper}
Man kann zeigen, dass je nach Konstruktion (z.B. diejenige von $\Q$) ein geordneter Körper Lücken haben kann. Es gibt daher noch eine weitere Eigenschaft, die ein Körper haben kann; sie kann \textbf{vollständig} sein resp. sie erfüllt das \textbf{Vollständigkeitsaxiom}. Dieses haben wir im Abschnitt \ref{cha_completeness} im Kontext von den reellen Zahlen definiert.
    %\begin{savequote}[60mm]
---Die natürlichen hat der liebe Gio geschaffen, alles andere ist Menschenwerk.
\qauthor{Konfuzius}
\end{savequote}


\chapter{Zahlenmengen}

In diesem Kapitel behandeln wir den axiomatischen Aufbau und die Eigenschaften verschiedener Zahlenmengen, insbesondere den reellen Zahlen.

\section{Die natürlichen Zahlen} \label{cha_natural_numbers}
Es beginnt wie im Kindergarten mit den Zahlen, welche mit mit den Fingern abzählen können: Die \textbf{natürlichen Zahlen}, geschrieben als $\N$. Wir wollen diese Zahlenmenge mithilfe der Axiome von \textsc{Giuseppe Peano} definieren:
\subsection{\textsc{Peano}-Axiome}\label{cha_peano_axioms}
\begin{enumerate}
    \item $1$ ist eine natürliche Zahl\footnote{In Rahmen der Analysis Vorlesung verwenden wir diese Konvention. Andere definieren 0 als die kleinste natürliche Zahl.}.
    \item  $1 \notin \nu(\N)$: Jede natürliche Zahl $n$ hat einen Nachfolger $\nu(n)$\footnote{$\nu(n)$ schreiben wir normalerweise als $n + 1$}, wobei $\nu: \N \to \N$ die \textbf{Nachfolgerfunktion} bezeichnet.
    \item $\nu$ ist injektiv: Es $n = m$ gilt, falls $\nu(n) = \nu(m)$ gilt.
    \item \textbf{Induktionsaxiom}: $A \subseteq \N \land 1 \in A \land \nu(A) \subseteq A \implies A = \N$
    
    In anderen Worten bedeutet das, dass eine Aussage $\mathcal{A}$ für alle natürlichen Zahlen gilt, wenn
    \begin{itemize}
        \item \textbf{Induktionsverankerung}: für die Aussage $\mathcal{A}(1)$ gilt und
        \item \textbf{Induktionsschritt}: $\mathcal{A}(\nu(n))$ aus $\mathcal{A}(n)$ folgt.
    \end{itemize} 
    Daraus folgt, dass jede beliebig grosse natürliche Zahl durch ein endlich-oftes Anwenden der Nachfolgerfuktion erreicht werden kann. Auf diesem Axiom beruht auf der Induktionsbeweis in Kapitel \ref{cha_induction}. Es ist zu bemerken, dass die ''Zahlen'' $\pm \infty$ nicht zu den natürlichen Zahlen gehören, da diese durch die Addition von 1 nie erreicht werden können.
\end{enumerate}
\begin{remark}
\textsc{Dedekind} hat gezeigt, dass es im wesentlichen nur eine Menge an natürlichen Zahlen gibt. Falls eine andere Zahlenmenge $\mathbb{A}$ ähnliche Eigenschaften hat wie $\N$, dann lässt sich eine eins-zu-eins-Abbildung (ein \textit{Isomorphismus}) $\Phi: \mathbb{A} \to \N$ finden, welche die Eigenschaften von beiden Mengen ineinander ''übersetzen'' kann.
\end{remark}

Wir wollen nun noch einige Sätze erwähnen, welche direkt aus den Axiomen folgen:

\begin{satz}{Sätze zu den Peano-Axiomen}{}
\begin{enumerate}
    \item Jede Zahl ausser 1 ist ein Nachfolger einer natürlichen Zahl
    \item Zu jedem $n\in \N$ gibt es genau eine Teilmenge $I_n$, sodass
    \begin{enumerate}
        \item $\forall k \in \N: \nu(k) \in I_n \implies k \in I_n$ und
        \item $n \in I_n, \nu(n) \notin I_n$ gilt.
    \end{enumerate}
    $I_1 = \{1\}$. Für alle $n \in \N$ gilt rekursiv $I_{\nu(n)} = I_n \cup \{\nu(n)\} = \{1, ..., n\}$ Wir nennen diese Mengen \textbf{Anfangsmengen} [initial set].
\end{enumerate}
\end{satz}

\begin{remark}
Die Anfangsmengen $I_n$ können auch mithilfe des Urbildes von $\nu$ definiert werden: (a) $\nu^{-1}(I_n) \subseteq I_n$ und (b) $n \in I_n, \nu(n) \notin I_n$
\end{remark}

\begin{proof}\label{prf_no_successor} {\ }
\begin{enumerate}
    \item Um das zu zeigen, konstruieren wir die Menge $A = \{1\} \cup \nu(\N)$. Wir wissen aus dem zweiten Axiom, dass $1 \notin \nu(\N)$ gilt und haben das somit manuell zu $\nu(\N)$ hinzugefügt. Wir wollen nun zeigen, dass $A = \N$ gilt, denn dann wissen wir, dass es für eine beliebige Zahl $n \neq 1 \in \N$ ein $m \in \N$ gibt, sodass $n = \nu(m)$ gilt:
    \newcommand{\ard}{\downarrow}
    \begin{center}
        \begin{tabular}{ccccccccccccc}
            $\N=\big\{$ &1&,&2&,&3&,&4&,&5&,& ... \big\}  \\
                        &&&$\ard$&&$\ard$&&$\ard$&&$\ard$&&\\
            $A =\big\{$ &$1$&,&$\nu(1)$&,&$\nu(2)$&,&$\nu(3)$&,&$\nu(4)$&,& ... \big\} 
        \end{tabular}
    \end{center}
    Wir verwenden hierfür das Induktionsaxiom: Für die Induktionsverankerung wissen wir bereits aus der Konstruktion von $A$, dass $1 \in A$ gilt. Wir nehmen an, dass $n \in A$ gilt. Da $A \subseteq \N$ ist, ist $n \in \N$. Daraus folgt wiederum aus der Konstruktion, dass $\nu(n) \in \nu(\N)$ also $\nu(n) \in A$ gelten muss. Der Induktionsschritt ist abgeschlossen und es folgt $A = \N$, was zu zeigen war. 
    \item Wir verzichten hier auf den Beweis, grundsätzlich kann man die Existenz und die Eindeutigkeit ebenfalls durch Induktion zeigen.
\end{enumerate}
\end{proof}

\begin{remark}(Mengentheoretische Konstruktion von $\N$) Zwar haben wir $\N$ axiomatisch eingeführt, jedoch haben wir die Existenz einer solchen Menge noch nicht gezeigt. Die folgende mengentheoretische Konstruktion stammt von \textsc{John von Neumann} und verwendet hierzu ausschliesslich die von den \textsc{Zermelo-Fraenkel}-Axiomen gegeben Mittel: leere Mengen und Mengen von Mengen:
\begin{align*}
    1 &:= \{\emptyset\}\\
    \nu(n) &:= n \cup \{n\}\\
    2 &:= \{\emptyset, \{\emptyset\}\}\\
    3 &:= \{\emptyset, \{\emptyset\}, \{\emptyset, \{\emptyset\}\}\}\\
    &...
\end{align*}
Das zeigt, dass sich die natürlichen Zahlen ebenfalls aus der Mengenlehre ableiten lässt. Wir werden jedoch diese ''Urform'' der natürlichen Zahlen praktisch nie brauchen; wir werden uns stattdessen (auch im Allgemeinen in anderen Theorien) auf die Axiome stützen und brauchen nur zu wissen, dass sie auf mengentheoretische Konstruktionen zurückzuführen sind.
\end{remark}

\subsection{Operationen auf $\N$}

Mit diesem Gerüst kann man nun Operationen wie $+,\cdot$ aus den Axiomen herleiten. Folgend werden wir die Addition und Multiplikation definieren:

\begin{definition}{Addition in $\N$}{}
\begin{align*}
    +: \N \times \N &\to \N\\
    (n,m) &\mapsto n + m
\end{align*}
Für $n+m$ definieren für jedes $n \in \N$ die Abbildung $m \mapsto n + m$ rekursiv:
$$n + m = \begin{cases}\nu(n)&\text{falls }m = 1 \\ \nu(n + \tilde{m}) & \text{falls } m\neq1 \implies \exists \tilde{m}: \nu(\tilde{m}) = m\end{cases}$$
\end{definition}
\begin{example}[ ] Es gilt also für die Rechnung $2+3$:
$$2 + 3 = \nu(2 + 2) = \nu(\nu(2 + 1)) = \nu(\nu(\nu(2))) = 2 + 1 + 1 + 1 = 5$$
wobei in den letzten beiden Schritten die gängigen Konventionen verwendet wurden.
\end{example}

\begin{definition}{Multiplikation in $\N$}{}
\begin{align*}
    \bullet: \N \times \N &\to \N\\
    (n,m) &\mapsto n \cdot m
\end{align*}
Auch hier definieren wir $n\cdot m$ für jedes $n \in \N$ die Abbildung $m \mapsto n \cdot m$ rekursiv:
$$n \cdot m = \begin{cases}n&\text{falls }m = 1 \\ n + (n \cdot \tilde{m}) & \text{falls } m\neq1 \implies \exists \tilde{m}: \nu(\tilde{m}) = m\end{cases}$$
\end{definition}
\begin{example}[ ] Für $2\cdot3$ gilt also:
\begin{align*}
    2 \cdot 3 &= 2 + (2 \cdot 2) = 2 + (2 + (2 \cdot 1)) = 2 + (2 + 2) = 2 + (\nu(2 + 1))= 2 + \nu(\nu(2))\\
              &= \nu(2 + \nu(2)) = \nu(\nu(2 + 2)) = \nu(\nu(\nu(2 + 1))) = \nu(\nu(\nu(\nu(2)))) = 6
\end{align*}
\end{example}

Wir könnten des Weiteren zeigen, dass die soeben definierten Abbildungen für $+$ und $\cdot$ die üblichen Eigenschaften wie die Assoziativität, Kommutativität, die Distributivität und das Kürzen erfüllen:

\begin{lemma}{Kürzen in $\N$}{}
Es folgt (aus der Injektivität von $\nu$, also ohne der Definition der Subtraktion resp. der Division):
\begin{align*}
    n + m = p + m \implies n = p\\
    n\cdot m = p \cdot m \implies n = p
\end{align*}
\end{lemma}

\subsection{Relationen}\label{cha_natural_number_ff}
Nun wollen wir eine Ordnung auf $\N$ definieren, also Relationen wie $<, >, \geq, \leq$. Da wir für Mengen bereits die Ordnungsrelation $\subseteq$ haben, wollen wir die Relationen auf $\N$ mit $\subseteq$ definieren:
\begin{definition}{Relationen auf $\N$}{}
Sei $I_n = \{1, ..., n\}$. Wir schreiben:
\begin{align*}
    n \leq m &\text{ falls } I_n \subseteq I_m &  n > m &\text{ falls } m < n\\
    n < m &\text{ falls } n \leq m \land n \neq m & n \geq m &\text{ falls } m \leq n
\end{align*}
\end{definition}
Es folgen daraus die üblichen Rechenregeln für alle natürlichen Zahlen $n, p, m$:
\begin{align*}
    n \leq p &\implies n + m \leq p + m\\
    n \leq p &\implies n \cdot m \leq p \cdot m
\end{align*}
Mit den Relationen können wir nun z.B. $I_n$ definieren als $\{ k \in \N \mid 1 \geq k \geq n\}$

\begin{lemma}{Anordnung von $\N$}{}
\begin{enumerate}[label=(\alph*)]
    \item $\forall n \in \N \implies n = 1 \lor n-1 \in \N$
    \item $\forall n,m \in \N: n \leq m \leq n+1 \implies m = n \lor m = n+1$
\end{enumerate}
\end{lemma}

\begin{proof}
\begin{enumerate}[label=(\alph*)]
    \item Falls $n \ne 1$ folgt $n \in \N \setminus \{1\}$. Aus den \textsc{Peano}-Axiomen folgt, dass jede natürliche Zahl ausser 1 Nachfolger einer Zahl ist: $\exists m \in \N: m+1 = n$, also gilt $ m = n - 1  \in \N$.
    \item Siehe Script Lemma 2.18 oder Notizen vom 1.10.2020
\end{enumerate}
\end{proof}

Hier wollen wir noch einige wichtige Eigenschaften von $\N$ erwähnen, welche auch aus den Axiomen folgen, jedoch werden wir sie nicht beweisen:
\begin{satz}{Eigenschaften von $\N$}{}
\begin{enumerate}[]
    \item $\N$ ist \textbf{wohlgeordnet}: Jede nicht leere Teilmenge $A \subseteq \N$ hat ein eindeutiges kleinstes Element:
    $$\forall A \subseteq \N: \exists! a_0 \in A, \forall a \in A: a_0 \leq a $$
    \item \textbf{Division mit Rest}: Sei $\N_0 = \N \cup \{0\}$, dann gibt es für alle $n \in \N_0$ und $d\in\N$ einen eindeutigen ''Quotienten'' $q \in \N_0$ und Rest $r \in \N_0$ mit $0 \leq r < d$, sodass gilt:
    $$n = qd + r$$
    \item \textbf{Primfaktorzerlegung}: Jede natürliche Zahl $n$ lässt sich eindeutig als Produkt von endlich vielen Primzahlen (bis auf Reihenfolge) schreiben:
    $$n = p_1\cdot ... \cdot p_k, \qquad p_i \text{ prim}$$
    Wir verwenden die Konvention, dass das Produkt aus 0 Zahlen 1 ist.
    $\forall n \in \N$ gibt es eindeutige $n_2, n_3, n_5, n_7,... \in \N_0$ (von denen nur endlich viele $\neq 0$ sind), sodass
    $$n = 2^{n_2} \cdot 3^{n_3} \cdot 5^{n_5} \cdot 7^{n_7} \cdot ...$$
\end{enumerate}
\end{satz}

\section{Die ganzen Zahlen}
Wir wollen nun die Zahlenmenge $\N$ zu den ganzen Zahlen $\Z$ erweitern. Da aber die Subtraktion nicht definiert ist, müssen wir sie durch eine geschickte Konstruktionen mit Elementen aus $\N$ schaffen.

Die Idee für die Konstruktion der ganzen Zahlen darauf, dass es sich dabei um die Differenz zweier ganzen Zahlen handelt. Wir bemerken noch kurz, dass die Darstellung durch die Differenz nicht eindeutig ist. Das werden wir gleich sehen.
\begin{definition}{ganze Zahlen $\Z$}{}
$\Z$ ist die Menge der Äquivalenzklassen für die Äquivalenzrelation $\sim$ auf $\N^2$:
$$(n,m) \sim (n',m') \iff n + m' = n' + m$$
\end{definition}
Wir sehen nun, dass die Subtraktion ganz elegant umschifft wurde. So können wir $-3 \in \Z$ schreiben als $(1, 4), (2, 5)$ oder $(420, 423)$. Da all diese drei Paare die Äquivalenzrelation erfüllen, sind sie Repräsentanten der selben Äquivalenzklasse, also laut Definition für die selbe ganze Zahl. Gedanklich subtrahieren wir die rechte Zahl von der linken, die Konstruktion jedoch ist frei von jedem Minus-Zeichen.

Wir haben in der Definition bereits erwähnt, dass $\sim$ ein Äquivalenzrelation ist. Dies wollen wir noch kurz beweisen:
\begin{proof} Zu zeigen sind die Reflexivität, Symmetrie und Transitivität von $\sim$:
\begin{enumerate}
    \item Es gilt $(n, m) \sim (n, m)$ da $n + m = n + m$ (Addition ist kommutativ.)
    \item Sei $(n, m) \sim (p, q)$, dann gilt:
    \begin{align*}
        (n, m) \sim (p, q)  &\iff n + q = p + m\\
                            &\iff p + m = n + q\\
                            &\iff (p, q) \sim (n, m)
    \end{align*}
    \item Seien $(n, m) \sim (p, q)$ und $(p, q) \sim (s, t)$, dann gelten $n + q = p + m$ und $p + t = s + q$. Daraus folgt durch Addition:
    \begin{align*}
     n + q + p + t &= p + m + s + q\\
        n + t &= m + s\\
        n + t &= s + m
    \end{align*}
    Also gilt $(n, m) \sim (s + t)$
\end{enumerate}
\end{proof}

Da $\sim$ nun eine Äquivalenzrelation ist, wir $\N^2$ partitioniert: $[(n,m)]$ ist dabei die Äquivalenzklasse von $(n,m)$, zu der alle anderen Paare mit der selben Differenz gehören. Man sieht schnell, dass z.B. $[(n,m)] = [(n + p,m + p)]$ gilt. Zudem finden wir für jede Äquivalenzklasse aus $\Z$ einen eindeutigen Repräsentanten der Form $(n+1, 1) =: n \in \Z$, $(1, 1) =: 0 \in \Z$ und $(1, 1+n) =: -n \in \Z$ ($n \in \N$).

\subsection{$\N \subseteq \Z$}
Wir können eine kanonische Abbildung $\Phi: \Z \to \N$ wie folgt definieren:  $n \mapsto [(n+1,1)]$ (Bemerke, dass der Versatz, hier 1, gleich sein muss, aber ansonsten frei wählbar ist, da die Äquivalenzklasse invariant ist unter Verschieben der Differenz.) Wir erhalten also $\Phi: n \in \N \mapsto n \in \Z$. Unter $\Phi$ können wir nun $\N$ als Teilmenge von $\Z$ auffassen\footnote{Genau ausgedrückt meinen wir mit ''als Teilmenge unter $\Phi$ auffassen'' $\Phi(\N) \subseteq \Z$.}:
$$ \Z = \N \cup \{0\} \cup \{-n \mid n \in \N\}$$


\subsection{Operationen auf $\Z$}
Wir werden nun wie für $\N$ die Addition als auch die Multiplikation auf $\Z$ definieren:
\begin{definition}{Addition auf $\Z$}{}
\begin{align*}
    +: \Z \times \Z &\to \Z\\
    \big([(n,m)], [(p, q)]\big) &\mapsto ([n + p, m + q])
\end{align*}
\end{definition}
Das diese Definition Sinn ergibt, ist nicht direkt offensichtlich. Wir müssen also zeigen, dass diese Abbildung wohldefiniert ist. In diesem Fall, da es eine Abbildung über Äquivalenzklassen ist, wollen wir v.a. die Repräsentantenunabhängigkeit prüfen: Wir müssen also zeigen, dass es nicht von der Wahl der Repräsentanten der Äquivalenzklassen abhängt:

Seien $(n,m) \sim (n',m')$, es gilt also $n + m' = n' + m$. Dann ist $(n,m) + (p,q) = (n+p,m+q)$ und $(n',m') + (p,q) = (n'+p, m'+q)$. Nun müssen wir zeigen, dass diese Summen gleich sind:
\begin{align*}
(n+p,m+q) \sim (n'+p',m+q) &\iff n+p+m'+q=n'+p+m+q\\
                        &\iff n+m'=n+m\\
                        &\iff (n,m) \sim (n',m')
\end{align*}
Da das letzte die Annahme ist, haben wir die Repräsentantenunabhängigkeit von der Addition bzgl. $[(n,m)]$ gezeigt. Ähnlich ist es auch für $[(p,q)]$ gezeigt, wodurch wir nun gezeigt haben, dass die Addition wie oben definiert wohldefiniert ist.

Für $\Phi: n \in \N \mapsto [(n+1,1)] \in \Z$ stimmt diese Definition der Addition mit der Addition der natürlichen Zahlen überein. Seien $n,m \in \N$. Es gilt:
$$\Phi(n)+\Phi(m) = [(n+1, 1)]+[(m+1, 1)] = [(n+m+1,1)] = [(n+m+2, 2)] = \Phi(n+m)$$

Die Negation ist mit dieser Konstruktion auch sehr schnell definiert, wir verzichten aber auf die Verifikation: $$-: \Z \to \Z, [(n, m)] \mapsto [(m, n)]$$ Die Subtraktion folgt demnach aus der Addition der Negation.

Wir zeigen nur noch kurz die Definition der Multiplikation:
\begin{definition}{Multiplikation auf $\Z$}{}
\begin{align*}
    \bullet: \Z \times \Z &\to \Z\\
    \big([(n,m)], [(p, q)]\big) &\mapsto ([np + mq, nq + mp])
\end{align*}
\end{definition}
Die Idee von der Definition ist, die Paare jeweils als Differenzen zu betrachten:
$$(n-m)(p-q) = np+mq-nq-mp = np+mq - (nq + mp)$$
\begin{exercise}[Verifizierung] Um die Definition abzuschliessen, wären folgende Eigenschaften noch zu verifizieren:
\begin{enumerate}
    \item wohldefiniert
    \item stimmt überein mit der Multiplikation auf $\N \subseteq \Z$
    \item übliche Rechenregeln: Kommutativität, Assoziativität, Distributiviät (also zeigen, dass $\Z$ ein kommutativer Ring mit eins ist) und Kürzen
\end{enumerate}
\end{exercise}

\section{Die rationalen Zahlen} \label{cha_rationals}
Wie auch bei der Konstruktion von $\Z$ werden wir für $\Q$ die Brüche $\frac{a}{b}$ als Äquivalenzklassen über ausdrücken. Wir wollen zudem, dass $\Q$ die gewohnten Eigenschaften eines Körpers besitzt, somit muss die Nullteilerfreiheit gewährleistet sein. Dies hat wie erwartet zur Folge, dass der Nenner ungleich 0 sein muss, also $a \in \Z$ und $b \in \Z^\times$

\begin{definition}{rationale Zahlen $\Q$}{}
Wir definieren die Äquivalenzrelation $\sim$ auf $\Z \times \Z^\times$ durch
$$(n, m) \sim (p, q) \iff nq = mp$$
und $\Q$ als die Menge der Äquivalenzklassen von $\sim$:
$$\Q = \big[\Z \times \Z^\times\big]_\sim$$
\end{definition}
Wir zeigen kurz, dass es sich dabei auch tatsächlich um eine Äquivalenzrelation handelt:
\begin{proof} Reflexivität und Symmetrie sind hier ähnlich bewiesen wie bei $\Z$. Seinen nun $(n, m) \sim (p, q)$ und $(p, q) \sim (s, t)$ in Relation auf $\Z \times \Z^\times$, also gelten  $nq = pm$ und $pt = sq$ und $m, q, t \neq 0$. Es folgt also unter Verwendung der Ringaxiome von $\Z$:
\begin{align*}
   (n, m) \sim (p, q) &\iff nq = pm&& \mid  \cdot t \neq 0\\
            &\iff t(nq) = t(pm)&& \mid  \text{$\cdot$ assoz. und kommut.}\\
            &\iff (tn)q = (pt)m&& \mid  pt = sq\\
            &\iff (tn)q = (sq)m&& \mid  \text{additive Inverse}\\
            &\iff (tn)q - (sm)q = 0&& \mid  \text{Distributivität}\\
            &\iff (tn - sm)q = 0 & &\mid  q \neq 0\\
            &\iff tn - sm = 0\\
            &\iff tn = sm\\
            &\iff (n, m) \sim (s, t)
\end{align*}\end{proof}

\begin{remark}
Man sieht anhand dieses Beweises, welche Bedeutung die Nullteilerfreiheit für die Konstruktion von $\Q$ hat: Ohne der Einschränkung ''Nenner $\neq 0$'' wäre $\sim$ keine Äquivalenzrelation (z.B. $(1,1) \sim (0, 0)$ und $(0,0) \sim (2, 1)$, aber $(1,1) \nsim (2,1)$), wodurch sich eine Äquivalenzklassen bilden lassen, welche die Elemente von $\Q$ sein sollen.
\end{remark}

Wir definieren daher nun $\Q$ als die Menge der Äquivalenzklassen und schreiben $n/m$ für die Äquivalenzklasse von $(n,m)$. (Diese Schreibweise soll noch nicht die Division sein, sie soll nur der Intuition helfen). Wir erkennen schnell die uns bekannte Kürzungsregel: $n/m = pn/pm$ da $(n, m) \sim (pn, pm)$. Wir definieren die Addition und die Multipliation wie folgt:

\begin{definition}{Operationen auf $\Q$}{}
\begin{align*}
    +: \Q \times \Q &\to \Q\\
    \big(n/m, p/q\big) &\mapsto (nq + pm)/ mq
\end{align*}
\begin{align*}
    \bullet: \Q \times \Q &\to \Q\\
    \big(n/m, p/q\big) &\mapsto np/mq
\end{align*}
\end{definition}

Mit diesen Definitionen wollen wir nun folgenden Satz behaupten:

\begin{satz}{$\Q$ ist ein Körper}{}
$(\Q, +, \cdot, 0/1, 1/1)$ erfüllt die Körperaxiome.
\end{satz}

Für das sind folgende Eigenschaften zu verifizieren:
\begin{itemize}
    \item Zeige, dass die Operationen $+, \cdot$ \textbf{wohldefiniert} (repräsentantenunabhängig) sind.
    \item \textbf{Kompatibilität zu $\Z$}: Wie bei $\Z$ können wir auch für $\Q$ eine injektive, kanonische Abbildung $\Phi: n \in \Z \mapsto n/1 \in \Q$ definieren, unter welcher wir $\Z$ wieder als Teilmenge von $\Q$ auffassen können. Aus dieser Definition folgt:
    \begin{enumerate}
        \item $\Phi(n)+\Phi(m)=n/1 + m/1 = (n+m)/1 = \Phi(n+m)$
        \item $\Phi(n)\Phi(m)=n/1 \cdot m/1 = (nm)/1 = \Phi(nm)$
        \item $\Phi(0) = 0/1,\ \Phi(1) = 1/1$
        \item $n/m + (-n)/m = (n+(-n))/m = 0/m = 0/1 = \Phi(0)$
        \item $n/m \cdot m/n = (nm)/(nm) = 1/1 = \Phi(1)$
    \end{enumerate}
    Wir erkennen in 1. und 2., dass die Operationen unter $\Phi$ mit $\Z$ kompatibel ist. Daher können wir ganze Zahlen ''implizit'' umwandeln zu rationalen Zahlen und schreiben $n$ amstelle von $n/1$ für ein $n\in \Z$. Auch erkennen wir in 3., 4. und 5. schon einige wichtige Eigenschaften, welche $\Q$ zu einem Körper machen: Nullelement und Einselement, die additive Inverse resp. die multiplikative Inverse.
    \item \textbf{$\Q$ ist ein Körper}: Dass $(\Q, +, \cdot, 0, 1)$ einen kommutativen Ring mit eins bildet, wissen wir bereits aus $\Z$. Die multiplikative Inverse von $n/m \in \Q^\times$ ist $(n/m)^{-1} = m/n$, somit ist $\Q$ ein Körper.
\end{itemize}
Mit der multiplikativen Inversen sind wir nun endlich in der Lage, die Division definieren  zu können:
\begin{definition}{Division}{}
    $$\div: \Q \times \Q^\times \to \Q\\$$
    $$\big(n/m, p/q\big) \mapsto n/m \cdot (p/q)^{-1} = n/m \cdot q/p = nq/mp$$
\end{definition}
\begin{remark}
Wir erkennen hier, wieso die Division durch 0 nicht erlaubt ist: Es gibt keine multiplikative Inverse von 0 wegen der Nullteilerfreiheit.
\end{remark}

Zu guter Letzt können wir noch zeigen, dass $\Q$ ein angeordneter Körper ist: Hierzu legen wir fest, dass rationale Zahlen der Form $\frac{n}{m}$, wobei $n,m \N$ gilt, positiv sind, es gilt also $\frac{n}{m} > 0$. Des Weiteren sollen $a, b \in \Q$ zueinander in Relation $a \leq b$ stehen , falls $a = b$ oder $ b-a >0$ gilt. Mit diesen Vorgaben ist zu zeigen, dass $\leq$ tatsächlich eine lineare Ordnung auf $\Q$ bildet, siehe Abschnitt \ref{cha_ordered_field}. Es folgt, dass $(\Q, +, \cdot,0, 1, \leq)$ ein geordneter Körper ist. Man kann sich diese Zahlenmenge deshalb auch als Zahlenstrahl vorstellen, welche von links nach rechts läuft, wobei $a < b$ soviel wie ''$a$ liegt links von $b$'' bedeuten würde.

\section{Die reellen Zahlen}
Wir könnten doch schon mit $\Q$ zufrieden sein, es stellt sich jedoch heraus, dass dieser Zahlenstrahl von $\Q$ Lücken hat: So ist z.B. die Zahl $x$, für welche $x^2 = 2$ gilt, nicht in $\Q$ enthalten:

\begin{example}[Irrationalität von $\sqrt{2}$]\label{ex_sqrt2_irrational}
Die Behauptung ist, dass $\sqrt{2}$ irrational ist und somit nicht als Bruch geschrieben werden kann.

Wir behaupten nun, dass $\sqrt{2}$ trotzdem als gekürzter Bruch $\frac{p}{q}$ geschrieben werden kann, d.h. $p$ und $q$ sind teilerfremd. Da $2 = (\sqrt{2})^2 = (\frac{p}{q})^2$ gilt, erhalten wir $2 q^2 = p^2$. Es folgt, dass $p^2$ gerade sein muss und wir wissen aus vorherigem Beispiel \ref{ex_counterpos}, dass somit auch $p$ gerade sein muss. Da aber nun $p$ gerade ist, ist $p^2$ durch 4 teilbar. Aus der Gleichheit $2 q^2 = p^2$ folgt somit aber auch, dass $p^2$ sowie $p$ gerade sein muss, wodurch $p$ und $q$ nicht teilerfremd sein können, welches ein Widerspruch zur Annahme ist. 
\end{example}

\subsection{Vollständigkeitsaxiom}\label{cha_completeness}

Wir brauchen also eine Eigenschaft, die es verspricht, dass all diese Lücken von gefüllt werden können. Um diese Eigenschaft (mengentheoretisch) präzise ausdrücken zu können, definieren wir das \textbf{Vollständigkeitsaxiom}:
\begin{definition}{Vollständigkeitsaxiom}{}
Sei $(K, + ,\cdot, 0, 1, \geq)$ ein geordneter Körper. Wir nennen den Körper \textbf{vollständig}, falls für alle nicht-leeren Teilmengen $X,Y \subseteq K$ mit $X \leq Y \iff \forall x\in X, y \in Y: x \leq y$ ein $c \in K$ existiert, welches zwischen diesen Mengen liegt:
$$\exists c \in K: \forall x\in X \forall y \in Y: x \leq c \leq y$$
Alternative Definition: alle Cauchy-Folgen konvergieren.
\end{definition}
Um es etwas besser verbildlichen zu können, verwenden wir nochmals das Beispiel von $\sqrt{2}$:
\begin{example}[$\Q$ erfüllt nicht das Vollständigkeitsaxiom] \label{ex_Q_not_complete} Wir suchen also eine Lösung für $x^2 = 2$. Wir können daher die zwei Teilmengen (von $K$) wie folgt definieren: $X$, die Teilmenge die darunter liegt, soll sein:
$$X = \{x \in K \mid x \leq 0, x^2 \leq 2 \}$$
und $Y$, diejenige, die darüber liegt
$$Y = \{y \in K \mid y \leq 0, y^2 \geq 2 \}$$
Bemerke die zusätzliche Bedingung $0 \geq x$ für beide Mengen, das stellt auch sicher, dass $X \leq Y$ gilt. Was haben wir nun konstruiert? Sprich können wir zeigen, dass $\Q$ diese Bedingung nicht erfüllt? Wir haben jetzt (plump gesagt) so etwas wie $X \leq \sqrt{2} \leq Y$. Wir wissen von vorher, dass $\sqrt{2}$ nicht in $\Q$ ist, trotzdem können wir aber diese zwei Mengen konstruieren: Wir nehmen ''jede'' Zahl aus $\Q$, vergleichen das Quadrat der Zahl mit 2 und teilen es der entsprechenden Menge zu. Wir werden aber zeigen können, dass dann eben kein solches $c \in \Q$ zwischen $X$ und $Y$ gefunden werden kann:

\begin{proof}[Beweis (Heron-Verfahren).]
Nehmen wir an, dass ein solches $\frac{a}{b} \in \Q$ mit $a,b \in \N$ existiert, für welches $X \leq \frac{a}{b} \leq Y$ gilt, jedoch ungleich $\sqrt{2}$ ist. Aus der Trichotomie folgt, dass es dann entweder grösser oder kleiner als $\sqrt{2}$ sein muss. Sei $\frac{a}{b} > \sqrt{2}$, dann gilt $\sqrt{2} < \frac{a}{b} \leq Y$. Da der linke Teil davon die Bedingung für $Y$ ist, muss $\frac{a}{b}$ das kleinste Element von $Y$ sein. Wir wollen nun zeigen, dass man eine Differenz $d \in \Q$ finden kann, für welches $\sqrt{2}< \frac{a}{b}-d<\frac{a}{b}$ gilt, also kleiner als $\frac{a}{b}$ ist aber die Bedingung von $Y$ trotzdem erfüllt, wodurch $\frac{a}{b}$ nicht das kleinste Element von $Y$ sein kann:
\begin{align*}
    \sqrt{2} + d &\stackrel{!}{<} \frac{a}{b} & &\mid ()^2\\
    2 + 2\sqrt{2}d + d^2 &\stackrel{!}{<}  \frac{a^2}{b^2} & & \mid \text{Mitternachtsformel}\\
    d &\stackrel{!}{<}  \frac{a-\sqrt{2}b}{b} && \mid \cdot \frac{a+\sqrt{2}b}{a+\sqrt{2}b}\\
    d &\stackrel{!}{<}  \frac{a^2 - 2 b^2}{ab+\sqrt{2}b^2} && \mid \text{Abschätzung } \sqrt{2} < \frac{a}{b}\\
    d &\stackrel{!}{\leq}  \frac{a^2 - 2 b^2}{2ab} < \frac{a^2 - 2 b^2}{ab+\sqrt{2}b^2} 
\end{align*}
Mit $d = \frac{a^2 - 2 b^2}{2ab} \in \Q$ ist $d$ positiv ($a^2 > 2b^2$) und erfüllt die obige Bedingung, wodurch $\sqrt{2} < \frac{a}{b}-\frac{a^2 - 2 b^2}{2ab} < \frac{a}{b}$ gilt. Somit haben wir eine kleinere Zahl in $Y$ gefunden, welches ein Widerspruch zur Annahme ist (ähnlich für $\frac{a}{b} < \sqrt{2}$).\footnote{Durch iteratives Anwenden dieser Funktion erhält man eine immer genauere Approximation für $\sqrt{2}$. Wir werden dieses Verfahren viel später mit der Differentialrechnung herleiten.}
\end{proof}
Daran erkennen wir also, dass $\Q$ das Vollständigkeitsaxiom für das Beispiel von $\sqrt{2}$ nicht erfüllt.
\end{example}

\begin{remark} Man könnte meinen, dass man die Mengen auch für die Bedingung $x^2 \geq -1$ resp. $x^2 \leq -1$ definieren kann, dann aber auch z.B. in $\R$ Lücken finden würde.
Mit den Mengen $X = \{x \in K \mid x^2 \leq -1 \}$ und $Y = \{y \in K \mid y^2 \geq -1 \}$ kann man aber keine Aussage über die Vollständigkeit machen, da $X = \emptyset$ und $Y = K$ gilt und die Mengen für das Axiom nicht leer sein dürfen.
\end{remark}

Wir werden später zeigen, wie $\R$ aus $\Q$ konstruiert wird (z.B. \textsc{Dedekind}-Schnitte) und dass $\R$ dieses Axiom auch erfüllt. For the time being nehmen wir einfach an, dass $\R$ die Axiome eines vollständig angeordneten Körpers erfüllt und auch existiert.

Es gibt eine ganze Liste aus Konsequenzen dieses Axioms, hierfür verweisen wir auf den Abschnitt \ref{cha_consequences_completeness}.
 
%\subsection{Konstruktion von $\R$ aus $\Q$}
%Wir haben soeben im Beispiel \ref{ex_Q_not_complete} gesehen, dass sich diese Teilmengen $X$ und $Y$ für $\sqrt{2}$ in $\Q$ bilden lassen, jedoch existiert kein Wert in $\Q$, der zwischen den Mengen liegt. Die Idee für die Konstruktion von $\R$ aus $\Q$ ist, diese Teilmengen $X$ und $Y$ von $\Q$ direkt als Definition der ''eingequetschten'' reellen Zahl zu verwenden. 

\subsection{$\R$ als Mutter von $\N$, $\Z$ und $\Q$}
Bis jetzt haben wir fast alle Zahlenmengen auf die Axiome der Mengenlehre zurückführen können: Aus Mengen können wir die Natürlichen konstruieren, aus denen die ganzen, etc. Anders wie diese ''Bottom-Up''-Methode, können wir auch die Körperaxiome eines vollständig angeordneten Körpers annehmen --- in diesem Fall werden wir $\R$ verwenden --- und können daraus ''Top-Down'' die anderen Zahlenmengen als Teilmengen mit 1 konstruieren, welche jeweils bezüglich $+$ und $\cdot$ abgeschlossen sind. Dazu wollen wir zuerst den Begriff einer \textbf{induktiven Teilmenge} definieren:

\begin{definition}{Induktive Teilmenge}{} Eine Teilmenge $M \subseteq \R$ heisst \textbf{induktiv}, falls:
\begin{enumerate}
    \item $1 \in M$
    \item $\forall x \in M \implies x + 1 \in M$
\end{enumerate}
\end{definition}

\begin{example}
So ist $\R$ selber eine induktive Menge. Andere Beispiele sind: $\R_\geq1 = \{x \in \R \mid x \geq 1\}$ oder $\{1, 2, 3, \pi, 4, \pi + 1, 5 , \pi + 2, ...\}$.
\end{example}

Mit induktiven Teilmengen von $\R$ können wir nun die Menge der natürlichen Zahlen definieren als:

\begin{definition}{natürliche Zahlen aus $\R$}{}
$\N$ soll die kleinste induktive Teilmenge von $\R$ sein:

$$\N = \bigcap_{\substack{M \subseteq \R \\ M \text{ induktiv}}} M = \{x \in \R \mid \forall M \subseteq \R \text{ induktiv}: x \in M\}$$
\end{definition}

Wir können des Weiteren zeigen, dass $\N$ dann wie oben definiert die \textsc{Peano}-Axiome (siehe Abschnitt \ref{cha_peano_axioms}) erfüllt, woraus wir dann wieder $\Z$ und $\Q$ konstruieren können. Man könnte sich dann fragen, wieso wir uns die Mühe gemacht habe, die Zahlenmengen aus $\N$ herzuleiten. Zwar sind diese beiden Methoden der Herleitung äquivalent, jedoch ist die ''Bottom-Up''-Methode notwendig, um die Existenz der Mengen zeigen zu können. Es nützt nichts, wenn die Axiome von $\N$ und $\R$ miteinander übereinstimmen, aber insgesamt widersprüchlich sind.

\begin{proof}\ 
\begin{enumerate}[label=(\arabic*)]
    \item $1 \in \N$: Folgt aus der Definition von induktiven Mengen.
    \item $\forall n \in \N \implies n + 1 \in \N$: Folgt aus der Definition von induktiven Mengen.
    \item $\forall m \in \N: 1 \neq 1 + m$: Gilt, da $\notin \N$
    \item $\forall n,m \in \N: n+1 = m+1 \implies n = m$: Folgt aus Körperaxiomen
    \item Induktionsaxiom $A \subseteq \N \land 1 \in A \land (n \in A \implies n + 1 \in A) \implies A = \N$: $\N \subseteq A$ folgt aus der Definition von $\N \implies A = \N$.
\end{enumerate}
\end{proof}

\begin{remark}
Da wir $\N$ nun als Teilmenge eines geordneten Körpers definiert haben, können wir aus $1 > 0$ folgern, dass $n \in \N: n + 1 > n$ gilt, also $1 < 2 < 3 < ...$
\end{remark}

\begin{satz}{$\N$ ist abgeschlossen bezüglich $+$ und $\cdot$}{}
Wenn $n,m \in \N$, dann gilt $n + m \in \N$ und $n \cdot m \in \N$
\end{satz}

\begin{proof} Wir machen für jedes $n$ eine Induktion über $m$:

($+$): Sei $m=1$: $n+1 \in \N$ gilt wegen (2). Sei nun $n+m \in \N$, dann gilt wegen (2) und der Assoziativität des Körpers  $\underbrace{n+(m+1)}_{IS} = \underbrace{(n+m)}_{IA: \in \N}+1 \in \N$.

(\ $\cdot$\ ): Sei $m = 1$: $n\cdot 1 = n \in \N$ gilt wegen K5). Sei $m \cdot n \in \N$, dann gilt mit K9) $(m + 1) \cdot n = \underbrace{(m \cdot n)}_{IA: \in \N} + n$. Aus der Abgeschlossenheit von $+$ folgt $(m + 1) \cdot n \in \N$.  
\end{proof}

Alle weiteren Eigenschaften von $\N$ sind im Abschnitt \ref{cha_natural_number_ff} erwähnt.

Weiter können wir nun die anderen Zahlenmengen einführen: $\Z = \N \cup \{0\} \cup \{-n \mid n \in \N\}$ und $\Q = \{ \frac{p}{q} \mid p,q, \in \Z, q \neq 0\}$.

\section{Die komplexen Zahlen}
Nun wollen wir die komplexen Zahlen aus $\R$ definieren. Eine Motivation kann sein, eine Lösung für $x^2 = -1$ zu finden. Wir haben sie stattdessen geometrisch/algebraisch motiviert hergeleitet: Wir wollen einen $\R^n$ mit der Vektoraddition definieren. Die Addition soll wie folgt aussehen: $x,y \in \R^2: x + y := (x_1+y_1,...,x_n+y_n)$. Man kann sehen, dass dies im Allgemeinen eine ablesche Gruppe bildet (siehe lineare Algebra), die Frage ist jedoch, ob ein $\R^n$ mit dieser Addition auch die Körperaxiome mit multiplikativer Inverse, etc. erfüllen kann. Es stellt sich heraus, dass dies nur für $n \in \{1 ,2\}$ der Fall sein kann\footnote{Falls man auf die multiplikative Kommutativität verzichtet, kann auch der $\R^4$ (Quarternionen) ein Körper sein. Verzichten wir auf beide Kommutativiäten, so ist auch der $\R^8$ (Oktonionen) ein Körper. Es gibt keine weiteren.}.

Statt die Elemente als Vektoren $z = (x, y)$ zu schreiben, schreiben wir $z = x + yi$. Wir definieren nun die komplexen Zahlen wie folgt:

\begin{definition}{komplexe Zahlen}{}
Die \textbf{imaginäre Einheit} $i$ definieren wir durch: $i\cdot i = -1$ oder vektoriell durch $i = (0, 1)$. Die Addition definieren wir durch
\begin{align*}
+: \C \times \C &\to \C:\\
((x, y), (x', y'))  &\mapsto (x + x', y + y')\\
((x + yi), (x' + y'i)) &\mapsto (x + x') + i(y + y')
\end{align*}
die Multiplikation durch
\begin{align*}
\bullet: \C\times \C  &\to \C:\\
((x, y), (x', y')) &\mapsto (xx'- yy', xy'+x'y)\\
((x + yi), (x' + y'i)) &\mapsto (xx'- yy') + i(xy'+x'y)
\end{align*}
Wir definieren also $\C := \R^2$.
\end{definition}

\begin{satz}{$\C$ ist ein Körper}{}
$(\C, +, \cdot, 1, 0)$ erfüllt mit
\begin{itemize}
    \item $0 = (0,0)$ als Nullelement
    \item $1 = (1,0)$ als Einselement
    \item $-(x, y) = (-x, -y)$ als additive Inverse
    \item $(x, y)^{-1} = (\frac{x}{x^2+y^2}, \frac{-y}{x^2+y^2})$ als multiplikative Inverse
\end{itemize}
alle Axiome eines vollständigen Körpers.
\end{satz}

\begin{example}
Die Körperaxiome gilt es nun zu zeigen, da solche Beweise bereits genug oft beschrieben wurden, kann man das als Übung machen. Für Lösungen siehe Notizen vom 1.10.2020.
\end{example}

to add: Intuition für Addition und Multiplikation in der Ebene.



\begin{satz}{Fundamentalsatz der Algebra}{}
Jede algebraische Gleichung in $\C$
$$a_nz^n + a_{n-1}z^{n-1}+...+a_1z+a_0 = 0$$
von Grad $n\geq1$ mit $a_1,...,a_n \in \C, a_n\neq0$ hat (mindestens) eine Lösung.
\end{satz}
Der Fundamentalsatz der Algebra besagt also, dass alle Polynome in $\C$ auch Lösungen in $\C$ besitzen, so auch $x^2 + 1 = 0$.

Wir wollen nun noch einige Funktionen auf $\C$ definieren:

\begin{definition}{Funtionen auf $\C$}{}Sei $z = x + yi \in \C, x,y \in \R$
\begin{enumerate}
    \item \textbf{Realteil}: $\Re(z) = x \in \R$, wir nennen $z$ (rein) imaginär, falls $\Re(z)=0$
    \item \textbf{Imaginärteil}: $\Im(z) = y \in R$, wir nennen $z$ reell, falls $\Im(z)=0$
    \item \textbf{komplexe Konjugation}: $\quer{\ \ }: \C \to \C: z \mapsto \quer{z}, \quer{x+yi} \mapsto x-yi$
\end{enumerate}
\end{definition}

Die komplexe Konjugation ist eine Spiegelung an der realen Achse. Es gelten:
\begin{itemize}
    \item $\quer{z + w} = \quer{z}+\quer{w}$
    \item $\quer{z \cdot w} = \quer{z}\cdot\quer{w}$
    \item $\quer{1} = 1, \quer{0} = 0, \quer{i} = -i$
    \item $z\quer{z} = x^2 + y^2 \geq 0 \in \R$, Gleichheit genau dann, wenn $z=0$
    \item $z^{-1}= \frac{\quer{z}}{z\quer{z}}=(\frac{x}{z\quer{z}}, \frac{-y}{z\quer{z}})$
    \item $\Re(z) = \frac{z+\quer{z}}{2}$
    \item $\Im(Z) = \frac{z-\quer{z}}{2i}$
    \item $z = \quer{z} \iff z$ reell, $z=-\quer{z} \iff z$ imaginär.
\end{itemize}

\section{Absolutbetrag und Intervalle}
\subsection{Absolutbetrag}
In der Analysis werden wir sehr oft das Konzept des Abstands zwischen zwei Werten gebrauchen. Der \textbf{Absolutbetrag} soll daher eine Funktion sein, welche die Distanz zwischen einer Zahl und der 0 zurück gibt. Der Funktionswert ist stets positiv und wird wie folgt definiert:
\begin{definition}{Absolutbetrag}{}
Der \textbf{Absolutbetrag/Betrag} ist die Funktion:
\begin{align*}
    |\ \ |: \R &\to \R_{\geq 0}\\
    x &\mapsto |x| = \begin{cases} \ \ x &, x\geq 0\\-x &, x<0\end{cases}
\end{align*}
Alternative Definition: $|x| := \sqrt{x^2}$
\end{definition}
Wir können daraus direkt einige sehr wichtige Gleich- und Ungleichheiten für die Analysis herleiten:
\begin{satz}{Folgerungen}{} Für alle $x,y \in \R$ gilt:
\begin{enumerate}[label=(\alph*)]
    \item $|x| \geq 0$, Gleichheit genau dann, wenn $x=0$.
    \item $|x| = |-x|$
    \item $|xy| = |x||y|$
    \item $|\frac{1}{x}| = \frac{1}{|x|}, x \neq 0$
    \item $|x| \leq r \iff -r\leq x \leq r$ für $r \geq 0$.
    \item $|x| < r \iff -r < x < r$ für $r \geq 0$.
    \item \textbf{Dreiecksungleichung}: $|x+y| \leq |x|+|y|$ \\ Varianten: (i) $|x-y| \leq |x|+|y|$, (ii) $|x-y| \leq |x+z|+|y+z|$
    \item \textbf{umgekehrte Dreiecksungleichung}: $\big||x|-|y|\big| \leq |x-y|$
\end{enumerate}
\end{satz}
\begin{proof} Beweise für (e) und (g) sind hier aufgezeigt. Alle anderen Beweise sind im Abschnitt 2.4.2 von Skript zu finden.
\begin{itemize}
    \item[(e)]($\Longrightarrow$): Sei $x \in \R$ mit $|x| \leq r$. Falls $x\geq0$ ist, gilt $|x| = x \leq r$. Falls $x <0$ ist, gilt  $|x| = -x \leq r$, also $x \geq -r$. Es folgt die Behauptung. 
    
    ($\Longleftarrow$): Sei $-r\leq x \leq r$. Falls $x \geq 0$, gilt $x = |x| \leq r$. Falls $x < 0$, gilt $-r \leq x$, also $-x = |x| = \leq r$. Es folgt die Behauptung.
    \item[(f)] Analog zu (e).
    \item[(g)] Varianten: (i) Ersetze $y$ mit $-y$ (ii) Ersetze $x$ und $y$ in (i) mit $x-z$ resp. $y-z$
\end{itemize}
\end{proof}

Alle Formeln der Dreiecksungleichung sind etwas mühsam zu merken, weswegen wir sie so zusammenfassen:
\begin{lemma}{Dreiecksungleichung}{} Für alle $x,y$ in $\R$ gilt:
$$\Big| |x|-|y|\Big| \le |x \pm y| \le |x|+|y|$$
Aus Induktion folgt aus der Dreiecksungleichung diejenige für beliebige Terme:
$$\Big| \sum_i{x_i} \Big| \leq \sum_i{|x_i|}$$
\end{lemma}

\subsection{Absolutbetrag von komplexen Zahlen}
\todo{}
Cauchy-Schwartz Uglg

\subsection{Intervalle}
Mit der mathematischen Formulierung der Distanz können wir nun Teilmengen mit einem ''Abstandskriterium'' bestimmen. Wir können also z.B. eine Teilmenge bilden aus Mittelpunkt und Radius. Wir nennen diese Mengen \textbf{Bälle} (auch wenn sie auf $\R$ vorerst nur eindimensional sind; in $\C$ wären das Kreise):

\begin{definition}{offener Ball}{}
Der \textbf{offene} Ball von Radius $r > 0$ um $a \in \R$ ist die Teilmenge
$$B_r(a) = \set{x \in \R }{|x-a| < r}$$
\end{definition}

Das sind also alle Punkte zwischen $a-r$ und $a+r$, ohne den Endpunkten, also ohne ''Rand''. Für eindimensionale Teilmengen wollen wir den Begiff der \textbf{Intervalle} einführen:

\begin{definition}{offenes Intervall}{} Sei $a \leq b \in \R$. Das \textbf{offene Intervall} $(a,b)$ ist die Teilmenge:
$$(a, b) = \set{x \in \R}{a < x < b}$$
$a$ und $b$ sind die \textbf{Endpunkte} vom Intervall $(a,b)$. Die \textbf{Länge} des Intervalls ist $b-a$. Wenn beide Endpunkte in $\R$ sind, nennen wir es ein \textbf{beschränktes Intervall}. Wir nennen die Mengen 
\begin{align*}
    (a, \infty) &= \set{x \in \R}{a < x} = \R_{>a}\\
    (-\infty, b) &= \set{x \in \R}{x < b} = \R_{<b}\\
    (-\infty, \infty) &= \R
\end{align*}
\textbf{unbeschränkte Intervalle}.
\end{definition}

Somit sind also die Mengen $B_r(a)$ und $(a-r, a+r)$ gleich. Wir haben nun schon zweimal eine ''offene'' Menge definiert. Da bedeutet soviel wie, dass die Endpunkte im eindimensionalen Fall oder der Rand im zwei- oder dreidimensionalen Fall nicht mehr zur Menge gehört:

\begin{definition}{offene und abgeschlossene Mengen}{}
Eine Teilmenge $U \subseteq \R$ heisst \textbf{offen} (in $\R$) falls für alle $x \in U$ ein $\varepsilon > 0$ existiert, sodass der gesamte Ball um $x$ mit Radius $\varepsilon$ in $U$ enthalten ist:
$$\forall x \in U: \exists \varepsilon > 0: B_\varepsilon(x) \subseteq U$$
Eine Teilmenge $A \subseteq \R$ heisst \textbf{abgeschlossen}, falls das Komplement $\R \setminus A$ offen ist.
\end{definition}

Man kann sich die Offenheit einer Menge vage als ''enthält keinen Rand'' übersetzen: Man kann für jeden Punkt eine Umgebung (die nicht nur aus dem Punkt selber besteht, da der Radius $\varepsilon$ nicht 0 sein darf) finden, die noch in der Menge enthalten ist.

Schau dir zudem die Definition von der Abgeschlossenheit genau an! Man schliesst direkt von der Offenheit des Komplements auf die Abgeschlossenheit der Menge. In anderen Worten schliessen sich diese Bezeichnungen nicht unbedingt gegenseitig aus, d.h. es gibt \textit{keine} Implikation ''$U$ nicht offen $\implies$ $U$ abgeschlossen'' oder dergleichen. Die Menge wie auch das Komplement davon können beide jeweils unabhängig voneinander offen oder nicht offen sein:

\begin{example}[offene und abgeschlossene Mengen]
$U = \R$ ist offen, da um jeden Punkt ein Ball gelegt werden kann, der komplett in $\R$ enthalten ist. Daraus folgt, dass $\emptyset$ abgeschlossen ist. Da diese Menge aber leer ist, ist jede Aussage bezüglich deren Elemente wahr, also ist $\emptyset$ auch offen, wodurch $\R$ ebenfalls abgeschlossen ist. 
\end{example}

Wir können nun selber verifizieren, dass offene Bälle und Intervalle tatsächlich offene Mengen sind:

\begin{proof} Sei $x_0 \in (a, b) \subseteq \R$, also gilt $a<x_0<b$. Wir wählen das $\varepsilon$ aus der Definition offener Mengen wie folgt:
$$\varepsilon = \min\{x_0-a, b-x_0\}$$
Somit gilt $B_\varepsilon = \set{x \in \R}{|x-x_0| < \varepsilon}$. Sei das $x_0$ in der rechten Hälfte des Balles, also gilt $\varepsilon = b-x_0$. Ein beliebiges $x \in B_\varepsilon$ erfüllt somit die Bedingung $|x-x_0| < b-x_0$ resp. $-(b-x_0) < x-x_0 < b-x_0$. Wir wollen also zeigen, dass $x < b$ gilt. Betrachten wir die rechte Ungleichung, erhalten wir durch Addition von $x_0$ direkt $x < b$. Ähnlich ist es für $a < b$ und den Fall $\varepsilon = x_0-a$ zu zeigen.
\end{proof}

Wenn es offene Intervalle  gibt, kann man durch Ersetzen der strickten Gleichheit auch abgeschlossene Intervalle definieren:

\begin{definition}{abgeschlossenes Intervall}{} Sei $a \leq b \in \R$. Das \textbf{abgeschlossene Intervall} $[a,b]$ ist die Teilmenge:
$$[a, b] = \set{x \in \R}{a \leq x \leq b}$$
\end{definition}

Es ist auch schnell verifiziert, dass solche Intervalle nicht offen sind: Sei $[a,b]$ ein abgeschlossenes Intervall, dann lässt sich kein Ball um die Endpunkte legen.

\todo{Vereinigung von Intervallen}

\todo{halboffene Intervalle}

\subsection{Teilmengen auf $\C$}
\todo{Bälle auf $\C$}

\section{Ungleichungen und Abschätzungen}
Für die Bestimmung von Stetigkeit, Divergenz, Konvergenz, etc. sind sehr oft Abschätzungen von grosser Hilfe, um einen unbekannten Ausdruck nach oben oder nach unten abzuschätzen. In anderen Worten sucht man einen bekannten/einfacheren Ausdruck, der immer grösser oder kleiner als der unbekannte ist.

Hier nun einige wichtige Ungleichungen und Abschätzungen:
\subsection{Dreiecksungleichung}
Die Dreiecksungleichung wurde aus der Definition des Absolutbetrags hergeleitet. Folgende Varianten können für Abschätzungen in  $\mathbb{R}$ und $\mathbb{C}$ sehr hilfreich sein:
\begin{itemize}
    \item $\Big| |x|-|y|\Big| \le |x \pm y| \le |x|+|y|$ für alle $x,y$ in $\mathbb{C}$
    \item $\Big| \sum_i{x_i} \Big| \leq \sum_i{|x_i|}$ für $x_i$ in $\mathbb{R}$ oder $\mathbb{C}$ (folgt aus Induktion)
\end{itemize}

\subsection{Cauchy-Schwarz Ungleichung}
Für den $\mathbb{R}^n$ gilt die \textbf{Cauchy-Schwarz Ungleichung}:
$$\forall x,y \in \mathbb{R}^n: |x \cdot y| \leq |x| \cdot |y|$$
wobei $x \cdot y = \sum_i{x_i + y_i}$ im $\mathbb{R}^n$ die Skalarmultiplikation und $|x| = \sqrt{\sum_i{x_i^2}}$ die Norm ist. Zur Gleichheit kommt es nur bei koliniaren Vektoren, also $\exists \lambda: \lambda x = y$.

Intuitiv lässt sich diese Ungleichung aus der Definition der Skalarmultiplikation von Vektoren im $\mathbb{R}^2$ herleiten: $x, y \in \mathbb{R}^2: |x \cdot y| = |x|\cdot|y|\cdot\cos{\theta}$, wobei $\theta$ der Winkel zwischen $x$ und $y$ ist. Da der Kosinus im Betrag stets $\leq 1$ ist, gilt: $|x \cdot y| \leq |x|\cdot|y|$ 

Falls das nicht genügt, hier noch ein weiterer Ansatz: Durch Division erhalten wir $\frac{|x \cdot y|}{|x|} \leq |y|$. Links erhalten wir die Projektion von $y$ auf den Vektor $x$. Nun folgt intuitiv (oder aus Pythagoras oder der Dreiecksungleichung), dass $y$ selber länger als seine Projektion sein muss.

Die Ungleichung lässt sich auch in der Bra-Ket Notation aufschreiben, welches eine fäncy Art ist, um (u.a.) Vektoren auszudrücken. Im Gegensatz zu vorher wurde die Ungleichung quadriert, ansonsten sind die Ausdrücke äquivalent.
$$|\left< v | u\right>|^2\leq \left< v | v\right>\cdot\left< u | u\right>$$

\textit{Wahrscheinlich wird diese Ungleichung nicht so wichtig sein; hier steht wahrscheinlich viel zu viel dazu...}

\subsection{Bernoulli-Ungleichung} \label{cha_bernoulli_uneq}
Für $\forall x > -1 \in \mathbb{R}$ und $\forall n \in \mathbb{N}$ gilt:
$$(1+x)^n \geq 1 + nx$$
\begin{proof} Wir zeigen die Ungleichung mittels Induktion über $n$:

\textbf{IV}: Für $n=1$ erhalten wir $(1 + x)^1 \geq 1 + 1 \cdot x$, was wahr ist.

\textbf{IA}: Es gilt $(1+x)^n \geq 1 + nx$

\textbf{IS}: Für den Fall $n+1$ erhalten wir unter Verwendung der Induktionsannahme:
$$ (1+x)^{n+1} = (1+x)(1+x)^n \stackrel{\text{IA}}{\geq} (1+x)(1 + nx) = 1 + (x+1)n + \underbrace{nx^2}_{\geq 0} \geq  1 + (x+1)n$$
Dabei wurde für die erste Ungleichheit neben der Induktionsannahme auch $x>-1$ verwendet (ansonsten wäre der Faktor $(1+x)$ negativ). Da wir mit dem Induktionsschritt die Behauptung für $n+1$ gezeigt haben, gilt die Ungleichung für alle $n \in \N$.
\end{proof}

\section{Maximum, Minimum, Supremum, Infimum}
\todo{(VL 8.10.2020) Definition, (kleinste) obere/ (grösste) untere Schranke, Eindeutigkeit, Bezug zum Vollständigkeitsaxiom, Folgerung: $\N$ wohlgeordnet}

\todo{(VL 12.10.2020, Seite 5) Konventionen $\sup(M) = \pm \infty$}


\section{Konsequenzen der Vollständigkeit} \label{cha_consequences_completeness}
\subsection{Archimedische Eigenschaft von $\mathbb{R}$}
Die Aussage vom \textbf{Archimedische Eigenschaft} ist so offensichtlich, und trotzdem können die Folgerungen sehr hilfreich sein:
$$\forall x,y \in \mathbb{R}_{>0} \exists n \in \mathbb{N}: ny\geq x$$
In Worten: Jede nicht-negative (reelle) Zahl lässt durch Addition mit sich selber beliebig gross machen (Beweis durch Widerspruch mit oberen Schranken). Die Implikationen sind:

\begin{satz}{Folgerungen}{}
\begin{itemize}
    \item $\forall \epsilon \in \mathbb{R}_{>0} \ \exists n \in \mathbb{N}: \frac{1}{n} < \epsilon$
    \item $\forall n \in \mathbb{N}: x \leq \frac{1}{n} \implies x = 0$ \hspace*{\fill} (\textit{keine positiven Infinitesimalgrössen})
    \item $\forall x \in \mathbb{R} \ \exists! n \in \mathbb{Z}: n \leq x < n+1$\hspace*{\fill} (\textit{ganzer Anteil/Abrundung ist eindeutig})
    \item \todo{$\Q$ liegt dicht in $\R$}
    \item \todo{Division mit Rest ist eindeutig}
\end{itemize} 
\end{satz}

Daraus folgt zum Beispiel, dass sich $\forall n \in \mathbb{N}: \frac{1}{2}-\frac{1}{n} \leq a \leq \frac{1}{2}+\frac{1}{n}$ auch ohne der Verwendung von Limes-Rechenregeln berechnen lässt (Anwendung z.B. bei der Berechnung vom Riemann-Integral):
$$-\frac{1}{n} \leq a - \frac{1}{2} \leq \frac{1}{n}$$
Nun folgt aus der Archimedischen Eigenschaft $\frac{1}{n} \geq -(a-\frac{1}{2})$ und $a-\frac{1}{2} \leq \frac{1}{n}$. Durch zweifache Anwendung der Eigenschaft erhalten wir $a - \frac{1}{2} = 0$, also $a = \frac{1}{2}$

\todo{(VL 12.10.2020) Beweise aller Folgerungen}

\section{Häufungspunkte}
Dieses Konzept haben wir in Abschnitt \ref{cha_limit_points} eingeführt.

\section{Intervallschachtelungsprinzip, Überabzählbarkeit von $\R$}
\todo{(VL 12.10.2020 Seite 6) Definition}
\todo{(VL 14.10.2020) $|[0, 1]| > |\N|$, Kontinuumshypothese}
\section{Konstruktion von $\R$}
\todo{(VL 14.10.2020) Konstr. von $\R$ aus $\Q$, def. Dedekind-Schnitt, Eindeutigkeit $\R$}

    \chapter{Differenzialrechnung}

\section{Grundlagen}\label{cha_grundlagen_diffrechnung}
Bevor wir uns Ableitungen, Integrale, usw. anschauen können, ist es wichtig, ein solides Fundament aus Grundbegriffen zu bauen, um die Definitionen möglichst präzis formulieren zu können. Die wichtigsten Grundlagen seien daher in den nachfolgenden Unterkapiteln erklärt.

Wie wir schon aus der Mittelschule gelernt haben, wollen wir mit der Ableitung einer Funktion eine Aussage über die Steigung machen können. Dazu haben wir mit immer kleineren Steigungsdreiecken Sekanten gebildet mit der Steigung $\frac{\Delta y}{\Delta x}$, welche dann im Grenzfall $\lim_{\Delta x \to 0}$ zur Tangente der Funktion wurde. Um diesen Grenzwert bilden zu können, benötigen wir für $\Delta x$ eine Nullfolge, die jedoch nie den Wert $0$ annimmt, da wir ansonsten durch $0$ teilen. Als \textbf{Häufungspunkte} wollen wir nun genau die Punkte bezeichnen, bei welchen man solche konvergierende Folgen finden kann, sodass sich dann der \textbf{Grenzwert einer Funktion} bilden lässt.

\subsection{Häufungspunkte}\label{cha_limit_points}
Um Häufungspunkte definieren zu können, müssen wir erst einmal das Konzept der \textbf{Umgebung} aus dem  Teilbereich der Topologie einführen:

\begin{definition}{Umgebung}{}
    Seien $x_0 \in \mathbb{R}$ und $ \varepsilon > 0$. Wir nennen 
    $$U_\varepsilon\,(x_0) := (x_0 - \varepsilon , x_0 + \varepsilon)\quad \text{ bzw. } \quad U_\varepsilon\,(x_0) := \{\ x \in \mathbb{R} \mid \abs{x-x_0} < \varepsilon \,\}$$
     die $\varepsilon$-Umgebung von $x_0$.
\end{definition}
    
 Die $\varepsilon$-Umgebung enthält alle Punkte $x$, die sich innerhalb eines gewissen Abstand  $\varepsilon$ von $x_0$ befinden, also auch $x_0$ selbst. Die Umgebung ist geometrisch das selbe die  bereits vorher verwendeten Bälle $B_\varepsilon(x_0)$, jedoch wird in diesem Kontext generell von Umgebungen gesprochen. Im Gegensatz dazu definieren wir die 
 
\begin{definition}{punktierte Umgebung}{}
$$\dot{U}_\varepsilon(x_0) := (x_0 - \varepsilon, x_0) \cap (x_0, x_0 + \varepsilon) \quad \text{ bzw. } \quad \dot{U}_\varepsilon(x_0) := U_\varepsilon(x_0)\setminus \{x_0\}$$
wobei $x_0 \in \R$ und $\varepsilon>0$
\end{definition}
als die \textbf{punktierte} $\varepsilon$-Umgebung von $x_0$. Der Unterschied zwischen der ''normalen'' und der punktierten Umgebung ist also, dass die punktierte Umgebung den Punkt $x_0$ selbst nicht enthält. Diese Eigenschaft wird sich später für einige Definitionen als sehr nützlich erweisen.

Mit diesen Grundbegriffen können wir nun Häufungspunkte definieren:

\begin{definition}{Häufungspunkt}{}
$x_0 \in \R$ ist ein Häufungspunkt einer Teilmenge $D \subseteq \R$, falls die in der Teilmenge $D$ enthaltenen punktierte Umgebung $\dot{U}_\varepsilon(x_0) \cap D$ mit einem $\varepsilon>0$ nicht leer ist:
$$\forall \varepsilon > 0 : \dot{U}_\varepsilon(x_0) \cap D \ne \emptyset $$
\end{definition}

In Worten: Für jede noch so kleinen Radius um den in Frage stehenden Punkt können wir unendlich weitere Punkte im Inneren des Radius finden, welche Teil der Menge sind. Am besten lässt sich das anhand von ein paar Beispielen zeigen:
\begin{example} \label{07ex_hp1}
\begin{enumerate}
    \item Sei $D = (a, b), a<b$, dann sind $[a, b]$ Häufungspunkte von $D$.
    \item Sei $D = [0, 1) \cap \{2\}$, dann sind $[0,1]$ Häufungspunkte von $D$.
    \item Sei $D = \{\frac{1}{n} \mid n \in \N\}$, dann ist nur $\{0\}$ ein Häufungspunkt von $D$.
\end{enumerate}
\end{example}


Wie man sieht, sind auch Randpunkte Häufungspunkte, isolierte Punkte jedoch (oben in Bsp. 2 die Menge $\{2\}$) können keine Häufungspunkte sein.

Da die Menge eines Häufungspunktes dieses $\varepsilon$-Kriterium erfüllt, ist sie sehr eng verwandt mit nach $x_0$ konvergente Folgen, darum hier noch ein Lemma, welches die beiden verbindet:

\begin{lemma}{Häufungspunkte und Folgen}{}
Sei $D \subseteq \R$, dann sind die folgenden Aussagen äquivalent:
\begin{enumerate}
    \item $x_0$ ist ein Häufungspunkt.
    \item Es gibt eine Folge $(x_n)_{n \in \N} \in D\setminus\{x_0\}$, sodass $\lim_{n \to \infty}{x_n} = x_0$
    \item $\forall \varepsilon > 0 : \dot{U}_\varepsilon(x_0) \cap D$ hat unendlich viele Punkte
\end{enumerate}
\end{lemma}


\subsection{Grenzwerte von Funktionen}
Nun haben wir mit einem Häufungspunkt $x_0 \in D$ eine Menge an Punkten, auf welchen wir beliebig nahe an $x_0$ gehen können. Jetzt brauchen wir noch unsere auf $D$ definierte Funktion $f(x)$, dann lassen sich mit der Definition vom Limes auch schon gleich ein Grenzwert bilden:

\begin{definition}{Grenzwert einer Funktion}{}
Sei $D \subseteq \R$ und $f: D \to \R$ eine Funktion und $x_0$ ein Häufungspunkt von $D$. Dann nennen wir $a \in \R$ den \textbf{Grenzwert} von $f$ für $x \to x_0$ falls für alle $\varepsilon>0$ ein $\delta>0$ existiert, sodass:
$$x\in\dot{U}_\delta(x_0) \cap D \implies | f(x) - a | < \varepsilon$$
oder
$$x\in\dot{U}_\delta(x_0) \cap D \implies  f(x) \in \dot{U}_\varepsilon(a)$$
Wir schreiben 
$$a = \lim_{x \to x_0}{f(x)}$$
\end{definition}

Somit können wir Grenzwerte nur in Häufungspunkten definieren, ansonsten ergibt die Definition des Grenzwertes keinen Sinn. Des Weiteren müssen wir auch zuerst garantieren, dass der Grenzwert eindeutig ist, bevor wir diesen als $a$ bezeichnen können. Dieser Begriff des Grenzwertes ist auch sehr nahe an der Definition der Stetigkeit und kann auch durch eine Umformulierung des obigen Kriteriums erhalten werden:

\begin{lemma}{Zusammenhang zur Stetigkeit}{Zusammenhang_zur_Stetigkeit}
Sei $x_0$ ein Häufungspunkt mit $x_0 \in D$, dann gilt für $f: D \to \R$
$$f(x_0) = \lim_{x \to x_0}{f(x)} \iff \text{f stetig in } x_0$$
\end{lemma}

Da wir jedoch bei der Definition der Stetigkeit keine punktierten Umgebungen verwendet haben, könnte man meinen, die Aussage des Lemmas oben etwas schwächer ist. Jedoch stellt sich heraus, dass trotz der punktierten Umgebung des Grenzwertes die Definitionen äquivalent sind (Beweis dem Dozenten überlassen).

\begin{example}\label{07b1}
Hier ein Beispiel aus der Vorlesung: Sei
$$f: \R \to \R, f(x)=\begin{cases}x & x \ne 1\\2 & x = 1\end{cases}$$
dann gilt für den Grenzwert $\lim_{x \to 1}{f(x)} = 1 \ne f(1)$, welches die Unstetigkeit von $f$ zeigt.
\end{example}

\subsubsection{Stetige Fortsetzung}

Wie wir aus dem obigen Beispiel gesehen haben, lässt sich diese Unstetigkeit beheben lässt. Man nennt diese Stellen \textbf{hebbare Unstetigkeitsstellen}. Wenn man also nur die unmittelbare Umgebung von $x_0$ betrachtet, dann lässt sich die Funktion durch die Redefinition von $f(x_0) = \lim_{x \to x_0}{f(x)}$ \textbf{stetig fortsetzen}:

\begin{definition}{stetige Fortsetzung}{}
Wenn $x_0 \notin D$, $x_0$ ein Häufungspunkt von $D$ und $f$ stetig ist mit $\lim_{x \to x_0}{f(x)} = a$, dann können wir eine Fortsetzung $\widetilde{f}: D \cup \{x_0\} \to \R$ durch
$$\widetilde{f}(x)=\begin{cases}f(x) & x \in D\\a & x = x_0\end{cases}$$
definieren, die stetig ist, da per Konstruktion folgendes gilt: $$\lim_{x \to x_0}{\widetilde{f}(x)} = \lim_{x \to x_0}{f(x)} = a = \widetilde{f}(x_0)$$
\end{definition}

\begin{example} Fortsetzung des Beispiels \ref{07b1}: Durch die Einschränkung von $f: \R \setminus \{1\}$ erhalten wir durch die stetige Fortsetzung $f(x) = x$.
\end{example}

\subsubsection{Grenzwerte von Funktionen und Folgen}

Da Grenzwerte vieles mit Folgen gemeinsam haben und wir aus den vorherigen Kapiteln bereits schon sehr viel über Folgen wissen, gibt es die folgende sehr nützliche/intuitive alternative Definition von Grenzwerten durch Folgen:

\begin{lemma}{Grenzwerte von Funktionen durch Folgen}{Grenzwerte_von_Funktionen_durch_Folgen}
Sei $x_0$ ein Häufungspunkt von $D \subseteq \R, f: D \to \R$ und $a\in \R$, dann gilt:
$$a = \lim_{x \to x_0} f(x) \iff \forall (x_n)_{n \in \N} \subseteq D \setminus \{x_0\}: a = \lim_{n \to \infty}{f(x_n)}$$
wobei $(x_n)_{n \in \N}$ Folgen sind, welche gegen $x_0$ konvergieren, aber $x_0$ nie annehmen.
\end{lemma}

\begin{remark}{1}
Es gibt nach $x_0$ konvergente Folgen $(x_n)_{n \in \N}$ in  $\subseteq D \setminus \{x_0\}$, da $x_0$ ein Häufungspunkt ist.
\end{remark}

Somit können wir viele Eigenschaften, welche wir bereits für die Folgen definiert haben, auch auf die Grenzwerte von Funktionen anwenden, so z.B. die Eindeutigkeit des Grenzwertes, das Sandwich-Lemmas, Linearität, Monotonie, etc.

\begin{remark}{2}
Auch unterscheidet sich Lemma \ref{lem:Zusammenhang_zur_Stetigkeit} von Lemma \ref{lem:Grenzwerte_von_Funktionen_durch_Folgen} durch die Definition von $x_0$, welches im letzterem Lemma nicht zwingend $\in D$ ist (Siehe Beispiel \ref{07ex_hp1}, wo auch offene Intervallsgrenzen Häufungspunkte der Menge sind). Des Weiteren wird in Lemma \ref{lem:Grenzwerte_von_Funktionen_durch_Folgen} auch nicht von einer stetigen Funktion ausgegangen.
\end{remark}

Aus den Sätzen zu Grenzwerten von Folgen folgt nun auch für die Grenzwerte von Funktionen:
\begin{satz}{Eigenschaften von Grenzwerten von Funktionen}{}
\begin{enumerate}
    \item Der Grenzwert $\lim_{x \to x_0} f(x)$ ist eindeutig, wenn er existiert.
    \item Linearität:\\Wenn $f,g:D \to \R, x_0$ ein Häufungspunkt von $D$ ist mit $\lim_{x \to x_0} f(x) = a$, $\lim_{x \to x_0} g(x) = b$, dann gilt:
    $$\lim_{x \to x_0}{\left( f(x) + g(x) \right) }= a + b$$
    $$\lim_{x \to x_0}{\left( f(x) g(x) \right) }= ab$$
    \item $f \leq g \implies \lim_{x \to x_0}{f(x)} \leq \lim_{x \to x_0}{g(x)}$
    \item Sandwich-Lemma:\\$f\leq g \leq h : D \to \R$, $x_0$ Häufungspunkt. Wenn $\lim_{x \to x_0}{f(x)}=\lim_{x \to x_0}{h(x)}=a$, dann existiert der Grenzwert auch für $\lim_{x \to x_0}{g(x)} = a$
\end{enumerate}
\end{satz}

\begin{example}[Sandwich-Lemma]
Nun wollen wir die von den Folgen übertragenen Sätze in einem Beispiel anwenden. Sei $f(x) = \frac{\sin{x}}{x}$, $f: \R \setminus \{0\} \to \R$, $0$ ein Häufungspunkt, dann ist$\lim_{x \to 0}{\frac{\sin{x}}{x}} = $ definiert, weil:
$$x -\frac{x^3}{3!} \leq \sin{x} \leq x \qquad \text{für } x > 0$$
$$1 -\frac{x^2}{3!} \leq \frac{\sin{x}}{x} \leq 1 \qquad \text{für } x > 0$$
Da diese Funktionen alle gerade sind, gilt dasselbe auch für $x<0$. Da $1-\frac{x^2}{6}$ und $1$ stetig auf $\R$ sind (Grenzwert nimmt Funktionswert bei $x=0$ an), lässt sich das Sandwich-Lemma für die Bestimmung des Grenzwertes verwenden:
$$\lim_{x \to 0}{\frac{\sin{x}}{x}} = 1$$
\end{example}

\begin{example}[Verschachtelte Funktionen]
Seien $f: D \to E \subseteq \R$ und $g: E \to \R$ Funktionen, sodass $g$ stetig ist bei $y_0 = \lim_{x \to x_0} f(x)$, dann gilt:
$$\limto{x}{g(f(x)) = g(y_0)}$$
\end{example}

\subsubsection{Links- und Rechtsseitige Grenzwerte}
Wie man vielleicht schon mal gesehen hat, kann einen Grenzwert auch ausschliesslich von links oder von rechts bilden. Um dies zu ermöglichen, müssen wir die zuvor eingeführten Grundbegriffe nochmals umdefinieren: Für links- und rechtsseitige Grenzwerte brauchen wir analog links- bzw. rechtsseitigen $\varepsilon$-Umgebungen resp. Häufungspunkte:

\begin{definition}{Einseitige Umgebungen und Häufungspunkte}{}
Analog zur punktierten Umgebung definieren wir die \textbf{linke/rechte punktierte Umgebung} von $x_0 \in \R$ als:
$$\dot{U}_\varepsilon^-(x_0) := (x_0-\varepsilon, x_0) \qquad\textit{linke punktierte Umgebung}$$
$$\dot{U}_\varepsilon^+(x_0) := (x_0, x_0+\varepsilon) \qquad\textit{rechte punktierte Umgebung}$$

$x_0$ nennen wir \textbf{links/rechtsseitiger Häufungspunkt}, falls $\forall \varepsilon >0$
$$\dot{U}_\varepsilon^-(x_0) \cap D \ne \emptyset \qquad\textit{linksseitiger Häufungspunkt}$$
$$\dot{U}_\varepsilon^+(x_0) \cap D \ne \emptyset \qquad\textit{rechtsseitiger Häufungspunkt}$$
\end{definition}

\begin{example}[einseitiger Häufungspunkt] Sei $a<b$, dann ist $b$ ein linksseitiger Häufungspunkt von $(a,b)$ und $(a,b]$, jedoch kein rechtsseitiger Häufungspunkt dieser Intervalle.
\end{example}

\begin{remark}
\begin{itemize}
    \item Links- und rechtsseitige Häufungspunkte von $D$ sind auch Häufungspunkte von $D$, wir nennen diese \textbf{einseitige Häufungspunkte}.
    \item Häufungspunkte sind links-seitige, rechts-seitig oder beidseitige Häufungspunkte.
    \item Für Intervalle $I = [a, b], [a, b), (a, b), [a , \infty), (-\infty, b)$ etc... sind rechte Endpunkte $<\infty$ linksseitige Häufungspunkte, linke Endpunkte $>-\infty$ rechtsseitige Häufungspunkte und die inneren Punkte sind beidseitige Häufungspunkte.
\end{itemize}
\end{remark}


\begin{definition}{Einseitige Grenzwerte}{}
Sei $f: D \to \R, x_0$ ein linksseitiger Häufungspunkt von $D$. Wir nennen $a$ den \textbf{linksseitigen Grenzwert} von $f$ für $x \to x_0$ und schreiben
$$a = \lim_{x \nearrow x_0}f(x)$$
falls
$$\forall \varepsilon>0 \ \exists \delta>0: x \in \dot{U}_\delta^-(x_0)\cap D \implies f(x) \in U_\varepsilon(a)$$
Es folgt:
$$a = \lim_{x \nearrow x_0} f(x) \iff \forall (x_n)_{n \in \N}: a = \lim_{n \to \infty}{f(x_n)}$$
wobei $(x_n)_{n \in \N}$ Folgen sind in $D \cap \R_{<x_0}$, welche gegen $x_0$ konvergieren.
\end{definition}
Dazu gibt es ein sehr nützliches Lemma, das wir noch öfters brauchen werden:
\begin{lemma}{Beidseitige Grenzwerte}{Beidseitige_Grenzwerte}
Sei $f:D \to \R$ und $x_0$ ein beidseitiger Häufungspunkt von $D$. Dann gilt:
$$a=\lim_{x \to x_0}{f(x)} \Longleftrightarrow \lim_{x \nearrow x_0}f(x) = \lim_{x \searrow x_0}f(x)=a$$
\end{lemma}

\begin{example}[Einseitige Grenzwerte]Berechne den Grenzwert von $x \log x$ und $x^x$:
\begin{align*}
\lim_{x \searrow 0} x \log x &= \lim_{x \searrow 0}e^{\log x} \log x\\
&= \lim_{x \searrow 0} \frac{\log x}{e^{- \log x}}\\
&= - \lim_{x \searrow 0} \frac{-\log x}{e^{- \log x}}
\end{align*}
Bemerke nun, dass im Nenner für kleine $x$ ein sehr grosser Exponent entsteht. Wir wollen also zeigen, das $e^{-\log x}$ schneller wächst als $-\log x$. Wir verwenden also zur Abschätzung die Reihen-Definition von $e^x$: Für ein $a>0$ gilt: $$e^a \geq 1 + a + \frac{a^2}{2} \geq \frac{a^2}{2}$$
$$0\leq \frac{1}{e^a} \leq \frac{2}{a^2}$$
Sei nun $a = -\log(x)$, es folgt somit:
\begin{align*}
    0 \leq \lim_{x \searrow 0} \frac{-\log x}{e^{- \log x}} & \leq \lim_{x \searrow 0} \frac{-2\log x}{(- \log x)^2}= \lim_{x \searrow 0} \frac{2}{- \log x} = 0
\end{align*}
Das heisst, wir haben $\lim_{x \searrow 0} x \log x = 0$ gezeigt. Daraus können wir nun auch den Grenzwert von $\lim_{x \searrow 0}x^x$ bestimmen:
$$\lim_{x \searrow 0} x^x = \lim_{x \searrow 0} e^{x \log x} = e ^0 = 1$$
\end{example}


\subsubsection{Uneigentlicher Grenzwert von $x_0$}
Nun wollen wir auch Grenzwerte von Stellen bestimmen können, welche z.B. durch Definitionslücken von Brüchen oder bei $f(x)=\log{x}$ mit $x_0 = 0$ entstehen und ins Unendliche divergieren. Diese Grenzwerte nennen wir \textbf{uneigentliche Grenzwerte}:
\begin{definition}{Uneigentliche Grenzwerte}{}
Wir schreiben $\lim_{x \to x_0}{f(x)} = \infty$ (analog für $-\infty$), falls zu jedem $M > 0$ ein $\delta > 0$ existiert, sodass:
$$x \in \dot{U}_\delta(x_0) \cap D \implies f(x) > M$$
Es ist äquivalent zu 
$$\forall (x_n)_{n \in \N}: \lim_{n \to \infty}{f(x_n)} = \infty $$
wobei $(x_n)_{n \in \N} \subseteq D \setminus \{x_0\}$ beliebige Folgen sind mit $\lim_{n \to \infty}{x_n} = x_0$
\end{definition}
In Worten: Für jedes noch so grosse $M$ finden wir nahe an $x_0$ liegende Argumente, sodass der Funktionswert $f(x)$ grösser als M ist.

\begin{example}[Uneigentlicher Grenzwert in Häufungspunkt] Sei $f: \R\setminus\{0\}$, $x \mapsto \frac{1}{x}$, dann gilt:
$$\lim_{x \nearrow 0} \frac{1}{x} = - \infty$$
Das bedeutet, dass für alle $M > 0$ ein $\delta >0$ existiert, sodass $\frac{1}{x} < -M$ für alle $x \in (-\delta, 0) = U_\delta^-(0)$ gilt.
Analog gilt:
$$\lim_{x \searrow 0} \frac{1}{x} = \infty$$
\end{example}

\subsubsection{Grenzwert von uneigentlichen Häufungspunkten}
Bis jetzt haben wir aber nur Grenzwerte von einem bestimmten Punkt $x_0$ in einer Menge $D$ bestimmt, welche einen endlichen oder unendlichen Funktionswert (also einen Grenzwert resp. uneigentlichen Grenzwert) haben. Was ist jedoch, wenn wir dieses $x_0$ nach $\pm \infty$ laufen lassen, also ein unendliches Funktionsargument haben? Für das benötigen wir zuerst ein nicht von oben beschränktes $D$, d.h. $D$ hat beliebig grosse Elemente. Wir können dann schreiben:
\begin{definition}{Grenzwert von $\pm \infty$}{}
$$a = \lim_{x \to \infty} f(x)$$
falls für alle $\varepsilon>0$ ein $M>0$ existiert, sodass:
$$x \in D, x>M \implies |f(x)-a| < \varepsilon$$
(analog für $-\infty$)
\end{definition}

In Worte gefasst, finden wir einen Grenzwert für immer näher an $\infty$ liegende Argumente. Wir haben somit gewissermassen eine $\varepsilon$-Umgebung von $\infty$. Diese können wir auch mathematisch rigoros definieren als ein

\begin{definition}{Uneigentlicher Häufungspunkt}{}
Wir definieren die $\varepsilon$-Umgebung von $\infty$ als:
$$\dot{U}_\varepsilon(\infty) := \left\{x \in \R \mid x > \frac{1}{\varepsilon}\right\}$$
(analog für $-\infty$)
\end{definition}
d.h. für den Grenzwert $a = \lim_{x \to \infty}f(x)$ bedeutet das in der Schreibweise mit der $\varepsilon$-Umgebung von $\infty$:
$$\forall \varepsilon>0 \ \exists \delta>0: x \in \dot{U}_{\frac{1}{\delta}}(\infty) \cap D \implies f(x) \in U_\varepsilon(a)$$

\begin{remark}
Felder nennt sowohl den \textit{uneigentlichen Grenzwert} wie auch den \textit{Grenzwert von uneigentlichen Häufungspunkten} einfach nur \textit{uneigentlicher Grenzwert}. Wir haben uns dafür entschieden, eine klare Differenzierung zwischen den beiden Begriffen zu machen, da bei Grenz\textbf{werten} von uneigentlichen Häufungspunkten der Wert der Funktion sehr wohl definiert ist, während die uneigentlichen Grenzwerte nicht in $\R$ sind.
\end{remark} 

\begin{example}[Uneigentlicher Grenzwert in uneigentlichem Häufungspunkt] Sei $\log{x}: (0, \infty) \to \R$, dann gilt:
$$ \limtoinf{x}{\log(x)}$$
Das bedeutet, dass $\forall M >0 \ \exists N>0$, sodass für jedes $x \in (N, \infty): \log x > M$ gilt. Des Weiteren können wir auch den uneigentlichen Grenzwert von $\lim_{x \searrow 0} \log{x}$ berechnen
$$\lim_{x \searrow 0} \log x = - \infty$$
indem wir $\delta \leq e^{-M}$ wählen. Denn dadurch gilt $|x - 0| = x < e^{-M}$ und somit erhält man durch Einsetzen von $x$ in $\log$: $\log x < \log e^{-M} = -M $, welches den uneigentlichen Grenzwert von $\lim_{x \searrow 0} \log x = -\infty$ zeigt.
\end{example}

\begin{tipp}{}{}
Wir haben in diesem Kapitel der Vorlesung folgende Identitäten gezeigt:
\begin{itemize}
    \item $\lim_{x \to 0}{\frac{\sin{x}}{x}} = 1$
    \item $\lim_{x \searrow 0} x \log x = 0$
    \item $\lim_{x \searrow 0} x^x = 1$
    \item $\lim_{x \searrow 0} \log x = -\infty$, $\lim_{x \to \infty} \log x = \infty$
\end{itemize}
Beachte, dass man auch Limites durch ''Substitution'' in eine bekannte Form umformen kann, z.B. $\limtoinf{x}{\frac{\sin{(x + 1)}}{x+1}}$ lässt sich mit der Substitution $y = x + 1$, also $\limtoinf{x}{\frac{\sin{(x + 1)}}{x+1}} = \limtoinf{y}{\frac{\sin y}{y}} = 1$ berechnen. %Maybe an eine bessere Stelle packen
\end{tipp}

\subsection{Landau Notation}\label{cha_landau_notation}
\subsection{Leibniz Notation}

\section{Ableitung}
Bei der Ableitung stellen wir uns die Frage, wie schnell sich eine gewisse Grösse (Funktionswert) relativ zu einer anderen (Funktionsargument) ändert. Derartige Betrachtungen sind beispielsweise in der Physik heute nicht mehr wegzudenken. Um sagen zu können, wie sich ein Funktionswert ändert, benötigen wir die zuvor definierten Häufungspunkte und Grenzwerte. Ohne Häufungspunkten wäre diese Betrachtung auch sehr sinnlos, da isolierte Punkte in sich auch keine Änderungsraten aufweisen können. 
\subsection{Differenzenquotient und Differenzierbarkeit}
Ähnlich wie wir es schon vor der Uni gemacht haben, werden wir uns erst kurz den Differenzenquotienten anschauen. Dieser berechnet die \textbf{durchschnittliche} bzw. \textbf{mittlere Änderungsrate} des Graphen einer Funktion $f$ zwischen zwei Punkten $f(x_0)$ und $f(x)$. Man könnte auch sagen: Der Differenzenquotient berechnet die Steigung der Sekante, die durch die zwei Punkte $f(x_0)$ und $f(x)$ geht. Ganz konkret ist diese Steigung gegeben durch:
$$\frac{f(x)-f(x_0)}{x-x_0}$$
Der Differenzenquotient ist schon eine relativ gute Approximation der Steigung der Funktion. Was ist jedoch, wenn wir es genauer wissen wollen? Intuitiv würden wir sagen, dass man die zwei Punkte einfach immer näher zueinander rücken lässt, z.B. $x$ zu $x_0$. Die Approximation wird so immer besser, bis man im Grenzfall die Steigung der Tangente durch $f(x_0)$ erhält. Um diesen Schritt zur Tangente auch machen zu können, muss $x_0$ ein Häufungspunkt sein. Dadurch ist es garantiert, dass man immer kleinere Schritte für den Differenzenquotienten betrachten kann und auch den Grenzwert davon bilden kann. So macht man den Sprung von der mittleren Änderungsrate hin zur tatsächlichen, \textbf{momentanen Änderungsrate}, welche wir dann \textbf{Ableitung} von $f$ bei $x_0$ nennen, notiert $f'(x_0)$ oder $\frac{df(x_0)}{dx}$.    
\begin{definition}{Ableitung und Differenzierbarkeit}{Ableitung_Differenzierbarkeit}
Seien $D \subseteq \R$, $f:D \to \R$ und $x_0 \in D$ ein Häufungspunkt von $D$. $f$ heisst \textbf{differenzierbar} bei $x_0$, falls der Grenzwert
$$f'(x_0) := \lim_{x \to x_0}{\frac{f(x)-f(x_0)}{x-x_0}} = \lim_{h \to 0}{\frac{f(x_0 + h)-f(x_0)}{h}}$$
existiert. Wir nennen dann $f'(x_0)$ die \textbf{Ableitung} bzw. \textbf{momentane Änderungsrate} von $f$ bei $x_0$. $f$ heisst \textbf{differenzierbar}, falls $f$ bei allen Häufungspunkten in $D$ differenzierbar ist.
\end{definition}
\begin{remark}
Den Term $\lim_{x \to x_0}{\frac{f(x)-f(x_0)}{x-x_0}}$ nennen wir nun nicht mehr Differenzenquotient, sondern \textbf{Differentialquotient}. Durch die alternative Schreibweise $\lim_{h \to 0}{\frac{f(x_0 + h)-f(x_0)}{h}}$, die manchmal auch \textbf{h-Methode} genannt wird, wird noch einmal mehr ersichtlich, wie man sich bei der Ableitung immer näher an $x_0$ annähert, um dort die Ableitung zu ermitteln. Bei Rechnungen kann diese auch nützlich sein.
\end{remark}
Ähnlich wie bei Häufungspunkten und Grenzwerten unterschieden wir auch bei der Differenzierbarkeit zwischen \textbf{rechts-} und \textbf{linksseitiger Differenzierbarkeit}. 
\begin{definition}{Einseitige Differenzierbarkeit}{}
Falls $x_0 \in D$ ein linksseitiger Häufungspunkt ist, dann heisst $f:D \to \R$ linksseitig differenzierbar, wenn
$$f_{-}^{\prime}\left(x_{0}\right)=\lim _{x \nearrow x_{0}} \frac{f(x)-f\left(x_{0}\right)}{x-x_{0}}=\lim _{h \nearrow 0} \frac{f\left(x_{0}+h\right)-f\left(x_{0}\right)}{h}$$
existiert. $f_-'(x_0)$ nennt man dann \textbf{linksseitige Ableitung}. Analog ist die \textbf{rechtsseitige Ableitung} $f_+'(x_0)$ definiert.
\end{definition}
\begin{remark}
Aufgrund von Lemma \ref{lem:Beidseitige_Grenzwerte} für Grenzwerte von Funktionen gilt auch hier: $$f'(x_0)=\lim_{x \to x_0}\frac{f(x)-f(x_0)}{x-x_0} \Longleftrightarrow \lim_{x \nearrow x_0}\frac{f(x)-f\left(x_{0}\right)}{x-x_{0}} = \lim_{x \searrow x_0}\frac{f(x)-f\left(x_{0}\right)}{x-x_{0}}=f'(x_0)$$
\end{remark}
\begin{lemma}{Differenzierbarkeit und Stetigkeit}{}
Wenn $f:D \to \R$ bei $x_0 \in D$ differenzierbar ist, dann ist $f$ stetig bei $x_0$.\\ Insbesondere gilt: Differenzierbare Funktionen sind stetig.
\end{lemma}
\begin{remark}
Die Umkehrung gilt nicht. Als Beispiel sei nur die Funktion $f(x)=\abs{x}$ genannt.
\end{remark}

\subsection{Extrempunkte}
Das Prinzip der Extrempunkte ist eigentlich altbekannt: Funktionen können Maxima und Minima annehmen. Mit den erarbeiteten Grundlagen können wir nun eine präzise Definition geben:

\begin{definition}{Lokale Maxima und Minima}{}
Sei $D \subseteq \R$, $f:D \to \R$. Wir sagen, dass f ein \textbf{lokales} Maximum bei $x_0 \in \R$ hat, wenn ein $\delta > 0$ existiert, sodass
\begin{center}
$f(x) \leq f(x_0)$ für alle $x \in U_\delta\,(x_0) \cap D$
\end{center}
Wir sprechen von einem \textbf{isolierten lokalen} Maximum bei $x_0$, falls sogar die strikte Ungleichung gilt:
\begin{center}
$f(x)<f(x_0)$ für alle $x\in\dot{U}_\delta(x_0) \cap D$    
\end{center}
Die Definition des Minimums erfolgt analog mit $\geq$ bzw. >.


Merke! Wir nennen den Funktionswert $f(x_0)$ \textbf{Extremum} bzw. \textbf{Extremwert} und die Stelle $x_0$ \textbf{Extremstelle}. Die Kombination aus Wert und Stelle nennt man \textbf{Extrempunkt}.  
\end{definition}
\begin{remark}
Man beachte, dass man beim isolierten lokalen Extremum unbedingt die punktierte Umgebung braucht, da diese den Punkt $x_0$ nicht enthält. Für diesen gilt nämlich selbstverständlich die strikte Ungleichung nicht. 
\end{remark}
Die Definition ist relativ selbsterklärend: Bei einem isolierten Extremum sind alle Funktionswerte in einer gewissen Umgebung ausschliesslich grösser bzw. kleiner als der Funktionswert an der Extremstelle selbst. Das typische Bild des ``Hügels'' bzw. ``Tals'' entsteht. Beim nicht-isolierten Extremum fordern wir nur, dass die Funktionswerte in einer gewissen Umgebung nicht grösser bzw. kleiner als der Extremwert sind. Es gibt also auch Punkte, die auf gleicher ``Höhe'' sind. Sprich: konstante Funktionen haben an jeder Stelle ein Maximum und Minimum.

Mithilfe der gerade eingeführten Ableitung können wir nun die weniger abstrakte, notwendige Bedingung für Extrema formulieren. 

\begin{lemma}{Notwendige Bedingung für Extrema}{}
Seien $D \subseteq \R$, $x_0 \in D$ ein beidseitiger Häufungspunkt bzw. innerer Punkt eines Intervalls und $f:D \to \R $ bei $x_0$ differenzierbar.\\
Wenn $f$ ein lokales Extremum bei $x_0$ hat, dann $$f'(x)=0.$$  
\end{lemma}
\begin{remark}
Wir haben das Lemma möglichst allgemein formuliert. Man beachte trotzdem, dass es sogar eine hinreichende Bedingung ist, wenn wir nicht fordern, dass es sich um ein isoliertes lokales Extremum handelt. Sobald nämlich $f'(x_0)=0$ gilt, wissen wir schon, dass sich bei $x_0$ zumindest ein lokales Extremum befindet. Das Extremum könnte jedoch ein Terassenpunkt sein. Daher brauchen wir noch eine hinreichende Bedingung für isolierte lokale Extrema, um dies auszuschliessen. 
\end{remark}

\subsection{Ableitungsregeln}
Da es viel zu lästig ist, für jede Funktion, die ich ableiten möchte, einen Differentialquotienten aufzustellen und diesen ewig umzuformen, gibt es einige Regeln, die uns beim Ableiten helfen. Da diese sehr fundamental sind, geben wir ausnahmsweise Beweise.
\subsubsection{Linearität des Differentialoperators}
Die Linearität der Ableitung haben wir eigentlich schon in der \textit{Lineare Algebra I} Vorlesung gesehen. Dort haben wir uns jedoch auf Polynome beschränkt und gesagt, dass die Abbildung
$$(\cdot)': K[X] \to K[X], \ \ p \mapsto p'$$
die ein Polynom $p$ auf seine Ableitung $p'$ abbildet, eine \textbf{lineare Abbildung} ist. Das heisst, dass die Ableitung von Polynomen \textbf{additiv} und \textbf{homogen} ist. Wir wollen diese besondere Eigenschaft nun auf alle differenzierbaren Funktionen ausweiten, uns dabei aber auch erst einmal auf reellwertige Funktionen beschränken.
\begin{satz}{Linearität der Ableitung}{Linearität_der_Ableitung}
Seien $\lambda \in \R$, $f:D \to \R$ und $g:D \to \R$ differenzierbare Funktionen. Dann gilt:
\begin{center}
    $(f(x) + g(x))' = f'(x) + g'(x)$ \hfill (Additivität) \\
    $(\lambda \cdot f(x))' = \lambda \cdot f'(x)$  \hfill (Homogenität)
\end{center}
Man sagt, der Differentialoperator bzw. die Ableitung ist \textbf{linear}.
\end{satz}
\begin{proof}
Zeigen wir zuerst mithilfe der allgemeinen Rechenregeln zu Grenzwerten die Additivität:
\begin{align*}
   (f(x)+g(x))'&=\limto{x}\frac{f(x)+g(x)-(f(x_0)+g(x_0))}{x-x_0}\\
   &= \limto{x}\frac{f(x)-f(x_0)+g(x)-g(x_0)}{x-x_0}\\
   &= \limto{x}\frac{f(x)-f(x_0)}{x-x_0}+\limto{x}\frac{g(x)-g(x_0)}{x-x_0}=f'(x)+g'(x)
\end{align*}
Analog erhalten wir die Homogenität:
\begin{align*}
    (\lambda \cdot f(x))' &=\limto{x}\frac{\lambda f(x) - \lambda f(x_0)}{x-x_0}\\
    &=\lambda \limto{x}\frac{f(x)-f(x_0)}{x-x_0} = \lambda \cdot f'(x)
\end{align*}

\end{proof}

%\subsubsection{Polynome}

%\subsubsection{Produktregel}



%\subsubsection{Kettenregel}
%\subsubsection{Quotientenregel}
%\subsubsection{Ableitung der Inversen Funktion}
%\subsubsection{Weitere Tricks}
%Es gibt noch einige Tricks mit denen sich einige Funktionen viel einfacher ableiten lassen, als es erst scheint.
%\begin{enumerate}
%\item \textbf{Der Exponentialtrick}\\
Diese Methode kommt immer dann zum Einsatz, wenn man Funktionen der Form $x^{g(x)}$ hat, wobei $g(x)$ keine konstante Funktion ist, da man in diesem Fall auch einfach die Methode für Polynome verwenden kann. \\ 
Forme wie folgt um:
%\begin{center}
%       $x^{g(x)} = e^{\log x^{g(x)}} = e^{g(x) \log x}$ 
%\end{center}
%Leite dann mit den üblichen Regeln ab.
%\begin{example}
%Berechne die Ableitung von $f(x)=x^{\sin{x}}$.

%Wir bemerken, dass $f(x)=x^{\sin{x}} = e^{\log x^{\sin{x}}} = e^{\sin{x}\log x}$.\\ Nun leiten wir ganz normal %mithilfe der Ketten- bzw. Produktregel ab: \\ $f'(x)=(\cos{x} \log x + \frac{\sin{x}}{x}) e^{\sin{x}\log x} = %(\cos{x} \log x + \frac{\sin{x}}{x}) x^{\sin{x}} $
%\end{example}
%\end{enumerate}
%
%
%\subsection{Mittelwertsatz}
%\subsubsection{Satz von Rolle}
%\section{Integral}
%\subsection{Hauptsatz der Differentialrechnung}
%\subsection{Integrationsmethoden}
\subsubsection{Partielle Integration}
Die Methode der partiellen Integration, Integration by Parts, lat. integratio per partes, macht so einige schwierige Integrale zu einem Kinderspiel. Sie ist als Analogon zu der Produktregel zu verstehen, und wird daher auch so hergeleitet.
\begin{satz}{Partielle Integration}{ibp}
Seien zwei stetig differenzierbare Funktionen $u(x)$ und $v(x)$. \\
Die Produktregel besagt, dass
$$( u(x) \cdot v(x) )' = u'(x) \cdot v(x) + u(x) \cdot v'(x)$$ 
Wenn man beide Seiten intergiert, erhält man 
$$\int ( u(x) \cdot v(x) )' \mathrm{d}x = \int ( u'(x) \cdot v(x) + u(x) \cdot v'(x) ) \mathrm{d}x$$ 
Mittels des Hauptsatzes der Integral- und Differentialrechnung folgt
$$u(x) \cdot v(x) = \int ( u'(x) \cdot v(x) ) \mathrm{d}x + \int ( u(x) \cdot v'(x) ) \mathrm{d}x$$
Wenn man dies nun umstellt, erhält man die eigentliche Integrationsregel:
$$\int ( u'(x) \cdot v(x) ) \mathrm{d}x = u(x) \cdot v(x) - \int ( u(x) \cdot v'(x) ) \mathrm{d}x$$
\end{satz}

\begin{example}[Stammfunktion des natürlichen Logarithmus]
\begin{align*}
\int \ln(x) \,\mathrm{d}x &= \int 1 \cdot \ln(x) \,\mathrm{d}x\\ 
& = x \cdot \ln(x) - \int x \cdot {1 \over x} \,\mathrm{d}x \\
& = x \cdot \ln(x) - \int 1 \,\mathrm{d}x \\
& = x \cdot \ln(x) - x + C\,.
\end{align*}
\end{example}
Wie man an diesem Beispiel sieht und wie so oft in der Analysis, ist auch hier das hinzufügen des neutralen Elements manchmal vom Vorteil.
\vspace{3mm}\\
Zur leichteren Verständlichkeit haben wir in \ref{satz:ibp} die Regel für das unbestimmte Integral hergeleitet, aber selbstverständlich funktioniert sie auch für das bestimmte:
\begin{align*} 
\int_a^b f'(x)\cdot g(x)\,\mathrm{d}x 
&= \Big[f(x)\cdot g(x)\Big]_{a}^{b} - \int_a^b f(x)\cdot g'(x)\,\mathrm{d}x\\
&=f(b) \cdot g(b) - f(a) \cdot g(a) - \int_a^b f(x)\cdot g'(x)\,\mathrm{d}x\,.
\end{align*}
\begin{example}[ ]
    
\end{example}

\subsubsection{Substitutionsregel}
Wo die partielle Integration das Gegenstück zur Produktregel der Ableitung, ist die Substitutionsregel das Gegenstück zur Kettenregel. Sie ist ebenfalls ein starkes Tool, das für manche Integrations-Aufgaben unerlässlich ist. 
\begin{satz}{Substitutionsregel}{}
Sei $F$ eine Stammfunktion von $f$. Nach der Kettenregel gilt
$$(F \circ g)'(x) = F'(g(x))\cdot g'(x) = f(g(x))\cdot g'(x)$$
Durch bilden des Integrals auf beiden Seiten erhält man
$$F(g(x)) = \int f(g(x))\cdot g'(x)\, \mathrm{d}x $$
\end{satz}


\subsubsection{Integration von rationalen Funktionen}
Nun wollen wir eine allgemeine Formulierung für Stammfunktionen von rationalen Funktionen betrachten (Def. rationale Funktionen: Polynome mit Strich dazwischen), d.h. wir haben ganz allgemein zwei reelle Polynome $P,Q \in \R[x]$, die mit $\frac{p(x)}{q(x)}$ eine rationale Funktion bilden.

Im ersten Schritt gilt es, durch Polynomdivision von $\frac{p(x)}{q(x)}$ einen \textbf{polynomialen Teil} $\widetilde{p}(x)\in \R[x]$ und einen \textbf{echt gebrochenen} $\frac{P(x)}{Q(x)}$ Teil zu erhalten, sodass $\frac{p(x)}{q(x)} = \widetilde{p}(x) + \frac{P(x)}{Q(x)}$ und $\deg P < \deg Q$ gelten. Der polynomiale Teil $p(x)$ ist mit der Linearität des Integrals und der Potenzregel einfach zu integrieren, der echt-gebrochene Teil ist jedoch etwas trickreicher zu integrieren.

Wir wollen uns dazu die Linearität des Integrals zu Nutze machen, somit müssen wir in einem ersten Schritt das Polynom durch Partialbruchzerlegung in eine Summe von Brüchen aufteilen, um sie dann dann einzeln integrieren zu können.

Erst einmal wollen wir ein Lemma für die Partialbruchzerlegung erwähnen:

\begin{lemma}{Partialbruchzerlegung}{Partialbruchzerlegung}
Für allgemeine rationale Funktionen gilt, dass für $P, Q \in \C[z]$, $\deg P < \deg Q$ die Funktion $\frac{P(z)}{Q(z)}$ als Linearkombination $$\frac{P(z)}{Q(z)} = \sum_{z_i \in \mathcal{N}}\sum_n^{\mu(Q, z_i)} \frac{A_{i,n}}{(x-z_i)^n}$$ geschrieben werden kann, wobei $\mathcal{N} \subseteq \C$ die Menge der Nullstellen von $Q$  ist und $\mu(Q, z_i)$ die Vielfachheit der jeweiligen Nullstelle bezeichnet.

Für reelle Polynome $P, Q \in \R[x]$ sind die Nullstellen immer entweder reell oder kommen in komplex-konjugierten Paaren vor ($Q(z) \implies \quer{Q(z)} = Q(\quer{z}) = 0$). Solche rationalen Brüche kann man ebenfalls als Linearkombination darstellen, wobei hier zuerst die reellen Nullstellen (einzeln) und dann die komplexen Nullstellen paarweise aufsummiert werden:

$$\frac{P(x)}{Q(x)} = \sum_{x_i \in \mathcal{N}_\R}\sum_n^{\mu(Q, x_i)} \frac{A_{i,n}}{(x-x_i)^n} +
\sum_{z_j, \quer{z_j} \in \mathcal{N}_\C}\sum_m^{\mu(Q, z_j)} \frac{B_{j,m}x + C_{j,m}}{((x-z_j)(x- \quer{z_j}))^m}$$

wobei $\mathcal{N}_\R$ die Menge der reellen Nullstellen und $\mathcal{N}_\C$ die Menge der komplexen Nullstellenpaare von $Q$ bezeichnet.
\end{lemma}

In anderen Worten gibt es 2 Kriterien, welche bei den Nullstellen von $Q$ zu beachten sind: Art der Nullstelle $a$ (reell- oder komplexwertig): Solche Nullstellen tragen jeweils zu einem Term der Form
$$\frac{A}{x-a} \qquad \textit{für }a \in \R$$
oder
$$\frac{B x + C}{(x-a)(x- \quer{a})} \qquad \textit{für }a \in \C$$
bei. Nun ist auch die Vielfachheit der Nullstelle zu beachten: Falls $a$ eine $n$-fache Nullstelle ist, dann kommen die Terme dieser Nullstelle mit bis zu $n$ potenzierten Nennern vor:
$$\frac{A_1}{x-a}+\frac{A_2}{(x-a)^2}+...+\frac{A_n}{(x-a)^n} \qquad \textit{für }a \in \R$$
resp.
$$\frac{B_1 x + C_1}{(x-a)(x- \quer{a})}+...+\frac{B_n x + C_n}{((x-a)(x- \quer{a}))^n} \qquad \textit{für }a \in \C$$
Natürlich muss man für ein $Q$, welches sowohl reelle als auch komplexe Nullstellen enthält, eine Linearkombination beider ''Arten'' von Termen bilden. Die Koeffizienten $A_i$, $B_i$ und $C_i$ lassen sich dann durch Koeffizientenvergleich mit $\frac{P(x)}{Q(x)}$ und dem Lösen eines linearen Gleichungssystems berechnen.

Wir erkennen aus dem Lemma, dass durch die Partialbruchzerlegung genau vier verschieden Typen von nicht weiter vereinfachbaren Brüchen entstehen können:
\begin{align*}
    \frac{A}{x-a} \qquad
    \frac{B}{(x-a)^n} \qquad
    \frac{Cx + D}{(x-z)(x- \quer{z})} \qquad
    \frac{Ex + F}{((x-z)(x- \quer{z}))^n}
\end{align*}

Wir wollen nun für die ersten beiden Fälle eine allgemeine Formulierung des Integrals finden, wofür wir eine rationale Funktion der Form $\frac{cx + d}{(x-a)(x-b)}$ betrachten wollen:

Seien $P, Q \in \R[x]$ mit Grad $\deg P < \deg Q \leq 2$ der Form $P(x) = cx + d$ und $Q(x) = (x-a)(x-b)$ dann lassen sich $A, B \in \R$ finden, sodass gilt:
\begin{align*}
    f(x) = \frac{P(x)}{Q(x)} = \frac{cx + d}{(x-a)(x-b)} &= \frac{A}{x-a} + \frac{B}{x-b}\\
    &=\frac{A(x-b)+B(x-a)}{(x-a)(x-b)}\\
    &=\frac{(A+B)x-(Ab+Ba)}{(x-a)(x-b)}
\end{align*}
Um das $A$ und $B$ zu finden, müssen wir ein $2\times2$ lin. Gleichungssystem mit $A+B = c$ und $bA+aB=-d$ lösen. Dieses Problem lässt sich auch als Matrixgleichung schreiben: $\big(\begin{smallmatrix}
  1 & 1\\
  b & a
\end{smallmatrix}\big)
\big(\begin{smallmatrix}
  A\\
  B
\end{smallmatrix}\big)=
\big(\begin{smallmatrix}
  c\\
  -d
\end{smallmatrix}\big)  $ Für diese Matrix ist die Determinante $a-b$.

Wir wollen nun den Fall betrachten, bei dem $a \ne b$ ist, also den Fall, bei dem $Q$ zwei einfache Nullstellen hat und eine eindeutige Lösung für $A$ und $B$ existiert. Das Integral von $f$ ist dann
\begin{align*}
    \int f(x) dx &= \int \frac{A}{x-a} dx + \int \frac{B}{x-b} dx \\
    &= A \log|x-a| + B \log |x-b|
\end{align*}
wobei für den letzten Schritt die Logarithmus-Regel für die Integration verwendet wurde.

Für den Fall $a=b$ hat $Q$ eine doppelte Nullstelle. Laut obigem Lemma \ref{lem:Partialbruchzerlegung} müssen die Nenner bis zur Ordnung der Nullstelle verwenden, d.h. wir erhalten:
$$f(x) = \frac{cx + d}{(x-a)^2} = \frac{c}{(x-a)} + \frac{ca + d}{(x-a)^2}$$
wobei für den letzten Schritt der Trick $x = x + a - a$ zur Umformung verwendet wurde. Das kann nun integriert werden; wir erhalten:
\begin{align*}
    \int f(x)dx &= \int\frac{c}{(x-a)}dx +\inf \frac{ca + d}{(x-a)^2}dx \\
    &= c\log|x-a|-\frac{ca+d}{x-a}
\end{align*}

Brüche des dritten Typus sind der Form
$$\int \frac{cx+d}{(x-z_i)(x-\quer{z_i})} dx$$
Solche Integrale lassen sich durch die Variablensubstitution $u = x + \lambda$ (Quadratisches Ergänzen) auf folgende Form bringen:
\begin{align*}
\int \frac{x}{x^2-a^2} &= \frac{1}{2} \int \frac{2x}{x^2+a^2} = \frac{1}{2} \log |x^2+a^2| + C\\
\int \frac{1}{x^2-a^2} &= \frac{1}{a} \arctan \frac{x}{a} + C
\end{align*}

\begin{example} Berechne das Integral 
$$\int \frac{1}{x^2(x+1)}$$
Wir erkennen, dass wir eine doppelte Nullstelle bei $x=0$ haben und eine bei $x=-1$. Somit muss die Zerlegung von folgender Form sein:
$$\frac{1}{x^2(x+1)} = \frac{A}{x^2}+\frac{B}{x}+\frac{C}{x+1}$$
Wir bringen nun alles auf einen gemeinsamen Nenner und erhalten:
\begin{align*}
    1 &= A(x+1)+Bx(x+1)+Cx^2\\
    &= x^2(B + C) + x(A+B)+ A
\end{align*}
Wir erhalten durch Koeffizientenvergleich die Gleichungen $0=B+C, 0 = A+B, 1 = A$, welche uns auf die Werte $A=1, B=-1$ und $C=1$ und somit die Partialbruchzerlegung
$$\frac{1}{x^2}-\frac{1}{x}+\frac{1}{x+1}$$
führt. Nun können wir die einzelnen Brüchen mit den bereits bekannten Integrationsmethoden integrieren und kommen auf
\begin{align*}
    \int \frac{1}{x^2(x+1)}dx &= \int \frac{1}{x^2}dx- \int \frac{1}{x}dx+\int \frac{1}{x+1}dx
    \\&= -\frac{1}{x} - \log|x| + \log|x+1| + C
\end{align*}
\end{example}

\begin{tipp}{}{}
Um die durch die Partialbruchzerlegung erhaltenen Brüche zu integrieren, sind folgende allgemein formulierten Integrale sehr nützlich:
\begin{align*}
\int \frac{A}{x-a} &= A \log|x-a| + C\\
\int \frac{A}{(x-a)^n} &= -\frac{A}{(n-1)(x-a)^{n-1}} + C \qquad \textit{für }n\geq2\\
\int \frac{x}{x^2-a^2} &= \frac{1}{2} \int \frac{2x}{x^2+a^2} = \frac{1}{2} \log |x^2+a^2| + C\\
\int \frac{1}{x^2-a^2} &= \frac{1}{a} \arctan \frac{x}{a} + C
\end{align*}

\end{tipp}

\section{Regeln von de L'Hôpital}
Bevor wir die Regeln von de L'Hôpital herleiten können, brauchen wir zuerst den \textbf{erweiterten Mittelwertsatz}:
\begin{satz}{Erweiterter Mittelwertsatz}{}
Seien $f,g : [a,b] \to \R$ stetig auf $(a,b)$ differenzierbar und sei $g(x)' \ne 0$ für alle $x \in [a,b]$. Dann gilt $g(x) \ne g(a)$ und gibt es ein $\xi \in (a,b)$, so dass:
$$\frac{f(b)f(a)}{g(b)-g(a)} = \frac{f'(\xi)}{g'(\xi)}$$
\end{satz}
\begin{remark}
Falls $g(x)=x$ ist, erhalten wir den Mittelwertsatz. Man könnte meinen, dass wir hier einfach zweimal den Mittelwertsatz verwendet haben, einmal für $f(x)$ und einmal für $g(x)$, welche wir dann durch einander geteilt haben, wodurch sich das $b-a$ im Nenner jeweils rauskürzen würde. Jedoch wären die $\xi$ nicht die selben wie hier im verallgemeinerten Mittelwertsatz.
\end{remark}
\begin{proof}
Wir zeigen (zuerst ohne der Annahme $g'(x)\ne0$), dass es ein gibt $\xi$ gibt, sodass:
$$g'(\xi)\Big(f(b)-f(a)\Big) = f'(\xi)\Big(g(b)-g(a)\Big)$$
Wir führen nun eine Hilfsfunktion $h$ ein:
$$h(x) = g(x)\Big(f(b)-f(a)\Big)-f(x)\Big(g(b)-g(a)\Big)$$
welche die Eigenschaften $h(a)=g(a)f(a)-f(a)g(b)$ und $h(b)=-g(b)f(a)+f(b)g(a) = h(a)$ aufweist.
Also haben wir eine Funktion, die den selben Wert auf beiden Endpunkten haben. Aus dem Satz von Rolle wissen wir, dass es dann ein $\xi \in (a,b)$ gibt, so dass
$$h'(\xi)=0$$
gilt. Daraus folgt 
$$g'(\xi)\Big(f(b)-f(a)\Big)-f(\xi)'\Big(g(b)-g(a)\Big) = 0$$
Aus der Kontraposition des Satzes von Rolle folgt nun:
$$ \forall x \ g'(x)\ne 0 \implies g(b) \ne g(a)$$
Nun können wir durch $g'(\xi)\big(g(b)-g(a)\big)$ dividieren und erhalten den erweiterten Mittelwertsatz, was zu zeigen war.
\end{proof}

\subsection{L'Hôpital-Regel}{}
Wir wollen eine Methode finden, um Limites, weche der ''Sorte'' oder Form:
$$\frac{0}{0}\qquad\frac{\infty}{\infty}\qquad0\cdot\infty\qquad 1^\infty \qquad \textit{etc...}$$
sind, bestimmen zu können. Also wenn z.B. $\limto{x}{f(x)}=\limto{x}{g(x)}=0$ gilt und wir $\limto{x}{\frac{f(x)}{g(x)}}$ berechnen wollen. Die Regel von de L'Hôptital besagt, dass wir das Verhältnis der Ableitungen von $f$ und $g$ bei $x_0$ betrachten können:

\begin{satz}{\textsc{Bernoulli - de l'Hôpital}}{}
Sei $-\infty \leq a < b \leq \infty$, $f,g:(a,b) \to \R$ differenzierbar, $g(x)\ne 0, g'(x)\ne 0 \forall x \in (a,b)$ und $\lim_{x \searrow a}\frac{f'(x)}{g'(x)} \in \R \cup \{\pm \infty\}$ existiert mit den folgenden Bedingungen:
\begin{enumerate}
    \item ''$\frac{0}{0}$'': $\lim_{x \searrow a}{g(x)}=0$, $\lim_{x \searrow a}{f(x)}=0$
    \item ''$\frac{\infty}{\infty}$'': $\lim_{x \searrow a}{g(x)}=\infty$, $\lim_{x \searrow a}{f(x)}=\infty$
\end{enumerate}
Dann existiert der uneigentliche Grenzwert $\lim_{x \searrow a}\frac{f(x)}{g(x)} \in \R \cup \{\pm \infty\}$ und es gilt:
$$\lim_{x \searrow a}\frac{f(x)}{g(x)} = \lim_{x \searrow a}\frac{f'(x)}{g'(x)}$$
Das gilt auch für $\lim_{x \nearrow b}$ und $\lim_{x \to x_0}, x_0 \in (a,b)$, wobei im letzteren Fall unter der Annahme dass $g(x)$,$g'(x)\ne 0$ für $x\ne x_0$
\end{satz}
Generell gilt, dass man diesen Satz für jegliche Kombination von Grenzwerten verwenden kann. Es kommt also nicht darauf ab, ob man den Limes von unten, von oben oder beidseitig für Grenzwerte nach $\pm \infty$ oder auch $0$ bildet. Der Beweis aus der Vorlesung entspricht mehr oder weniger dem des Skripts (7.2.5, Seite 367), darum wird hier auf die \TeX-ification davon verzichtet. Wir werden nun einige Beispiele behandeln:

\begin{example} Wir wollen den Grenzwert von folgender Funktion errechnen:
$$\lim_{x \searrow 0}\frac{\sin x}{\log (x + 1)}$$
Zuerst prüfen wir die Annahmen, welche erfüllt sind, da $\sin 0 = \log 1 = 0$. Wir haben $f'(x) = \cos x$ und $g'(x)= \frac{1}{x+1}$. Somit erhalten wir:
$$\lim_{x \searrow 0} \frac{f'(x)}{g'(x)} = \frac{\cos x}{\frac{1}{x+1}} = 1$$
Wir müssen wir nun ein Intervall wählen, sodass wir uns von oben annähern können, z.B. $a = 0$ und $b = 1$, wodurch wir folgendes erhalten (welches in diesem Fall auch den Limites von unten oder von beiden Seiten entspricht):
$$\lim_{x \searrow 0}\frac{\sin x}{\log (x+1)} = 1 \quad \Big( = \lim_{x \nearrow 0}\frac{\sin x}{\log (x+1)} = \lim_{x \to 0}\frac{\sin x}{\log (x+1)} \Big)$$
\end{example}
\begin{example} In diesem Beispiel ist erst eine Umformung in die Form ''$\frac{\infty}{\infty}$'' nötig:
$$\lim_{x \searrow 0} x \log x = -\lim_{x \searrow 0}\frac{-\log x}{\frac{1}{x}}$$
Nun sehen wir, dass die obere Funktion $f(x) = -\log x$ und die untere Funktion $g(x) = \frac{1}{x}$ ist, welche jeweils auch gegen $\infty$ divergieren. Nun können wir die Regel von de L'Hôpital anwenden und erhalten:
$$\lim_{x \searrow 0} \frac{f'(x)}{g'(x)} = \lim_{x \searrow 0} \frac{-\frac{1}{x}}{-\frac{1}{x^2}}=\lim_{x \searrow 0}x = 0$$
\end{example}

\section{Parameterdarstellung von Kurven}
Je nach Anwendung kann es von Nutzen sein, eine (mehrdimensionale) Funktion in Abhängigkeit von einem Parameter darzustellen. So wäre das z.B. der Fall, wenn man die Flugbahn oder den \textbf{Weg} von einem Objekt in Abhängigkeit von der Zeit modellieren will. Um das in eine Funktion fassen zu können, benötigen wir zuerst einen Parameter, der auf einem bestimmten Intervall ''läuft''. Wir werden in unseren Betrachtungen meist die Zeit $t$ mit diesem Parameter ausdrücken. Nun benötigen wir nur noch für jede (Raum)dimension eine Funktion, welche die Projektion des zu parametrisierenden Ortsvektors auf die jeweilige Dimension zu einer bestimmten Zeit $t$ gibt. Wir definieren die Parameterdarstellung folgendermassen:

\begin{definition}{Parameterdarstellung}{}
Ein \textbf{Weg} in $\R$ ist eine Abbildung auf dem Intervall $I = [a,b], a<b$ der Form:
$$\gamma: I \to \R^n, t \mapsto (\gamma_1(t), \gamma_2(t),...,\gamma_n(t))$$
Wir nennen $t\mapsto\gamma(t)$ auch eine Parameterdarstellung der Kurve $\gamma(I)\subseteq\R^n$
\end{definition}
(Kann auch $a,b\in \R\cup\pm\infty$ sein?)
Wir sagen nun, dass $\gamma$ stetig/differenzierbar/stetig differenzierbar ist, falls auch $\gamma_i:[a,b] \to \R$ für jedes $i$ stetig/differenzierbar/stetig differenzierbar ist. Falls $\gamma$ differenzierbar ist, dann lässt sich auch die Ableitung berechnen, indem jedes $\gamma_i$ abgeleitet wird:
\begin{definition}{Geschwindigkeitsvektor}{}
Da die zeitliche Ableitung des Ortes physikalisch gesehen der Geschwindigkeit entspricht, nennen wir $\dot{\gamma}(t)$ auch den \textbf{Geschwindigkeitsvektor} zur Zeit $t$, welchen wir definieren als
$$\dot{\gamma}(t) = (\dot{\gamma}_1(t),...,\dot{\gamma}_n(t)) = \lim_{h \to 0} \frac{\gamma(t+h) - \gamma(t)}{h}$$
wobei mit $\dot{\gamma}_i(t)$ (Newton-Notation) die Ableitung nach der Zeit $\frac{d}{dt}\gamma_i(t)$ (Leibniz-Notation) gemeint ist und der Limes für Abbildungen der Art $u: I \to \R^n$ allgemeint definiert ist als:
$$\lim_{h \to 0}(u_1(h),...,u_n(h)) := (\lim_{h \to 0}u_1(h),...,\lim_{h \to 0}u_n(h))$$
\end{definition}
\begin{definition}{Geschwindigkeit}{}
Nimmt man nun die Norm des Geschwindigkeitsvektors
$$|\dot{\gamma}(t)| = \sqrt{\dot{\gamma}_1(t)^2,...,\dot{\gamma}_n(t)^2} \quad \Big(= \|\dot{\gamma}(t)\| = \|\dot{\gamma}(t)\|_2 \Big)$$
, erhalten wir die \textbf{Geschwindigkeit} zur Zeit $t$.
\end{definition}
\begin{remark}
Auf Englisch nennt man den Geschwindigkeitsvektor resp. die Geschwindigkeit \textit{velocity} bzw. \textit{speed}.
\end{remark}
Wenn wir nun auch schon die Geschwindigkeit haben, können wir sie ganz analog zur bereits im Physikunterricht kennengelernten Bewegungsgleichung integrieren, wodurch wir zurückgelegte Strecke erhalten:
\begin{definition}{Bogenlänge}{}
Die \textbf{Bogenlänge} eines stetig differenzierbaren Wegs $\gamma: [a,b] \to \R^2$ ist:
$$L(\gamma) = \int_a^b |\dot{\gamma}(t)| dt =: \int_\gamma ds$$
wobei $ds = |\dot{\gamma}(t)| dt$ die in einem infinitesimalen Zeitintervall $[t,t+dt]$ zurückgelegte Distanz ist.
\end{definition}
\begin{example}[Kreisbogen] Sei $\gamma: t \mapsto (cos(t), sin(t)$ für $t\in [a,b]$, dann beschreibt $\gamma$ einen Kreisbogen auf dem Einheitskreis vom Winkel $a$ (in Radianten) bis $b$.
Nun wollen wir die Bogenlänge $L(\gamma)$ berechnen.

Für die Geschwindigkeitsvektor erhalten wir $\dot{\gamma}(t) = (-\sin t, \cos z)$. Daraus folgt für die Geschwindigkeit $|\dot{\gamma}(t)| = \sqrt{\sin^2 t + \cos^2 t}$. Durch Einsetzen in die Formel der Bogenlänge erhalten wir:
$$L(\gamma) =  \int_a^b \sqrt{\sin^2 t + \cos^2 t} \ dt =\int_a^b 1 \ dt = b-a$$
Also ist die Bogenlänge für $[0, \theta] = \theta$
\end{example}
\begin{example}[Zykloide] Wir wollen den Weg eines Punktes parametrisieren, welcher sich auf dem Radius eines sich abrollendes Rades befindet. Wir finden für die Parametrisierung
$$\gamma: [0, T]\to \R^2, t\mapsto \begin{pmatrix}
  t-R\sin \frac{t}{R}\\
  R-R\cos \frac{t}{R}
\end{pmatrix}$$
Wir wollen nun die zurückgelegte Strecke für eine Umdrehung berechnen. Zuerst finden wir Ausdrücke für den Geschwindigkeitsvektor/die Geschwindigkeit:
\begin{align*}
    \dot{\gamma}(t) &= \left(1-\cos \frac{t}{R}, \sin \frac{t}{R}\right)
\end{align*}
\begin{align*}
    |\dot{\gamma}(t)| &= \sqrt{\left(1-\cos \frac{t}{R}\right)^2+\left(\sin^2 \frac{t}{R}\right)}&\\
    &= \sqrt{2- 2\cos \frac{t}{R}}& \bigm\vert 1-\cos \frac{t}{R} &= 1 - \cos^2\frac{t}{2R} + \sin^2\frac{t}{2R}\\
    &&&= 2 \sin^2\frac{t}{2R}\\
    &=2 \sin\frac{t}{2R}\\
\end{align*}
Nun wollen wir den gefundenen Ausdruck für eine Umdrehung integrieren. Wir finden aus der Parametrisierung, dass $0\leq T \leq 2\pi R$, also der Endpunkt des Integrals bei $T = 2\pi R$ ist:
$$L(\gamma) = \int_0^{2\pi R}2 \sin\frac{t}{2R} \ dt = \Big[-4R\cos \frac{t}{2R}\Big]^{2\pi R}_0=8R\\$$
\end{example}
\subsection{Reparametrisierung}\label{cha_reparametrisierung}
%\subsection{Bogenlänge-Parameter-Darstellung}
%\subsection{Wegintegrale von Vektorfeldern}
%\subsection{Volumen von Rotationskörpern}
%\subsection{Taylorpolynome}

    %\chapter{Analysis II}

\section{Potenzreihen}
Wir haben bis jetzt unter anderem die Grundlagen für Folgen und Reihen behandeln können und haben viele einige Funktionen wie z.B. $\exp$, $\sin$ und $\cos$ gesehen, welche wir durch Potenzreihen (je nach dem beliebig genau) approximieren konnten (i.e. Taylorapproximation). Wir können das ganze aber auch umdrehen und aus Potenzreihen ''neue'' Funktionen definieren.

Wir wollen nun also Funktionen betrachten, welche von folgender Form sind:
\begin{align}\label{eq_fcn_potenzreihe}
    f(x) = \sum^\infty_{n=0} a_n (x-a)^n
\end{align}
wobei die Koeffizienten $a_n \in \R$ oder $\C$ sind und $a \in \R$ den Entwicklungspunkt der Funktion bezeichnen soll. Überzeuge dich zuerst davon, dass diese Form von Potenzreihen die Allgemeinheit derer nicht einschränkt und für $a_0 = 0$ oder für $a = 0$ eindeutig ist.

Wir wollen uns nun zwei Fragen zu Potenzreihen stellen:
\begin{enumerate}
    \item Für welche $x$ konvergiert $f$?
    \item Was sind die Eigenschaften (Stetigkeit, Periodizität, Verhalten im Unendlichen, etc.) von $f$?
\end{enumerate}
Wie wir bereits erkannt haben, müssen Reihen, aus welchen Potenzreihen effektiv auch bestehen, nicht immer konvergieren. Da diese Reihen im Fall von Potenzreihen vom Argument $x$ abhängig sind, müssen wir für jedes $x$ entscheiden, ob die Potenzreihe konvergiert, i.e. ob die Potenzreihe einen eindeutigen Wert zurückgibt, i.e. was der Definitionsbereich von der Potenzreihe ist. Wir werden sehen, dass Potenzreihen in einem Bereich $-R < x-a < R$ also $|x-a| < R$ absolut konvergieren\footnote{Die Reihe (i.e. partielle Summen) der einzelnen Terme konvergiert.}. Wir nennen $R$ dann auch den \textbf{Konvergenzradius} und werden sehen, dass dieser durch den $\limsup$ konstruiert wird.

\subsection{Einschub: Der Limes superior}
Bis jetzt haben wir keine grossen Aussagen zu Folgen wie z.B. 
\begin{align}\label{eq_examble_limsup}
    (c_n)_{n=0}^\infty = (1.1, -0.9, 1.01, -0.99, 1.001, -0.999, 1.0001, ...)
\end{align} gemacht, ausser dass sie begrenzt sind aber trotzdem nicht konvergieren (i.e. divergieren?). Wir sehen aber trotzdem, dass eine Teilfolge nach $1$ und eine andere nach $-1$ konvergiert und im Limit auf den Bereich $[-1, 1]$ begrenzt sein wird. Hierfür wollen wir nun den Begriff des \textbf{Limes superior} $\limsup$ und des \textbf{Limes inferior} $\liminf$ einführen. Dabei betrachten wir für ein $n$ jeweils das Supremum (resp. Infimum) der Folgeglieder von $n$ (also den ''Rest'' der Folge ohne den ersten $n$ Elementen). Wir bilden aus den Suprema eine Folge und nehmen davon den Grenzwert.

\begin{definition}{Limes superior}{}
Sei $(c_n)_{n=0}^\infty$ eine beschränkte Folge in $\R$. Der \textbf{Limes superior} dieser Folge ist
$$\limsup_{n \to \infty}{c_n} = \lim_{n \to \infty}{\big(\sup\{c_k\mid k\geq n\}\big)}$$
\end{definition}

Wir können zeigen, dass \textit{jede} beschränkte Folge ein solches $\limsup$ besitzt: Betrachten wir die Mengen der Folgeglieder eines bestimmten Indexes $S_n = \{c_k \mid k \geq n\}$, welche einerseits nicht-leer und beschränkt sein muss. Aus der Konstruktion gilt nun:
$$S_1 \supseteq S_2 \supseteq S_3 \supseteq ... \implies \sup S_1 \geq \sup S_2 \geq \sup S_3 \geq ...$$
Da $(\sup S_n)_{n=0}^\infty$ eine monoton fallende, beschränkte Folge ist, erkennen wir durch
$$\limsup_{n \to \infty}  S_n = \lim_{n \to \infty} (\sup S_n) = \inf((\sup S_n)_{n=0}^{\infty})$$
dass der $\limsup$ existiert.

Analog ist der \textit{Limes inferior} definiert, jedoch können wir auch die Folge negieren und davon den $\limsup$ nehmen:
\begin{definition}{Limes inferior}{}
$$\liminf_{n \to \infty}{c_n} = -\limsup_{n \to \infty}{-c_n}$$
\end{definition}

\begin{example}In unserem Beispiel von vorher (\ref{eq_examble_limsup}) gilt somit $\limsup_{n \to \infty} c_n = 1$ und $\liminf_{n \to \infty} c_n = -1$
\end{example}

\begin{lemma}{Folgerungen}{limsup_consequences}
Sei $c_{n \to \infty}$ eine beschränkte Folge in $\R$. Folgende Aussage über $c \in \R$ sind äquivalent:
\begin{enumerate}
    \item Der $\limsup_{n \to \infty} c_n$ existiert
    \item \begin{enumerate}
        \item \textit{(Obere Grenze)} Falls für ein $a \in \R: a > c$ gilt, dann gilt für alle (ausser endlich viele) Nachfolgerelemente $c_n < a$.
        \item \textit{(Untere Grenze)} Falls für ein $b \in \R: b < c$ gilt, dann gilt für unendlich viele Nachfolgerelemente $c_n > b$
    \end{enumerate}
\end{enumerate}
\end{lemma}
In Worte gefasst: Da $c$ der $\limsup$ von $c_{n \to \infty}$ ist, gibt es Teilfolgen, welche nach $c$ konvergieren. Nun haben wir ein $a > c$; man kann es sich als $c + \varepsilon$ vorstellen. Da wir aus der Definition wissen, dass es über $c$ zu keiner Konvergenz kommen kann, müssen alle Glieder sich ab einem bestimmten $n$ im Streifen unter $a$ befinden. Für das $b<c$, welches wir uns als $c - \varepsilon$ vorstellen können, kann man dieselbe Aussage im Allgemeinen nicht mehr machen. Da es aber Teilfolgen gibt, die nach $c$ konvergieren, müssen sicher unendlich viele Glieder grösser als $b$ sein. 

\begin{proof} (1. $\Longrightarrow$ 2.) ist als Übung zu lösen (Lösung im Skript Abschnitt 5.2.2)

(2. $\Longrightarrow$ 1.): Seien $a > c > b$, wähle $\varepsilon_a = 2(a - c) > 0$ und $\varepsilon_b = c - b > 0$.
\begin{enumerate}[label=(\alph*)]
    \item Es existiert ein $n_0$ gibt, sodass für alle Suprema der darauf folgenden Glieder kleinergleich\footnote{wegen der Definition des Supremums} $c + \frac{\varepsilon_a}{2}$ gilt, also:  $\sup \{c_k \mid k \geq n\} \leq c + \frac{\varepsilon_a}{2}$ mit $n \geq n_0$. Wir erhalten somit: $\sup \{c_k \mid k \geq n\} \leq c + \frac{\varepsilon_a}{2} < c + \varepsilon_a < a$.
    \item Da $c$ ein Konvergenzpunkt von Teilfolgen ist, gibt es für jedes $n$ ein $k \geq n$, sodass $c_k > c - \varepsilon_b = b$ gilt. Das impliziert $\sup \{c_k \mid k \geq n \}> c - \varepsilon = b$
\end{enumerate}
Sei nun $\varepsilon = \min\{\varepsilon_a, \varepsilon_b\}$, dann gilt $c-\varepsilon < \sup\{c_k \mid k \geq n\} < c+\varepsilon$ für $n \geq n_0$. Lassen wir $n$ nun nach Unendlich laufen, erhalten wir die Behauptung:
$$\lim_{n \to \infty}{\sup \{c_k \mid k \geq n\}} = c$$
\end{proof}

\begin{remark}
\begin{enumerate}
    \item Falls nun die gesamte Folge $(c_n)$ konvergiert, dann gibt es zu jedem $\varepsilon >0$ ein $n_0$ so dass $c-\varepsilon < c < c + \varepsilon$, also a) und b) gelten. Daraus folgt $\limsup_{n \to \infty} c_n = \lim_{n \to \infty} c_n$
    \item Aus dem Lemma folgt, dass im Allgemeinen $c = \limsup_{n \to \infty} c_n$ der grösste Häufungspunkt (also der grösste Grenzwert aller konvergierenden Teilfolgen von $(c_n)_{n=0}^\infty$)
    \item Analog gelten alle Eigenschaften auch für das Infimum: $\liminf{c_n} = -\limsup{-c_n}$
    \item Es gilt eine (auf positive Werte begrenzte) Homogenität: $\limsup{\lambda c_n} = \lambda \limsup{c_n}$ für $\lambda \in \R_{>0}$
\end{enumerate}
\end{remark}

Zu guter Letzt verwenden wir für nicht von oben beschränkte Folgen $(c_n)_{n=0}^\infty$ die Konvention
$$\limsup_{n \to \infty} c_n = \infty$$
In diesem Fall gilt 2b) aus obigem Lemma für alle $b \in \R$.

\subsection{Einschub: Wurzelkriterium und Limes Superior}
Nun nochmal einen kurzen Theorieeinschub um die Brücke zwischen dem soeben definierten $\limsup$ und dem \textsc{Cauchy}-Wurzelkriterium zu schlagen. Hier nochmals das Wurzelkriterium zur Erinnerung:

Sei $(c_n)_{n=0}^\infty$ eine Folge in $\R$ oder $\C$:
\begin{enumerate}
    \item Falls $\sqrt[n]{|c_n|} < 1$ für alle (ausser endlich) viele $n$ gilt, dann konvergiert die Reihe von $(c_n)_{n=0}^\infty$ absolut, d.h. die Summe der Absolutbeträge $\sum_{n=0}^{\infty}|c_n|$ konvergiert.
    \item Falls $\sqrt[n]{|c_n|} > 1$ für unendlich viele $n$ gilt, divergiert die Reihe $\sum_{n=0}^{\infty}|c_n|$ (da $(c_n)_{n=0}^\infty$ keine Nullfolge ist.) Insbesondere konvergiert sie nicht absolut.
\end{enumerate}

Die Konvergenz der Reihe von einer Folge $(c_n)_{n=0}^\infty$ unter dem Wurzelkriterium besagt in Worten ausgedrückt, dass ''die Reihe von $(c_n)_{n=0}^\infty$ höchstens so schnell wächst wie die geometrische Reihe''.

\begin{satz}{Umformulierung des Wurzelkriteriums}{limsup_root_crit}
\begin{enumerate}
    \item Falls $\limsup_{n \to \infty} \sqrt[n]{|c_n|} < 1$ gilt, dann konvergiert $\sum_{n=0}^{\infty}|c_n|$
    \item Falls $\limsup_{n \to \infty} \sqrt[n]{|c_n|} > 1$ gilt, dann divergiert $\sum_{n=0}^{\infty}|c_n|$
\end{enumerate}
\end{satz}

\begin{remark}
Über $\limsup (...) = 1$ kann wie beim Wurzelkriterium keine Aussage gemacht werden
\end{remark}

\begin{proof}
Verwende obiges Lemma \ref{lem:limsup_consequences}. Für 1. wähle $a$, sodass $\sqrt[n]{|c_n|} < a < 1$ gilt. Für 2. wähle $b = 1$
\end{proof}

\subsection{Konvergenzverhalten}
Nun haben wir die nötigen Begriffe erarbeitet und können zu den Potenzreihen zurückkehren, von denen wir nun das Konvergenzverhalten bestimmen wollen. Wir werden die obige Form des \textsc{Cauchy}-Wurzelkriteriums verwenden, um den \textbf{Konvergenzradius} um $0$ herum zu bestimmen, d.h., dass wir vorerst den Entwicklungspunkt oder den ''Mittelpunkt'' $a = 0$ setzen. Die Behauptung zur Konvergenz von Potenzreihen ist folgende:

\begin{satz}{Konvergenzradius, Formel von Hadamard}{konvergenz_von_potreihen}
Sei $(a_n)_{n=0}^\infty$ eine Folge in $\R$ oder $\C$, dann gilt für den \textbf{Konvergenzradius} $R$ von $f(z) = \sum_{n=0}^\infty a_n z^n$:

$$R = \frac{1}{\limsup_{n \to \infty}\sqrt[n]{|a_n|}} \in [0, \infty]$$

d.h. für ein beliebiges $z \in \C$ die Reihe $\sum_{n=0}^\infty a_n z^n$ absolut genau dann, wenn $|z|< R$ gilt. Für $|z|>R$ divergiert die Reihe.
\end{satz}

\begin{remark} Auch hier können wir keine Aussage über das Konvergenzverhalten auf dem Rand machen. Dieses muss für Punkt des Kreises auf andere Weise berechnet werden.
\end{remark}

\begin{proof}
Wir wenden Satz \ref{satz:limsup_root_crit} an auf $c_n = a_nz^n$ und erhalten $\sqrt[n]{|c_n|} = \sqrt[n]{|a_nz^n|} = \sqrt[n]{|a_n|}\cdot|z|$. Wir erkennen, dass die Reihe immer konvergiert für $|z| = 0 = z$. Dies kann man auch intuitiv leicht am Polynom $f(0)$ erkennen: Alle Terme sind 0 ausser der konstante Term $a_0$, welcher endlich ist. Nun sei $|z| \neq 0$, dann muss für die Konvergenz mit dem Wurzelkriterium gelten:
$$\limsup_{n \to \infty} \sqrt[n]{|c_n|} = \limsup_{n \to \infty} \sqrt[n]{|a_nz^n|} \stackrel{Bem. 4.}{=} |z| \limsup_{n \to \infty} \sqrt[n]{|a_n|} < 1$$
Wir erhalten durch Umstellung den folgenden Ausdruck und definieren damit den Konvergenzradius:
$$|z| < \frac{1}{\limsup_{n \to \infty} \sqrt[n]{|a_n|}} =: R$$
Für die Konvergenz gilt also die Beziehung: $\limsup_{n \to \infty} \sqrt[n]{|c_n|} = \frac{|z|}{R} < 1$. Falls im anderen Fall also  $\frac{|z|}{R} > 1$ gilt, divergiert die Reihe.
\end{proof}

\begin{example} Hier einige Beispiele:
\begin{enumerate}[label=\alph*)]
    \item Exponentialfunktion $e^z = 1 + \sum_{n = 1}^\infty \frac{z^n}{n!}$. Das Supremum eines Elements $n$ ist jeweils $\frac{1}{n!}$, somit erhalten wir also für den Konvergenzradius:
    $$R = \frac{1}{\limsup_{n \to \infty}\sqrt[n]{\frac{1}{n!}}} = \frac{1}{\lim_{n \to \infty}\sqrt[n]{\frac{1}{n!}}} = \infty$$
    \item Geometrische Reihe $\sum_{n = 1}^\infty z^n = \frac{1}{1-z}$: Es folgt direkt, dass $R = 1$ gilt. Somit konvergiert es nur absolut für $|z| < 1$
    \item $\sum_{n = 1}^\infty n! z^n$: Die Reihe ist nicht begrenzt: $\lim_{n \to \infty} \sqrt[n]{|n!|} = \infty$, also ist das Supremum für alle $n$ gleich $\infty$ und somit gilt für den Limes Superior auch $\limsup_{n \to \infty}\sqrt[n]{|n!|} = \infty$. Es folgt $R =$ ''$\frac{1}{\infty}$'' $ = 0$. $z = 0$ nennen wir in diesen Fall einen \textbf{konvergenten Randpunkt}.
\end{enumerate}
\end{example}

Bis jetzt haben wir alles für Potenzreihen ''um den Ursprung'' mit $a = 0$ gezeigt. Wir wollen nun in einen letzten Schritt diesen Mittelpunkt um $a$ verschieben, um die Form aus (\ref{eq_fcn_potenzreihe} zu erhalten:

\begin{lemma}{}{}
Die Potenzreihe
$$\sum_{n = 1}^\infty a_n (x-a)^n$$
mit Mittelpunkt $a \in \R$ konvergiert absolut im Intervall $(a-R, a+R)$, wobei 
$$R = \frac{1}{\limsup_{n \to \infty}\sqrt[n]{|a_n|}} \in [0, \infty]$$
\end{lemma}

\begin{proof}
Hierfür setzen setzen wir $z = x - a$ in die obigen Definitionen ein.
\end{proof}

Der Konvergenzbereich ist also eine Kreisscheibe in $\C$ um $a$ mit Radius $R$. 

\subsection{Summen und Produkte von Potenzreihen}
Wir können des Weiteren auch Summen und Produkte zweier Potenzreihen bilden: Seien $\sum a_n z^z$ und $\sum b_n z^z$ zwei Potenzreihen mit Konvergenzradien $R_a$ und $R_b$, somit konvergieren beide für $z \in \C$ im Bereich $|z| < R$, wobei $R = \min\{R_a, R_b\}$ gilt. Es folgen also für Summen und Produkte:
\begin{align*}
    \sum_{n = 0}^\infty a_n z^n + \sum_{n = 0}^\infty b_n z^n &= \sum_{n = 0}^\infty (a_n + b_n) z^n\\
    (\sum_{n = 0}^\infty a_n z^n)(\sum_{n = 0}^\infty b_n z^n) &= \sum_{n = 0}^\infty (a_0b_n + a_1b_{n-1} + ... + a_nb_0 )z^n
\end{align*}
welche absolut konvergieren für $|z| < R$. Für die Multiplikation wurde das Cauchy-Produkt zweier absolut konvergenter Reihen verwendet.
$$*\ *\ *$$

Nun haben wir unsere erste Frage bezüglich der Konvergenz beantworten können. Um die zweite Frage zu den Eigenschaften von Potenzreihen zu beantworten wollen wir etwas ausholen:

Eine Potenzreihe sieht ja im Allgemeinen in etwa so aus:
$$f(x) = a_0 + a_1(x-a) + a_2(x-a)^2 + a_3(x-a)^3...$$
Der Funktionswert $f(x)$ ist also der Grenzwert der Reihe der einzelnen Terme oder anders ausgedrückt der Grenzwert aus der Folge der Partialsummen:
$$f(x) = \lim_{n \to \infty}\Big( \sum_{k=0}^n a_k(x-a)^k \Big)$$
Wir können daher also eine Funktion $f_n$ definieren, welche nur die ersten $n$ Terme von $f$ aufweist. Diese Funktion $f_n$ wird nur bis zu einem gewissen Grad Ähnlichkeiten mit $f$ haben, lässt man aber den Index $n$ nach $\infty$ laufen, so inkludieren wir immer mehr Terme von $f$, wobei wir im Limit (im Konvergenzbereich versteht sich) $\lim_{n \to \infty} f_n = f$ erhalten.

Wir könnten nun aber auch eine Folge von diesen Partialsummenfunktionen $f_n$ konstruieren, also $(f_1, f_2, f_3, ...)$. Werten wir alle Funktionen bei einem $x$ aus, so erhalten wir eine Folge in $\R$, dessen Grenzwert $f(x)$ sein muss. Wir können also Potenzreihen als Limit von Funktionenfolgen schreiben. Wir wollen also kurz allgemeine Eigenschaften von solchen Funktionenfolgen zeigen:

\subsection{Einschub: Konvergenz von Funktionenfolgen}\label{cha_funktionenfolgen}
Wir wollen also eine Folge konstruieren, dessen Elemente Funktionen sind: $(f_1, f_2, f_3, ...)$. Sie sollen alle den selben Definition- und Wertebereich besitzen, wir verwenden $f_i: D \to \R$, wobei $D \subseteq \R$ eine nicht-leere Teilmenge von $\R$ ist. Im Allgemeinen sind alles verschiedene Funktionen, im Fall von Partialsummen von Potenzreihen jedoch unterscheiden sie sich nur in der vom Index $i \in \N$ abhängigen Länge.

Diese Folge von Funktionen liefert für jedes Argument $x \in D$ eine ''normale'' Folge in $\R$: $(f_1(x), f_2(x), f_3(x), ...)$, somit können wir sie und die daraus entstehenden Reihen auf Konvergenz untersuchen und erhalten daraus je nach dem einen Grenzwert. Macht man das nun für alle $x \in D$, so erhält man quasi eine ''Grenzfunktion'' $f$, welche ausgewertet bei einem $x$ den Grenzwert der Folge von $\sum_{i=1}^\infty f_n(x)$ gibt. Hierzu wollen wir nun zwei Begriffe der Konvergenz definieren:

\begin{definition}{Punktweise Konvergenz}{}
Sei $D \subseteq \R, D \neq \emptyset$. Die Folge $(f_n:D \to \R)^\infty_{n=0}$ \textbf{konvergiert punktweise} [pointwise] gegen $f: D \to \R$ falls
$$\lim_{n \to \infty} f_n(x) = f(x)$$
für alle $x \in D$.
\end{definition}

\begin{definition}{Gleichmässige Konvergenz}{}
Sei $D \subseteq \R, D \neq \emptyset$. Die Folge $(f_n:D \to \R)^\infty_{n=0}$ \textbf{konvergiert gleichmässig} [uniformly] gegen $f: D \to \R$, falls zu jedem $\varepsilon >0$ ein Index $n_0$ existiert, sodass:
$$\forall n\geq n_0: |f_n(x) - f(x)| < \varepsilon $$
für alle $x \in D$ gilt.
\end{definition}
\begin{remark}
Die Definitionen sind sehr ähnlich, beachte aber die Reihenfolge der Quantoren: Für die punktweise Konvergenz betrachten wir zuerst ein $x$ und wählen dann das passende $n$, sodass $|f_n(x)-f(x)| < \varepsilon$ (kommt aus der Definition des Limes). Bei der gleichmässigen Stetigkeit hingegen soll das $n_0$ --- nach der Wahl eines $\varepsilon$ --- für alle $x \in D$ gelten. Also darf das $n_0$ nur von $\varepsilon$ abhängen, nicht aber von $x$. Somit ist die gleichmässige Konvergenz einschränkender und die punktweise Konvergenz folgt aus der gleichmässigen Konvergenz.
\end{remark}

\begin{example}\label{ex_grenzfunktion} Wir definieren die Funktionenfolge wie folgt:

\begin{minipage}[c]{0.4\linewidth}
$$f_n: D=[0,1] \to \R$$
$$x \mapsto \begin{cases} 0 & x >\frac{1}{n} \\ 1-nx & x \leq \frac{1}{n}\end{cases}$$\\
\end{minipage}
\hspace{2em}
\begin{minipage}[]{0.4\linewidth}
\begin{tikzpicture}
    \begin{axis}[
        axis lines = left,
        xlabel = $x$,
        ylabel = {$f_n(x)$},
        xtick={0, 0.125,0.25,1},
        xticklabels={$0$,$\frac{1}{2n}$,$\frac{1}{n}$,$1$},
        ytick={0,0.5, 1},
        yticklabels={$0$,$\frac{1}{2}$,$1$},
        width=7cm,height=4cm,
    ]
    \addplot [domain=0:0.25, samples=5, color=red, style={thick}] {1 - 4*x};
    \addplot [domain=0.25:1, samples=5, color=red, style={thick}] {0};
    \addplot[mark=*, color=red] coordinates {(0.125,0.5)};
    \end{axis}
\end{tikzpicture}
\end{minipage}\hfill

Der punktweise Grenzwert dieser Funktionenfolge ist:
$$ \lim_{n \to \infty} f_n(x) = f(x) = \begin{cases} 0 & x >0 \\ 1 & x = 0\end{cases}$$
Wir erkennen zudem auch, dass diese Funktionenfolge nicht gleichmässig konvergiert, da es z.B. für $\varepsilon = \frac{1}{2}$ kein $n_0$ gibt, sodass für alle $x > 0$ die Bedingung $|f_{n_0}(x)-f(x)| < \varepsilon$ gilt. Betrachte hierfür z.B. den Punkt bei $\frac{1}{2n_0}$, welcher für jedes beliebig grosse $n_0$ stets eine Differenz von $\frac{1}{2}$ zu $f(\frac{1}{2n}) = 0$ aufweist.
\end{example}

\subsection{Folgerungen der gleichmässigen Konvergenz}

Die gleichmässige Konvergenz ist speziell, da einige Eigenschaften der Funktionen $f_n$ der Funktionenfolge an die Grenzfunktion ''weitervererbt'' werden können:

So gilt für die Grenzfunktion $f$ von Funktionenfolgen, bei der alle (ausser endlich viele) Elemente/Funktionen $f_n$ stetig sind, ebenfalls die Stetigkeit\footnote{Da wir jetzt von zwei verschiedenen ähnlichen Eigenschaften reden, muss man sich immer im Klaren sein, was nun konvergiert/stetig ist. Die Elemente $f_n(x)$ der Funktionenfolge sind für ein $x$ im Konvergenzradius absolut konvergent. Das sagt aber nichts über die Stetigkeit der Grenzfunktion $f$ aus, ausser dass sie dort definiert ist. Wie wir gesehen haben, kann diese trotz stetiger $f_n$ wie in obigem Beispiel \ref{ex_grenzfunktion} selber nicht stetig ist.}. Wir wollen also folgenden Satz zeigen:

\begin{satz}{Stetigkeit von Grenzfunktion}{stetigkeit_grenzfcn}
Ist $f_n: D \to \R$ (oder $\C$) eine Folge stetiger Funktionen auf $D \subseteq \R$ oder $\C$, die gleichmässig gegen die Funktion $f: D\to \R$ konvergieren, dann ist die Grenzfunktion $f$ stetig.
\end{satz}

\begin{proof} Sei $a \in D \subseteq \R, \varepsilon > 0$. Wir wollen zeigen, dass ein $\delta > 0$ existiert, sodass
$$\abs{x-a} < \delta \implies \abs{f(x) - f(a)} < \varepsilon$$
Wegen der gleichmässigen Konvergenz gibt es also ein $n_0$, sodass
$$\abs{f(x)-f_{n_0}(x)}<\frac{\varepsilon}{3}$$
gilt für alle $x\in D$. Aus der Stetigkeit von $f_{n_0}$ wissen wir zudem, dass es ein $\delta$ gibt, sodass 
$$\abs{f_{n_0}(x)-f_{n_0}(a)} < \frac{\varepsilon}{3}$$
gilt für $\abs{x-a} < \delta$. Nun formen wir den zu zeigenden Ausdruck mit der Dreiecksungleichung um, sodass wir die uns bekannten Ausdrücke erhalten:
\begin{align*}
    \abs{f(x) - f(a)} &= \abs{f(x) - f_{n_0}(x) + f_{n_0}(x) - f_{n_0}(a)+f_{n_0}(a) - f(a)}\\
    &\leq \abs{f(x) - f_{n_0}(x)} + \abs{f_{n_0}(x) - f_{n_0}(a)} + \abs{f_{n_0}(a) - f(a)}\\
    &< \frac{\varepsilon}{3} +  \frac{\varepsilon}{3} + \frac{\varepsilon}{3} = \varepsilon
\end{align*}
falls $\abs{x - a} < \delta$ gilt. Das zeigt also die Behauptung, dass wir eine stetige Grenzfunktion aus einer gleichmässig stetigen Funktionenfolge erhalten.
\end{proof}

Eine andere Eigenschaft, die die Grenzfunktion $f$ mit der gleichmässigen Konvergenz aus den Funktionen $f_n$ der Funktionenfolge ''erbt'', ist die \textsc{Riemann}-Integrierbarkeit\footnote{Erinnerung \textsc{Riemann}-Integrierbarkeit: $f:{a,b} \to \R$ ist \textsc{Riemann}-integrierbar genau dann, wenn es zu jedem $\varepsilon > 0$ Treppenfunktionen $u,o$ gibt mit $u \leq f \leq o$ und $\int_a^b(o-u)(x)dx < \varepsilon$}:

\begin{satz}{\textsc{Riemann}-Integrierbarkeit von Grenzfunktion}{fcnfolge_riem_inbarkeit}
Sei $a < b \in R$. Falls $f_n:[a,b] \to \R$ eine Folge aus \textsc{Riemann}-integrierbaren Funktionen ist, die gleichmässig gegen $f[a,b] \to \R$ konvergiert, dann ist $f$ Riemann-integrierbar und es gilt:
$$\lim_{n \to \infty} \int_a^b{f_n(x)}dx =  \int_a^b \lim_{n \to \infty}{f_n(x)}dx = \int_a^b {f_n(x)}dx$$
\end{satz}
Es wird also in anderen Worten behauptet, dass die Vertauschung des Limes-Operators und des Integrals erlaubt ist.

\begin{proof} (Teil I: \textsc{Riemann}-Integrierbarkeit) Die Beweisidee ist, ein $f_n$ zu finden, damit man aus deren (obere und untere) Treppenfunktionen entsprechende Treppenfunktionen für $f$ konstruieren können:

Wir legen im ersten Schritt einen $\tilde{\varepsilon}$-''Schlauch'' um $f$ (Das entspricht also der Fläche zwischen $f-\tilde{\varepsilon}$ und $f+\tilde{\varepsilon}$). Nun wollen wir ein $n$ finden, sodass $f_n$ ganz in diesem Schlauch liegt: Es soll also $f-\tilde{\varepsilon} < f_n(x) < f+\tilde{\varepsilon}$ resp. $\abs{f_n(x) - f(x)} < \tilde{\varepsilon}$ für alle $x \in [a,b]$ gelten.

Dann können wir wegen der \textsc{Riemann}-Integrierbarkeit von $f_n$ eine obere und eine untere Treppenfunktion $o_n$ und $u_n$ von $f_n$ finden, für welche $\int_a^b o_n(x) - u_n(x) dx < \tilde{\varepsilon}$ gilt. Da für $o_n$ und $u_n$ per Definition $u_n \leq f_n \leq o_n$ gilt und die Differenz zwischen $f_n$ und $f$ kleiner als $\tilde{\varepsilon}$ ist, können wir folgenden Ausdruck herleiten:
$$\underbrace{u_n(x) - \tilde{\varepsilon}}_{\tilde{u}_n(x)} \leq f_n(x) - \tilde{\varepsilon} < f(x) < f_n(x) + \tilde{\varepsilon} \leq \underbrace{o_n(x) + \tilde{\varepsilon}}_{\tilde{o}_n(x)}$$
Somit haben wir eine obere und eine untere Treppenfunktion für $f$ konstruieren können:
$$\tilde{u}_n(x) \stackrel{(<)}{\leq} f(x) \stackrel{(<)}{\leq} \tilde{o}_n(x)$$
Integriert man nun die Differenz von $\tilde{u}_n(x)$ und $\tilde{o}_n(x)$, so erhält man:
\begin{align*}
    \int_a^b(\tilde{u}_n(x)-\tilde{o}_n(x))dx &= \int_a^b(o_n(x) + \tilde{\varepsilon} - u_n(x) + \tilde{\varepsilon})dx\\
    &=\int_a^b(o_n(x) - u_n(x))dx + 2\tilde{\varepsilon}(b-a)\\
    &<\tilde{\varepsilon}+ 2\tilde{\varepsilon}(b-a)
\end{align*}
Nun können wir $\tilde{\varepsilon} = \frac{\varepsilon}{1+2b-2a}$ wählen, wodurch wir die \textsc{Riemann}-Integrierbarkeit von $f$ zeigen:
$$\forall \varepsilon > 0: \exists \tilde{u}_n, \tilde{o}_n: \Big(\tilde{u}_n \leq f \leq \tilde{o}_n\Big) \land \Big(\int_a^b(\tilde{u}_n - \tilde{o}_n)(x)dx < \varepsilon\Big)$$

(Teil II: Gleicher Grenzwert) Nachdem wir die Integrabilität gezeigt haben, wollen wir noch zeigen, dass das Integral von $f$ auch den selben Wert annimmt wie der vom Limit der Funktionenfolge. Wir wollen also zeigen, dass für ein $\varepsilon>0$
$$\abs{\int_a^b f_n(x) dx - \int_a^b f(x) dx} < \varepsilon$$
gilt. Ähnlich wie vorher (nur direkter ohne $\tilde{\varepsilon}$) können wir wegen der gleichmässigen Konvergenz $n$ wie folgt wählen:
$$\abs{f_n(x) - f(x)}<\frac{\varepsilon}{b-a}$$
Wir erhalten also für die Differenz der Integrale:
\begin{align*}
    \abs{\int_a^b f_n(x) dx - \int_a^b f(x) dx} &= \abs{\int_a^b f_n(x) - f(x) dx}\\
    &\leq \int_a^b \abs{f_n(x)- f(x)} dx\\
    &<\int_a^b \frac{\varepsilon}{b-a}dx\\
    &< \varepsilon
\end{align*}
Somit haben wir die Gleichheit der Integrale gezeigt, woraus die Behauptung folgt.
\end{proof}

\subsection{Stetigkeit von Potenzreihen}
Nun wollen wir die Eigenschaften der Funktionenfolgen auf Potenzreihen anwenden und zeigen, dass die Grenzfunktion auch stetig ist. Sei also $\sum_{n=0}^\infty a_n x^n$ eine Potenzreihe mit Koeffizienten $a_n \in \C$ und Konvergenzradius $R \in [0, \infty]$. Dann definiert $$f(x) = \sum_{n=0}^\infty a_n x^n$$
eine Funktion $(-R, R) \to \C$. Wir wollen $f$ nun auch als Grenzwert der Partialsummen betrachten, deswegen definieren wir die Partialsumme bis zum $n$-ten Glied als:
$$s_n(x) = \sum_{k=0}^n a_k x^k$$
Es gilt also per Konstruktion $\lim_{n \to \infty} s_n(x) = f(x)$. Wir wollen nun einige Eigenschaften zeigen:

\begin{lemma}{gleichmässige Konvergenz im Konvergenzradius}{glm_konv_konvergenradius}
Die Folge der Partialsummen $(s_n)_{n = 0}^\infty$ konvergiert gleichmässig gegen $f$ auf jedem kompakten Intervall $[a, b] \subseteq (-R,R) $, d.h. $s_n\mid_{[a,b]}$ konvergiert gegen $f\mid_{[a,b]}$.
\end{lemma}

\begin{proof}
Es genügt das Lemma für $[a,b] = [-c,c]$ mit $0\leq c < R$ zu beweisen, da jedes kompakte Intervall $[a,b] \subseteq (-R,R)$ in einem Intervall $[-c,c] \subseteq (-R,R)$ enthalten ist.

Da $c \in (-R,R)$ im Konvergenzradius $R$ liegt, folgt aus der absoluten Konvergenz, dass
$\sum_{n=0}^\infty |a_n| c^n$ konvergiert, d.h. für jedes $\varepsilon > 0$ gibt es ein $n_0$, sodass man die Reihe ab einem $n \geq n_0$ abtrennen und aufteilen kann:
$$\Big|\underbrace{\sum_{k=0}^\infty |a_k| c^k}_{\text{abs. Grenzwert}} - \underbrace{\sum_{k=0}^{n_0} |a_k| c^k}_{\text{abs. Part.summe}}\Big| = \underbrace{\sum_{k=n_0+1}^\infty |a_k| c^k}_{\text{''Rest''}} < \varepsilon$$
Mit diesem Ausdruck können wir gleich die gleichmässige Konvergenz zeigen. Für das obige $\varepsilon$ und $n_0$ genügt es also zu zeigen, dass $\abs{f(x) - s_n(x)} < \varepsilon$ für alle $n \geq n_0$ gilt:
\begin{align*}
    \abs{f(x) - s_{n_0}(x)} &= \abs{\sum_{k=0}^\infty a_k x^k - \sum_{k=0}^{n_0} a_k x^k}\\
                &= \Bigg|\underbrace{\sum_{k=n_0+1}^\infty a_k x^k}_{\text{''Rest''}}\Bigg|\\
                &\leq \ \sum_{k=n_0+1}^\infty\abs{a_k}\abs{x^k}\\
                &\stackrel{\mathclap{\mathrm{\abs{x}\leq c}}}{\leq} \ \sum_{k=n_0+1}^\infty \abs{a_k}c^k < \varepsilon
\end{align*}
\end{proof}

\begin{remark}
Dasselbe gilt (mit dem selben Beweis) für komplexe Argumente innerhalb des Konvergenzradius, also für Funktionen $f: B_R(0) \to \C$, wobei $B_R(0) = \{z \in \C \mid |z| < R\}$ der Ball um $0$ mit Radius $R$ ist. Das bedeutet, dass die Partialsummen $s_n(z)$ gleichmässig gegen $f(z)$ konvergieren, falls $z$ auf der abgeschlossenen Kreisscheibe $\quer{B_c(0)} = \{z \in \C \mid |z| \leq c\}$ mit $c<R$ liegt. Sie konvergieren also gleichmässig auf jedem kompakten Intervall in $(-R,R)$.
\end{remark}

Aus der gleichmässigen Konvergenz von den Partialsummen $s_n$ zur Grenzfunktion $f$ können wir nun mit Satz \ref{satz:stetigkeit_grenzfcn} auf folgende Aussage schliessen:

\begin{satz}{Stetigkeit der Potenzreihe}{}
Sei $\sum_{n=0}^\infty a_n x^n$ eine Potenzreihe mit Konvergenzradius $R$, dann ist
$$f: (-R, R) \to \C, x \mapsto f(x)= \sum_{n=0}^\infty a_n x^n$$ stetig.
\end{satz}

\subsection{Stetigkeit am Rand}
Wir haben gesehen, dass das Wurzelkriterium nicht anwendbar ist für den Rand und wir im Allgemeinen nur wenig über den Rand wissen. \textsc{Niels Abel} konnte aber durch geschicktes Anwenden einer Identität folgenden Satz zeigen: 

\begin{satz}{Grenzwertsatz von Abel}{abel_grenzwertsatz}
Sei $f: (-R, R) \to \C, f(x) = \sum_{n=0}^\infty a_n x^n$ eine Potenzreihe mit Konvergenzradius $R < \infty$, und sei $\sum_{n=0}^\infty a_n R^n$ konvergent, dann gilt:
$$\lim_{x \nearrow R}\sum_{n=0}^\infty a_n x^n = \sum_{n=0}^\infty a_n R^n$$
d.h. die Reihe definiert eine stetige Funktion auf $(-R, R] \to \C$. Analog gibt es dieselbe Aussage für $-R$.
\end{satz}

Bevor wir den Grenzwertsatz beweisen, wollen uns zuerst diese Identität kurz zeigen. Diese behauptet nämlich, dass für ein $x \in \R$ mit $|x|<1$ und $(a_n)_{n=0}^\infty$ eine Folge, deren Potenzreihe Konvergenzradius $1$ besitzt, folgendes gilt:
\begin{align}\label{eq_identity_abel}
    \frac{1}{1-x}\sum_{n=0}^\infty a_n x^n = \sum_{n=0}^\infty (a_0 + a_1 + ... + a_n) x^n
\end{align}

\begin{proof}[Beweis Identität] Links erkennen wir die geometrische Reihe, d.h. sie ist konvergent auf $(-1,1)$. Die Potenzreihe mit Koeffizienten $(a_n)_{n=0}^\infty$ ist ebenfalls konvergent auf $(-1,1)$. Es ist also ein Produkt zweier konvergenter Reihen, also können wir die rechte Seite direkt durch das \textsc{Cauchy}-Produkt beider Reihen erhalten:
$$\Big(\sum_{n=0}^\infty x^n\Big) \Big(\sum_{n=0}^\infty a_n x^n\Big) = \sum_{n=0}^\infty (a_0 + a_1 + ... + a_n) x^n$$
Somit ist die Identität gezeigt.
\end{proof}

\begin{proof}[Beweis Grenzwertsatz]
Wir nehmen o.B.d.A. an, dass $R = 1$ gilt\footnote{Ansonsten können wir alles skalieren indem wir $f(x)$ durch $f(xR) = \sum_{n=0}^\infty(a_nR^n)x^n$ ersetzen, siehe Satz \ref{satz:konvergenz_von_potreihen}.}. Aus der Annahme wissen wir zudem, dass
$$\sum_{n=0}^\infty a_n := A$$
konvergiert. Beachte, dass wir nicht ''$f(R) = A$'' setzen, da mit $f(R)$ die stetige Fortsetzung von $f$ im Punkt $R$ und mit $A$ der Grenzwert bezeichnet wird. Zu zeigen ist also, dass $\abs{f(x) - A} < \varepsilon$ beliebig klein wird.

Da $A$ per Annahme konvergiert, muss es für ein $\varepsilon > 0$ ein $n_0$ geben, sodass 
$$\Big|\underbrace{A-(a_0+a_1+a_2+...+a_{n_0})}_{-b_n}\Big| < \frac{\varepsilon}{2}$$
für alle $n \geq n_0$ gilt. $-b_n$ ist also der ''Rest'' der Reihe. Wir verwenden nun die Identität (\ref{eq_identity_abel}) und erkennen rechts die Partialsummen vom Grenzwert $A = \sum_{n=0}^\infty a_n$ als Koeffizienten. Wir leiten also für $|x| < 0$ her:
\begin{align*}
    f(x) - A &= \Big(\sum_{n=0}^\infty a_n x^n\Big) - A\\
        &\stackrel{\mathclap{\mathrm{(\ref{eq_identity_abel})}}}{=}\Big( (1-x) \sum_{n=0}^\infty (a_0 + a_1 + ... + a_n) x^n\Big) - A\\
        &=\Big( (1-x) \sum_{n=0}^\infty (b_n+A) x^n\Big) - A\\
        &=\Big( (1-x) \sum_{n=0}^\infty b_n x^n\Big) + \cancel{(1-x)\frac{1}{1-x}A} - \cancel{A}\\
        &= (1-x) \Big(\sum_{n=0}^{n_0} b_n x^n\Big) +  (1-x) \Big(\sum_{n=n_0+1}^\infty b_n x^n\Big)
\end{align*}
wobei wir im letzten Schritt die Summe bei $n_0$ aufgeteilt haben. Nun folgt mit der Dreiecksungleichung:
\begin{align*}
\abs{f(x)-A}&\leq\abs{(1-x) \sum_{n=0}^{n_0} b_n x^n} + \abs{(1-x)  \Big(\sum_{n=n_0+1}^\infty b_n x^n\Big)}\\
        &<\abs{(1-x) \sum_{n=0}^{n_0} b_n x^n} + \frac{\varepsilon}{2}\abs{(1-x) \Big(\sum_{n=0}^\infty x^n\Big)}\\
        &<\Big|\underbrace{(1-x) \sum_{n=0}^{n_0} b_n x^n}_{\text{Polynom }P(x)}\Big| + \frac{\varepsilon}{2}\Big|\underbrace{\sum_{n=0}^\infty x^n - \sum_{n=1}^\infty x^n}_{= 1}\Big|
\end{align*}
Den rechten Term haben wir durch Ausweiten der Grenzen nach oben abgeschätzt, wobei sich durch Ausklammern der $1$ die Summen wegkürzen. Wir erkennen, dass das (endliche) Polynom $P(x)$ eine Nullstelle bei $x=1$ hat, also gegen $0$ geht für $x \to 1$ und somit einfach abzuschätzen ist: Wir können also ein $\delta > 0$ finden, sodass das Polynom wie folgt abschätzen können: $\abs{x-1} < \delta \implies |P(x)| < \frac{\varepsilon}{2}$. Setzen wir alles zusammen, erhalten wir die Behauptung:
$$\abs{f(x)-A}<\frac{\varepsilon}{2} + \frac{\varepsilon}{2} = \varepsilon$$
\end{proof}

\subsection{Integration von Potenzreihen}
Wir haben die Stetigkeit von Potenzreihen im Konvergenzradius zeigen können. Dann ergibt es intuitiv aus Sinn, dass man diese Funktion in ihrem Definitionsbereich auch integrieren und ableiten können soll. Dies wollen wir nun also zeigen:

\begin{satz}{Stammfunktionen von Potenzreihen}{potenzreihen_stammfcn}
Sei $f(x) = \sum_{n=0}^\infty$ eine Potenzreihe mit Konvergenzradius $R$, dann hat die Potenzreihe
$$F(x) = \sum_{n=0}^\infty \frac{a_n}{n+1} x^{n+1}$$
denselben Konvergenzradius $R$ und es gilt für alle $a,b$ mit $-R < a \leq b < R$:
$$\int_a^bf(t)dt = F(b) - F(a)$$
Insbesondere ist $F(x) = \int_0^x f(t)dt$ eine Stammfunktion von f.
\end{satz}

\begin{proof} (Teil I: Konvergenzradius) 
Sei $S$ der Konvergenzradius von $F(x)$. Da für jeden Term von $F$: $\abs{\frac{a_{n}}{n+1}x^{n+1}} \leq \abs{x}\abs{a_{n}x^n}$ gilt und $f(x)$ mit $\abs{x} < R$ absolut konvergiert, muss auch $F(x)$  mit $x \in (-R, R)$ konvergieren. Daraus folgt $S \geq R$.

Nun wollen wir $S \leq R$ mithilfe des Majorantenkriteriums zeigen. Durch Umschreiben erhalten wir:
$$F(x) = a_0 + x\sum_{n=1}^\infty\frac{a_n}{n+1} x^n$$
Wir wissen, dass $a_0$ wie auch der Vorfaktor $x$ endlich sind und nichts am Konvergenzverhalten ändern. Also muss $\sum_{n=1}^\infty\frac{a_n}{n+1} x^n$ konvergieren. Aus der Definition des Konvergenzradius gilt deshalb für $S$:
$$S = \frac{1}{\limsup_{n \to \infty}\sqrt[n]{\abs{\frac{a_n}{n+1}}}}$$
Wenn wir $\limsup_{n \to \infty}\sqrt[n]{\abs{\frac{a_n}{n+1}}}$ betrachten, können wir erkennen, dass der Nenner $\sqrt[n]{n+1} \geq 1$ für grosse $n$ nach $1$ konvergiert. Wir können diesen Term mit einer Abschätzung unabhängig von $n$ machen: Es gibt für jedes $\varepsilon > 0$ ein $n_0$, sodass für alle grösseren $n \geq n_0$ gilt: $\abs{\sqrt[n]{n+1}} \leq 1+\varepsilon$, somit gilt also:
$$S \leq \frac{1}{\frac{1}{1+\varepsilon}\limsup_{n \to \infty}\sqrt[n]{\abs{a_n}}} = R(1+\varepsilon)$$
Da $\varepsilon$ nun beliebig klein gewählt werden kann, gilt $S = R$.

(Teil II: Stammfunktion) Wir betrachten die Folge der Partialsummen $s_n(x) = \sum_{k=0}^na_kx^k$:
$$\int_a^bf(x) dx = \int_a^b \lim_{n \to \infty} s_n(x) dx$$
Aus Satz \ref{satz:fcnfolge_riem_inbarkeit} wissen wir, dass, wenn die Funktionenfolge $s_n$ stetig ist und gleichmässig konvergiert (siehe Lemma \ref{lem:glm_konv_konvergenradius}), man das Integral und den Limes vertauschen kann. Wir erhalten durch das Integrieren der (endlichen) Polynome:
\begin{align*}
    \int_a^bf(x) dx &= \lim_{n \to \infty} \int_a^b s_n(x) dx\\
        &= \lim_{n \to \infty}\Big( \sum_{k=0}^n a_k \frac{b^k}{k+1} -\sum_{k=0}^n a_k \frac{a^k}{k+1} \Big)\\
        &= \sum_{k=0}^\infty a_k \frac{b^k}{k+1} -\sum_{k=0}^\infty a_k \frac{a^k}{k+1}\\
        &= F(b) - F(a)
\end{align*}
Mit $a=0, b=x$ erhalten wir schliesslich die Stammfunktion wie behauptet:
$$F(x) = \int_0^xf(t) dt$$
\end{proof}
Nun wissen wir, dass wir Potenzreihen im Konvergenzradius integrieren dürfen. Hier einige Anwendungsbeispiele dazu:

\begin{example}[$\log 2$ aus Reihe] Wir wollen den Grenzwert der alternierenden harmonischen Reihe berechnen:
$$A := 1 - \frac{1}{2} + \frac{1}{3} - \frac{1}{4} + ... = \sum_{n=0}^\infty (-1)^{n} \frac{1}{n+1}$$
Wir wollen zuerst eine allgemeinere Potenzreihe finden, nämlich die vom $\log(x+1)$:
\begin{align*}
    \log(x+1)&=\int_0^x\frac{1}{1+t}dt &&\text{geom. Reihe mit }(-t)\\
        &=\int_0^x\Big(\sum_{n=0}^\infty (-t)^n \Big) dt\\
        &=\sum_{n=0}^\infty (-1)^n \int_0^xt^n dt\\
        &=\sum_{n=0}^\infty (-1)^n \frac{x^{n+1}}{n+1}
\end{align*}
Da wir wissen, dass die alternierende harmonische Reihe konvergiert (\textsc{Leibniz}-Kriterium), können wir aus dem \textsc{Abel}'schen Grenzwertsatz \ref{satz:abel_grenzwertsatz} und der Erhaltung des Konvergenzradius (?) schliessen, dass die Reihe bei $x=1$ konvergiert. Wir erhalten also:
$$\log(2)=\sum_{n=0}^\infty (-1)^n \frac{1}{n+1} = A$$
\end{example}

\begin{example}[\textsc{Gregory}-Reihe, \textsc{Leibniz}-Formel für $\pi$] Wir wissen, dass $\arctan(1) = \frac{\pi}{4}$ und dass die Ableitung davon wieder nach einem schönen Fall für die geometrische Reihe mit $(-t^2)$ aussieht. Wir leiten als für $\abs{x} < 1$ her:
\begin{align*}
    \frac{d}{dx}\arctan x &= \frac{1}{1+x^2}\\
    \arctan x &= \int_0^x \frac{1}{1+t^2}\\
        &= \int_0^x \sum_{n=0}^\infty (-1)^n t^{2n}\\
        &= \sum_{n=0}^\infty (-1)^n \frac{x^{2n+1}}{2n+1}\\
        &= x - \frac{x^3}{3} + \frac{x^5}{5} - \frac{x^7}{7} +...
\end{align*}
Setzen wir nun $x=1$, so erhalten wir die alternierende harmonische Reihe. Da diese Reihe konvergent ist (\textsc{Leibniz}-Kriterium für alternierende Nullfolgen), können wir mit dem \textsc{Abel}'schen Grenzwertsatz auf die Stetigkeit schliessen und $\arctan 1 = x - \frac{x^3}{3} + \frac{x^5}{5} -... = \frac{\pi}{4}$ behaupten. Daraus erhalten wir mit $x=1$ die \textsc{Madhava-Leibniz}-Formel für $\pi$:
$$\frac{\pi}{4} = 1 - \frac{1}{3} + \frac{1}{5} - \frac{1}{7} +...$$
Man merkt aber, dass diese Formel nur sehr langsam nach $\frac{\pi}{4}$ konvergiert (Fehler jeweils in der Grössenordnung des letzten Terms, also sind z.B.  für ein $\Delta \leq 0.001$ die Terme bis $\frac{1}{1001}$ nötig). Das liegt daran, dass sich der Punkt direkt auf dem Rand des Konvergenzbereichs befindet. Viel schneller konvergieren andere Reihen, welche z.T. dieselbe Reihe wie oben verwenden, diese aber im Inneren des Konvergenzbereichs z.B. bei $\frac{1}{\sqrt{3}}$ auswerten, da $\arctan \frac{1}{\sqrt{3}} = \frac{\pi}{6}$ gilt.
\end{example}

\subsection{Ableitung von Potenzreihen}
Wie schon angesprochen sollen Potenzreihen mit den üblichen Eigenschaften differenzierbar sein. Wir wollen also folgenden Satz zeigen:
\begin{satz}{Differenzierung von Potenzreihen}{}
Sei $f(x) = \sum_{n=0}^\infty$ eine Potenzreihe mit Konvergenzradius $R$, dann ist $f$ stetig differenzierbar auf $(-R,R)$ und es gilt:
$$f'(x) = \sum^\infty_{n=1} na_nx^{n-1}$$
\end{satz}
\begin{proof}
Wir haben bereits gezeigt, dass Potenzreihen integrierbar sind. Das heisst, wenn wir eine Funktion finden, welche integriert $f$ ergibt (bis auf eine Konstante), dann wissen wir, dass das die Ableitung von $f$ ist. Wir definieren daher die Funktion $g$ wie folgt:
$$g(x):=\sum_{n=1}^\infty n a_n x^{n-1} = \sum_{n=0}^\infty (n+1) a_{n+1} x^n$$
Dann folgt für die Stammfunktion $G$ von $g$:
\begin{align*}
    G(x) &= \int_0^x g(t) dt\\
        &= \sum_{n=0}^\infty \frac{\cancel{(n+1)}a_{n+1}}{\cancel{n+1}}x^{n+1}\\
        &= \sum_{n=1}^\infty a_nx^n = f(x) - a_0
\end{align*}
Mit dem Satz für die Integrierbarkeit \ref{satz:potenzreihen_stammfcn} folgt aus $G'(x) = g(x)$ die Behauptung $f'(x) = g(x)$. Des Weiteren folgt daraus auch, dass $f$ und $f$ denselben Konvergenzradius $R$ besitzen.
\end{proof}

Mit dieser Differenzierbarkeit von Potenzreihen können wir nun einige weitere Eigenschaften folgern:

\begin{korollar}{Glattheit von Potenzreihen}{}
Die Funktion $f(x) = \sum_{n=0}^\infty$ ist \textbf{glatt}\footnote{beliebig oft stetig differenzierbar} im Konvergenzbereich $(-R,R)$ von $f$ und es gilt für die $n$-te Ableitung:
$$f^{(n)}(x) = \sum_{k=n}^\infty k(k-1)...(n-k+1)a_nx^{n-k}$$
für $x \in (-R, R)$. Insbesondere gilt
$$f^{(n)}(0) = n! a_n$$
\end{korollar}
\begin{proof}
Die erste Aussage folgt induktiv durch $n$-faches Ableiten, die zweite Aussage durch einsetzen von $0$, wodurch alle nicht-konstanten Terme wegfallen.
\end{proof}
\begin{korollar}{Koeffizientenvergleich}{potreihen_koeffizientenvergleich}
Seien $f(x) = \sum_{n=0}^\infty a_nx^n$ und $g(x) = \sum_{n=0}^\infty b_nx^n$ zwei Potenzreihen mit überlappenden Konvergenzbereichen.

Wenn für alle $x \in (-\varepsilon, \varepsilon) \subseteq \{ \text{Schnitt beider Konvergenzbereiche}\}$ mit $\varepsilon > 0$
$$\sum_{n=0}^\infty a_nx^n =\sum_{n=0}^\infty b_nx^n$$
gilt, dann gilt $a_n = b_n$ für alle $n$.
\end{korollar}
\begin{proof}
Da $f(x) = g(x)$ gilt, müssen die Ableitungen ebenfalls gleich sein:
$$a_n = \frac{f^{(n)}(0)}{n!} = \frac{g^{(n)}(0)}{n!} = b_n$$
für alle $n$.
\end{proof}

\section{Taylorreihen}
Wir haben in Korollar \ref{kor:potreihen_koeffizientenvergleich} gesehen, dass zwei Potenzreihen gleich sind, wenn alle Funktionswerte resp. die Ableitungen bei 0 ausgewertet gleich sind. Die Hoffnung bei der Taylor-Approximation ist nun, dass man eine beliebige Funktion $f: D \subseteq \R \to \C$ so approximieren kann. Beim Approximieren von Potenzreihen $p$ kann man sich das gut vorstellen: Man kann durch Ableiten und Einsetzen $p^{(k)}(0)$ jeweils den $k$-ten Koeffizienten von $p$ ausfindig machen. Macht man das für die ersten $n$ Koeffizienten, so erhält man eine \textbf{n-te Taylor-Approximation}:
$$f(x) = f(0) + f'(0)x + \frac{f''(0)}{2!}x^2+...+\frac{f^{(n)}(0)}{n!}x^n + R_n$$
wobei $f$ natürlich auf $n$-mal ableitbar sein muss (Wir schreiben hierfür $f \in C^n(D)$). Die Stellen bei $x=0$ haben nichts spezielles an sich, weswegen wir die Approximation auch in einem beliebigen \textbf{Entwicklungspunkt} $a \in D$ entwickeln können:
$$f(x) = f(a) + f'(a)(x-a) + \frac{f''(a)}{2!}(x-a)^2+...+\frac{f^{(n)}(a)}{n!}(x-a)^n + R_n(x)$$
Das \textbf{Restglied} $R_n(x)$ bezeichnet den Fehler der Approximation, d.h. falls wir das Restglied nach oben abschätzen können, gibt es uns die Güte der Approximation. Wir schreiben es und schätzen es ab wie folgt:
\begin{align*}
    R_n(x)&=\int_a^x\frac{f^{(n+1)}(t)}{n!}(x-t)^n dt\\
    \abs{R_n(x)}&=\frac{\abs{f^{(n+1)}(\xi)}}{n!}\abs{x-t}^{n+1}\\
        &=\mathcal{O}(x-a)^{n+1}
\end{align*}

wobei für das $\xi$ gilt $a<\xi<x$. Bei einer Fehlerabschätzung wir das $\xi$ so gewählt, damit der Fehler maximal wird und uns die oberste Schranke für den Fehler angibt. Dazu später mehr.

Falls unser $f$ glatt ist, wir schreiben $f \in C^\infty(D)$, können wir die Funktion beliebig oft auf $D$ ableiten und somit die \textbf{Taylorreihe} von $f$ (in $a \in D$) bilden:
$$\sum_{n=0}^\infty \frac{f^{(n)}(a)}{n!}(x-a)^{n}$$
Es stellen sich nun einige Fragen:
\begin{enumerate}[label=(\alph*)]
    \item Konvergiert die Taylorreihe von $f$ überhaupt?
    \item Und falls ja, konvergiert sie auch gegen $f$?
\end{enumerate}
Wir haben darauf folgende, mehr oder weniger zufriedenstellende Antworten:
\begin{enumerate}[label=(\alph*)]
    \item Wenn wir nur die Konvergenz betrachten, können wir die Taylorreihe wie jede andere Potenzreihe auf deren Konvergenzradius untersuchen. Somit konvergiert die Taylorreihe von $f$ für $x \in (a-R,a+R)$, mit dem Konvergenzradius $R$:
    $$R= \frac{1}{\limsup_{n \to \infty}\sqrt[n]{\frac{\abs{f^{(n)}}}{n!}}}$$
    wenn sich der Term unter der Wurzel im $\limsup$ höchstens wie eine geometrische Reihe verhält, d.h. wenn man diesen Term durch zwei Konstanten $\lambda_c, C > 0$ wie folgt nach oben abschätzen kann:
    $$\frac{\abs{f^{(n)}}}{n!} \leq \lambda_c C^n$$
    Es folgt also für den Konvergenzradius $R$ der Taylorreihe:
    $$R \geq \frac{1}{\limsup_{n \to \infty}\sqrt[n]{\lambda_c C^n}} = \frac{1}{C}$$
    Also konvergiert die Taylorreihe sicher für $\abs{x-a} < \frac{1}{C}$
    \item Wir können Gegenbeispiele finden, bei denen die Taylorreihe von einer allgemeinen glatten Funktion nicht zu dieser konvergiert, siehe das Beispiel \ref{ex_taylor_series_counterex} unten.
    \item[(b')] Falls $f$ jedoch durch eine Potenzreihe gegeben ist, dann stimmt die Taylorreihe mit der Potenzreihe überein (siehe Korollar \ref{kor:potreihen_koeffizientenvergleich}).
    \item[(b'')] Falls wir das Restglied mit Konstanten $\lambda_c, C$ wie folgt abschätzen können: $\frac{\abs{f^{(n+1)}}}{n!} \leq \lambda_c C^{n+1}$ mit $x\in D$ \todo{für alle x? für allgemeine $f$?} Daraus folgt, dass das Restglied $Rn(x) \to 0$ nach 0 konvergiert für $x \in (a-\frac{1}{C}, a+\frac{1}{C})$, also konvergiert die Taylorreihe auf für $\abs{x-a} < \frac{1}{C}$
\end{enumerate}


\begin{example}[Gegenbeispiel]\label{ex_taylor_series_counterex}
    Sein $f: \R \to \R$ folgende Funktion:
    
\begin{minipage}[c]{0.4\linewidth}
$$f(x) = \begin{cases}0 & x\leq 0 \\ e^{-\frac{1}{x}} & x>0 \end{cases}$$\\
\end{minipage} 
\hspace{2em}
\begin{minipage}[]{0.4\linewidth}
\begin{tikzpicture}
    \begin{axis}[
        axis lines = left,
        xlabel = $x$,
        ylabel = {$f(x)$},
        width=7cm,height=4cm,
    ]
    \addplot [domain=-1:0.01, samples=5, color=red, style={thick}] {0};
    \addplot [domain=0.01:2, samples=50, color=red, style={thick}] {exp(-1/x)};
    \end{axis}
\end{tikzpicture}
\end{minipage}\hfill

Man kann sich leicht davon überzeugen, dass $f$ eine glatte Funktion ist, denn man erhält durch Ableiten jeweils die Negation der Funktion. Falls wir nun die Taylorreihe von $f$ in $0$ entwickeln wollen, dann werden wir die Nullreihe $0+0x+0x^2+... \neq f(x)$ erhalten.
\end{example}
\begin{example}[Binomische Reihe] Die binomische Reihe von \textsc{Newton} ist für $x \in (-1, 1)$ (also auch für irrationale Zahlen!) und $a \in \C$ wie folgt definiert:

$$f_a(x) = \sum_{n=0}^\infty \binom{a}{n} x^n = (1+x)^a$$

\begin{figure}[hbt!]
    \centering
    \begin{tikzpicture}
    \begin{axis}[
        axis lines = left,
        xlabel = $x$,
        ylabel = {$f(x)$},
        ymax=3, ymin=-0,
        width=11cm,height=7cm, legend columns=3, 
        legend cell align={left},
    ]
    \addplot [domain=-0.8:0.99, samples=150, style={thick}, color=orange] {(1+x)^(-2)};
    \addlegendentry{$a=-2$}
    \addplot [domain=-0.9:0.99, samples=150, style={thick}, color=red] {(1+x)^(-1.5)};
    \addlegendentry{$a=-1.5$}
    \addplot [domain=-0.9:0.99, samples=150, style={thick}, color=purple] {(1+x)^(-1)};
    \addlegendentry{$a=-1$}
    \addplot [domain=-0.9:0.99, samples=150, style={thick}, color=violet] {(1+x)^(-0.5)};
    \addlegendentry{$a=-0.5$}
    \addplot [domain=-0.99:0.99, samples=150, style={thick}, color=blue] {(1+x)^(0)};
    \addlegendentry{$a=0$}
    \addplot [domain=-0.99:0.99, samples=150, style={thick}, color=cyan] {(1+x)^(0.5)};
    \addlegendentry{$a=0.5$}
    \addplot [domain=-0.99:0.99, samples=150, style={thick}, color=lime] {(1+x)^(1)};
    \addlegendentry{$a=1$}
    \addplot [domain=-0.99:0.99, samples=150, style={thick}, color=yellow] {(1+x)^(1.5)};
    \addlegendentry{$a=1.5$}
    \addplot [domain=-0.99:0.8, samples=150, style={thick}, color=yellow] {(1+x)^(2)};
    \addlegendentry{$a=2$}
    \end{axis}
\end{tikzpicture}
\end{figure}

wobei der \textbf{Binomialkoeffizient} $\binom{a}{n}$ wie folgt definiert ist:
$$\binom{a}{n} = \frac{a(a-1)...(a-n+1)}{n(n-1)...1} = \frac{a!}{n!(a-n)!}$$
Es gilt $\binom{a}{0} = 1$. 

Wir wollen nun also zeigen, dass die explizite Form von $f$ und die Reihe für $f$ äquivalente Formulierungen sind für $f: (-1,1) \to \R$:
\begin{proof}
Zuerst bilden wir die Taylorreihe von der expliziten Form von $f$ um 0 und erhalten:
\begin{align*}
    f(x) &= (1+x)^{a}\\
    f'(x) &= a(1+x)^{a-1}\\
    f''(x) &= a(a-1)(1+x)^{a-2}\\
    &...\\
    f^{(n)}(x)&= a(a-1)(a-2)...(a-n)(1+x)^{a-n}\\
    f^{(n)}(0)&= a(a-1)(a-2)...(a-n+1)\\
    \frac{f^{(n)}(0)}{n!} &= \frac{a(a-1)(a-2)...(a-n+1)}{1\cdot2\cdot...\cdot n} \stackrel{def.}{=} \binom{a}{n}
\end{align*}
für $a\in \C$ und $n \in \N$, wobei $n$ den Index des Gliedes der Taylorreihe bezeichnen soll. Wir müssen nun einige Fallunterscheidungen machen. Betrachten wir zuerst den Fall, in welchem $a \in \N$ gilt. Wir erkennen im Zähler, dass einige Glieder gleich $0$ sein können, nämlich genau die, dessen Index grösser gleich $a$ ist. Des Weiteren wissen wir, dass wegen $n \in N$ für $a \in \C \setminus \N$ jeder Faktor des Zählers ungleich 0 sein muss. Wir erhalten also zusammengefasst:
\begin{align}\label{eq_binomial_cases}
    \binom{a}{n} &= \begin{cases} 0 & , a \in \N, a \leq n \\
    \frac{a!}{n!(a-n)!} & , a \in \N, 0 \leq n < a \\
    \frac{a(a-1)(a-2)...(a-n+1)}{1\cdot2\cdot...\cdot n} & , a \in \C \setminus \N \end{cases}
\end{align}

Wir möchten nun die üblichen zwei Fragen beantworten: 
\begin{enumerate}[label=(\alph*)]
    \item Konvergiert die Taylorreihe von $f$?
    \item Und falls ja, konvergiert sie auch gegen $f$?
\end{enumerate}

Wir zeigen also:

(a) für den Fall $a \notin \N$ mit dem Quotientenkriterium\footnote{Der Quotient von zwei aufeinander folgenden Gliedern konvergiert gegen 0.}:
\begin{align}\label{eq_binomial_identity}
    \frac{\binom{a}{n+1}}{\binom{a}{n}} &= \frac{a(a-1)...(a-n+1)(a-n)}{1\cdot2\cdot...\cdot n \cdot (n+1)} \cdot \frac{1\cdot2\cdot...\cdot \nonumber n}{a(a-1)...(a-n+1)}\\
    &= \frac{a-n}{n+1}
\end{align}
also erhalten wir mit $x$ als Variable:
$$\abs{ \frac{\binom{a}{n+1}x^{n+1}}{\binom{a}{n}x^n}} = \frac{\abs{a-n}}{\abs{n+1}}\abs{x} \xrightarrow{n \to \infty} \abs{x}$$
woraus wir mit dem Quotientenkriterium erkennen, dass die Reihe für $\abs{x} < 1$ konvergiert und für $\abs{x} > 1$ divergiert.

Für den Fall $a \in \N$ bricht die Reihe bei Index $n = a$ ab (siehe \ref{eq_binomial_cases}) und ist somit ein endliches Polynom, welches für alle $x$ konvergiert.

(b) Um die Gleichheit zu zeigen, wollen wir einen gängigen Trick verwenden: Wir wollen zeigen, dass $f$ und die Reihe davon die eindeutige Lösung einer Differentialgleichung mit gegebenen Anfangsbedingungen ist. Unsere Behautung ist also folgende:

(Behauptung) $f(x) = (1+x)^a$ ist die eindeutige Funktion, die
\begin{align*}
    (1+x)f'(x) &= af(x) & x &\in (-1,1)\\
    f(0) &= 1
\end{align*}
erfüllt.

Um diese Behauptung zu beweisen, haben wir zwei Eigenschaften zu zeigen: Die Eindeutigkeit der Funktion, die durch die Einschränkungen der Differentialgleichung gegeben wird und dass die Reihe von $f$ die Differentialgleichung ebenfalls erfüllt:

(Eindeutigkeit) Wir können schnell durch Differenzieren und Einsetzen verifizieren, dass $f$ die Eigenschaften der Behauptung erfüllt. Sei nun $g$ eine weitere Funktion, die $(1+x)g'(x) = a\cdot g(x)$ erfüllt. Wir wollen nun den Quotienten aus $f$ und $g$ bilden und diesen ableiten. Intuitiv sollte klar sein, dass der Quotient konstant sein wird, falls $f$ und $g$ ein Vielfaches voneinander sind. Die Ableitung davon wird also 0 sein:
\begin{align*}
    \frac{d}{dx} \Big(\frac{g(x)}{f(x)}\Big) &= \frac{d}{dx} \big(\frac{g(x)}{(1+x)^a}\big)\\
    &=\frac{g'(x)}{(1+x)^a}-\frac{ag(x)}{(1+x)^{a+1}}\\
    &= \frac{(1+x)g'(x)-ag(x))}{(1+x)^{a+1}} = 0\\
    \implies \frac{g(x)}{f(x)} &= const
\end{align*}
Setzen wir die Anfangsbedingung $g(x) = 1$ ein, so ist die Konstante gleich 1, welches $g(x) = f(x)$ impliziert.

(Reihe erfüllt Bedingungen) Es bleibt zu zeigen, dass $g(x) = \sum_{n=0}^\infty \binom{a}{n} x^n$ die Bedingungen erfüllt:
{\allowdisplaybreaks
\begin{align*}
(1+x) g'(x) &= (1+x) \frac{d}{dx} \Big(\sum_{n=0}^\infty \binom{a}{n} x^n\Big)\\
    &= (1+x) \Big(\sum_{n=1}^\infty \binom{a}{n}n x^{n-1}\Big)\\
    &= \sum_{n=1}^\infty \binom{a}{n}n x^{n-1} + \sum_{n=1}^\infty \binom{a}{n}n x^{n}\\
    &= \sum_{n=0}^\infty \binom{a}{n+1}(n+1) x^{n} + \sum_{n=0}^\infty \binom{a}{n}n x^{n}\\
    &= \sum_{n=0}^\infty \Big( \underbrace{\binom{a}{n+1}}_{(\ref{eq_binomial_identity})}(n+1) + \binom{a}{n}n \Big) x^n\\
    &= \sum_{n=0}^\infty \Big( \binom{a}{n}\frac{a-n}{\cancel{n+1}}\cancel{(n+1)} + \binom{a}{n}n \Big) x^n\\
    &= \sum_{n=0}^\infty \Big( \binom{a}{n} (a-n+n) \Big) x^n\\
    &= \sum_{n=0}^\infty a \binom{a}{n} x^n = a \cdot g(x)
\end{align*}}
Da per Definition $g(0) = \binom{a}{0} = 1$ die Anfangsbedingung erfüllt ist, gilt $g(x) = (1+x)^a$.
\end{proof}
\end{example}

    \setcounter{chapter}{8}
\chapter{Metrische Räume}
Wir haben schon des Öfteren den Begriff des ''Abstands'' verwendet, wobei wir uns direkt die Betragsfunktion oder den Pythagoras zunutze gemacht haben. Dies ist jedoch eine von vielen Weisen, einen Abstand mathematisch zu formulieren. 

Wie genau dieser Abstand bestimmt werden kann, hängt von den zugrundeliegenden mathematischen Strukturen ab. Je nach dem, wie viel gegeben ist, können wir eine der folgenden Strukturen betrachten: Die \textbf{Norm}, die \textbf{Metrik} und die \textbf{Topologie}, die eine abstrakter als die vorherige:

\begin{itemize}
    \item Bei \textbf{Normen} halten wir uns in den uns bekannten Vektorräumen auf, in welcher wir die Länge der Vektoren bestimmen können.
    \item Wollen wir nun unabhängig von Vektorräumen Abstände zwischen Elementen definieren, so können wir dies auf der nächsten Abstraktionsebene der \textbf{Metriken} mit einer Abstandsfunktion oder eben \textbf{Metrik} machen.
    \item Das ist Mathematikern aber zu wenig abstrakt, denn was ist, wenn wir keine Ordnung haben, mit der wir die Distanz ausdrücken und vergleichen können? Man kann also zu guter Letzt die Distanz rein mengentheoretisch als \textbf{Topologie} von einer Menge $X$ definieren. Dabei ist die Topologie eine Menge, die \textit{offene} Teilmengen von $X$ besitzt. Sehr hand-wavy ausgedrückt werden Elemente der ''ähnlichen Distanz'' zu Elementen der Topologie gruppiert.
\end{itemize}

Man könnte erkannt haben, dass eine Norm gewissermassen auch eine Metrik und Topologie ist resp. eine Metrik auch eine Topologie ist -- es gibt also eine Hierarchie zwischen den Begriffen:

$$\text{Norm} \xrightarrow{\text{induziert}}\text{Metrik} \xrightarrow{\text{induziert}}\text{Topologie}$$

Falls also eine Eigenschaft für eine zugrundeliegende Struktur gilt, i.e. Metrik, dann gilt es auch für die darüberliegenden, i.e. Norm. 

\section{Norm}
Beginnen wir also mit den uns bereits bekannten Vektorräumen. Wenn wir mit vektoriellen Werten rechnen wollen wir je nach dem eine Funktion haben, die die Länge der Vektoren bestimmen kann. Solche Funktionen nennen wir \textbf{Normen} und sie sollen folgende Eigenschaften besitzen:

\begin{definition}{Norm}{}
Sei $V$ ein Vektorraum über den Körper $K = \R$ oder $\C$. Eine \textbf{Norm} (Länge) auf $V$ ist eine Abbildung
\begin{align*}
    \Nrm: V &\to \R_{\geq 0}\\
    v &\mapsto \norm{v}
\end{align*}
sodass gilt:
\begin{enumerate}[label=N\arabic*)]
    \item $\norm{\lambda x} = \abs{\lambda} \cdot \norm{x}$ für alle $\lambda \in K, x \in V$ \hfill \textit{(Homogenität)}
    \item $\norm{x} = 0 \iff x = 0$ \hfill \textit{(Definitheit)}
    \item $\forall x,y \in K: \norm{x+y} \leq \norm{x} + \norm{y}$ \hfill \textit{(Dreiecksungleichung)}
\end{enumerate}
Einen Vektorraum mit einer Norm $(V, \Nrm)$ nennen wir einen \textbf{normierten Vektorraum}.
\end{definition}

\begin{example} \label{ex_norm_vr}Hier nun einige Beispiele von verschiedenen Normen auf Vektorräumen:
\begin{enumerate}[label=(\alph*)]
    \item Euklidische Norm auf $\R^n, \C^n$: $\norm{x}_2 = \sqrt{\abs{x_1}^2+...+\abs{x_n}^2}$
    \item Maximumsnorm: $\norm{x}_\infty = \max\{\abs{x_1},...,\abs{x_n}\}$
    auf $\R^n, \C^n$. Axiome folgen aus den Axiomen für die Betragsfunktion.
    \item Eins-Norm: $\norm{x}_1 = \sum_{i=1}^n \abs{x_i}$ 
\end{enumerate}
\end{example}

Im Beispiel haben wir bereits die Notation $\Nrm_p$ für Normen verwendet. Das $p \in \N$ spezifiziert dabei, zu welcher Potenz die Norm berechnet wird:
\begin{definition}{$p$-Norm}{}
 Sei $x \in K^n$, dann gilt:
$$\norm{x}_p = \sqrt[p]{x_1^p+x_2^p+...+x_n^p}$$
Für unendlich-dimensionale Vektoren, z.B. Funktionen $f: [a,b] \to \R$ oder $\C$ gilt:
$$\norm{f}_p = \sqrt[p]{\int_a^b\abs{f(x)}^p dx}$$
Der Fall $p=\infty$ entspricht dabei dem Supremum des Koordinateneinträge resp. der Funktion.
\end{definition}
Somit bezeichnet also $p=2$ die uns bekannte euklidische Norm.

Wir können aber auch unendlich-dimensionale Vektorräume betrachten, zum Beispiel \textbf{Funktionenräume}:
\subsection{Funktionenräume}
Wir wollen also die Menge $C([a, b]) = \{\text{Menge aller stetigen Funktionen } f : [a,b] \to \R \text{(oder $\C$)}\}$  mit einem Begriff für Distanz versehen. Dabei können wir, wie wir bei den endlichen Vektorraum-Normen bereits gesehen haben, die Norm unterschiedlich definieren. Wir wollen zwei gängige Varianten für Funktionenenräume einführen:

\begin{definition}{Supremum-Norm}{}
Für $f \in C([a,b])$ und $x \in [a.b]$ ist die $\sup$-Norm definiert als:
\begin{align*}
    \norm{f}_\infty &= \sup\{\abs{f(x)}, x \in [a,b]\} =: \sup\abs{f(x)}\\
    &= \max\abs{f(x)} \geq 0
\end{align*}
Wobei die zweite Gleichheit aus der Eigenschaft von stetigen Funktionen auf kompakten Intervallen folgt.
\end{definition}
Wir können zeigen, dass $\norm{f}_\infty$ auch tatsächlich eine Norm ist:
\begin{enumerate}[label=N\arabic*)]
    \item $\lambda \in \R : \norm{\lambda f}_\infty = \abs{\lambda} \cdot \norm{f}_\infty$ 
    \item $\norm{f} = 0 \iff \max \abs{f(x)} = 0 \iff f=0$
    \item $\norm{f + g}_\infty = \max \abs{f(x) + g(x)} \leq \max\left(\abs{f(x) + g(x)}\right) \leq \max \abs{f(x)} + \max \abs{g(x)}  = \norm{f}_\infty + \norm{g}_\infty$
\end{enumerate}

\subsubsection{Gleichmässige Konvergenz}\label{cha_funktionenraum_glm_konvergenz}
Mit einer Norm können wir auch eine Konvergenz von Funktionen definieren. Was wir also ganz mühsam im Kapitel \ref{cha_funktionenfolgen} definiert haben, ist hier dasselbe Konzept wie das berechnen der Länge von Vektoren:

Sei $(f_n)$ eine Folge im Funktionenraum $C([a,b])$.
$$ f_n \to f \quad (n \to \infty)$$
bedeutet dann
\begin{align*}
    \norm{f_n - f}_\infty &\longrightarrow 0 \quad (n \to \infty)\\
    \max \abs{f_n(x) -f(x)} &\longrightarrow 0 \quad (n \to \infty)
\end{align*}
In der uns vertrauten $\varepsilon$-Terminologie kann man das auch wie folgt ausdrücken:
$$\forall \varepsilon >0\ \exists n_0 \in \N : \forall n\geq n_0, \forall x \in [a,b]: \abs{f_n(x) - f(x)} < \varepsilon$$
Wir sehen also, dass $f_n$ gegen $f$ bezüglich der Norm $\Nrm_\infty$ genau dann konvergiert, wenn $f_n$ gleichmässig gegen $f$ konvergiert. Somit ist es also eine alternative, kompaktere Definition der gleichmässigen Folgenstetigkeit.

Eine andere Norm für Funktionenräume bildet die folgende:
\begin{definition}{Eins-Norm}{}
Für $f \in C([a,b])$ und $x \in [a,b]$ ist die Eins-Norm definiert als:
$$\norm{f}_1 = \int_a^b \abs{f(x)} dx$$
\end{definition}
Es wir also der Betrag der Fläche unter der Funktion auf $[a, b]$ als Norm verwendet. Auch hier können wir zeigen, dass es sich um eine Norm handelt:

\begin{enumerate}[label=N\arabic*)]
    \item $\norm{\lambda f}_1 = \int_a^b \abs{\lambda f(x)}dx = \int_a^b \abs{\lambda} \cdot \abs{f(x)}dx = \abs{\lambda} \cdot \norm{f}_1$ 
    \item $\norm{f}_1 = \int_a^b \abs{f(x)}dx = 0 \iff f=0$
    \item $\norm{f + g}_1 = \int_a^b \abs{f(x) + g(x)}dx \leq \int_a^b \left(\abs{f(x)} + \abs{g(x)}\right) dx = \int_a^b\abs{f(x)}dx + \int_a^b \abs{g(x)}dx = \norm{f}_1 + \norm{g}_1$
\end{enumerate}

\subsubsection{Konvergenz im Mittel}
Die Konvergenz, die durch die Eins-Norm definiert wird, nennen wir die \textbf{Konvergenz im Mittel}:
\begin{definition}{Konvergenz im Mittel}{}
Sei $(f_n)$ eine Folge im Funktionenraum $C([a,b])$. Bei der Konvergenz im Mittel gilt:
$$f_n \to f  \quad (n \to \infty)\iff \int_a^b\abs{f_n - f(x)} dx \to 0 \quad (n \to \infty)$$
\end{definition}
Diese Konvergenz ist weniger einschränkend als die gleichmässige Konvergenz von der $\sup$-Norm und folgt auch aus dieser:
$$\text{gleichmässige Konvergenz} \implies \text{Konvergenz im Mittel}$$
da mit $\norm{f_n - f}_\infty < \frac{ \varepsilon}{b-a}$ von der gleichmässigen Konvergenz folgende Aussage gilt:
$$\varepsilon > (b-a) \norm{f_n - f}_\infty \geq \int_a^b \abs{f_n(x) -f(x)}dx = \norm{f_n-f}_1$$
$\norm{f_n - f}$ bezeichnet per Definition dieser Norm die grösste Differenz zwischen $f_n$ und $f$, somit ist die Differenz des Integrals auf dem gesamten Intervall höchstens $(b-a)$-mal so gross wie die grösste Differenz $\norm{f_n - f}$; daher die entsprechene Wahl des $\varepsilon$.

Auch hier sehen wir, dass $f_n$ gegen $f$ bezüglich der Norm $\Nrm_1$ genau dann konvergiert, wenn $f_n$ punktweise gegen $f$ konvergiert. Somit ist es also eine alternative, kompaktere Definition der punktweisen Folgenstetigkeit.

Das Beispiel \ref{ex_grenzfunktion} konvergiert daher im Mittel, jedoch wie bereits gezeigt, nicht gleichmässig. Intuitiv lässt sich das daran erkennen, dass die Fläche unter der Funktion beliebig klein gemacht werden kann.

\subsection{Äquivalenz von Normen}\label{cha_eq_norms}
Wir haben gesehen, dass wir verschiedene Normen definieren können, die an der Oberfläche in unterschiedlichen Konvergenzverhalten äussert. Wir können aber zeigen, dass von einigen Normen die zugrundeliegende Mechanismus der Konvergenz derselbe ist. Wir wollen daher die folgende Äquivalenz zwischen Normen definieren:
\begin{definition}{}{}
Zwei Normen $\Nrm, \Nrm'$ auf $V$ heissen \textbf{äquivalent} $\Nrm \sim \Nrm'$, falls es Zahlen $c_1, c_2 > 0$ gibt, so dass
$$c_1 \norm{x}' \leq \norm{x} \leq c_2 \norm{x}'$$
für alle $x \in V$ gilt.
\end{definition}
Wir können zeigen, dass es sich bei $\sim$ auch um eine Äquivalenzrelation handelt:
\begin{exercise}[Äquivalenzrelation] $\sim$ ist eine Äquivalenzrelation auf der Menge der Normen auf $V$. \todo{Beweis einfügen}
\end{exercise}

\begin{example} Hier einige bereits bekannte Normen im Vektorraum $\R^n$, welche äquivalent zueinander sind:
\begin{itemize}
    \item $\Nrm_2 \sim \Nrm_\infty$ (Es gilt mit $x_i = \max \{x_1, ..., x_n\} = \norm{x}_\infty $: $\norm{x}_\infty \leq \sqrt{x_1^2 +...+x_n^2} \leq \sqrt{n} \norm{x}_\infty$)
    \item $\Nrm_1 \sim \Nrm_\infty$ (Es gilt: $\norm{x}_\infty \leq \abs{x_1}+...+\abs{x_n} \leq n \norm{x}_\infty$)
\end{itemize}

\end{example}

Follow-up: Wir werden im  Abschnitt \ref{cha_eq_norms_topology} zeigen, dass eine solche Äquivalenz auf dieselbe Topologie zurückzuführen ist.

\section{Metrik}
Nun wollen wir den ersten Schritt der Abstraktion wagen und uns vom Konzept der Vektorräume entfernen. Wir wollen stattdessen eine allgemeinere Abstandsfunktion einführen, welche die Elemente einer Menge mit folgenden Eigenschaften auf $\R$ (oder einen anderen geordneten Körper) abbildet:
\begin{definition}{Metrik}{}
Sei $X$ eine Menge, dann ist eine \textbf{Metrik} eine Abbildung
$$d: X\times X \to \R_{\geq 0}$$
sodass Folgende Axiome erfüllt sind:
\begin{enumerate}[label=M\arabic*)]
    \item Für alle $x,y \in X : d(x,y) = d(y,x)$ \hfill \textit{(Symmetrie)}
    \item $d(x,y) = 0 \iff x=y$
    \item $\forall x,y,z \in X: d(x,z) \leq d(x,y) + d(y,z)$ \hfill \textit{(Dreiecksungleichung)}
\end{enumerate}
Eine Menge $X$ die mit einer Metrik $d$ versehen ist, nennen wir \textbf{metrischen Raum} und notieren dies als Tupel: $(X, d)$
\end{definition}

Man erkennt sofort nur schon durch Vergleichen der Axiome von Normen und Metriken, dass jede Norm auch eine Metrik ist. Deswegen werden uns einige der folgenden Beispiele auch sehr bekannt vorkommen:

\begin{example}[Euklidische Metrik] Die uns bekannte Metrik ist die \textbf{Euklidische Metrik}. Für die Menge $X = \R^n$ mit ist die Abstandsfunktion $d$ definiert als:
$$d_E(x,y) = \sqrt{(x_1-y_1)^2+...(x_n-y_n)^2}$$
wobei $x = (x_1, x_2, ..., x_n) \in \R^n$ und $y = (y_1, y_2, ..., y_n) \in \R^n$ gilt.
\end{example}

\begin{example}[Manhattan-Metrik] Befindet man sich in einer Stadt mit schachbrettartig angeordneten Blöcken, i.e. Manhattan, kann man sich nur in orthogonal zueinander liegende Richtungen bewegen. Man enthält also folgende Metrik für die zurückgelegte Distanz (mit $X = \R^n$ und $x,y \in X$):
$$d_M(x,y) = \abs{x_1 - y_1} + ... + \abs{x_n - y_n}$$
M1 und M2 kann man selber schnell verifizieren, wir zeigen M3:
$$d_M(x,z) = \sum_{i=1}^n \abs{x_i-z_i} \leq \sum_{i=1}^n \abs{x_i-y_i} + \abs{y_i-z_i} = d_M(x,y) + d_M(y,z)$$

\end{example}

\begin{example}[Französische Eisenbahnmetrik] Im Frankreich der Industrialisierung gab es vor allem Eisenbahnlinien, welche direkt aus der Agglomeration ins Zentrum Paris führten, i.e. waren radial mit Paris als Zentrum. Wollte man von einer Station zur anderen, musste man in Paris umsteigen --- es sei denn, die Endstation liegt genau auf der selben radialen Bahnlinie, dann ist der Weg kürzer. Wir können diese Metrik daher als Metrik von $X = \C$ mit $z,w \in \C$ definieren:
$$d_{SNCF}(z,w) = \begin{cases} \abs{z-w} &, \text{falls $z,w$ auf derselben Gerade durch 0 liegen} \\ \abs{z} + \abs{w} & , \text{sonst} \end{cases}$$
\end{example}

\begin{example}[Bogenlänge auf Kugel] Sei $X = \set{(x,y,z) \in \R^3}{x^2+y^2+z^2=1}$ die Einheitskugel im $\R^3$. Wir definieren Metrik wie folgt:
$$d_K(a,b) = \text{Länge eines kürzesten Grosskreisbogen zwischen $a$ und $b$}$$
wobei der Grosskreisbogen von $a$ und $b$ die Bogenlänge des Kreises ist, der durch den Schnitt von $X$ und der Ebene durch $a,b$ und 0 ist:
$$d_K(a,b) = \arccos{a\cdot b}$$
\end{example}
\begin{exercise} Verifiziere M1 -- M3 für die $d_{SNCF}$ und für $d_K$.\end{exercise}

\subsection{Teilraum}
Wir haben bis jetzt stets Räume direkt mit einer Metrik definiert, jedoch können wir auch einen metrischen Raum auf einen \textbf{Teilraum} einschränken, wobei die Metrik selber auf diese Teilmenge ''vererbt'' wird:

\begin{definition}{Teilraum}{}
Ein \textbf{Teilraum} oder \textbf{Unterraum}\footnote{Nicht zu verwechseln mit dem Untervektorraum der linearen Algebra.} von einem metrischen Raum $(X,d)$ ist eine Teilmenge $Y \subseteq X$ mit der Metrik
$$d\vert_{Y\times Y}: Y \times Y \to \R_{\geq 0}$$
\end{definition}

\begin{example}[Euklidische Metrik auf Kreis] Sei die Menge $S^1 \subseteq \R^2$, $S^1 = \set{(x,y)}{x^2+y^2=1}$ der Einheitskreis mit der Euklidischen Metrik aus dem $\R^2$. Dann bezeichnet $d(x,y)$ mit $x,y \in S^1$ die Länge der Sehne zwischen $a$ und $b$.
\end{example}

Wir werden hauptsächlich Fälle betrachten, welche Teilräume von den normierten Vektorräumen $\R^n$ oder $\C^n$ mit der euklidischen Norm/Metrik sind. Zwar sind diese Teilmengen selber nicht unbedingt Vektorräume, jedoch kann man die Metrik des Vektorraums darauf verwenden.

\subsection{Induzierte Metrik}

Wie wir bereits angedeutet haben, induziert eine Norm stets eine Metrik. Dies bedeutet also, dass man mit der Norm eines Vektorraums auch eine Metrik auf dem Raum definieren kann:

\begin{satz}{Induzierte Metrik}{}
Sei $(V, \norm{\ \cdot \ })$ ein normierter Vektorraum. Dann definiert
$$d(x,y) = \norm{x - y}$$
mit $x,y \in V$ eine Metrik auf $V$.
\end{satz}
Wir wollen nun verifizieren, dass eine Norm resp. normierte Vektorräume die Axiome der Metrik M1 -- M3 erfüllen: 
\begin{proof} Sei $d: X \times X \to \R_{\geq 0}$

(M1) Es gilt: 
$$d(y,x) = \norm{y-x} = \norm{(-1)(x-y)} \stackrel{N1}{=} \underbrace{\abs{-1}}_{1} \norm{x-y} = d(x,y)$$
(M2) Es gilt:
$$d(x,y) = 0 \iff \norm{x-y}=0 \stackrel{N2}{\iff} x-y = 0 \iff x= y$$
(M3) Es gilt:
$$d(x,z) = \norm{x-z} = \norm{x-y+y-z} \stackrel{N3}{\leq} \norm{x-y} + \norm{y-z} = d(x,y) + d(y,z)$$
\end{proof}

\begin{lemma}{Translationsinvarianz}{}
Sei $\Nrm$ eine Norm vom Vektorraum $V$, dann ist die induzierte Metrik $d_{\Nrm}$ translationsinvariant:
$$\forall a \in V: d_{\Nrm}(x,y)=d_{\Nrm}(x+a,y+a)$$
\end{lemma}
\begin{proof}
$d_{\Nrm}(x,y) = \norm{x-y} = \norm{(x+a)-(y+a)}=d_{\Nrm}(x+a,y+a)$
\end{proof}

\begin{remark}
Wir erkennen, dass die Manhattan-Metrik $d_M$ durch die Eins-Norm induziert wird, siehe Beispiel \ref{ex_norm_vr} c). Hingegen wird die Französische Eisenbahnmetrik durch keine Norm induziert, denn sie erfüllt im Allgemeinen nicht die Translationsinvarianz.
\end{remark}

\subsection{Konvergenz}
Wir haben die Konvergenz bereits in beliebigen Mengen definiert, nun wollen wir aber allgemeiner den Begriff der Konvergenz für beliebige metrische Räume einführen, um z.B. auch Konvergenz in Funktionenräumen oder in höherdimensionalen Vektorräumen betrachten zu können. Dabei werden uns sehr viele Begriffe bekannt vorkommen, da diese schon sehr ähnlich im Rahmen der bereits besprochenen Konvergenz definiert wurde.

\begin{definition}{Konvergenz im metrischen Raum}{}
Sei $(X, d)$ ein metrischer Raum und $(x_n)_n$ eine Folge/Abbildung von $n \in \N \to x_n \in X$. Die Folge $(x_n)_n$ konvergiert gegen $x \in X$, falls es zu jedem $\varepsilon > 0$ ein $n_0 \in \N$ gibt, sodass
$$\forall n \geq n_0: d(x_n,x) < \varepsilon$$
Wir nennen $x$ den \textbf{Grenzwert} und schreiben
$$x = \lim_{n \to \infty} x_n \qquad \text{oder} \qquad x_n \to x \quad (n \to \infty)$$
\end{definition}

\begin{remark}
Dabei gilt $x = \lim_{n \to \infty}x_n$ genau dann, wenn $\lim_{n \to \infty} d(x_n,x) = 0$ gilt.
\end{remark}

Wie bei den uns bereits bekannten Grenzwerten, sind diese Grenzwerte ebenfalls eindeutig:
\begin{lemma}{Eindeutigkeit des Grenzwertes}{}
Der Grenzwert einer Folge ist eindeutig.
\end{lemma}

\begin{proof}
Im Grunde ist das derselbe Beweis wie für Grenzwerte in $\R$ oder $\C$: Seien $x\neq y$ beides Grenzwerte der Folge $(x_n)$, dann gilt für ein $\varepsilon > 0$ und ein $n \in \N$ gross genug sowohl $d(x_n,x) < \frac{\varepsilon}{2}$ als auch $d(x_n,y) < \frac{\varepsilon}{2}$. Dies ist aber ein Widerspruch mit der Dreieckungleichung, wenn wir $\varepsilon = d(x,y) > 0$ wählen:
$$\varepsilon = d(x,y) \leq d(x_n, x) + d(x_n, y) < \frac{\varepsilon}{2} + \frac{\varepsilon}{2} = \varepsilon$$
\end{proof}

Wir werden in folgendem Beispiel sehen, dass die Wahl der Norm einen wesentlichen Einfluss auf den Begriff der Konvergenz hat:

\begin{example}[Normen und Konvergenzen] Wir betrachten die Folge $(e^{i\frac{1}{n}})_n$ in $\C$, die eine Folge auf dem Einheitskreis bezeichnet, die sich der $1$ nähert. Mit der euklidischen Norm $\Nrm_2$ erhalten wir als Grenzwert:
$$\lim_{n \to \infty} e^{i\frac{1}{n}} = 1$$
da $\abs{e^{i\frac{1}{n}} - 1} = \sqrt{(\cos \frac{1}{n} - 1)^2 + \sin^2\frac{1}{n}} \to 0$ gilt. 

Für die Metrik $d_{SNCF}$ jedoch konvergiert diese Folge nicht, da immer $d_{SNCF}(e^{i\frac{1}{n}}, 1) = 2$ gilt.
\end{example}

\section{Topologie von metrischen Räumen}
Nun wollen wir uns der letzten Abstraktionsebene, der \textbf{Topologien}, widmen. Wir werden dabei --- wie der Titel des Abschnittes schon vermuten lässt --- immer wieder auf Konzepte von metrischen Räumen zurückgreifen. Es wäre aber möglich, all diese topologischen Begrifflichkeiten ausschliesslich mengentheoretisch zu definieren. 

Zuerst wollen wir jedoch noch die Grundbegriffe der \textbf{Umgebung} und der \textbf{Offenheit von Mengen} einführen:

\subsection{Umgebungen}
\begin{definition}{Umgebung}{}
Sei $(X,d)$ ein metrischer Raum und $\varepsilon > 0$. Die \textbf{$\varepsilon$-Umgebung} von $x \in X$ ist
$$B_\varepsilon(x) = \set{y \in X}{d(x,y)<\varepsilon}$$
Wir haben diese Umgebung auch als \textbf{offenen Ball} von Radius $\varepsilon$ um $x$ bezeichnet. 

Die \textbf{Umgebung} [neighbourhood] von $x \in X$ ist eine Teilmenge von $X$, die für ein $\varepsilon > 0$ eine $\varepsilon$-Umgebung enthält. 
\end{definition}
Dabei kommt es bei Umgebungen nicht darauf an, ob der Rand in der Menge ist oder nicht.

\begin{example} Einige Beispiele zu Umgebungen verschiedener Metriken:
\begin{itemize}
    \item $X=\R, B_\varepsilon(x) = (x-\varepsilon, x+\varepsilon)$
    \item $X=\R^2$ mit euklidischer Metrik. $B_\varepsilon(x)$ bezeichnet die offene Kreisscheibe mit Radius $\varepsilon$ um $x$
    \item $X=\R^2$ mit Manhattan-Metrik. $$B_\varepsilon(x) = \{y \in \R^2 \text{ mit } \norm{y-x}_1 = \abs{y_1-x_1} + \abs{y_2-x_2} < \varepsilon\}$$ bezeichnet einen offenen Rhombus mit Breite und Höhe $2\varepsilon$ um den Punkt $x$.
    \item $X = \R^2$ mit $\sup$-Metrik. $$B_\varepsilon(x) = \{y \in \R^2\text{ mit }\norm{y-x}_\infty = \max\{\abs{y_1-x_1}, \abs{y_2-x_2} < \varepsilon\}$$ bezeichnet ein Quadrat mit Seitenlänge $2\varepsilon$ um den Punkt $x$.
\end{itemize}
\end{example}

\subsection{Offene Mengen}\label{cha_offenheit}
Nun wollen wir die Begriffe der Offenheit in metrischen Räumen definieren:

\begin{definition}{Offene Teilmenge}{}
Sie $(X, d)$ ein metrischer Raum, dann ist $U \subseteq X$ \textbf{offen}, falls für alle $x \in U$ eine $\varepsilon$-Umgebung $B_\varepsilon(x)$ mit $\varepsilon >0$ existiert, sodass $B_\varepsilon(x) \subseteq U$ gilt.
\end{definition}
Ein einfaches Beispiel wäre die Teilmenge von $\varepsilon$-Umgebungen. Es folgen daraus einige Lemmata:

\begin{lemma}{Folgerungen}{folgerungen_offenheit}
\begin{enumerate}
    \item $\emptyset, X$ sind offen.
    \item endliche Durchschnitte und beliebige Vereinigungen von offenen Mengen sind offen:
    \begin{enumerate}[label=(\alph*)]
        \item Seien $U_1, ... , U_n$ offen in $X$, dann ist der endliche Schnitt $U_1 \cap ... \cap U_n $ offen.
        \item Sei $(U_i)_{i \in I}$ eine Familie von offenen Mengen, dann ist die Vereinigung offen.
        $$\bigcup_{i\in I}U_i = \set{x \in X}{\exists i:x \in U_i}$$ 
    \end{enumerate}
\end{enumerate}
\end{lemma}
\begin{proof}
\begin{enumerate}
    \item $\emptyset$ ist offen, da jede Aussage für die Elemente der leeren Menge gilt. $X$ ist offen, da jede $\varepsilon$-Umgebung (auch auf dem Rand) per Definition eine Teilmenge von $X$ sein muss, somit ist $X$ auch offen.
    \item \begin{enumerate}[label=(\alph*)]
        \item Seien $U_1, ... , U_n$ offen und $x \in U_1 \cap ... \cap U_n $. Da $U_i$ offen ist, existiert ein $B_{\varepsilon_i}(x) \subseteq U_i$ mit $\varepsilon_i > 0$. Wähle $\varepsilon = \min\{\varepsilon_1, ..., \varepsilon_n\} > 0$, dann gilt $B_\varepsilon(x) \subseteq B_{\varepsilon_i} \subseteq U_i$ für alle $0 \leq i \leq n$. Es gilt $B_\varepsilon \subseteq U_1 \cap ... \cap U_n$, also ist der Schnitt offen.
        \item Für ein $x \in \bigcup_{i\in I}U_i$ finden wir ein $i \in I$, sodass $x \in U_i$ gilt. Wir können also ein $\varepsilon >0$ finden, sodass $B_\varepsilon(x) \subseteq U_i \subseteq \bigcup_{i\in I}U_i$ gilt. Also ist die Vereinigung offen.
    \end{enumerate}
\end{enumerate}
\end{proof}

\begin{remark}
Beim Schnitt von offenen Mengen ist im Allgemeinen die endliche Anzahl Mengen von Wichtigkeit. Dies widerspiegelt sich auch im Beweis, da wir das Minimum für das $\varepsilon$ bestimmen müssen. Bei einem unendlichen Schnitt müssen wir je nach dem das Infimum wählen, wodurch der Schnitt nicht mehr offen ist:
\end{remark}

\begin{example}[unendlicher Schnitt] Sei $X = \R$ und $U_n = (-\frac{1}{n}, \frac{1}{n})$, dann ist der Schnitt aller Mengen nicht offen:
$$\bigcap_{n \in \N} U_n = \{0\}$$
\end{example}

\begin{remark} Des Weiteren ist die Offenheit einer Menge im Allgemeinen abhängig von der Wahl der Metrik:
\begin{example}
Die offene Menge $B_1(1)$ von $(\C, d_{SNCF})$ ist das offene Intervall $(0,2)$. Diese Menge ist jedoch nicht offen bezüglich $(\C, d_E)$, da jeder $\varepsilon$-Schritt entlang der imaginären Achse nicht mehr in der Menge liegt.
\end{example}
Wir werden aber im Abschnitt \ref{cha_eq_norms_topology} sehen, dass äquivalente Normen dieselbe Offenheit beschreiben.
\end{remark}

\subsection{Topologie}
Mit der Offenheit von Mengen können wir nun den Begriff der \textbf{Topologie} definieren:

\begin{definition}{Topologie}{}
Die Menge $\mathcal{U}$ aller offenen Teilmengen eines metrischen Raums $(X,d)$ heisst \textbf{Topologie} von $(X,d)$.

$X$ versehen mit einer Topologie heisst \textbf{topologischer Raum}.
\end{definition}
\begin{remark}
Unabhängig von metrischen Räumen definiert heisst eine Menge $\mathcal{U}$ von Teilmengen einer Menge $X$ \textbf{Topologie}, falls sie die Eigenschaften 1. und 2. des Lemmas \ref{lem:folgerungen_offenheit} zur Offenheit von Mengen erfüllt. Auf diese Weise definierte Topologien müssen nicht unbedingt durch eine Metrik induziert sein, so sind die triviale und die \textsc{Zariski}-Topologie Beispiele für Topologien, denen keine Metrik zugrunde liegt.
\end{remark}

\subsection{Äquivalente Normen und Topologien}\label{cha_eq_norms_topology}
Wir haben im Abschnitt \ref{cha_eq_norms} den Begriff der äquivalenten Normen eingeführt. Nun werden wir zeigen, dass äquivalente Normen auf dieselbe zugrundeliegende Topologie zurückzuführen sind resp. dieselbe Topologie induzieren/definieren:
\begin{lemma}{Äquivalente Normen und Topologien}{}
Äquivalente Normen definierten dieselbe Topologie. Seien $\Nrm, \Nrm'$ äquivalente Normen auf einem Vektorraum $V$, dann gilt:

\centering $U$ offen bezüglich $\Nrm \iff U $ offen bezüglich $\Nrm'$
\end{lemma}

\begin{proof} Wir wollen zeigen, dass offene Mengen bezüglich einer Norm $\Nrm$ auch offen bezüglich einer anderen, äquivalenten Norm $\Nrm'$ sind. Hierfür wollen wir folgende Notation für $\varepsilon$-Umgebungen verschiedener Normen um einen Punkt $x \in V$ verwenden:
\begin{align*}
    B_\varepsilon(x) &= \set{y \in V}{\norm{x-y} < \varepsilon}\\
    B'_\varepsilon(x) &= \set{y \in V}{\norm{x-y}' < \varepsilon}
\end{align*}
Des Weiteren gilt aufgrund der Äquivalenz zwischen $\Nrm$ und $\Nrm'$, dass es zwei Konstanten $c_1$ und $c_2$ gibt, mit welchen für alle $v \in V$ gilt:
$$c_1 \norm{v}' \leq \norm{v} \leq c_2 \norm{v}'$$
Sei also $y \in B'_\varepsilon(x)$, dann folgt aus der Äquivalenz: $\norm{x-y} \leq c_2\norm{x-y}' < c_2 \varepsilon$. Wir erhalten die Bedingung $y \in B_{c_2 \varepsilon}(x)$, also folgt: $B'_\varepsilon(x) \subseteq B_{c_2 \varepsilon}(x)$. Analog dazu folgt aus $y \in B_\varepsilon(x)$ und der Äquivalenz $c_1\norm{x-y}' \geq \norm{x-y} < \varepsilon$, also erhalten wir durch Division mit $c_1$ die Bedingung für $y \in B'_{\frac{\varepsilon}{c_1}}$, also $B_\varepsilon \subseteq B'_{\frac{\varepsilon}{c_1}}$

Sei nun $U\subseteq V$ eine offene Menge bezüglich $\Nrm$, dann gilt für ein $x \in U$:
$$\exists B_\varepsilon(x) \subseteq U \implies B'_{\frac{\varepsilon}{c_2}} \subseteq B_\varepsilon(x) \subseteq U$$
Da aber $x$ ein beliebiges Element aus U ist, ist U auch offen bezüglich $\Nrm'$. Analog dazu ist eine offene Menge $U'\subseteq V$ bezüglich $\Nrm'$ offen bezüglich $\Nrm$. Also definieren äquivalente Normen dieselbe Topologie.
\end{proof}

\subsection{Konvergenz auf Topologien}
Auch auf der Stufe der Topologie können wir wieder den Begriff der Konvergenz definieren --- und das komplett unabhängig von der Metrik oder der Norm. Wir wollen also folgenden Satz zeigen:

\begin{satz}{Konvergenz auf Topologien}{}
Eine Folge $(x_n)_n$ in einem metrischen Raum $(X, d)$ konvergiert gegen $x \in X$ genau dann, wenn für alle offenen Teilmengen $U$, die $x$ enthalten, ein $n_0 \in \N$ existiert, sodass alle Folgeglieder $x_n$ mit $n\geq n_0$ in $U$ liegen:
$$\lim_{n \to \infty} x_n = x \iff \forall U \ni x, U \text{ offen}: \exists n_0 \in \N: \forall n \geq n_0 : x_n \in U$$
\end{satz}
\begin{proof}
($\Longrightarrow$) Die Folge $(x_n)$ konvergiert gegen $x$, also gibt es für jedes $\varepsilon >0$ ein $n_0$, sodass für alle Folgeglieder $x_n$ mit $n \geq n_0$ gilt: $d(x_n, x) < \varepsilon$ oder mit der Umgebung formuliert: $x_n \in B_\varepsilon(x)$. Sei nun die Teilmenge $U$, welche $x$ enthält, offen. Somit finden wir auch eine $\varepsilon$-Umgebung $B_\varepsilon(x)$, sodass für alle $n\geq n_0$ gilt: $x_n \in B_\varepsilon(x) \subseteq U$.

($\Longleftarrow$) Wähle für ein $\varepsilon > 0$ die Umgebung $U = B_\varepsilon(x)$.
\end{proof}

\subsection{Abgeschlossene Mengen}
Mit der Offenheit einer Menge können wir auch die \textbf{Abgeschlossenheit} definieren:
\begin{definition}{Abgeschlossenheit}{}
Die Teilmenge $A \subseteq X$ eines metrischen Raumes $(X,d)$ ist abgeschlossen, wenn das Komplement $X\setminus A = A^C$ offen ist.
\end{definition}
\begin{example}Einige Beispiele mit der euklidischen Metrik im $\R^2$:
\begin{itemize}
    \item $(0,1) \times (0,1)$ ist offen und nicht abgeschlossen.
    \item $(0,1) \times (0,1]$ ist weder offen noch abgeschlossen.
    \item $[0,1] \times [0,1]$ ist nicht offen aber abgeschlossen.
\end{itemize}
\begin{remark}
Es ist zu beachten, dass dies eine ausschliesslich durch die Offenheit des Komplements bestimmte Eigenschaft ist. Es gibt keine Implikation \textit{''$A$ offen $\implies A$ nicht abgeschlossen''} oder dergleichen. Von der Offenheit der Menge selber kann man also nicht auf deren Abgeschlossenheit schliessen und umgekehrt.
\end{remark}
\end{example}

Mit dem Begriff der Konvergenz können wir für die Abgeschlossenheit folgenden Satz formulieren:
\begin{satz}{Grenzwert in abgeschlossenen Mengen}{grenzwert_abgeschlossenen_mengen}
Eine Teilmenge $A \subseteq X$ ist genau dann abgeschlossen, wenn für alle konvergenten Folgen $(x_n)_n$ deren Folgenglieder $x_n$ alle in $A$ liegen, auch der Grenzwert in $A$ liegt:
$$A \text{ abgeschlossen} \iff \forall (x_n)_n \land \set{x_n}{n\in \N} \subseteq A: \lim_{n \to \infty} x_n \in A $$
\end{satz}
Kurz für die Intuition: Wir wissen bereits, dass Folgen in einem offen Intervall, z.B. $x_n = \frac{1}{n} \in (0,2)$, zu einem Grenzwert ausserhalb des Intervalls konvergieren können, im Fall vom Beispiel $\lim_{n \to \infty}x_n = 0 \notin (0,2)$. Wenn das Intervall aber abgeschlossen ist, ist der Rand bereits dabei und man kann eine Grenzwerte ausserhalb davon erreichen. Hier der formale Beweis:
\begin{proof}
Sei $A$ abgeschlossen resp. das Komplement $A^C$ offen und $(x_n)_n$ eine konvergente Folge mit $x_n \in A$ für alle $n$ mit Grenzwert $\lim_{n \to \infty}x_n = x$.

($\Longrightarrow$) Wir nehmen an, dass $x \in A^C$ und führen es zu einem Widerspruch: Da $x \in A^C$ ist und $A^C$ offen ist, kann man ein $n$ finden, sodass $x_n \in A^C$ liegt, also ein Widerspruch zur Annahme.

($\Longleftarrow$) Wir nehmen an, dass $A^C$ nicht offen ist. Somit gibt es ein $x \in A^C$ (auf dem Rand), sodass der Schnitt $A \cap B_{\frac{1}{n}}(x)$ für alle $n \in \N$ nicht-leer ist (d.h. nicht, dass $A^C$ abgeschlossen ist! Falls unklar, siehe Definition von offenen Mengen \ref{cha_offenheit}). Wir konstruieren nun die Reihe mit $x_n \in A \cap B_{\frac{1}{n}}(x)$, somit sind alle Folgenglieder $x_n \in A$, jedoch ist der Grenzwert per Konstruktion $x \in A^C$, also ein Widerspruch.
\end{proof}

\subsection{Innere Punkte, Randpunkte und Abschluss}
Wir wollen nun die Punkte einer Menge anhand ihrer Umgebung charakterisieren:

\begin{definition}{Innerer Punkt, Inneres und Abschluss}{}
Sei $Y \subseteq X$ eine Teilmenge des metrischen Raums $X$. Dann heisst $y \in Y$ \textbf{innerer Punkt} von $Y$, falls $y$ eine komplett in $Y$ enthaltene Umgebung besitzt:
$$Y \text{ abgeschlossen} \iff \forall y \in Y: \exists\varepsilon>0: B_\varepsilon(y) \subseteq Y$$
Das \textbf{Innere} [interior] von $Y$ bezeichnen wir als die Menge aller inneren Punkte von $Y$ und schreiben
$$\mathring{Y} = \{\text{innere Punkte von Y}\}$$
Den \textbf{Abschluss} [closure] von $Y$ bezeichnen wir als die Menge aller Grenzwerte von Folgen in $Y$
$$\quer{Y} = \{\text{Grenzwerte von Folgen in }Y\}$$
\end{definition}
\begin{lemma}{Folgerungen}{}
Wir können zeigen, dass folgende Behauptungen gelten:
\begin{enumerate}[label=\alph*)]
    \item $\mathring{Y}$ ist offen
    \item $\quer{Y}$ ist abgeschlossen
    \item Für alle offenen Mengen $A \subseteq X$ gilt: $A \subseteq Y \iff A \subseteq \mathring{Y}$
    \item Für alle abgeschlossenen Mengen $A \subseteq X$ gilt: $A \supseteq Y \iff A \supseteq \quer{Y}$
\end{enumerate}
\end{lemma}
\begin{proof}
\begin{enumerate}[label=\alph*)]
    \item Folgt direkt aus der Definition von inneren Punkten.
    \item Sei $(x_n)_n$ eine konvergente Folge in $\quer{Y}$, also gilt für alle Folgenglieder $x_n \in \quer{Y}$. Wir wollen zeigen, dass der Grenzwert $\lim_{n \to \infty}x_n = x$ in $\quer{Y}$ liegt. Wir können aufgrund der Konstruktion des Abschlusses $\quer{Y}$ für jedes $n$ ein $y_n \in Y$ finden, sodass $d(x_n, y_n) < \frac{1}{n}$ gilt. Die Folge $(y_n)_n$ liegt in $Y$ und konvergiert ebenfalls gegen $x$:
    $$d(x,y_n) \stackrel{DU}{\leq} \underbrace{d(x,x_n)}_{\to 0} + \underbrace{d(x_n + y_n)}_{<\frac{1}{n} \to 0}$$
    Da $(y_n)_n$ nach $x$ konvergiert, giltju per Definition des Abschlusses $x \in \quer{Y}$.\footnote{Beachte, dass wir Satz \ref{satz:grenzwert_abgeschlossenen_mengen} für diese Behauptung nicht direkt anwenden können, da sich a priori nicht von der Menge der Folgenglieder $Y$ auf die Menge der Grenzwerte $\quer{Y}$ schliessen lässt.}
    \item Übung \todo{...}
    \item ($\Longrightarrow$) Falls $A$ abgeschlossen ist, dann gilt nach Satz \ref{satz:grenzwert_abgeschlossenen_mengen}, dass jede konvergente Folge in $Y$ auch in $A$ liegt, also liegt auch der Abschluss $\quer{Y}$ in $A$, es folgt $A \supseteq \quer{Y}$.
    
    ($\Longleftarrow$) Für jedes $x \in Y$ lässt sich die Trivialfolge $(x,x,x...)$ bilden, welche den Grenzwert $\lim_{n \to \infty} x_n = x$ besitzt, somit gilt $x \in \quer{Y}$, also $Y \subseteq \quer{Y}$. Es folgt $Y \subseteq \quer{Y} \subseteq A$, also $Y \supseteq A$.
\end{enumerate}
\end{proof}
\begin{definition}{Rand}{}ez
Der \textbf{Rand} [boundary] von einer Teilmenge $Y \subseteq X$ ist
$$\delta Y = \quer{Y} \setminus \mathring{Y}$$
\end{definition}
\begin{definition}{Dichtheit}{}
Eine Teilmenge $Y \subseteq X$ heisst \textbf{dicht} [dense], falls
$$\quer{Y} = X$$
\end{definition}
\begin{remark}
Von hier kommt die Aussage ''$\Q$ liegt dicht auf $\R$''.
\end{remark}
In der Übung haben wir gezeigt, dass auch $\mathring{Y}$, $\quer{Y}$ und $\delta Y$ durch Mengenoperationen definiert werden können und somit auch topologische Begriffe sind:
\begin{korollar}{$\mathring{Y}$, $\quer{Y}$ und $\delta Y$ als topologische Begriffe}{}
Es gilt:
$$\mathring{Y} = \bigcup_{\substack{\mathcal{U}\subseteq Y \\ \mathcal{U} \text{ offen}}} \mathcal{U} \qquad \text{und} \qquad  \quer{Y} = \bigcap_{\substack{A\supseteq Y \\ A \text{ abgeschl.}}} A$$
\end{korollar}

\section{Relative Topologie}
Wie bei den Metriken mit Teilräumen wollen wir nun auch die Eigenschaften einer Topologie auf eine Teilmenge davon ''vererben'' können. Wir definieren daher den Begriff der \textbf{relativen Topologie}.

\begin{definition}{Relative Offenheit}{}
Sei $(X,d)$ ein metrischer Raum und $Y \subseteq X$ eine Teilraum mit $(Y, d\vert_{Y\times Y})$. Die Teilmenge $Y' \subseteq Y$ heisst \textbf{relativ offen}, falls es eine offene Menge $X' \subseteq X$ gibt, sodass $Y' = Y \cap X'$ gilt.
\end{definition}

\begin{example}\label{ex_relative_offenheit}
Sei $Y = \set{(x,y)}{x^2+y^2=1} \subseteq \R^2 = X$ die Teilmenge des Einheitskreises im $(\R^2, d_E)$. Die Teilmenge des Kreisbogens $Y' = \set{(\cos \phi, \sin \phi)}{\phi \in (0, \pi)} \subseteq Y$ ist natürlich offen bezüglich $(Y, d_E\vert_{Y\times Y})$, sie ist aber zudem \textbf{relativ offen} als Teilmenge von $Y$ in $X$, da wir $Y'$ als Schnitt von $Y$ mit der offenen Menge $X' = \set{(x,y)}{x \in \R_{>0}, y \in \R}$ konstruieren können:
$$\set{\binom{\cos \phi}{\sin \phi}}{\phi \in (0, \pi)} = \set{\binom{x}{y}}{x^2+y^2=1} \cap \set{\binom{x}{y}}{x \in \R, y \in \R_{>0}}$$
\end{example}

Wir können nun die Brücke schlagen zwischen dieser formalen Definition von relativer Offenheit und der Topologie, die auf die Teilmenge übertragen wird:

\begin{satz}{Offenheit in Teilraum und relative Offenheit}{}
Die offenen Teilmengen von einem Teilraum $(Y, d\vert_{Y \times Y})$ sind relativ offene Mengen:

Sei $(Y, d\vert_{Y \times Y}) \subseteq (X, d)$ ein Teilraum und $Y' \subseteq (Y, d\vert_{Y \times Y})$:
$$Y' \text{ offen in } (Y, d\vert_{Y \times Y}) \iff Y' \text{ relativ offen in } X$$
Insbesondere bilden sie eine Topologie auf $Y$.
\end{satz}
\begin{proof}
Um den Satz zu beweisen, werden wir die verschiedene $\varepsilon$-Umgebungen von $X$ resp. $Y$ betrachten. Wir bezeichnen daher diejenigen $\varepsilon$-Umgebungen in $X$ als $B_\varepsilon(x)$ und diejenigen in $Y \subseteq X$ als $B^Y_\varepsilon{x} = \set{y \in Y}{d(y,x) < \varepsilon}$. Aus dieser Notation folgt also direkt $B^Y_\varepsilon(x) = B_\varepsilon(x) \cap Y$.

($\Longrightarrow$) Wir wollen zeigen, dass wir für eine in $(Y, d\vert_{Y \times Y})$ offene Menge $Y'$ eine offene Menge $X'$ in $X$ konstruieren kann: Da $Y'$ offen ist, gibt es zu jedem $x \in Y'$ ein $\varepsilon(x) > 0$, sodass $B^Y_{\varepsilon(x)} \subseteq Y'$ gilt. Wir konstruieren also
$$X' = \bigcup_{x \in Y} B_{\varepsilon(x)}(x)$$
$X'$ ist somit eine offene Menge in $X$ und es gilt $X' \cap Y = Y'$

($\Longleftarrow$) Sei nun $X'$ und die Menge $Y'$ der Schnitt $Y' = Y \cap X'$. Für jedes $x \in Y'$ gibt es also ein $\varepsilon > 0$, sodass $B_\varepsilon(x) \subseteq X$ gilt. Es folgt direkt $B^Y_\varepsilon(x) = B_\varepsilon(x) \cap Y \subseteq X' \cap Y = Y'$, also sind relativ offene Teilmengen auch offen für $(Y, d\vert_{Y \times Y})$.
\end{proof}

\begin{remark} Relativ offene Teilmengen von $Y$ sind im Allgemeinen \textbf{nicht} offen als Teilmengen von $X$. Man erkennt am Beispiel \ref{ex_relative_offenheit} oben schnell, dass der Halbkreis $Y'$ oder auch der Einheitskreis $Y$ selber keine offene Menge bezgl. $\R^2$ sind, hingegen aber relativ offen sind.
\end{remark}

\section{Stetigkeit}\label{cha_stetigkeit_metr}
Wir haben bereits die Konvergenz für die verschiedenen Strukturen formulieren können, nun wollen wir auch die Stetigkeit von Abbildungen zwischen metrischen Räumen definieren:

\begin{definition}{Stetigkeit}{}
Seien $(X, d_X)$ und $(Y, d_Y)$ zwei metrische Räume und $f: X \to Y$ eine Abbildung. $f$ heisst \textbf{stetig} bei $x_0 \in X$, falls für alle $\varepsilon > 0$ ein $\delta > 0$ existiert, sodass:
$$d_X(x,x_0) < \delta \implies d_Y(f(x), f(x_0)) < \varepsilon$$
oder
$$x \in B_\delta(x_0) \implies f(x) \in B_\varepsilon(f(x_0))$$
Zudem nennen wir $f$ \textbf{stetig}, falls $f$ in jedem Punkt $x_0 \in X$ stetig ist.
\end{definition}

Wir können diese Stetigkeit wie in $\R$ auch als Folgenstetigkeit ausdrücken:

\begin{lemma}{Folgenstetigkeit}{}
$f: X \to Y$ ist genau dann stetig bei $x_0$, falls für alle Folgen $(x_n)_n$ in $X$ mit dem Grenzwert $\lim_{n \to \infty}x_n = x_0$ gilt:
$$\lim_{n \to \infty} f(x_n) = f(x_0)$$
\end{lemma}
Den Beweis für dieses Lemma haben wir bereits im letzten Semester gezeigt, wobei die Betragsfunktion $\abs{\:\cdot\:}$ mit der Metrik $d_X$ resp. $d_Y$ zu ersetzen ist.

\subsection{Vektorwertige Funktionen}
Wir wollen nun die Stetigkeit von Funktionen betrachten, welche anstelle eines Skalars einen Vektor von $\R^n$ oder $\C^n$ produzieren. Dabei können wir die Funktionen der einzelnen Komponenten separat betrachten:

\begin{satz}{Stetigkeit von vektorwertigen Funktionen}{stetigkeit_vektor_fcns}
Sei $(X, d_X)$ der Definitionsbereich mit einer Metrik und
$$f: X \to \R^n \text{ oder } \C^n \text{ mit } x \mapsto (f_1(x), ..., f_n(x))$$
eine Abbildung, wobei wir für den Zielrraum die euklidische Metrik verwenden. $f$ ist genau dann stetig in $x_0 \in X$, falls alle $f_i$ stetig in $x_0$ sind:
$$f \text{ stetig} \iff \forall i: 1 \leq i \leq n: f_i \text{ stetig}$$
\end{satz}
\begin{proof} ($\Longrightarrow$) Sei $f$ stetig bei $x_0 \in X$, also gibt es für eine $\varepsilon > 0$ ein $\delta > 0$, sodass
$$\norm{f(x)-f(x_0)}_2 = \sqrt{\sum_{i=1}^n \abs{f_i(x) - f_i(x_0)}^2} < \varepsilon$$
gilt. Es folgt also für jedes $1 \leq i \leq n$:
$$\abs{f_i(x) - f_i(x_0)} \leq \norm{f(x) - f(x_0)}_2 < \varepsilon$$

($\Longleftarrow$) Sei nun jedes $f_i$ stetig, also gibt es ein $\varepsilon > 0$, sodass für alle $1 \leq i \leq n$ gilt:
$$x\in B_\delta(x_0) \implies \abs{f_i(x) - f_i(x_0)} < \frac{\varepsilon}{\sqrt{n}}$$
wobei wir die rechte Seite so gewählt haben, damit sich der Faktor $\sqrt{n}$ wegkürzt. Wir erhalten  also für die Norm
$$\norm{f(x)-f(x_0)}_2 = \sqrt{\sum_{i=1}^n \abs{f_i(x) - f_i(x_0)}^2} < \sqrt{\frac{\varepsilon^2}{n} + ... + \frac{\varepsilon^2}{n}} = \varepsilon$$
\end{proof}

\begin{remark}[Warnung] Wir haben nun Funktionen betrachtet, deren Zielraum ein Vektorraum ist: $X \to K^n$. Betrachten wir nun umgekehrt Funktionen $f: K^n \to Y, (x_1,...,x_n) \mapsto y$, welche einen Vektor als Argument aufnehmen, so gilt die Rückrichtung der Implikation nicht mehr, i.e. aus der Stetigkeit der Funktionen, für welche alle ausser eine $x_i$-Komponente festgehalten wird, lässt sich nicht mehr auf die Stetigkeit von $f$ schliessen:
$$f \text{ stetig } \implies f(x_i)\big\vert_{x_j \textit{ const}, i\neq j} \text{ stetig }$$
\begin{example}\label{ex_partiell_diffbar_gegenbsp} Sei $f: \R^2 \to \R$ mit
$$f(x_1,x_2) = \begin{cases} \frac{x_1x_2}{x_1^2+ x_2^2} & ,(x_1, x_2) \neq (0,0) \\ 0 & , (x_1, x_2) = (0,0)\end{cases}$$
Es gilt zwar die Stetigkeit für
$$a \neq 0: x \mapsto f(x,a) = f(a,x) = \frac{xa}{x^2+a^2}$$
resp.
$$a = 0: x \mapsto f(x,0) = f(0,x) = 0$$
jedoch können wir zeigen, dass $f$ nicht stetig ist, wenn wir uns nicht entlang einer Koordinatenachse bewegen:
$$\lim_{n \to \infty} f\left(\frac{1}{n}, \frac{1}{n}\right) = \frac{1}{2} \neq f(0,0) = 0$$
\begin{center}
    \centering
    \begin{tikzpicture}
    \begin{axis}[%colormap/viridis,
    colormap/cool,
    height=7cm,
    surf, fill opacity=0.75,
    shader=flat,
    grid style=dashed,
    ]
    \addplot3[surf,domain=-2:2,samples=30, shader=interp] {x*y/(x*x + y*y)};
    \addplot3[domain=0:2, samples = 2, samples y=0] ({x}, {0}, {0});
    \addplot3[domain=0:2, samples = 2, samples y=0] ({0}, {-x}, {0});
    \addplot3[domain=-2:2, samples = 2, samples y=0] ({x}, {x}, {0.5});
    \addplot3[domain=0:2, samples = 2, samples y=0] ({x}, {-x}, {-0.5});
    \addplot3[domain=0:2, samples = 2, samples y=0, dashed] ({-x}, {x}, {-0.5});
    \addplot3[domain=0:-2, samples = 2, samples y=0, dashed] ({x}, {0}, {0});
    \addplot3[domain=0:-2, samples = 2, samples y=0, dashed] ({0}, {-x}, {0});
    \addplot3[mark=*, mark size=1.5] coordinates{(0,0,0)}; 
    \end{axis}
    \end{tikzpicture}
\end{center}


\end{example}
\end{remark}

\subsection{Eigenschaften von stetigen Funktionen}
\begin{satz}{Urbild von stetigen Funktionen}{urbild_stetige_fcns}
Seien $(X, d_X), (Y, d_Y)$ metrische Räume. Die Abbildung $f: X \to Y$ ist genau dann stetig, wenn für jede offene Teilmenge im Zielraum $Y' \subseteq Y$ auch das Urbild davon $f^{-1}(Y')$ offen ist:
$$f: X \to Y \text{ stetig} \iff \forall Y' \subseteq Y \text{ offen}: f^{-1}(Y') \text{ offen}$$

Es gilt äquivalent dazu auch die Aussage für abgeschlossene Mengen:
$$f: X \to Y \text{ stetig} \iff \forall Y' \subseteq Y \text{ abgeschlossen} \implies f^{-1}(Y') \text{ abgeschlossen}$$
\end{satz}
\begin{proof}
($\Longrightarrow$) Sei $f$ stetig und $Y' \subseteq Y$ offen und sei $x_0 \in X$ im Urbild $f^{-1}(Y')$, also $f(x_0) \in Y'$. Wir wollen nun zeigen, dass die Offenheit vom Urbild $f^{-1}(Y')$ gilt, also dass man für jedes $x_0$ eine $\delta$-Umgebung ($\subseteq X$) finden kann, welche komplett nach $Y'$ abgebildet wird. $Y'$ ist offen, also gibt es eine $\varepsilon$-Umgebung, sodass $B_\varepsilon(f(x_0)) \subseteq Y'$ gilt. Da $f$ stetig ist, können wir nun ein $\delta > 0$ wählen, sodass die gesamte $\delta$-Umgebung $B_\delta(x_0)$ nach $B_\varepsilon(f(x_0))$ abgebildet wird. Somit erhalten wir $B_\delta(x_0) \subseteq B_\varepsilon(f(x_0)) \subseteq Y'$, also die Behauptung.

($\Longleftarrow$) Sei nun das Urbild $f(Y') \subseteq X$ offen für alle offenen Teilmengen $Y' \subseteq Y$. Also gilt auch, dass für ein $\varepsilon > 0$ und ein $x_0 \in X$ die Menge $f^{-1}(B_\varepsilon(x_0)$ offen ist. Somit können wir eine $\delta$-Umgebung für $x_0$ wählen, sodass $B_\delta(x_0) \subseteq f^{-1}(B_\varepsilon(x_0)$, also die Stetigkeit $x \in B_\delta(x_0) \implies f(x) \in B_\varepsilon(f(x_0))$ gilt.

Die letzte Behauptung folgt direkt aus der Definition der Abgeschlossenheit: Falls $Y'$ abgeschlossen ist, ist das Komplement $(Y')^C$ offen. Zudem kann man sich schnell davon überzeugen, dass $f^{-1}(Y'^C) = (f^{-1}(Y'))^C$ gilt. Ersetzt man nun in den Beweisen oben $Y'$ mit $(Y')^C$ für geschlossene $Y'$, so erhält man die Behauptung.
\end{proof}

In anderen Worten können wir nun die Stetigkeit unabhängig von der Metrik und nur mit einer Topologie definieren: Nämlich sind Abbildungen stetig, falls das Urbild einer offenen Menge ebenfalls offen ist.

Diesen Satz können wir verwenden, um z.B. Aussagen über Lösungsmengen von Gleichungssystemen oder Ungleichungssystemen machen zu können:

\begin{example}[Offenheit von Lösungsmengen] Sei $X = \R^n$ und $L$ die Lösungsmenge einer stetigen, vektorwertigen Funktion $f: \R^n \to \R^m$ mit den Komponenten $x \mapsto (f_1(x_1,...,x_n), ..., f_m(x_1,...x_n))$:
$$ L = \setr{x \in \R^n }{ f(x)=0\quad \text{resp.}\quad \begin{matrix}f_1(x_1, ..., x_n) = 0\\ \vdots \\ f_m(x_1, ..., x_n) = 0 \end{matrix}}$$
Da alle $f_i$ mit $1 \leq i \leq n$ stetig sind und $L = f^{-1}(\{0\})$ das Urbild von der abgeschlossenen Menge $\{0\}$ ist, wissen wir aus dem Satz, dass auch $L$ abgeschlossen sein muss.

Analog dazu ist z.B. die Menge 
$$\setr{x \in \R^n }{ f(x)>0} = f^{-1}(\R_{>0})$$
offen, falls $f$ stetig ist.
\end{example}

Wir können mit dem obigen Satz \ref{satz:urbild_stetige_fcns} sehr elegant auch die Stetigkeit von Verknüpfungen stetiger Abbildungen zeigen:
\begin{satz}{Stetigkeit von Verknüpfungen}{}
Seien $(X, d_X), (Y, d_Y)$ und $(Z, d_Z)$ metrische Räume und $g: X \to Y, f: Y \to Z$ zwei stetige Abbildungen, dann ist auch die Verknüpfung $f \circ g : X \to Z$ stetig.
\end{satz}
\begin{proof}
Sei $Z'$ offen, dann gilt für das Urbild davon:
\begin{align*}
    (f\circ g)^{-1} &= \set{x \in X}{f(g(x)) \in Z'}\\
    &= \set{x \in X}{g(x) \in f^{-1}(Z')}\\
    &= g^{-1}(f^{-1}(Z'))
\end{align*}
Mit der Stetigkeit von $f$ und $g$  wissen wir, dass $f^{-1}(Z')$ offen und somit auch $g^{-1}(f^{-1}(Z'))$ offen ist. Nun folgt aus Satz \ref{satz:urbild_stetige_fcns}, dass die Verknüpfung $f\circ g$ stetig ist.
\end{proof}

\subsection{Beispiele zu Klassen von stetigen Funktionen}
Wir wollen nun die Stetigkeit von einigen Funktionen zeigen:

\begin{lemma}{konstante Funktion, Identität und Kehrwert}{}
\begin{enumerate}[label=(\alph*)]
    \item konstante Funktionen $f: X \to Y, x \mapsto a$ mit $a \in Y$
    \item Identität $id: X \to X, x \mapsto x$
    \item Kehrwert $(\:\cdot\:)^{-1}: K^\times \to K^\times, x \mapsto \frac{1}{x}$
\end{enumerate}
sind stetige Funktionen
\end{lemma}
\begin{proof}
Wähle für (a) $\delta$ beliebig, für (b) wähle $\delta = \varepsilon$ und für (c) wähle $\delta = \frac{1}{\varepsilon}$.
\end{proof}

\begin{lemma}{Addition und Multiplikation}{}
Die Addition und die Multiplikation $\R \times \R \to \R$ oder $\C \times \C \to \C$ sind stetig (mit der euklidischen Metrik).
\end{lemma}
\begin{proof} Wir wollen die Konvergenz von $+: \C^2 \to \C, (x,y) \mapsto x + y$ mittels der Folgenkonvergenz zeigen: Sei $(x_n,y_n)_n$ eine Folge in $\C^2$, die gegen $(x,y)$ konvergiert. Wir setzen diese in $+(x,y)$ ein und erhalten für das Stetigkeitskriterium:
\begin{align*}
    \abs{+(x_n,y_n) - +(x, y)} &= \abs{x_n - x + y_n - y}\\
    &\leq \abs{x_n-x} + \abs{y_n - y}\\
    &= \norm{(x_n-y_n) - (x,y)}_1 \longrightarrow 0 \quad (n \to \infty)
\end{align*}
Dabei haben wir im letzten Schritt die Konvergenz von  $(x_n,y_n)_n$ verwendet. Es folgt also, dass $\lim_{n \to \infty}(x_n + y_n) = x+y$ gilt. Analog wird die Stetigkeit für die Multiplikation gezeigt.
\end{proof}
Wir folgern aus diesem Lemma den allgemeineren Satz für die Stetigkeit von Linearkombinationen von Funktionen:
\begin{satz}{Summen, Produkte und Linearkombinationen}{}
Summen, Produkte und Linearkombinationen von Funktionen $X \to \R$ oder $\C$ sind stetig.
\end{satz}
\begin{proof}
Seien $f,g : X \to \C$ stetig, dann ist die Verknüpfung durch die Multiplikation $f \cdot g : X \to \C^2 \to \C$ gegeben durch
$$ x \mapsto (f(x), g(x)) \mapsto f(x) \cdot g(x)$$
wobei die erste Abbildung nach $\C^2$ wegen der Stetigkeit von vektorwertigen Funktionen (siehe Satz \ref{satz:stetigkeit_vektor_fcns}) stetig ist und die Stetigkeit von $\cdot$ im vorherigen Lemma gezeigt wurde. Insbesondere gilt das auch, wenn man $f(x) = a$ konstant wählt. Analog können wir das auch für die Addition von stetigen Funktionen zeigen. Durch Induktion können wir zeigen, dass auch Linearkombinationen aus stetigen Funktionen stetig sind.
\end{proof}
\begin{lemma}{Koordinatenabbildung/Projektionen}{}
Die Projektion $\pi_i : \R^n \to \R$ oder $\C^n \to \C$ mit $(x_1,...,x_n) \mapsto x$ ist stetig.
\end{lemma}
\begin{proof}
Wir verwenden für diesen Beweis die Notation $x = (x_1, ..., x_n)$ und $x_0 = (x_{0_1}, ..., x_{0_n})$. Sei $\varepsilon > 0$, dann erhalten wir für $\delta = \varepsilon$ mit $d(x, x_0) =    \norm{x-x_0}_2 < \delta$:
\begin{align*}
    \sqrt{\sum_{i=1}^n \abs{x_i - x_{0_1}}^2} &< \delta\\
    \abs{x_i - x_{0_i}} &< \varepsilon
\end{align*}
es folgt daraus die Stetigkeit von $\pi_i$:
$$d(\pi_i(x) , \pi_i(x_0)) = \abs{x_i - x_{0_i}} < \varepsilon$$
\end{proof}

\begin{lemma}{Lineare Abbildungen}{}
Lineare Abbildungen/Homomorphismen $\C^n \to \C^m \text{ oder } \R^n \to \R^m$ sind stetig.
\end{lemma}
\begin{proof} Da Linearkombinationen sowie die Koordinatenabbildung die Stetigkeit erhalten, können wir auch auf die Stetigkeit von linearen Endomorphismen schliessen: Sei $A$ eine lineare Abbildung
\begin{align*}
    A : \C^n &\to \C^m \text{ oder } \R^n \to \R^m\\
    x &\mapsto A(x)\\
    x_i &\mapsto  (a_{i1} \pi_1 + ... + a_{in} \pi_n)(x) \pi_n\\
    \begin{pmatrix} x_1\\\vdots\\x_n\end{pmatrix}&\mapsto\begin{pmatrix}a_{11} x_1 + ... + a_{1n} x_n\\\vdots\\a_{m1} x_m + ... + a_{mn}\end{pmatrix}
\end{align*}
wobei $a_{ij}$ die Koeffizienten der Matrix $[A]_\mathcal{B}$ bezüglich einer Basis $\mathcal{B}$ bezeichnet.
\end{proof}

\begin{lemma}{Polynomfunktionen}{}
Polynomfunktionen $P : \C^n \to \C$ sind stetig.
\end{lemma}
\begin{proof} Monome für vektorwertige Argumenten sind von der Form:
$$x = (x_1, ..., x_n) \mapsto x_1^{a_1}\cdot...\cdot x_n^{a_n}$$
wobei $a_i \in \N_0 = \{0,1,2,...\}$ die jeweiligen Potenzen bezeichnen. Da Monome Produkte von Projektionen $\pi_i$ sind, sind sie auch stetig. Polynome sind nun Linearkombinationen von Monomen, also folgt aus vorherigem Beispiel, dass Polynomfunktionen\footnote{Wir bezeichnen Polynomfunktionen umgangssprachlich häufig auch als ''Polynome'', wobei das Polynom erst durch das Einsetzen eines Arguments aus einem Ring entsteht.} ebenfalls stetig sind.
\end{proof}
\begin{example}Das Polynom $P: \C^3 \to \C$ mit $P(x,y,z) = x^2y + 3z$ ist stetig.\end{example}

\begin{satz}{Einschränkung von Abbildungen}{Einschraenkung_von_Abbildungen}
Seien $(X, d_X), (Y, d_Y)$ metrische Räume $X' \subseteq X$ eine Teilmenge und $f: X \to Y$ eine stetige Abbildung.

Die Einschränkung $f\vert_{X'} : X' \to Y$ ist stetig als Abbildung von $X' \to Y$.
\end{satz}
\begin{proof} $(X', d_X\vert_{X' \times X'})$ bildet einen metrischen Teilraum, wobei die offenen Mengen in $X'$ relativ offen sind in $X$. Sei nun $Y'$ eine offene Teilmenge in der Zielmenge $Y$, dann folgt für das Urbild der eingeschränkten Funktion:
\begin{align*}
    (f\vert_{X'})^{-1}(Y') &= \set{x \in X'}{f(x) \in Y'}\\
    &= f^{-1}(Y') \cap X'
\end{align*}
Aus der Stetigkeit von $f$ wissen wir, dass $f^{-1}(Y')$ offen ist. Also ist das Urbild relativ offen resp. offen als Teilmenge von $X'$, wobei aus Satz \ref{satz:urbild_stetige_fcns} folgt, dass $f\vert_{X'}$ stetig ist.
\end{proof}

\begin{lemma}{Rationale Funktionen}{}
Seien $P,Q$ Polynomfunktionen auf $\R^n \to \R$ und $D = \set{x \in \R^n}{Q(x) \neq 0}$ eine Teilmenge von $\R^n$ ohne den Nullstellen von $Q$, dann ist die rationale Funktion
$$f: D \to \R\text{ mit }f(x) = \frac{P(x)}{Q(x)}$$ stetig.
\end{lemma}
\begin{proof}
Wir erkennen, dass $D = Q^{-1}(\{0\})^C$ das Komplement von der abgeschlossenen Lösungsmenge $\set{x \in \R^n}{Q(x) = 0}$ ist, also $D$ folglich offen ist. $f$ ist die Multiplikation von $P(x)$ mit $\frac{1}{Q(x)}$ wobei $P$ und $Q$, die Multiplikation sowie der Kehrwert auf $Q(D) \subseteq X^\times$ stetig sind.
\end{proof} 

\begin{lemma}{Weitere stetige Funktionen}{}
\begin{itemize}
    \item Exponentialfunktion $\exp: \C \to \C$
    \item Trigonometrische Funktionen 
    \begin{itemize}
        \item $\sin, \cos: \C \to \C$
        \item $\tan: \C \setminus \set{x \in \C}{\cos(x) = 0} \to \C$
        \item $\arcsin, \arccos: [-1, 1] \to \R$
        \item $\arctan: \R \to \R$
    \end{itemize}
\end{itemize}
\end{lemma}

\section{Zusammenhang}
In $\R$ haben wir bereits den Begriff der Intervalle verwendet. Wir haben sie so charakterisiert, dass für jedes Paar von Punkten aus dem Intervall alle Punkte dazwischen ebenfalls im Intervall sind. Wir wollen davon nun den allgemeineren topologischen Begriff des Zusammenhangs definieren:
\begin{definition}{Zusammenhang}{}
Sei $(X, d)$ ein metrischer Raum, dann heisst $X$ \textbf{nicht zusammenhängend}, falls es zwei offene, nicht leere und disjunkte Teilmengen $U, V \subseteq X$ gibt, welche vereint $X$ ergeben:
$$X \text{ nicht zusammenhängend} \iff U \sqcup V = X$$
$X$ heisst \textbf{zusammenhängend}, falls für offene und disjunkte Teilmengen $U, V$ gilt:
$$X \text{ nicht zusammenhängend} \land U \sqcup V = X \iff V = \emptyset \lor U = \emptyset$$
\end{definition}
Wir haben in der Definition der Offenheit gesehen, dass jeweils die Menge $X$ und $\emptyset$ offene Mengen sind, welche gleichzeitig auf abgeschlossen sind, weil sie das Komplement voneinander sind. Diese Definition besagt nun, dass zusammenhängende metrische Räume nur genau diese zwei offenen \textbf{und} abgeschlossenen (\textbf{abgeschloffenen}) Mengen besitzen. Denn falls wir eine weitere solche Menge $U \neq X$ und $U \neq \emptyset$ hätten, so wäre das Komplement $U^C$ ebenfalls offen und abgeschlossen und $U \sqcup U^C = X$ wäre erfüllt.

\begin{example}
$X = (-\infty, 0) \cup (0, \infty) \subseteq \R$ wie auch $Y = [0,1] \cup \{2\}$ sind nicht zusammenhängend. Für $X$ können wir $U = (-\infty, 0)$ und $V = (0, \infty)$ wählen, für $Y$ analog $V' = [0,1]$ und $U' = \{2\}$. Man beachte, dass $V'$ und $U'$ als Teilmengen von $Y$ offen sind, auch wenn sie als abgeschlossene Mengen von $\R$ geschrieben sind.
\end{example}

Hingegen sind alle Intervalle $(a,b)$, $[a,b)$, etc. zusammenhängend. Wir sehen, dass das für $\R$ auch in die andere Richtung gilt:

\begin{satz}{Zusammenhängende Mengen auf $\R$}{zusammenhaengend_intervall_aufR}
Sei $A \subseteq \R$ eine Teilmenge, dann ist $A$ genau dann zusammenhängend, wenn es ein Intervall ist:
$$A \subseteq \R \text{ zusammenhängend} \iff A \text{ ist ein Intervall}$$
\end{satz}
\begin{proof}($\Longrightarrow$) Wir wollen diese Richtung mit der Kontraposition ''$A$ kein Intervall $\implies A$ nicht zusammenhängend'' beweisen. Da $A$ kein Intervall ist, gibt es für $a_1 < a_2 \in A$ ein $X \notin A$ mit $a_1 < x < a_2$ dazwischen. Wir konstruieren nun also die Mengen $U = (-\infty, x) \cap A$ und $V = (x, \infty) \cap A$. Überprüfen wir die Konditionen für eine nicht zusammenhängende Menge: $U,V$ sind relativ offen, das folgt aus der Definition der relativen Offenheit. Des Weiteren sind sie nicht leer, da $a_1 \in U$ und $a_2 \in V$ gilt. Auch sind $U,V$ disjunkt, da $U < V$, und es gilt $V \sqcup U = A$, da $x \notin A$ ist. Also ist $A$ keine zusammenhängende Menge.

($\Longleftarrow$) Wir nehmen an, dass $A$ ein nicht zusammenhängendes Intervall ist und führen es zu einem Widerspruch. Aus der Annahme wissen wir, dass disjunkte, nicht leere $U,V$ existieren, sodass $U \sqcup V = A$ gilt. Seien also $a_1 \in U$ und $a_2 \in V$ mit $a_1 < a_2$, dann können wir $x$ ''an der/einer Grenze'' wie folgt konstruieren:
$$x = \sup \set{y \in U}{[a_1, y] \subseteq U}$$
Mit dem Supremum ertasten wir also von $a_1$ aus die nächste rechte Grenze von $U$. $x$ wird zudem grössergleich $a_1$ sein per Definition des Supremums. Da $A$ per Annahme ein Intervall ist und $a_1 \leq x \leq a_2$ gilt, muss $x \in A$, also wegen $U \sqcup V = A$ entweder $x \in U$ oder $x \in V$ gelten. Beachte, dass das Supremum auch in $U \subseteq A$ liegen kann (i.e. das Maximum ist) und trotzdem noch (relativ) offen sein kann, weswegen wir beide Fälle betrachten müssen:

(Fall $x \in U$) Da $U$ offen ist, existiert ein $\varepsilon > 0$, sodass $B_\varepsilon(x) \in U$, also ist auch das Intervall $[x, x+\frac{\varepsilon}{2}]$ in $U$, was der Definition von $x$ widerspricht.

(Fall $x \in V$) Da $V$ offen ist, existiert ein $\varepsilon > 0$, sodass $B_\varepsilon(x) \in V$, also ist auch das Intervall $[x-\frac{\varepsilon}{2}, x]$ in $V$ und wegen der Disjunktheit nicht in $U$, welches auch ein Widerspruch zur Definition von $x$ ist.

Also ist $A$ zusammenhängend.
\end{proof}

Für stetige Abbildungen auf zusammenhängenden Mengen kann man sich nun vorstellen, dass das Bild davon ebenfalls zusammenhängend sein muss. Wir wollen daher folgenden Satz zeigen:

\begin{satz}{Abbildungen auf zusammenhängenden Mengen}{abb_auf_zusammenhaengenden_mengen}
Seien $(X,d_X), (Y, d_Y)$ zwei metrische Räume mit einer stetigen Abbildung $f : X \to Y$, dann gelten:
\begin{enumerate}[label=(\alph*)]
    \item Falls $X$ zusammenhängend und $f$ surjektiv\footnote{Ist nötig, da nicht getroffene Punkte vom Bild abgespalten sein könnten.} ist, dann ist auch $Y$ zusammenhängend.
    \item Falls $A \subseteq X$ zusammenhängend ist, dann ist auch das Bild $f(A) \subseteq Y$ zusammenhängend.
\end{enumerate}
\end{satz}
\begin{proof}
\begin{enumerate}[label=(\alph*)]
    \item Wir nehmen an, dass $Y$ nicht zusammenhängend ist, somit gibt es offene, nicht-leere und disjunkte $U, V \subseteq Y$, sodass für die Vereinigung $U \sqcup V = Y$ gilt. Da $f$ stetig ist, sind die Urbilder $\tilde{U} = f^{-1}(U) \subseteq X$ und $\tilde{V} = f^{-1}(V)$ offen als auch disjunkt, da $f^{-1}(U) \cap f^{-1}(V) = f^{-1}(U \cap V) = f^{-1}(\emptyset) = \emptyset$ gilt. Zudem wissen wir auch aufgrund der Surjektivität (und Wohldefiniertheit) von $f$, dass $\tilde{U} \sqcup \tilde{V} = X$ gelten muss, da $U \sqcup V = Y$ den gesamten Wertebereich abdeckt. Zudem sind $\tilde{U}$ und $\tilde{V}$ nicht leer, da $V$ und $U$ nicht leer sind und $f$ surjektiv ist. Also ist $X$ nicht zusammenhängend.
    \item folgt aus (a), weil $f\vert_A : A \to f(A) \subseteq Y$ gilt und $f$ auf sein Bild surjektiv ist.
\end{enumerate}
\end{proof}

Wir können aus dieser Erkenntnis den verallgemeinerten Zwischenwertsatz formulieren:
\begin{korollar}{verallgemeinerter Zwischenwertsatz}{}
Sei $X$ eine zusammenhängender metrischer Raum und $f: X \to \R$ eine stetige Funktion. Seien zudem $a,b \in X$, dann gibt es für jedes $y$ zwischen den Bildern $f(a)$ und $f(b)$ ein $x \in X$, sodass $f(x) = y$ gilt.
\end{korollar}
\begin{proof}
Das Bild $f(X)$ ist nach obigem Satz \ref{satz:abb_auf_zusammenhaengenden_mengen} (b) ebenfalls zusammenhängend. Da es eine zusammenhängende Menge auf $R$ ist, ist $f(X)$ zudem ein Intervall nach Satz \ref{satz:zusammenhaengend_intervall_aufR}, wodurch für alle Punkte zwischen $f(a)$ und $f(b)$ getroffen werden müssen.
\end{proof}

\subsection{Wegzusammenhang}
Wir haben den Begriff des Weges Ende letztes Semester eingeführt. Kurz gesagt ist es eine stetige Funktion zwischen zwei Punkten $a$ und $b$.

%\begin{definition}{Weg}{} ein \textbf{Weg} von $x \in X$ nach $y \in X$ ist eine stetige Abbildung $\gamma : [0, 1] \to X$, sodass $\gamma(0) = x$ und $\gamma(1) = y$.
%\end{definition}
\begin{definition}{Wegzusammenhang}{}
Wir nennen eine Menge $X$ \textbf{wegzusammenhängend}, falls es für alle $x,y \in X$ einen Weg von $x$ nach $y$ gibt.
\end{definition}
\begin{example}
Der $\R^n$ ist wegzusammenhängend: Seien $x,y \in \R^n$, dann ist $\gamma: [0,1] \to \R^n$ mit $t\mapsto x + t(y-x)$ ein Weg von $x$ nach $y$.

Hingegen ist $\R\setminus \{0\}$ nicht wegzusammenhängend, da aus dem Zwischenwertsatz folgt, dass jeder Weg z.B. von $-1$ nach $1$ durch $0$ gehen muss. $\R^n \setminus \{0\}$ mit $n\geq 2$ ist jedoch wieder wegzusammenhängend, da man einen Weg um die $0$ finden kann.
\end{example}

\begin{satz}{Wegzusammenhang und Zusammenhang}{Wegzusammenhang_und_Zusammenhang}
Falls $X$ eine wegzusammenhängende Menge ist, so ist $X$ auch zusammenhängend.

\centering $X$ wegzusammenhängend $\implies X$ zusammenhängend
\end{satz}
\begin{proof}
Sei $X$ wegzusammenhängend aber nicht zusammenhängend. Dann gibt es offene, nicht-leere und disjunkte $U, V \subseteq Y$, sodass für die Vereinigung $U \sqcup V = X$ gilt. Wähle nun $x \in U$ und $y \in V$, dann existiert wegen dem Wegzusammenhang ein Weg $\gamma$. Da $[0,1]$ zusammenhängend und $\gamma$ stetig ist, folgt aus Satz \ref{satz:abb_auf_zusammenhaengenden_mengen} (b), dass das Bild $A$ auch zusammenhängend ist. Da $A \subseteq X = U \sqcup V$ gilt, müsste auch $A = (U \cap A) \cup (V \cup A)$ gelten. Da Schnitte von offenen Mengen aber offen sind und $(U \cap A)$ resp. $(V \cap A)$ jeweils offen sind, kann $A$ nicht zusammenhängend sein, also ein Widerspruch.
\end{proof}

Wir können im folgenden Beispiel erkennen, dass der Wegzusammenhang eine stärkere Eigenschaft ist als der Zusammenhang selber:

\begin{example}\label{ex_topologists_curve}
Die Menge $T = \set{(x,\sin \frac{1}{x}) \in \R^2}{x \in \R_{\neq 0}} \cup \{(0,0)\}$\footnote{Mehr dazu unter \href{https://en.wikipedia.org/wiki/Topologist\%27s_sine_curve}{diesem} Link} ist zusammenhängend, denn jeder $\varepsilon$-Ball um $(0,0)$ schneidet sich mit dem Rest der Menge. Aber es lässt sich kein Weg zu $(0,0)$  oder von der linken Hälfte in die rechte Hälfte finden, da ein solcher Weg ''unendlich'' wäre.
\begin{figure}[hbt!]
    \centering
    \begin{tikzpicture}
    \begin{axis}[
        axis lines = left,
        ytick={-1, 0, 1},
        xtick={-0.2, -0.1, 0, 0.1, 0.2},
        xlabel = $x$,
        ylabel = {$\sin \frac{1}{x}$},
        width=14cm,height=4cm,
    ]
    \addplot [domain=-0.05:-0.2, samples=200, color=red, style={thick}] {sin(deg(1/x))};
    \addplot [domain=-0.015:-0.05, samples=250, color=red, style={thick}] {sin(deg(1/x))};
    \addplot [domain=0.015:0.05, samples=250, color=red, style={thick}] {sin(deg(1/x))};
    \addplot [domain=0.05:0.2, samples=200, color=red, style={thick}] {sin(deg(1/x))};
    \addplot [fill, color=red] coordinates {(-0.016,1) (0.016,1) (0.016,-1)(-0.016,-1) } --cycle;
    \node at (axis cs:0.17,0.5) {$T$};
    \end{axis}
\end{tikzpicture}
\end{figure}
\end{example}
Unter einigen Einschränkungen können wir aber eine Äquivalenz formulieren:
\begin{satz}{(Weg)zusammenhang von offenen Teilmengen in $\C^n$}{}
Sei $U \subseteq \R^n$ oder $\C^n$ eine offene Teilmenge mit der euklidischen Metrik, dann gilt:
$$U \text{ zusammenhängend} \iff U \text{ wegzusammenhängend}$$
\end{satz}
Beachte hier genau, welches ''offen'' wir verwenden. Hier, im Gegensatz zur Definition des Zusammenhangs, reden wir wieder von offenen Teilmengen $\subseteq \R^n$ oder $\C^n$. Im vorherigen Beispiel \ref{ex_topologists_curve} konnten wir keine zwei offenen, disjunkte Teilmengen von $T$ finden, wobei $T$ als offen in sich selber gilt. $T$ als Teilmenge von $\R^2$ ist jedoch keine offene Menge.
\begin{proof}
($\Longrightarrow$) Sei $x_0$ ein Punkt aus $U$. Wir wollen nun für den Wegzusammenhang zeigen, dass es zu jedem Punkt aus $U$ einen Pfad zu $x_0$ gibt. Falls das erfüllt ist, kann auch ein Pfad zwischen zwei beliebigen Punkten gefunden werden. Wir definieren die Menge$$G = \set{x \in U}{\text{es existiert ein Pfad zu }x_0}$$ für welche Punkte es einen Pfad zu $x_0$ gibt. Wir wollen zeigen, dass $G$ offen und abgeschlossen ist, welches in einer zusammenhängenden Menge nach Definition bedeutet, dass $G = U$ gilt.

A priori wissen wir aber nur folgendes über diese Menge: Sie ist nicht leer, da wir für $x_0$ bereits den ''konstanten'' Pfad kennen. Sei also $x$ ein Element aus diesem nicht-leeren $G$ (notfalls $x_0$ selber), dann wissen wir aufgrund der Offenheit von $U$, dass ein $\varepsilon > 0$ existiert, sodass der Ball $B_\varepsilon(x)$ komplett in $U$ enthalten ist. Wir wollen zeigen, dass dieser Ball aber auch in $G$ liegt. Wir können von einem Punkt aus dem Ball $y \in B_\varepsilon(x)$  einen Pfad $\gamma_y$ zu $x$ finden, indem wir (\todo{was ist die Begründung?}). Da $x \in G$ ist, gibt es auch bereits einen Pfad $\gamma_x$ von $x_0$ zu $x$. Wir können diese nun wie folgt zusammensetzen:
$$\gamma(t) = \begin{cases} \gamma_x(2t) & 0 \geq t \geq \frac{1}{2} \\ \gamma_y(2t-1) & \frac{1}{2} \geq t \geq 1\end{cases}$$
Wir erkennen, dass der zusammengesetzte Pfad $\gamma$ auch tatsächlich stetig ist, da $\gamma_x(1) = \gamma_y(0) = x$ gilt. Also ist $y \in B_\varepsilon(x)$, wodurch $G$ offen ist.

Nun ist zu zeigen, dass $G$ auch abgeschlossen ist, also das Komplement $U \setminus G$ offen ist. Dies können wir analog mit Punkten zeigen, die keinen Pfad au $x_0$ haben, wodurch deren $\varepsilon$-Umgebung ebenfalls keinen Pfad zu $x_0$ haben kann. Wir schliessen daraus, dass $U\setminus G$ ebenfalls offen sein muss und, da $U$ zusammenhängend ist, gleich der leeren Menge sein muss. Somit sehen wir, dass $G = U$ gilt.

($\Longleftarrow$) Dies haben wir bereits im Allgemeinen gezeigt in Satz \ref{satz:Wegzusammenhang_und_Zusammenhang}
\end{proof}

Wir wollen noch einen interessanten topologischen Begriff einführen:
\begin{definition}{Homemorphie}{}
Zwei metrische Räume $X,Y$ heissen \textbf{homöomorph/homeomorph}, falls es eine bijektive stetige Funktion $f: X \to Y$ gibt, welche auch eine stetige Inverse $Y \to X$ besitzt.
\end{definition}
\begin{example} Wir können mit dem Begriff des Zusammenhangs zeigen, dass der Einheitskreis $S^1=\set{x \in \R^2}{\norm{x}_2 = 1}$ und die reellen Zahlen $\R$ nicht homeomorph sind:
\begin{proof}
Sei $f : S^1 \to \R$ eine homeomorphe Abbildung. Wähle ein beliebiges $x \in S^1$ und betrachte die Mengen $U = S^1 \setminus \{x\}$ und $V = \R \setminus \{f(x)\}$. Wir erkennen, dass $U$ nach wie vor zusammenhängend ist (\todo{Beweis}) und $V = (-\infty, f(x)) \cup (f(x), \infty)$ nicht mehr zusammenhängend ist. Da aber $f\vert_U: U \to V$ stetig ist (Satz \ref{satz:Einschraenkung_von_Abbildungen}) und stetige Funktionen auf zusammenhängenden Mengen ein zusammenhängendes Bild besitzen müssen (Satz \ref{satz:abb_auf_zusammenhaengenden_mengen}), erhalten wir einen Widerspruch, da $V$ nicht zusammenhängend sein kann.
\end{proof}
\end{example}

\section{Vollständige metrische Räume}
Wir haben die Existenz vom Supremum resp. die Konvergenz von Cauchy-Folgen als Kriterium verwendet, um die Vollständigkeit zu charakterisieren. Wir möchten nun nochmals dieselben Begriffe mit dem topologischen Hintergrund definieren.

\begin{definition}{Cauchy-Folge}{}
Sei $(X,d)$ ein metrischer Raum, dann ist eine Folge $(x_n)_{n \in \N}$ in $X$ eine \textbf{Cauchy-Folge}, falls zu jedem $\varepsilon >0$ ein $n_0 \in \N$ existiert, sodass
$$n,m \geq n_0 \implies d(x_n, x_m) < \varepsilon$$
\end{definition}
Die einzige Änderung an dieser Definition ist die Verwendung der Metrik anstelle der Betragsfunktion.

Der Vorteil von dieser Definition ist, dass wir den Grenzwert nicht zu wissen brauchen, um eine Aussage über die Konvergenz einer Folge machen zu können. Denn wir haben konvergente Folgen so definiert, dass die Differenz zum Grenzwert beliebig klein gemacht werden kann resp. dass der Grenzwert der Limes der Folge ist. Mit dem Cauchy-Kriterium können wir also ohne Limes und Grenzwert zeigen, dass eine Folge cauchy ist und somit neue Objekte, z.B. $\pi$ oder $e$ definieren.

\begin{lemma}{}{}
Jede konvergente Folge ist eine Cauchy-Folge:
\begin{gather*}
    \exists x: \forall \varepsilon >0\ \exists n_0 \in \N: n \leq n_0 \implies d(x, x_n) < \varepsilon \\ \big\Downarrow \\ \forall \varepsilon >0\ \exists n_0 \in \N: n,m \geq n_0 \implies d(x_n, x_m) < \varepsilon
\end{gather*}
\end{lemma}
\begin{proof}
Sei $x = \lim_{n \to \infty} x_n$ der Grenzwert der Folge $(x_n)_{n\in \N}$. Dann gibt es für ein $\varepsilon > 0$ ein $n_0$, sodass $d(x_n,x) < \frac{\varepsilon}{2}$ für alle $ n \geq n_0$ gilt. Daraus folgt für alle $n,m \geq n_0$:
\begin{align*}
    d(x_n,x_m) &\leq d(x_n,x) + d(x,x_m)\\
    &\leq d(x, x_n) + d(x, x_m)\\
    &\leq \frac{\varepsilon}{2} + \frac{\varepsilon}{2} = \varepsilon
\end{align*}
\end{proof}
Die umgekehrte Implikation, dass jede Cauchy-Folge auch konvergiert, gilt jedoch nicht im Allgemeinen. Falls das jedoch in einem Raum gilt, so nennen wir diesen \textbf{vollständig}:

\begin{definition}{Vollständigkeit}{}
Ein metrischer Raum $X$ heisst \textbf{vollständig}, falls alle Cauchy-Folgen in $X$ konvergieren.
\end{definition}
Wir wissen bereits, dass $\R$ vollständig ist, also können wir diesen Satz formulieren:
\begin{satz}{Vollständigkeit von $\R^N$ und $\C^N$}{}
$\R^N$ mit der euklidischen Metrik ist vollständig für alle $N \geq 1$.

Daraus folgt auch die Vollständigkeit von $\C^N \cong \R^{2N}$.
\end{satz}
\begin{proof} Wir verwenden Stetigkeit und Linearität der Projektionen $\pi_i: \R^N \to \R$, $(x_1, ..., x_n) \mapsto x_i$, u.a. auch um Doppelindeces zu vermeiden. Wir erkennen, dass im allgemeinen
$$\abs{\pi_i(x)} \leq \sqrt{x_1^2+...+x_N^2} = \norm{x}_2$$
gilt. Sei nun $(x_n)_{n \in \N}$ eine Cauchy-Folge mit $x_n \in \R^N$. Für das $n$-te Folgenglied gilt also $x_n = (\pi_1(x_n),..., \pi_N(x_n)) \in \R^N$. Da $(x_n)_{n \in \N}$ eine Cauchy-Folge ist, gilt für $n,m \in \N$
$$\abs{\pi_i(x_n) - \pi_i(x_m)} = \abs{\pi_i(x_n-x_m)} \leq \norm{x_n - x_m}_2$$
Somit ist auch die Folge der Projektionen $(\pi_i(x_n))_{n \in x}$ eine Cauchy-Folge in $\R$ für jede Dimension $i = 1,...,N$. Also existiert für jedes $i$ ein Grenzwert $y_i = \lim_{n \to \infty} \pi_i(x_n)$, da wir wissen, dass Cauchy-Folgen in $\R$ konvergieren.

Wir wollen nun zeigen, dass $y = (y_1,...,y_N) \in R^N$ der Grenzwert $\lim_{n \to \infty}x_n$ der ursprünglichen Folge ist. Wir finden für die Distanz zwischen einem Folgenglied $x_n$ und $y$ den Ausdruck
$$d(x_n, y) = \sqrt{\sum_{i = 1}^N \abs{\pi_i(x_n) - y_i}^2}$$
welcher beliebig klein gemacht werden kann, da alle Terme der Summe nach $0$ gehen. Also ist $\R^N$ ebenfalls vollständig.
\end{proof}
\begin{satz}{Vollständigkeit von abgeschlossenen Teilmengen}{Vollstaendigkeit_abgeschlossene_Teilmengen}
Abgeschlossene Teilmengen von vollständigen metrischen Räumen sind als metrische Teilräume ebenfalls vollständig.
\end{satz}
\begin{proof}
Übungsserie 3
Idee: Charakterisierung einer abgeschlossenen Menge: genau dann abgeschossen, wenn der Grenzwert einer konvergenten Folge in der Teilmenge liegt.
\end{proof}

\begin{example}
$[a, b] \subseteq \R$ ist vollständig, jedoch ist $(0,1]$ nicht vollständig, da die Cauchy-Folge $(\frac{1}{n})_{n \in \N}$ ihre Elemte in $(0,1]$ hat, aber nicht als Folge des metrischen Teilraums $(0,1]$ konvergiert.
\end{example}

\subsection{Klassen von Beispielen}
Hier einige Beispiele von vollständigen Mengen 
\begin{enumerate}
    \item abgeschlossene Teilmengen $A \subseteq \R^N$
    \item $C(K)$ stetige Funktionen auf einem kompakten Intervall $K = [a,b]$
\end{enumerate}
\begin{proof}
Siehe allgemeineren Beweis für beliebige kompakte metrische Räume \todo{ref Beweis}
\end{proof}

\begin{remark}
Falls $X$ ein normierter Vetorraum mit der induzierten Metrik $d(x,y) = \norm{x-y}$ ist und $\Nrm'$ eine zu $\Nrm$ äquivalente Norm ist (siehe Definition in Abschnitt \ref{cha_eq_norms}), dann sind Cauchy-Folgen bezüglich $\Nrm$ auch Cauchy-Folgen bezüglich $\Nrm'$:
$$\norm{x_n-x_m} < \varepsilon \implies \norm{x_n-x_m}' \leq c_2 \varepsilon$$
wobei wegen der Äquivalenz der Normen $c_1 \norm{x} \leq \norm{x}' < c_2 \norm{x}$ gilt. 
\end{remark}

\subsection{\textsc{Banach}'scher Fixpunktsatz}
Nehmen wir eine Weltkarte, welches eine skalierte und gestreckte Version der realen Welt ist, dann gibt es unabhängig von der Orientation der Karte immer einen Punkt darauf, der genau über dem Punkt in der realen Welt liegt, den er abbildet/auf der Karte bezeichnet. Es ist also ein \textbf{Fixpunkt}, der durch die Abbildung von der realen Welt auf die Karte nicht verschoben hat.

Wir wollen diese Eigenschaft nun topologisch in Formeln fassen. Hierfür wollen wir zuerst einige Begriffe einführen:

\begin{definition}{Fixpunkt}{}
Sei $f: X \to X$ eine (Selbst)abbildung, dann heisst $x \in X$ ein Fixpunkt, falls $f(x) =x$ gilt.
\end{definition}

Auf einem Funktionsgraphen sind die Fixpunkte einer Funktion die Projektion der Schnittpunkte zwischen der Funktion und der Geraden $x = y$ auf die $x$-Achse.
\begin{example}
Der Goldene Schnitt $\varphi = \frac{1+\sqrt{5}}{2}$ und $\psi = \frac{1-\sqrt{5}}{2}$ Fixpunkte der Funktion $1 + \frac{1}{x}$.
\begin{figure}[hbt!]
    \centering
    \begin{tikzpicture}
    \begin{axis}[
        axis lines = center,
        xlabel = $x$,
        ylabel = {$1 + \frac{1}{x}$},
        ytick = {-2,-1,1,2,3},
        xtick = {-2,-1,1,2,3},
        xmin=-2,xmax=3,ymin=-2,ymax=3,
        width=7cm,height=7cm,
        xticklabels={-2,,1,2,3},
        yticklabels={-2,,1,2,3},
    ]
    \addplot [domain=-2:-0.3, samples=200, color=red, style={thick}] {1+(1/x)};
    \addplot [domain=0.4:3, samples=200, color=red, style={thick}] {1+(1/x)};
    \addplot [domain=-2:3, samples=5, color=blue, dashed] {x};
    \node[label={270:{$\varphi$}},circle,fill,inner sep=2pt] at (axis cs:1.618,1.618) {};
    \node[label={0:{$\psi$}},circle,fill,inner sep=2pt] at (axis cs:-0.618,-0.618) {};
    \end{axis}
\end{tikzpicture}
\end{figure}
\end{example}

Wir wollen und nun die Frage stellen, wie man solche Fixpunkte findet. Man könnte sie versuchen, iterativ vorzugehen: Man beginnt mit einem $x_0 \in X$ und definiert $x_{n+1}$ rekursiv durch $x_{n+1} = f(x)$ für $n \geq 1$. Falls die Folge $(x_n)$ konvergiert mit dem Grenzwert $x = \lim_{n \to \infty}$, dann gilt 
\begin{align}\label{eq_fixpunkt}
    f(x) = \lim_{n \to \infty} f(x_n) = \lim_{n \to \infty} x_{n+1} = x    
\end{align}

\begin{example} Also ist $\varphi = 1+\frac{1}{1+\frac{1}{1+...}}$ ein Fixpunkt.
\end{example} 

Man kann mit dieser Methodik auch Gleichungen $g(x) = 0$ lösen. Die Stetigkeit von $g(x)$ ist dabei notwendig, da wir einen Grenzwert bilden.
\begin{enumerate}
    \item Schreibe die Gleichung als Fixpunktgleichung mit $x = f(x)$, also z.B. $f(x) = x+g(x)$
    \item Wähle einen Anfangswert $x_0$ und definiere $x_{n+1} = f(x_n)$
    \item Man ''hofft'', dass die Folge $(x_n)$ konvergiert. Wenn diese Folge konvergiert, wird der Grenzwert ein Fixpunkt sein.
\end{enumerate}

Newton hat diesen Such-Algorithmus mit der Verwendung der Ableitung verbessert. Die durch $f(x) = x - \frac{g(x)}{g'(x)}$ definierte Folge konvergiert, falls das initiale $x_0$ nahe genug an einem Fixpunkt ist.

Wir können mit solchen Suchalgorithmen Fixpunkte finden, jedoch wollen wir mithilfe der Vollständigkeit eine allgemeine Aussage über Fixpunkte machen können. Wir Wollen, wie im Beispiel der Karte am Anfang des Abschnittes einen Begriff haben, für stetige ''verkleinernede'' Abbildungen haben:

\begin{definition}{Lipschitz-Kontraktion}{}
Sei $(X, d)$ ein metrischer Raum, dann ist eine Abbildung $f: X \to X$ eine \textbf{Lipschitz-Kontraktion}, wenn es eine Konstante $L \in [0, 1)$ gibt, so dass
$$d(f(x), f(x')) \leq L \cdot d(x,x')$$
für alle Punkte $x, x' \in X$ gilt. Lipsitz-Kontraktionen sind Lipschitz-stetig.
\end{definition}
\begin{proof}
Verwende $L$ als Lipschitz-Konstante. Man erhält direkt per Definition einer Lipschitz-Kontraktion $d(f(x), f(x')) \leq L \cdot d(x,x')$, das die Lipschitz-Stetigkeit zeigt.
\end{proof}
Nun können wir den \textsc{Banach}'schen Fixpunktsatz formulieren:
\begin{satz}{\textsc{Banach}'scher Fixpunktsatz}{}
Sei $f: X \to X$ eine Lipschitz-Kontraktion eines vollständigen metrischen Raums $X$, dann hat $f$ einen eindeutigen Fixpunkt $x\in X$.
\end{satz}
Mit diesem Satz werden später auch zeigen können, dass Differentialgleichungen immer eine Lösung haben.
\begin{proof}
(Existenz) Wir wählen ein $x_0 \in X$ und konstruieren eine Folge $(x_n)_{n \in \N}$ mit Gliedern $x_n = f^{\circ n}(x_0) = f\circ f...\circ f (x_0)$ also $x_{n+1} = f(x_n)$. Wir wollen zeigen, dass die Folge $(x_n)_{n \in \N}$ cauchy ist. Da Lipschitz-Kontraktionen stetig sind folgt daraus, dass die Folge konvergiert und der Grenzwert davon nach der Erkenntnis \ref{eq_fixpunkt} oben ein Fixpunkt ist. Wir betrachten nun also $d(x_n,x_m)$ zuerst für $m=n+1$:
\begin{align*}
    d(x_n,x_{n+1}) &= d(f(x_{n-1}), f(x_n))\\
    &\leq L \cdot d(x_{n-1}, x_n)\\
    &\leq L^n \cdot d(x_0, f(x_0))
\end{align*}
wobei wir beim letzten Schritt induktiv vorgegangen sind. $L^n$ wird für $L \in [0,1)$ beliebig klein, jedoch benötigen wir für Cauchy-Folgen die Aussage für ein $n$ und $m$. Sei also $n < m$, dann gilt:
\begin{align*}
    d(x_n,x_m) &\leq d(x_n, x_{n+1}) + d(x_{n+1}, x_{n+2}) + ... + d(x_{m-1}, x_m)\\
    &\leq (L^n + L^{n+1} + ... + L^{m-1}) \cdot d(x_0, f(x_0))\\
    &\leq (L^n + L^{n+1} + L^{n+2} + ...) \cdot d(x_0, f(x_0))\\
    &\leq \frac{L^n}{1-L} \cdot d(x_0, f(x_0)) \longrightarrow 0 \quad (n \to \infty)
\end{align*}
Wir können nun die Cauchy-Eigenschaft zeigen: Sei $\varepsilon > 0$ beliebig, dann können wir ein $n_0$ finden, sodass für $n_0 \leq n \leq m$ gilt:
$$d(x_n, x_m) \leq \frac{L^n}{1-L} \cdot d(x_0, f(x_0)) < \varepsilon$$
Der Fall $m = n$ ist trivial, da $d(x_n, x_n) = 0 < \varepsilon$ per Definition einer Metrik gilt. Da Metriken zudem symmetrisch sind, also $d(x_n, x_m) = d(x_m, x_n)$, gilt die Aussage für beliebige $n,m \geq n_0$, also ist die Folge eine Cauchy-Folge und der Grenzwert resp. der Fixpunkt existiert.

(Eindeutigkeit) Seien nun $x$ und $x'$ zwei Fixpunkte, dann gilt
\begin{align*}
    d(x,x') &= d(f(x), f(x'))\\
    &\leq L \cdot d(x, x')\\
    \underbrace{(1-L)}_{\leq 0}\cdot \underbrace{ d(x, x')}_{\geq 0} &\leq 0
\end{align*}
Diese letzte Ungleichung ist nur dann möglich, wenn $d(x,x') =0$, also $x = x'$ gilt.
\end{proof}

\subsection{Markow-Ketten und Wahrscheinlichkeitsmatrizen}
Wir möchten nun den \textsc{Banach}'schen Fixpunktsatz für das folgende Problem anwenden:

Wir haben System mit $N$ verschiedenen Zuständen und einen Begriff von Zeitschritten, wobei wir in jedem Schritt zufällig entscheiden, welchen nächsten Zustand wir annehmen (z.B. ''Random-Walk'' auf einem Schachbrett). Dies gibt uns eine Wahrscheinlichkeitsverteilung der Zustände, welche gegeben ist durch die Werte $x_1, ..., x_N$. Da es sich um Wahrscheinlichkeiten handelt, ist jedes $x_i$ positiv und die Summe aller $x_i$ ist immer 1: $\sum_{i=1}^N x_1 = 1$. Wir können also die Wahrscheinlichkeitsverteilung auch als Vektor im $R^N$ darstellen.

\begin{definition}{$(N-1)$ Simplex}{}
Wir nennen die Menge aller Vektoren aus $\R^N$ mit positiven Einträgen und $\Nrm_1 = 1$ den \textbf{$(N-1)$-Simplex}:
$$X = \set{x \in \R^N}{\forall i: x_i \geq 0, \sum_{i=1}^N x_i = 1}$$
$X$ ist also die Menge aller Wahrscheinlichkeitsverteilungen. Zudem ist $X$ eine abgeschlossene und vollständige Menge. 
\end{definition}
\begin{proof}
Wir bemerken, dass $X$ als Schnitt zwischen der Lösungsmenge $$L = \set{\R^N}{\norm{x}_1 = 1}$$ und dem rein positiven ''Quadranten'' von $\R^N_{x_i \geq 0}$ definiert ist. Da die Bedingung für $L$ eine Gleichung ist, gilt nach Satz \ref{satz:urbild_stetige_fcns}, dass $L$ abgeschlossen ist. Der Schnitt aus zwei abgeschlossenen Mengen ist ebenfalls abgeschlossen, also ist $X = L \cap \R^N_{x_i \geq 0}$ abgeschlossen.

Die Vollständigkeit von $X$ folgt aus Satz \ref{satz:Vollstaendigkeit_abgeschlossene_Teilmengen}.
\end{proof}

Nun können wir also die Wahrscheinlichkeit aller Zustände als Vektor eines $(N-1)$-Simplex' ausdrücken. Am Anfang eines Random-Walks wäre also der Eintrag des Startfeldes gleich $1$ und die restlichen gleich 0.

Wenn wir nun die Wahrscheinlichkeit, um in einem beliebigen Zeitschritt $(t \to t+1)$ vom Zustand $j$ in Zustand $i$ zu gelangen mit $p_{ij} \in [0,1]$ bezeichnen, dann gilt für die Wahrscheinlichkeit vom Zustand $i$:
$$x_i' = \sum_{j=1}^N p_{ij}$$
Wir können das auch als Skalarprodukt mit dem Vektor $x_i' = (p_{i1}, ..., p_{iN})\cdot x$ bezeichnen resp. kann man durch Matrixmultiplikation mit der \textbf{Wahrscheinlichkeitsmatrix} $P$ den Wahrscheinlichkeitsvektor $x'$ nach einer beliebigen Verteilung $x$ berechnen.
\begin{definition}{Wahrscheinlichkeitsmatrix}{}
Seien $p_{ij}$ \textbf{Übergangswahrscheinlichkeiten}, um vom Zustand $j$ in Zustand $i$ zu wechseln, dann ist
$$P = \big(p_{ij}\big)_{\substack{1 \leq i \leq N\\1 \leq j \leq N}} = \begin{pmatrix}
p_{11} & p_{12} & \cdots & p_{1N}\\
p_{21} & p_{22} & \cdots & p_{2N}\\
\vdots & \vdots & \ddots & \vdots\\
p_{N1} & p_{N2} & \cdots & p_{NN}
\end{pmatrix}
$$
die \textbf{Wahrscheinlichkeitsmatrix}. Es gilt $p_{ij} > 0 \forall i,j$ und $\sum_{i = 1}^N p_{ij} = 1$.
\end{definition}
Also gilt $$Px = \big(p_{ij}\big)_{\substack{1 \leq i \leq N\\1 \leq j \leq N}} x = x'$$
Betrachten wir $P$ also als Abbildung in $X$:
$$\norm{Px}_1 = \sum_{i=0}^N\sum_{j=0}^N p_{ij} x_j =  \sum_{j=0}^N x_j\left(\sum_{i=0}^N p_{ij}\right) = \sum_{j=0}^N x_j = 1$$
Wir erkennen, dass $P$ eine in $X$ abgeschlossene Abbildung $P: X \to X, x \mapsto Px$ auf der vollständigen Menge $X$ ist, also können wir den Langzeitverlauf $\lim_{n \to \infty} P^nx$ dieser stochastischen Dynamik betrachten. Wahrscheinlichkeitsmatrizen sind zudem Lipschitz-Kontraktionen für $\Nrm_1$:
$$\norm{Px-Px'}_1 \leq L \norm{x-x'}_1$$
mit $L= 1- \min\{p_ij\} < 1$ Aus der Abgeschlossenheit und der Vollständigkeit von $X$ und dem Banach'schen Fixpunktsatz lässt sich also ein eindeutiger Fixpunkt $x \in X$ finden. $x$ ist eine ''stationäre Verteilung'', also eine Wahrscheinlichkeitsverteilung, die invariant unter einem Zeitschritt ist. Aus Banach folgt, dass jeder beliebige Punkt aus $X$ zu diesem Fixpunkt konvergiert, also $\forall x_0 \in X$ gilt:
$$x = \lim_{n \to \infty} P^nx_0$$
Der Beweis, dass $P$ eine Lipschitz-Kontration ist, ist im Skript zu finden.

\begin{example}[Biased Coin]
Sei $p$ die Wahrscheinlichkeit von Zustand $1$ zu $2$ zu wechseln und $q$ für $2 \to 1$. Für die Übergangsmatrix erhalten wir:
$$P = \begin{pmatrix}
1-p & q \\ p & 1-q
\end{pmatrix}$$
Für die ersten beiden Schritte $P^0x_0$ und $P^1x_0$
$$x_0 = \binom{1}{0}, \quad x_1 = \binom{1-p}{p}$$
Im Unendlichen erhalten wir die Gleichgewichtsverteilung
$$x = \lim_{n \to \infty}P^nx_0 = \binom{\frac{q}{p+q}}{\frac{q}{p+q}}$$
\end{example}

\section{Kompakte metrische Räume}
Wir haben bereits in $\R$ gesehen, dass sich mit der Kompaktheit einer Menge sehr viele Aussagen machen lassen können. Wir wollen diesen Begriff daher auch Topologisch einführen\footnote{Siehe Skript für weitere Definitionen und Begriffe. Diese Einführung beruht auf Königsberger Analysis 2, Seiten 1-44.}. Wir werden folgende Begriffe verwenden:

\begin{definition}{Überdeckung \& Teilüberdeckung}{}
Sei $(X, d)$ ein metrischer Raum und $Y \subseteq X$ eine Teilmenge. Eine offene \textbf{Überdeckung} von $Y$ ist eine Menge $\mathcal{U}$ von offenen Teilmengen, die $Y$ überdecken, d.h. $\forall y \in Y \exists U \in \mathcal{U}$, sodass $y \in U$ gilt.
$$Y \subseteq \bigcup_{U \in \mathcal{U}}$$

Eine \textbf{Teilüberdeckung} von $\mathcal{U}$ ist eine Teilmenge von $\mathcal{U}$, die immer noch eine offene Überdeckung von $Y$ ist.
\end{definition}
Nun kommen wir zur topologischen Definition der Kompaktheit:
\begin{definition}{Kompaktheit/Überdeckungskompaktheit}{}
Eine Teilmenge $Y \subseteq X$ ist \textbf{kompakt}, falls jede offene Überdeckung eine endliche Teilüberdeckung besitzt:
$$\forall \mathcal{U} \text{ offene Überdeckung: } \exists U_1, ..., U_n \subseteq \mathcal{U}: Y \subseteq \bigcup_{i = 1}^n U_i $$
\end{definition}
In Worte gefasst können wir aus \textit{jeder} offenen Überdeckung von $Y$ \textit{endlich} viele Mengen $U_i$ wählen, welche vereint immer noch $Y$ enthalten.

\begin{example}
$\mathcal{U} = \set{U_n}{n \in \Z}$ mit $U_n = (n-1, n+1)$ ist eine offene Überdeckung von $\R$, jedoch gibt es keine endliche Teilüberdeckung in $\mathcal{U}$.

$\mathcal{U}$ enthält für eine abgeschlossene Menge $[a,b] \subseteq \R$ eine endliche Teilüberdeckung, z.B. $\set{U_n}{n \in [a, b]} \subseteq \mathcal{U}$. Dies ist jedoch kein Beweis, da wir das für alle Überdeckungen zeigen müssten.
\end{example}

\begin{lemma}{Kompaktheit der Referenzmenge}{}
Die Menge $X$ heisst kompakt, falls $X$ als Teilmenge von sich selbst kompakt ist.
$$\forall \mathcal{U} \text{ offene Überdeckung: } \exists U_1, ..., U_n \subseteq \mathcal{U}: X \subseteq \bigcup_{i = 1}^n U_i $$
\end{lemma}
\begin{remark}
Man kann mit diesem Lemma die Kompaktheit eines Teilraums $Y \subseteq X$ auch formulieren, ohne den ''Mutterraum'' $X$ betrachten zu müssen. Denn offene Mengen in $Y$ sind relativ offene Mengen in $X$, also kann man eine Überdeckung von $Y$ auch auf $X$ erweitern, resp. von $X$ auf $Y$ einschränken und dieselbe Aussage über die Kompaktheit machen.
\end{remark}

\begin{example}[Wichtiges Beispiel]
$[a,b] \subseteq \R$ ist kompakt, allgemeiner sind Quader $[a_1, b_1] \times ... \times [a_n, b_n] \subseteq \R^n$ kompakt.
\begin{proof}[Heine-Borell]
(für Würfel in $n$ Dimensionen) Sei $W = [-\frac{L}{2}, \frac{L}{2}]^n \subseteq \R^n$ und $\mathcal{U}$ eine offene Überdeckung von $W$. Wir nehmen an, dass es keine endliche Teilüberdeckung von $\mathcal{U}$ gibt, die $W$ überdeckt und wollen es zum Widerspruch führen. Wir teilen also $W$ in $2^n$ Würfel mit halber Seitenlänge. Da es keine komplette Teilüberdeckung gibt, muss daher mindestens einer dieser $2^n$ Würfel keine Überdeckung haben (ansonsten würde die Vereinigung $W$ überdecken). \todo{Beweis verstehen}

Also muss mindestens ein Würfel $W_i$ mit Seitenlänge $L\cdot 2^{-i}$ keine Teilüberdeckung haben.

Maximale Distanz zwischen zwei Punkten (in $\Nrm_2$) ist $\sqrt[2]{n} \cdot ...$
\end{proof}

(Gegenbeispiel) Sei $(0 ,1] \subseteq \R$, dann ist auch $U_j = (\frac{1}{j}, \infty)$ mit $\set{U_j}{j\in \N}$, eine offene Überdeckung von $(0,1]$, jedoch gibt es keine endliche Teilüberdeckung.
\end{example}

Wir können eine weitere Art von Kompaktheit definieren:
\begin{definition}{Folgenkompaktheit/Bolzano-Weierstrass Kompaktheit}{}
Ein metrischer Raum $(X, d)$ heisst \textbf{folgenkompakt}, falls jede Folge in $X$ eine in $X$ konvergente Teilfolge besitzt.
\end{definition}
Alternativ kann man auch sagen, dass die Menge der Punkte jeder Folge $M = \set{x_i}{i \in \N}$ einen Häufungspunkt $x \in X$ besitzt:
$$\forall \varepsilon > 0: B_\varepsilon(x) \cap M \neq \emptyset$$

Im Allgemeinen gilt die Äquivalenz von Folgenkompaktheit und Überdeckungskompaktheit nicht, jedoch gilt sie, wie wir im nächsten Abschnitt sehen werden, in allen metrischen Räumen.

\subsection{Eigenschaften von kompakten Mengen}

Wir haben einige Eigenschaften von stetigen Funktionen auf kompakten Mengen gesehen: Sie nehmen ihr Minimum resp. Maximum an, sie sind begrenzt, sie sind Lipschitz-stetig, etc...

\begin{lemma}{Kompaktheit $\implies$ Folgenkompaktheit}{kompaktheit_folgkmpkt}
Sei $X$ ein kompakter metrischer Raum, dann ist $X$ folgenkompakt.
\end{lemma}
\begin{remark}
Die Umkehrung gilt auch, jedoch werden wir diese in \ref{satz:heine_borel} nur für $\R^n$ zeigen.
\end{remark}
\begin{proof}
Sei $X$ kompakt und $(x_n)_{n \in \N}$ eine Folge in $X$. Nehmen wir nun an, dass $(x_n)_{n \in \N}$ \textit{keine} konvergente Teilfolge besitzt, also dass es keinen Grenzwert $x \in X$ gibt. Also darf es für jeden Punkt $x \in X$ nur endlich viele Folgenglieder geben, die in der $\varepsilon_x$-Umgebung liegen: $\abs{\set{x_n}{x_n \in B_{\varepsilon_x}, n \in \N}} < \infty$. Wir bilden nun die Überdeckung $\mathcal{U} = \set{B_{\varepsilon_x}}{x \in X}$. Da $X$ kompakt ist, wissen wir auch, dass es eine endliche Teilüberdeckung $B_{\varepsilon_1}(y_1),..., B_{\varepsilon_n}(y_n) \in \mathcal{U}$ mit $B_{\varepsilon_1}(y_1) \cup ... \cup B_{\varepsilon_n}(y_n) = X$ gibt. Für die Folge bedeutet das aus der Konstruktion, dass nur endlich viele Folgenglieder in $X$ liegen können, also ein Widerspruch.
\end{proof}

\begin{definition}{Beschränktheit}{}
Der Teilraum $A \subseteq X$ heisst \textbf{beschränkt}, wenn ein Punkt $a \in X$ und ein Radius $r > 0$ existieren, sodass $A \subseteq B_r(a)$ gilt.
\end{definition}

\begin{lemma}{Folgenkompaktheit $\implies$ Abgeschlossenheit}{folgkmpkt_abgeschl}
Sei die Teilmenge $A \subseteq X$ folgenkompakt, dann ist $A$ abgeschlossen und beschränkt.
\end{lemma}

\begin{proof}
(Abgeschlossenheit) Wir wollen zeigen, dass jede konvergente Folge einen Grenzwert in $A$ besitzt. Sei also $(x_n)_{n \in \N}$ eine konvergente Folge in $X$ mit $x_n \in A$ für alle $n \in \N$ und $x = \lim_{n \to \infty} x_n$ der Grenzwert. Aus der Folgenkompaktheit folgt, dass eine Teilfolge $(x_{n_k})$ den Grenzwert in $A$ hat. Da $(x_n)$ konvergiert, gilt:
$$\lim_{k \to \infty}x_{n_k} = \lim_{n \to \infty} x_n = x \in A$$

(Beschränktheit) Dies wollen wir per Widerspruch beweisen. Sei also $A$ nicht beschränkt und $a \in A$, dann ist $A \nsubseteq B_n(a)$ für beliebig grosse $n \in \N$. Es gibt also jeweils ein $x_n \in A \setminus B_n(a)$ mit $d(a,x_n) \geq n$. Sei nun nach Annahme $(x_{n_k})_{n \in \N}$ eine konvergente Teilfolge mit Grenzwert $x = \lim_{k \to \infty} \in A$, dann gilt widersprüchlicherweise:
$$\underbrace{n_k}_{\to \infty} \leq d(a, x_{n_k}) \leq d(a,x) + \underbrace{d(x,x_{n_k})}_{\to 0}$$
Wenn also $A$ nicht beschränkt ist, dann gibt es beliebig weit entfernte Punkte von $a$. 
(Falls $A$ leer gibt es nichts zu beweisen.)
\end{proof}

\begin{lemma}{Kompaktheit von Teilmengen}{}
Abgeschlossene Teilmengen von kompakten metrischen Räumen sind kompakt.
\end{lemma}
\begin{proof}
Sei $A \subseteq X$ eine abgeschlossene Teilmenge und $\mathcal{U}$ eine offene Überdeckung von $A$. Wir konstruieren mit $\mathcal{U} \cup \{X \setminus A\}$ eine offene Überdeckung für $X$. Da $X$ per Annahme kompakt ist, gibt es eine endliche Teilüberdeckung $U_1, ..., U_n \in \mathcal{U}$ von $X$, sodass
$$U_1 \cup ... \cup U_n \cup (X \setminus A) = X$$
Also gilt auch $A \subseteq U_1 \cup ... \cup U_n$, wodurch $A$ kompakt ist.
\end{proof}

\begin{satz}{Heine-Borel}{heine_borel}
Sei $K \subseteq \R^n$, dann sind folgende Aussagen äquivalent:
\begin{enumerate}[label=(\alph*)]
    \item $K$ ist kompakt
    \item $K$ ist folgenkompakt
    \item $K$ ist abgeschlossen und beschränkt
\end{enumerate}
\end{satz}
\begin{proof}
Wir haben bereits gezeigt:
(a) $\stackrel{\text{\ref{lem:kompaktheit_folgkmpkt}}}{\implies}$ (b) $\stackrel{\text{\ref{lem:folgkmpkt_abgeschl}}}{\implies}$ (c). Also bleibt (c) $\implies$ (a) zu zeigen:

Sei $K \subseteq \R^n$ abgeschlossen und beschränkt, dann gibt es einen Würfel $[-\frac{L}{2}, \frac{L}{2}]$, der $K$ enthält. Da der Würfel kompakt ist, wissen wir aus vorherigem Lemma, dass auch $K$ kompakt ist.
\end{proof}

\begin{exercise}Man kann ($\implies$) als Übung zeigen:

$X$ kompakt $\iff X$ vollständig und \textbf{total beschränkt}\footnote{Für jeden Radius $r > 0$ kann $X$ durch endlich viele Bälle von Radius $r$ überdeckt werden.}
\end{exercise}

\subsection{Eigenschaften von Funktionen auf kompakten Mengen}
Wir haben bis jetzt bei stetigen Funktionen nur Aussagen machen können über das Urbild. Auf kompakten Mengen können wir ähnliche Eigenschaften für das Bild einer stetigen Funktion finden:
\begin{satz}{Bild von Funktionen auf kompakten Mengen}{bild_kompakt}
Seien $(X,d_X), (Y, d_Y)$ metrische Räume und sei $X$ kompakt. Dann ist das Bild einer stetigen Abbildung $f: X \to Y$ kompakt.
$$X \text{ kompakt} \land f: X \to Y \text{ stetig} \implies f(X) \text{ kompakt}$$
\end{satz}
\begin{proof}
Wir wollen also zeigen, dass eine offene Überdeckung vom Bild $f(X) \subseteq Y$ eine endliche Teilüberdeckung besitzt. Sei also $\mathcal{U}$ eine offene Überdeckung von $f(X)$. Es gibt also für jedes $x \in X$ eine offene Teilmenge $U \in \mathcal{U}$, welches $f(x)$ enthält. Aus der Stetigkeit von $f$ folgt, dass das Urbild $f^{-1}(U) \subseteq X$ auch offen ist, also ist die Menge $\Tilde{\mathcal{U}} = \set{f^{-1}(U)}{U \in \mathcal{U}}$ eine offene Überdeckung vom Definitionsbereich $X$. Da $X$ per Annahme kompakt ist, ist eine endliche Teilüberdeckung $\{f^{-1}(U_1), ..., f^{-1}(U_n)\}$ darin enthalten. Für jedes $x \in X$ gibt es also ein $U_i$, sodass $f(x) \in U_i$ enthalten ist. Somit ist $\{U_1, ..., U_n\}$ eine offene Überdeckung von $f(X)$, also ist das Bild $f(X)$ kompakt.
\end{proof}

\begin{korollar}{Beschränktheit auf kompakten Mengen}{beschr_kompakt}
Jede stetige Abbildung $f: K \to \R^n$ oder $\C^n$ ist auf einem kompakten $K$ beschränkt
$$\exists C:\norm{f(x)}_2\leq C$$
\end{korollar}
\begin{proof}
Das Bild $f(X)$ ist nach obigem Satz \ref{satz:bild_kompakt} kompakt, also ist es auch beschränkt.
\end{proof}

\begin{korollar}{Extremalwerte auf kompakten Mengen}{extr_komapkt}
Jede stetige Funktion $f: K \to \R$ auf einem kompakten metrischen Raum $K \neq \emptyset$ nimmt immer ein Minimum und ein Maximum an.
$$\exists x_{min}, x_{max} \in K: \forall x \in K: f(x_{min}) \leq f(x) \leq f(x_{max})$$
\end{korollar}
\begin{proof}
Wir wissen, dass $f(K)$ beschränkt und nicht leer ist, also müssen ein $\sup f(X)$ und $\inf f(X)$ existieren. Das Supremum (resp. Infimum) ist der Grenzwert einer Folge im Bild $f^{-1}(K)$ ist und $f^{-1}(K)$ ist nach obigem Satz \ref{satz:bild_kompakt} abgeschlossen, also liegt der Grenzwert im Bild. Es folgt $\sup f(X) = \max f(X) = f(x_{max})$ und  $\inf f(X) = \min f(X) = f(x_{min})$
\end{proof}

Wir wollen nun noch die Implikationen der Kompaktheit auf die Stetigkeit einer Funktion betrachten. Hierfür zuerst die entsprechenden Begriffe:

\begin{definition}{gleichmässige Stetigkeit \& Lipschitzstetigkeit}{}
Seien $(X, d_X), (Y, d_Y)$ metrische Räume, dann heisst eine Abbildung $f: X \to Y$ \textbf{gleichmässig stetig}, falls es für jedes $\varepsilon>0$ ein $\delta>0$ gibt, sodass gilt
$$\forall x,x' \in X, d_X(x,x') < \delta \implies d_Y(f(x), f(x')) < \varepsilon$$
$f$ heisst \textbf{Lipschitz-stetig}, wenn es eine Konstante $L>0$ gibt, sodass gilt
$$\forall x,x' \in X: d_Y(f(x), f(x')) \leq L \cdot d_X(x,x')$$
Es gilt $$f  \text{ lipschitz} \implies f \text{ gleichmässig stetig}$$
wähle $\delta = \frac{\varepsilon}{L}$
\end{definition}
%cgucci 8:52
\begin{example}[ein dummes, aber wichtiges]
Sei $a \in X$, dann ist die Abstandsfunktion $X \to \R$ mit $x \mapsto d(x,a)$ zu lipschitz stetig:

Mit $L = 1$ und $x, x' \in X$ gilt für den Abstand:
$$d_\R(f(x), f(x')) = \abs{d_X(x,a) - d_X(x',a)} \leq 1 \cdot d_X(x,x')$$
Wir erhalten also die umgekehrte Dreiecksungleichung, welche durch die Metrik-Axiome gegeben ist.
\end{example}

\begin{satz}{gleichmässige Stetigkeit auf kompakten Mengen (Heine)}{glm_stetig_kompakt}
Sei $K$ ein kompakter Raum und $Y$ eine Menge, dann ist jede stetige Funktion $f: K \to Y$ auch gleichmässig stetig.
\end{satz}
\begin{proof}
Sei $f: K \to X$ stetig auf einem kompakten $K$, jedoch nicht gleichmässig stetig. Es gibt also ein $\varepsilon$, zu dem jedes $\delta = \frac{1}{n}, n \in \N$ die gleichmässige Stetigkeitsbedingung nicht erfüllt, also existieren ein $x_n, x_n' \in K$, sodass gilt
$$d_K(x_n, x_n') < \frac{1}{n} \quad \text{ aber }\quad d_Y(f(x_n), f(x_n')) \geq \varepsilon$$
Wir bilden daraus die Folgen $(x_n)_{n \in \N}$ und $(x'_n)_{n \in \N}$. Da $K$ kompakt ist, gibt es eine konvergente Teilfolge $(x_{n_k})_{k \in \N}$. Sei also $x = \lim_{k \to \infty}x_{n_k}$ der Grenzwert dieser Teilfolge. Für $(x'_{n_k})_{k\in\N}$ gilt aber wegen $d_K(x_n, x_n') < \frac{1}{n}$, dass $d_K(x'_{n_k}, x'_{n_k}) < \frac{1}{n_k}$ gilt, also konvergiert auch $\lim_{k \to \infty} x'_{n_k} = x$ zu diesem Grenzwert. Wir erhalten dann den Widerspruch, dass $d_Y(f(x_{n_k}), f(x'_{n_k}))$ beliebig klein gemacht werden kann:
$$\lim_{k \to \infty}f(x_{n_k}) = f(x) = \lim_{k \to \infty}f(x'_{n_k})$$
\end{proof}

\subsection{Verwendung der Kompaktheit: Funktionenraum}
Wir wollen die oben gewonnenen Erkenntnisse verwenden, um die Vollständigkeit vom Funktionenraum $C(K, \R) = \set{f: K \to \R}{f \text{ stetig}}$ bezüglich der $\Nrm_\infty$ zu zeigen, wobei $K$ eine  kompakter metrischer Raum sein soll.

Wir können einige Eigenschaften von Elementen $f \in C(K, \R)$ bereits notieren:
\begin{itemize}
    \item Korollar \ref{kor:extr_komapkt}: $\norm{f}_\infty = \sup_{x \in K}\abs{f(x)} = \max_{x \in K}\abs{f(x)}\geq 0$
    \item Homogenität der Norm: $\norm{\lambda f}_\infty = \abs{\lambda} \norm{f}_\infty$ für $\lambda \in \R$ oder $\C$ 
    \item Definitheit der Norm: $\norm{f}_\infty = 0 \iff f=0$
    \item Dreiecksungleichung: $\norm{f+g}_\infty \leq \norm{f}_\infty + \norm{g}_\infty$
    \item Induzierte Metrik: $d_\infty(f,g) = \norm{f - g}_\infty$
\end{itemize}
Wir behaupten nun folgenden Satz:
\begin{satz}{Vollständigkeit von Funktionenräumen}{}
Die folgenden Funktionenräume sind auf einem kompakten metrischen Raum $K$ bezüglich der $\Nrm_\infty$-Norm vollständig:
\begin{enumerate}[label=(\alph*)]
    \item $C(K, \R)$
    \item $C(K, \R^n)$ resp. $C(K, \C^n)$ mit
        $$\norm{f}_\infty = \max_{x \in K}\norm{f(x)}_2 = \max_{x \in X} \sqrt{\abs{f_1(x)}^2 + ... \abs{f_n(x)}^2}$$ 
    \item $C_b = \set{f: K \to \R^n}{f \text{ stetig und beschränkt}}$
\end{enumerate}
\end{satz}
\begin{remark}
Wir haben in Abschnitt \ref{cha_funktionenraum_glm_konvergenz} gesehen, dass die Konvergenz bezüglich der $\infty$-Norm die gleichmässige Konvergenz ist: 
$$\lim_{n \to \infty}f_n = f \iff \lim_{n \to \infty}\norm{f-f_n}_\infty = 0 \iff \limsup_{n \to \infty, x \in K} \abs{f(x) - f_n(x)} = 0$$
also gilt die gleichmässige Konvergenz:
$$\forall \varepsilon >0, \exists n_0: n \geq n_0 \implies \forall x \in K: \abs{f(x)-f_n(x)} < \varepsilon$$
\end{remark}
Wir nun wollen wir die Vollständigkeit zeigen, doch was bringt uns das? Der Vorteil davon ist, dass man bei vollständigen Räumen neue Elemente durch Grenzwerte von Folgen definieren kann und sich sicher sein kann, dass dieser Grenzwert auch existiert. So kann man z.B. zeigen, dass jede Differenzialgleichung eine Lösung hat, welche sich dann z.B. numerisch approximieren lässt.

\begin{proof} Wir wollen (a) beweisen. Sei $(f_n)_n$ eine Cauchy-Folge in $C(K, \R)$, d.h.
$$\forall \varepsilon > 0 \ \exists n_0: \forall m,n\geq n_0: \max_{x \in K}\abs{f_n(x)-f_m(x)} < \varepsilon$$
Betrachten wir die Folge ausgewertet bei einem $x \in X$, so ist $(f_n(x))_{n \in \N}$ eine Cauchy-Folge in $\R$. Da wir bereits wissen, dass $\R$ vollständig ist, hat $\lim_{n\to \infty}f_n(x)$ einen Grenzwert, also konvergiert $(f_n)_{n \in \N}$ sicher punktweise zur Grenzfunktion
$$f(x) := \lim_{n \to \infty} f_n(x)$$

(gleichmässige Konvergenz) Sei $\varepsilon > 0$, dann gibt es aufgrund der gleichmässigen Stetigkeit eines jeden $f_n$ ein $n_0$, sodass $\norm{f_n-f_m}_\infty < \frac{\varepsilon}{2}$ für $n,m \geq n_0$ gilt. Wir erhalten $\forall x, \forall n,m \geq n_0$:
$$\abs{f(x) - f_n(x)} \leq \abs{f(x) - f_m(x)} + \underbrace{\abs{f_m(x) - f_n(x)}}_{< \frac{\varepsilon}{2}}$$
Da $f_m$ punktweise konvergiert, können wir $m$ so gross wählen, sodass auch $\abs{f(x) - f_m(x)} < \frac{\varepsilon}{2}$ gilt. Wir erhalten also  $\abs{f(x) - f_n(x)}<\varepsilon$ für alle $x \in X$, also gilt, da $(f_n)_{n \in \N}$ cauchy ist, auch $\sup \abs{f(x) - f_n(x)} < \varepsilon$ für alle $n \geq n_0$, also konvergieren Folgen gleichmässig.

(Stetigkeit) Es bleibt zu zeigen, dass $f$ stetig ist. Sei $\varepsilon > 0$ beliebig und $x, x' \in X$, dann gilt
$$\abs{f(x) -f(x')} \leq \abs{f(x) -f_n(x)} + \abs{f_n(x) - f_n(x')} + \abs{f_n(x') - f(x')}$$
Wir wollen also zeigen, dass das beliebig klein gemacht werden kann. Wir können aufgrund der gleichmässigen Konvergenz $n$ so wählen, dass $\abs{f(x) -f_n(x)} < \frac{\varepsilon}{3}$ und $\abs{f(x) -f_n(x)} < \frac{\varepsilon}{3}$ gilt. Da zudem $f_n$ stetig ist, gilt auch $\abs{f_n(x) - f_n(x')} < \frac{\varepsilon}{3}$ für $d_K(x, x')$ genügend klein gewählt. Wir erhalten also $\abs{f(x) -f(x')} < \varepsilon$, also ist $f$ stetig.

Also konvergieren Cauchy-Folgen $(f_n)_{n \in \N}$ zu $f \in C(K, \R)$ bezüglich der $\Nrm_\infty$ Norm, also ist $\bk{C(K, \R), d_\infty}$ ist vollständig.
\end{proof}

\subsection{Verwendung der Kompaktheit: Endlichdimensionale Vektorräume}
Betrachten wir nun endlichdimensionale Vektorräume. Diese sind alle, wie wir in der linearen Algebra gesehen haben, isomorph sind zu $\R^n$. Für beliebige Normen wollen wir also folgenden Satz zeigen:

\begin{satz}{Äquivalenz von Normen auf $\R^n$}{equiv_real_norms}
Alle Normen auf $\R^n$ sind äquivalent zueinander.
\end{satz}
Dies impliziert des Weiteren, dass alle endlichdimensionalen Vektorräume ein und dieselbe Topologie induzieren, eine ''ausgezeichnete'' Topologie.
\begin{proof}
Wir wissen, dass die Äquivalenz von Normen eine Äquivalenzrelation ist (S2A5). Es genügt also zu zeigen, dass alle Normen zu $\Nrm_1$ äquivalent sind. Es gilt also zwei Konstanten zu finden, sodass $\norm{x} \leq C_1 \norm{x}_1$ und $C_2 \norm{x}_1 \leq \norm{x}$ gilt.

($\norm{x} \leq C_1 \norm{x}_1$) In einem endlichdimensionalen Vektorraum können wir eine Basis $(e_1, ..., e_n)$ definieren. Sei also $x \in \R^n$ mit $x = \sum_{i=1}^n x_i e_i$, dann gilt für eine Norm $\Nrm$:
$$ \norm{x} = \norm{\sum_{i=1}^n x_i e_i} \leq \sum_{i=1}^n \norm{ x_i e_i} = \sum_{i=1}^n \abs{x_i} \norm{e_i} \leq C_1 \sum_{i=1}^n \abs{x_i} = C_1 \norm{x}_1$$
wobei wir im zweitletzten Schritt $C = \max\{\norm{e_i}\}$ abgeschätzt haben und im letzten Schritt die Definition von $\Nrm_1$ verwendet haben.

($C_2 \norm{x}_1 \leq \norm{x}$) Wir wollen hierfür die Kompaktheit der Einheitssphäre $S = \set{x \in \R^n}{\norm{x}_1 = 1}$ bezüglich der Manhattan-Norm verwenden: $S$ ist abgeschlossen, denn $S$ ist das Urbild der abgeschlossenen Menge $\{1\}$ bezüglich der stetigen \textbf{Normabbildung} $N_1: \R^n \to \R, x \mapsto \norm{x}$, also $S = N_1^{-1}(\{1\})$ nach Satz \ref{satz:urbild_stetige_fcns}. Des Weiteren gilt für jedes $x \in S$ per Definition $\norm{x}_1 \leq 1$, also ist $S$ beschränkt. Also ist $S$ kompakt nach Heine-Borel \ref{satz:heine_borel}.

Wir sehen aus dem zuvor gezeigten $\norm{x} \leq C_1 \norm{x}_1$, das die Normabbildung $N: x \mapsto \norm{x}$ Lipschitz-stetig ist bezüglich $\Nrm_1$, also können wir formulieren:
$$\abs{\norm{x}-\abs{x'}} \leq \norm{x-x'} \leq C_1 \norm{x-x'}_1$$
Betrachten wir nun $N\vert_S$ auf der Einheitssphäre, dann wird $N\vert_S$ nach Korollar \ref{kor:extr_komapkt} ein Minimum $x_{min}$ mit $\norm{x} \leq \norm{x_{min}} > 0$ für alle $x \in S$ annehmen.

Sei nun $x \in R^n\setminus\{0\}$, dann gilt $\frac{x}{\ \norm{x}_1} \in S$ per Konstruktion. Wir erhalten:
$$\norm{\frac{x}{\ \norm{x}_1}} \leq \norm{x_{min}} =: C_2 > 0$$
Es folgt $C_2 \norm{x}_1 \leq \norm{x}$, also haben wir gezeigt, dass alle Normen äquivalent zu $\Nrm_1$ sind.

Da jeder endlichdimensionale Vektorraum (über einen nicht-endlichen Körper) zu einem $\R^n$ isomorph ist (durch die Wahl einer Basis), folgt, dass alle Normen auf einem endlichdimensionalen Vektorraum äquivalent zueinander sind.
\end{proof}
\begin{remark}
Wir haben im Beweis von der Endlichkeit Nutzen gemacht. Diese ist notwendig, wie wir im folgenden unendlichdimensionalen Gegenbeispiel über den Funktionenraum $X = C([a,b])$ sehen können:

\begin{itemize}
    \item (Äquivalenz) Es gilt $\norm{f}_\infty = \sup_{x\in[a,b]}\abs{f(x)}$ und $\norm{f}_1 = \int_a^b\abs{f(x)}dx$, jedoch sind diese Normen nicht äquivalent, wie wir in Beispiel \ref{ex_grenzfunktion} erkennen können.
    \item (Heine-Borel) Die Sphäre $S = \set{f\in C([a,b])}{\norm{f}_\infty = 1} = \{f$ stetig auf $[a,b]$ mit $\max\abs{f(x)} = 1\}$ ist zwar abgeschlossen wegen $S=N^1{\{1\}}$ und beschränkt wegen $\norm{f}_\infty\leq 1$, jedoch ist sie nicht kompakt:
    
    Sei $S = C([0, 2\pi])$ und die Folge $f_n(x) = \cos(2^n\cdot x)$, dann gilt für alle Folgeglieder $f_n$: $\norm{f_n}_\infty = 1$, also $f_n \in S$. Jedoch gilt für beliebige $n,m$: $\norm{f_n - f_m}_\infty = 2$: Sei o.B.d.A. $n>m$, dann erhalten wir bei $2^{-m}\pi$ ausgewertet 
    $$f_n(2^{-m}\pi) - f_m(2^{-m}\pi) = \cos(2^{n-m}\pi) - \cos(\pi) = 1 + 1 = 2$$
    Also gibt es insbesondere keine konvergente Teilfolge, wodurch $S$ nicht kompakt sein kann.
\end{itemize}
Aus dem \href{https://en.wikipedia.org/wiki/Riesz\%27s_lemma}{Lemma von \textsc{Riesz}} folgt, dass in jedem unendlichdimensionalen normierten Vektorraum $S$ nicht kompakt ist, also es existieren Folgen $(x_n)_{n\in \N} \subseteq S$, sodass $\norm{x_n - x_m} \geq 1$ für alle $n\neq m$ gilt.
\end{remark}
    \setcounter{chapter}{9}
\chapter{Mehrdimensionale Differenzialrechnung}

Wir wollen in diesem Kapitel den Begriff des Grenzwertes und der Ableitung für Funktionen von $\R^n$ in $\R^m$ definieren. Wie auch bei der eindimensionalen Ableitung müssen wir zuerst den Begriff der (punktierten) Umgebung und den des Häufungspunktes definieren:
\section{Begriffe}
\begin{definition}{punktierte Umgebung}{}
Sei $X$ ein metrischer Raum und $x_0\in X$. Die \textbf{punktierte Umgebung} von $x$ ist
$$\Dot{D}_\varepsilon(x_0) = B_\varepsilon(x_0) \setminus \{x_0\} = \set{x\in X}{0 < d(x, x_0) < \varepsilon}$$
\end{definition}
\begin{definition}{Häufungspunkt}{}
Sei $D \subseteq X$ eine Teilmenge des metrischen Raums $X$. $x_0 \in X$ heisst \textbf{Häufungspunkt} von $D$, falls für alle $\varepsilon > 0$ der Schnitt der punktierten Umgebung $\Dot{B}_\varepsilon(x_0)$ mit $D$ nicht-leer ist:
$$x_0 \text{ Häufungspunkt von } D \iff \forall \varepsilon > 0: \Dot{B}_\varepsilon(x_0) \cap D \neq \emptyset$$
\end{definition}
\begin{definition}{Grenzwert einer Funktion}{}
Sei $f:D \to Y$ mit $D \subseteq X$ ein Teilraum und $x_0$ ein Häufungspunkt von $D$, dann bedeutet
$$\lim_{y\to x_0} f(y) = a \iff \forall \varepsilon > 0 \ \exists \delta>0: y \in D \cap \Dot{B}_\delta(x_0) \implies f(y) \in B_\varepsilon(a)$$
Äquivalent dazu: Für alle Folgen $(x_n)_{n \in \N} \subseteq D$ mit $\lim_{n \to \infty}x_n = x_0$ gilt:
$$\lim_{n \to \infty} f(x_n) = a$$
\end{definition}
\begin{korollar}{Stetigkeit}{}
Eine Funktion $f: D \to Y$ mit $D\subseteq X$ ein Teilraum ist genau dann stetig, wenn für alle Häufungspunkte $x_0 \in D$ gilt:
$$\lim_{x \to x_0} f(x) = f(x_0)$$
\end{korollar}
\begin{proof}
Dies folgt direkt aus der Definition der Stetigkeit \ref{cha_stetigkeit_metr} und der obigen Definition des Grenzwertes.
\end{proof}
Für eine ausführlichere Erklärung und Motivation dieser Begriffe siehe Abschnitt \ref{cha_grundlagen_diffrechnung}.

\section{Die Ableitung}
Wir erinnern uns nochmal an die Eigenschaften der Ableitung in $\R$: Sei $D \subseteq \R$ offen, dann ist $f: D \to \R$ bei $x_0 \in D$ differenzierbar, wenn es eine lineare Funktion $f'(x_0): D \to \R, h \mapsto f'(x_0) \cdot h$ gibt, sodass gilt:
$$f(x_0 + h) = f(x_0) + f'(x_0)h + o(\abs{h})$$
Daraus können wir die (affin)lineare Approximation von $f$ durch $f(x_0) + f'(x_0)(x-x_0)$ herleiten, welche $f$ bei Punkten nahe an $x_0$ beliebig genau approximieren kann.
Dadurch ist $f'$ eindeutig bestimmt und es gilt
$$\frac{o(\abs{h})}{\abs{h}} = \abs{\frac{f(x_0+h)- f(x_0)}{h} - f'(x_0)} \longrightarrow 0 \quad (h \to 0)$$
Siehe Abschnitt \ref{cha_landau_notation} für Details zur Landau-Variante der Ableitung.

Wir wollen nun einen allgemeinen, koordinatenunabhängigen Begriff der Ableitung definieren (also ohne Basis):
\begin{definition}{Ableitung}{}
Seien $V,W$ endlichdimensionale normierte Vektorräume über $\R$ und sei $U \subseteq V$ ein offener Teilraum. Die Abbildung $f: U \to W$ nennen wir \textbf{differenzierbar} bei $x_0 \in U$, wenn es eine eindeutige lineare Abbildung $f'(x_0): V \to W, h \mapsto f'(x_0)\cdot h$ gibt, sodass für $x_0, x_0+h \in U$ gilt
\begin{align}\label{eq_totales_differential}
    f(x_0+h) = f(x_0) + f'(x_0) h + o(\norm{h}) \longrightarrow 0 \quad (h \to 0)
\end{align}
wobei $o(\norm{h}): \R \to W$ hierbei eine Abbildung ist, die $\lim_{h \to 0} \frac{\norm{o(\norm{h}_V)}_W}{\norm{h}_V} = 0$ erfüllt.

Dies ist die (totale) \textbf{Ableitung}/das Differenzial/Tangentialabbildung und wir schreiben $D_{x_0}f$ oder $Df(x_0), df(x_0),f'(x_0),...$. Zudem heisst $f: U \to W$ \textbf{differenzierbar}, wenn $f$ bei allen $x_0 \in U$ differenzierbar ist.
\end{definition}
Alternativ könnte man auch eine Folge $(h_{i})_{i \in \N}$ in $\R^n\setminus\{0\}$ mit $x_0 + h_i \in U$ für alle Folgenglieder und $\lim_{i \to \infty} h_i = 0$ betrachten, dann gilt für die Ableitung:
$$\lim_{i \to \infty}\frac{\norm{f(x_0 + h_i) - f(x_0)  - f'(x_0) h_i}_W}{\norm{h_i}_V} = 0$$
\begin{remark}
Beachte, dass $f'$ eine Abbildung von $U \to \text{Hom}(V, W)$ mit $x_0 \mapsto f'(x_0) \in \text{Hom}(V, W)$ ist, also ist $f'(x_0)$ eine lineare Abbildung von $V \to W$, welches ein Element $h \in V$ nach $W$ abbildet. Wir werden $f(x_0)$ als Matrix darstellen, welche einen Vektor $h \in U$ aufnimmt und einen anderen Vektor $f(x_0)(h) \in W$ zurückgibt. Daher können wir den Definitionsbereich von $f'(x_0)$ auch ganz $V$ verwenden, da wie in $\R$ die Tangente über ganz $\R$ definiert ist.
\end{remark}

\subsection{Richtungsableitung}
Schränken wir die Ableitungsrichtung durch einen Vektor ein, so erhalten wir den Begriff der \textbf{Richtungsableitung}:
\begin{definition}{Richtungsableitung}{}
Sei $U \subseteq V$ und $f: U \to W$ differenzierbar in $x_0$, dann ist die Ableitung von $f$ entlang eines Vektors $v \in V$ bei $x_0 \in U$ die \textbf{Richtungsableitung} in Richtung $v$
$$\partial_v f(x_0) = \lim_{t \to 0} \frac{f(x_0 + tv) - f(x_0)}{t} = \diff{}t f(x_0+ tv) \Big\vert_{t=0}$$
\end{definition}
\begin{remark}
Beachte, dass auch hier die Richtungsableitung auf für Vektoren $v \notin U$ definiert ist. 
\end{remark}
\begin{lemma}{Richtungsableitung in beliebige Richtungen}{}
Falls $f$ bei $x_0 \in U$ differenzierbar ist, dann existiert auch die Richtungsableitung
$$\partial_v f(x_0) = f'(x_0)v$$
für beliebige $v \in V$. Insbesondere ist die lineare Abbildung $f'(x_0)$ eindeutig durch (\ref{eq_totales_differential}) bestimmt.
\end{lemma}
Die Ableitungsmatrix $f'(x_0)$ enthält also sozusagen alle Informationen für beliebige Richtungsableitungen.
\begin{proof}
Sei $v \neq 0 \in V$ und $h = tv$ mit $t \in \R$, dann gilt mit der Definition der (totalen) Ableitung:
$$f(x_0 + tv) = f(x_0) + f'(x_0)tv + \underbrace{o(\norm{tv})}_{=: R(t)} = f(x_0) + tf'(x_0)v + R(t)$$
Es folgt mit $\lim_{t \to 0} \frac{\norm{R(t)}}{tv} = 0$
\begin{align*}
    \lim_{t\to 0} \frac{\norm{f(x_0+ tv) -f(x_0) -f'(x_0)tv}}{\abs{t}\norm{v}} &= 0\\ 
    \lim_{t\to 0} \frac{1}{\norm{v}}\norm{\frac{f(x_0+ tv) -f(x_0)}{t}-f'(x_0)v} &= 0
\end{align*}
Also erhalten wir wie behauptet
$$\underbrace{\lim_{t \to 0} \frac{f(x_0 + tv) - f(x_0)}{t}}_{\partial_v f(x_0)}= f'(x_0) v$$
Für $v=0$ sind beide Seiten $=0$, da das eine Eigenschaft von linearen Abbildungen ist. Die Eindeutigkeit von $f'(x_0)$ folgt, weil wir für einen beliebigen Vektor $v$ zeigen konnten, dass die lineare Abbildung $f'(x_0)v = \partial_v f(x_0)$ eindeutig ist (siehe Definition von $\partial_v f(x_0)$, diese ist nicht abhängig von $f'(x_0)$).
\end{proof}

\begin{example}
Sei $f: \R^2 \to \R$ eine Abbildung mit $(x,y) \mapsto xy + \cos(y)$, dann erhalten wir für die Richtungsableitung in Richtung $v = \binom{v_1}{v_2}$:
\begin{align*}
    \partial_vf(x_0,y_0) &= \diff{}{t} f\binom{x_0 + tv_1}{y_0 + tv_2}\Big\vert_{t=0}\\
    &= \diff{}{t} (x_0 + tv_1)(y_0 + tv_2) + \cos(y_0 + tv_2) \Big\vert_{t=0}\\
    &= v_1 y_0 + x_0v_2 - v_2 \sin (y_0)
\end{align*}
\end{example}

\subsection{Partielle Ableitung}
Wir möchten nun den Begriff der Ableitung im Bezug auf eine Basis von $V = \R^n$ resp. $W = \R^m$ betrachten.

\begin{lemma}{}{}
Sei $U \subseteq \R^n$ eine offene Menge und $f: U \to \R^m$ eine differenzierbare Funktion mit $f(x) = (f_1(x), ..., f_m(x))^T$ und $v\in \R^n$, dann gilt
$$\partial_vf(x) = \mat{\partial_vf_1(x)\\ \vdots \\ \partial_v f_m(x)}$$
\end{lemma}
\begin{proof} Wir setzten in die Definition ein und erhalten direkt die Behauptung:
\begin{align*}
    \partial_vf(x) &= \lim_{t\to 0} \frac{f(x+tv) -f(x)}{t}\\
    &= \mat{\lim_{t\to 0} \frac{f_1(x+tv) -f_1(x)}{t}\\ \vdots \\ \lim_{t\to 0} \frac{f_m(x+tv) -f_m(x)}{t}}
\end{align*}
\end{proof}
Da $f'$ eine lineare Abbildung i.e. Homomorphismus $V \to W$ ist, brauchen wir für die explizite Konstruktion von $f'(x)$ nur zu wissen, wohin die Basisvektoren von $U = \R^n$ abgebildet werden, und können diese in die Spalten einer Matrix eintragen. Die Richtungsableitung von Basisvektoren nennen wir die \textbf{partielle Ableitung}:
\begin{definition}{partielle Ableitung}{}
Für Einheitsvektoren $v = e_i$ einer Basis $(e_1, ..., e_n)$ von $\R^n$ definieren wir die \textbf{partielle Ableitung}
$$\diffp{f}{x_i}(x) := \partial_{e_i}f(x) = \lim_{t\to 0}\frac{f(x_1, ..., x_{i}+t, ..., x_n) - f(x_1, ..., x_{i}, ..., x_n)}{t}$$
\end{definition}
Also ist $\diffp{f}{x_i}(x) \in W$ die Ableitung von $f$ nach der Variablen $x_i$ im Punkt $x$ ist, wobei alle anderen $x_j, j\neq i$ festgehalten werden.

\subsection{Jacobi-Matrix}
Tragen wir nun die partiellen Ableitungen $f'(x)e_i = \diffp{f}{x_i}(x)$ der Basisvektoren der Standardbasis in die Spalten der Matrix von $f'(x)$ ein, so konnten wir $f'(x)$ explizit berechnen und wir erhalten die $m \times n$ \textbf{Jacobi-Matrix}: 

\begin{definition}{Jacobi-Matrix}{}
\begin{align*}
    D_xf = \mat{\vert& & \vert \\\diffp{f}{x_1}(x) & \cdots & \diffp{f}{x_n}(x) \\ \vert & & \vert } = \mat{\diffp{f_1}{x_1}(x) & \cdots & \diffp{f_1}{x_n}(x)\\
    \vdots & \ddots & \vdots \\
    \diffp{f_m}{x_1}(x) & \cdots & \diffp{f_m}{x_n}(x) } =
    \mat{(\nabla f_1)^T\vert_{x}  \\ \vdots  \\  (\nabla f_m)^T\vert_{x} } \in M_{m \times n}
\end{align*}
\end{definition}
Es gilt also
$$f'(x) \cdot v = \partial_v f(x) = \mat{\partial_vf_1(x)\\ \vdots \\ \partial_v f_m(x)} = \mat{v_1\diffp{f_1}{x_1}(x) & \cdots & v_n\diffp{f_1}{x_n}(x)\\
    \vdots & \ddots & \vdots \\
    v_1\diffp{f_m}{x_1}(x) & \cdots & v_n\diffp{f_m}{x_n}(x) } = D_xf \cdot v$$
Beachte die Verwendung des Subskripts der Richtungsableitung resp. der Jacobi-Matrix: Ersteres verwendet es als Ableitungsrichtung, letzteres als Auswertungsstelle.

\begin{example}
Sei $V = \R^2$ und $W= \R^1$, dann ist der Graph einer Funktion $f: \R^2 \to \R$ eine Fläche im $R^3$, hier in blau. In einem Punkt $P = (x,y, f(x,y))$ bildet $z = f(x_0, y_0) + D_{x_0, y_0}f\cdot \binom{x-x_0}{y-y_0}$ eine Tantentialebene in rot an den Graphen:
\begin{center}
    \centering
    \begin{tikzpicture}
    \begin{axis}[%colormap/viridis,
    height=7cm,
    surf, fill opacity=0.75,
    grid style=dashed,
    legend pos=outer north east
    ]
    \addplot3[surf,domain=-2:2,samples=30, opacity=0.01, fill opacity = 0.5, blue, shader=flat] {8 - x*x - y*y};
    \addplot3[mark=*, mark size=1.5] coordinates{(1,-1,6)}; 
    \addplot3[domain=-2:2, samples = 30, samples y=0] ({x}, {-1}, {7 - x*x});
    \addplot3[domain=-2:2, samples = 30, samples y=0] ({1}, {x}, {7 - x*x});
    \addplot3[surf,domain=0:2,samples=30, opacity=0.01, fill opacity = 0.5, red, shader=flat] ({x}, {-y}, {10 - 2*x - 2*y});
    \addlegendentry{$(x,y,f(x,y))$}
    \end{axis}
    \end{tikzpicture}
\end{center}
\end{example}

Wir können für ein stetiges $f$ die Konvergenzbedingung der Differenzierbarkeit auch so formulieren:
$$\mat{f_1(x+h)  \\ \vdots  \\  f_m(x+h) } =
\mat{f_1(x)  \\ \vdots  \\  f_m(x) } +
(D_xf)\mat{h_1  \\ \vdots  \\  h_n} + o(\norm{h}_V)  \longrightarrow 0 \quad (h \to 0)$$
Ganz wichtig ist jetzt aber, dass die umgekehrte Implikation im Allgemeinen \textbf{nicht} gilt! Falls also eine Funktion obige Konvergenz aufweist, heisst das nicht, dass sie auch differenzierbar ist (Das gilt auch, wenn alle Richtungsableitungen möglichen definiert sind):
\begin{example}
Sei $f: \R^2 \to \R$ mit
$$f(x_1,x_2) = \begin{cases} \frac{x_1x_2}{\sqrt{x_1^2+ x_2^2}} & ,(x_1, x_2) \neq (0,0) \\ 0 & , (x_1, x_2) = (0,0)\end{cases}$$
Wir erkennen, dass $\diffp{f}{x}, \diffp{f}{y}$ für alle Punkte $(x,y)$ ausser $(0,0)$ existieren. Wir berechnnen nun  $\diffp{f}{x}(0,0)$:
$$\diffp{f}{x} = \lim_{t \to 0} \frac{f(t,0) - f(0,0)}{t} =  \lim_{t \to 0} \frac{0 - 0}{t} = 0$$
Wir erhalten $0$, da $f(t,0) = 0$ für jedes $t$ gilt. Analog erhalten wir $\diffp{f}{y}(0,0) = 0$. Somit existieren beide partiellen Ableitungen für jedes $x \in \R^2$.

Betrachten wir nun aber die Richtungsableitung entlang von z.B. $v = \binom{1}{1}$, so sehen wir, dass $\partial_vf$ bei $0$ nicht definiert ist:
$$f(0+t, 0+t) = \frac{t^2}{\sqrt{2t^2}} = \frac{\abs{t}}{\sqrt{2}}$$
Es folgt, dass $f$ bei $(0,0)$ nicht differenzierbar ist:
$$\lim_{t \searrow 0}\frac{f(t,t)-f(0,0)}{t} = \lim_{t \searrow 0} \frac{\abs{t}}{\sqrt{2}t} = \frac{1}{\sqrt{2}} \neq -\frac{1}{\sqrt{2}} = \lim_{t \nearrow 0} \frac{\abs{t}}{\sqrt{2}t}$$
\end{example}

Wir können jedoch folgenden Satz formulieren, mit welchem wir auf die Differenzierbarkeit von $f$ schliessen können:
\begin{satz}{Differenzierbarkeit und partielle Ableitungen}{}
Sei $f: U \to \R^m$ mit $U \subseteq \R^n$ eine Abbildung. Falls alle partiellen Ableitungen $\diffp{f_i}{x_j}$ existieren und falls für alle $i\leq m, j \leq n$ die Abbildungen
$$x \mapsto \diffp{f_i}{x_j}(x)$$
stetig sind, dann ist $f$ differenzierbar.
\end{satz}
\begin{definition}{stetige Differenzierbarkeit}{}
Sei $U \subseteq\R^n$ offen und $f:U \to \R^m$ eine Funktion, dann heisst $f$ \textbf{stetig differenzierbar}, wenn alle partiellen Ableitungen $\diffp{f_i}{x_j}$ existieren und stetig sind auf $U$.

Also ist die vektorwertige Abbildung, die einem Punkt die Ableitungsmatrix zuordnet, stetig:
$$x \mapsto D_xf = \bk{\diffp{f_i}{x_j}}_{ij} \in M_{m \times n}(\R)$$
\end{definition}
Falls also $f$ \textbf{stetig differenzierbar} ist, so lässt sich die Ableitung finden.
\begin{remark}
Wir können die Stetigkeit mit einer beliebigen Norm von $\R^n$ zeigen, da sie alle dieselbe Topologie induzieren (Satz \ref{satz:equiv_real_norms}), jedoch ist die \textbf{Operatornorm} $\Nrm_{op}$ natürlich/kanonisch für $\text{Hom}(\R^n, \R^m)$:
$$\norm{A}_{op} = \sup \set{\norm{Ax}}{x \in \R^n, \norm{x} \leq 1}$$
Die Operatornorm erfüllt $\norm{Ax} \leq \norm{x} \norm{A}_{op}$ und $\norm{AB}_{op} \leq \norm{A}_{op} \norm{B}_{op}$
\end{remark}

\subsection{Spezielle Jacobi-Matrizen}
Wir wollen noch einige spezielle Jacobi-Matrizen betrachten:
\begin{itemize}
    \item Mit $n=1$ betrachten wir Wege in $\R^m$ von der Form $t\mapsto \gamma(t) = \bk{\gamma_1(t), \cdots, \gamma_m(t)}^T \in \R^m$. $D_t\gamma$ ist somit einen $m \times 1$ Spaltenvektor:
    $$D_t\gamma = \mat{
    \Dot{\gamma}_1(t)\\ \vdots \\ \Dot{\gamma}_m(t)
    }$$
    wobei wir mit $\Dot{\cdot} = \diff{}{t}$ den Geschwindigkeitsvektor bezeichnen. Es gilt also mit $h \in \R$:
    $$\gamma(t+h) = \gamma(t) + \Dot{\gamma}(t)h + o(\abs{h})$$
    \item Mit $m = 1$ betrachten wir ein Skalarfeld $x\in \R^n \mapsto f(x) = f(x_1,...,x_n) \in \R$. Also ist $D_xf$ gegeben durch einen $1\times n$ Zeilenvektor:
    $$D_xf = \mat{ \diffp{f}{x_1}(x) & \hdots & \diffp{f}{x_n}(x)  }$$
    Wir erhalten also
    $$f(x+h) = f(x) + (D_xf)h + o(\norm{h}) = f(x) + \sum_{i=1}^n \diffp{f}{x_i}(x)h_i + o(\norm{h})$$
    Dabei spricht man oft vom \textbf{Gradient} $\nabla f(x)$, welcher in der Literatur als Spaltenvektor definiert ist:
    $$\nabla f(x) = \mat{ \diffp{f}{x_1}(x) \\ \vdots \\ \diffp{f}{x_n}(x)  }$$
    wobei dann mit $\nabla f(x) \cdot h$ auch das Standardskalarprodukt auf $\R^n$ gemeint ist.
    
    Wenn wir die Richtungsableitung $\partial_vf$ eines normierten Vektors $v \in \R^m, \norm{v} = 1$ betrachten, können wir mit der Cauchy-Schwarz Ungleichung
    $$a\cdot b = \norm{a}\norm{b} \cos \theta$$
    wobei $\theta = \angle(a,b)$ der Winkel zwischen den Vektoren ist, folgende Aussage herleiten: Sei $a = v$ und $b = \partial_vf = \nabla f\cdot v$:
    $$\partial_vf(x) = \norm{\nabla f(x)} \cos \theta$$
    Das bedeutet, dass der Anstieg $\partial_v f(x)$ maximal ist, wenn der Winkel $\theta = 0$ ist, also zeigt $\nabla f(X)$ in die Richtung des grössten Zuwachs der Funktion $f$.
\end{itemize}

\section{Kettenregel}
\begin{lemma}{Summen- und Produktregel}{}
Seien $f,g : U \to \R^m$ differenzierbar, dann ist auch $f+g$ differenzierbar und es gilt
$$(f+g)'(x) = f'(x) + g'(x)$$
Für $m=1$ gilt zudem, dass $fg$ differenzierbar ist mit
$$(fg)'(x) = f'(x)g(x) + f(x)g(x)' \qquad (m = 1)$$
\end{lemma}
\begin{proof}
Folgen aus den Regeln für eindimensionale Ableitungen mit Richtungsableitungen.
\end{proof}
\begin{lemma}{Kettenregel}{}
Seien $U \subseteq \R^n$ und $V \subseteq \R^m$ offen mit $f(U) \subseteq V$ und den Abbildungen $f: U \to V, g: V \to \R^k$. Dann ist die Verknüpfung $g \circ f: U \to \R^k$ definiert:
$$x \stackrel{f}{\mapsto} f(x) \stackrel{g}{\mapsto} g(f(x))$$
\end{lemma}
\begin{satz}{Differenzierbarkeit von Verknüpfungen}{}
Sei $f$ bei $x_0 \in U$ und $g$ bei $f(x_0) \in V$ differenzierbar, dann ist die Verknüpfung $g\circ f$ bei $x_0$ differenzierbar und es gilt:
\begin{align*}
    (g\circ f)'(x_0) &= \underbrace{g'(f(x_0)) \circ f'(x_0)}_{\text{Verkn. von lin. Funktionen}}\\
    D_{x_0}(g\circ f) &= \underbrace{\bk{D_{f(x_0)}g} \cdot \bk{D_{x_0}f}}_{\text{Matrixprodukt}}
\end{align*}
\end{satz}
\begin{proof}
Wir verwenden eine Verkettung der Definition von $f'$: Es gilt 
$$f(x_0+h) = f(x_0) + f'(x_0)h + R_f(x_0, h) \quad (h \to 0)$$
wobei für den Rest $\lim_{h \to 0} \frac{\norm{R_f(x_0, h)}}{\norm{h}} = 0$ per Definition gilt. Wir erhalten also:
\begin{align*}
    (g \circ f)(x_0 + h) &= g(f(x_0+h))\\
    &= g(\underbrace{f(x_0)}_{y_0} + \underbrace{f'(x_0)h + R_f(x_0,h)}_{=:k})\\
    &= g(y_0 + k)\\
    &= g(y_0) + g'(y_0)k + R_g(y_0, k)\\
    &= g(y_0) + g'(y_0) f'(x_0) h + \underbrace{ g'(y_0)R_f(x_0, h) + R_g(y_0,k)}_{\text{z.z.: } \to 0 }
\end{align*}
Wir müssen also zeigen, dass $g'(y_0) R_f(x_0,h) + R_g(y_0,k) = o(\norm{h})$ gilt. Wir schätzen den ersten Teil ab durch:
\[\frac{\norm{g'(y_0) R_f(x_0,h)}}{\norm{h}} \leq \norm{g'(y_0)}_{op} \underbrace{\frac{\norm{R_f(x_0,h)}}{\norm{h}}}_{\to 0 } \longrightarrow 0\]
Für die Abschätzung von $R_g(y_0,k)$ definieren wir eine Hilfsfunktion $\phi$ für welche gilt:
$$\phi(\Tilde{h}) = \frac{\norm{R_g(y_0, \Tilde{h})}}{\norm{\Tilde{h}}} \longrightarrow 0 \quad (\Tilde{h} \to 0)$$
Also folgt:
\begin{align*}
 \frac{\norm{ R_g(y_0,k)}}{\norm{h}} &= \frac{\norm{ R_g(y_0,f'(x_0) h +  R_f(x_0,h))}}{\norm{h}}\\
 &=\phi(f'(x_0)h+R_f(x_0,h))\cdot\frac{\norm{ f'(x_0)h+R_f(x_0,h)}}{\norm{h}}
\end{align*}
Für das Argument von $\phi$ gilt $\lim_{h \to 0}\norm{f'(x_0)h + R_f(x_0,h)} = 0$, also gilt auch $\lim_{h \to 0} \phi(\cdots) = 0$. Den zweiten Term können wir mit der Operatornorm abschätzen und erhalten:
\begin{align*}
 \frac{\norm{ R_g(y_0,k)}}{\norm{h}} &\leq \norm{f'(x_0)}_{op}\cdot\frac{\norm{R_f(x_0,h)}}{\norm{h}} \longrightarrow \norm{f'(x_0)}_{op} \quad (h\to 0)
\end{align*}
und ist somit begrenzt, wodurch der gesamte Ausdruck nach 0 strebt. Wir erhalten also wie behauptet
$$(g \circ f)(x_0 + h) = (f \circ g)(x_0) + g'(y_0)f'(x_0)h + o(\norm{h})$$
also ist $g \circ f$ bei $x_0$ differenzierbar mit der Ableitung $g'(f(x_0))\circ f'(x_0)$.
\end{proof}

\begin{example}[Funktionen auf Wegen ausgewertet] Betrachten wir eine Funktion $f: \R^n \supseteq U \to \R$ mit  und einen Weg $\gamma: (a, b) \to \R^n$, dann beschreibt die Verkettung eine Funktion von $[a,b] \to \R$ mit:
$$t \mapsto (\gamma_1(t), ..., \gamma_n(t)) \mapsto f(\gamma_1(t), ...,\gamma_n(t))$$
Wir erhalten für die Ableitung $\diff{}{t}f(\gamma(t))$:
\begin{align*}
    \diff{}{t}f(\gamma_1(t), ..., \gamma_n(t)) &= \mat{
    \diffp{f}{x_1}(\gamma(t)) & \cdots & \diffp{f}{x_n}(\gamma(t))
    }\cdot\mat{
    \Dot{\gamma}_1(t) \\ \vdots \\ \Dot{\gamma}_n(t)
    }\\
    &= \diffp{f}{x_1}(\gamma(t))\Dot{\gamma}_1(t) + ... + \diffp{f}{x_n}(\gamma(t))\Dot{\gamma}_n(t)\\
    &= \nabla f(\gamma(t)) \cdot \Dot{\gamma}(t) = \bk{\partial_{\Dot{\gamma}(t)}f}\bk{\gamma(t)}
\end{align*}
Für beliebige Wege $\gamma : (-a, a) \to \R^n$ mit $\gamma(0) = x$ und $\Dot{\gamma}(0) = v$ gilt also für die Ableitung von $f: U \to \R$ 
$$\partial_vf(x) = \diff{}{t}f(\gamma(x))\big\vert_{t=0}$$
Seien also $y_1,...,y_l$ Grössen, die von $x_1,...,x_m$ mit $y_i = g_i(x_1,...,x_m)$ abhängen, welche selber durch $t_1, ..., t_n$ gegeben werden  $x_j = f_j(t_1,...,t_n)$, dann steht $\diffp{y_i}{x_j}$ für $\diffp{}{x_j}g_i(x_1,...,x_m)$. Aus der Kettenregel folgt:
$$\diffp{y_i}{t_j} = \sum_{k=1}^m \diffp{y_i}{x_k}\diffp{x_k}{t_i}$$
\end{example}

\subsection{Intuition für die Verkettung}
\begin{definition}{Tangentenbündel}{}
Sei $U \subseteq \R^n$, die \textbf{Tangentenbündel} von $U$ sind
$$TU = U \times \R^n = \bigsqcup_{x \in U} T_xU$$
wobei $T_xU =\{x\} \times \R^n \in TU$ ein Tangentenbündel ist.
\end{definition}
Für jeden Punkt $x \in U$ ist das dazugehörige Tangentenbündel $T_xU = \set{(x, v)}{v \in \R^n}$ eine Menge mit Vektorraumstruktur. Für eine stetige Abbildung $f: U \to V$ gilt:
$$\set{(x, v)}{v \in \R^n} \stackrel{f}{\mapsto} \set{(f(x), D_xf(v) )}{v \in \R^n}$$
für die Verkettung $f\circ g$ gilt
$$(x,v) \stackrel{f}{\mapsto} \bk{f(x), f'(x) v} \stackrel{g}{\mapsto} \bk{g\circ f(x), g'(f(x))f'(x) v}$$

\section{Mittelwertsatz}
Wie im eindimensionalen können wir einen Mittelwertsatz formulieren:
\begin{satz}{Mittelwertsatz}{}
Sei $f: U \to \R$ differenzierbar mit $x, y \in U$ und $\gamma: [0,1] \to U$ mit $\gamma(0)=x$ und $\gamma(1)=y$, dann existiert ein $\xi \in \gamma([0,1])$, sodass
$$f(y) - f(x) = \underbrace{f'(\xi)\cdot(y-x)}_{\in \R}$$
\end{satz}
\begin{proof}[Beweisskizze] Wir betrachten einen stetigen Pfad zwischen $x$ und $y$, welcher das Problem auf eine Dimension reduziert. Nun können wir den eindimensionalen Mittelwertsatz anwenden und erhalten die Behauptung.

\end{proof}
\subsection{Anwendungen des Mittelwertsatzes}
Intuitiv ist klar, dass eine differenzierbare Funktion auf einer zusammenhängenden Menge konstant ist, falls die Ableitung stets $0$ ist:

\begin{definition}{Gebiet}{}
Ein \textbf{Gebiet} [domain] in $\R^n$ ist eine offene zusammenhängende Teilmenge von $\R^n$.
\end{definition}

\begin{satz}{konstante Funktion}{}
Sei $f: U \to \R^m$ eine differenzierbare Funktion auf einem Gebiet $U \subseteq \R^n$. Falls $D_xf=0$ für alle $x \in U$ gilt, dann ist $f$ konstant.
$$\forall x \in U: D_xf=0 \iff \forall x \in U: f(x) = c \in \R^m$$
\end{satz}
\begin{proof}
Wir betrachten für $m=1$ das abgeschlossene Urbild eines $f(x_0) = a$, also $V = \set{f(x) = a}{x \in U}$ und zeigen, dass es auch offen ist. Für jedes $x \in V$ gilt $B_\varepsilon(x) \subseteq V$, da für die Strecke zwischen $x$ und einem $y \in B_\varepsilon(x)$ gilt:
$$ f(y) - f(x) = \underbrace{f'(\xi)}_{=0}(y-x) = 0$$
also gilt $f(x) = f(y) = 0$ und $V$ offen. Für $m \geq 2$ können wir $f$ als vektorwertige Funktion betrachten und die obige Erkenntnis für jede Dimension separate anwenden.
\end{proof}

Eine weitere Anwendung zeigt die Lipschitz-Stetigkeit auf konvexen Mengen:
\begin{definition}{konvexe Mengen}{konvexe_menge}
Eine Menge $U \subseteq \R^n$ heisst \textbf{konvex}, wenn
$$x,y \in U \implies \quer{xy} \subseteq U$$
\end{definition}

\begin{satz}{Lipschitzstetigkeit auf konvexen Mengen}{}
Sei $U \subseteq \R^n$ offen und konvex und $f: U \to \R^m$ eine Funktion mit beschränkter Ableitung, dann ist $f$ lipschitz-stetig:
$$\exists M \forall x \in U: \norm{D_f}_{op} \leq M \implies f \text{ lipschitz-stetig}$$
\end{satz}

\section{Höhere Ableitungen}
Wir wollen Ableitungen höherer Ordnung betrachten.

$f: \R^n \supseteq U \to \R$ mit $U$ offen heisst \textbf{2-mal stetig differenzierbar}, falls alle zweiten partiellen Ableitungen 
$$\forall i,j \leq n: \partial_i \partial_j f(x) = \diffp{}{x_i}\diffp{}{x_j} f(x)$$
existieren und stetige Funktionen von $x \in U$ sind.

\begin{satz}{Satz von Schwarz}{satz_von_schwarz}
Sei $f: U \to \R$ eine 2-mal stetig differenzierbare Funktion, dann gilt
$$\partial_i \partial_j f(x) = \partial_j \partial_i f(x)$$
für alle $x \in U$.
\end{satz}
Per Induktion gilt dies auch für beliebige $n$.

\begin{example}
Sei $f(x,y) = xy^2$, dann gilt
\begin{align*}
    \partial^2x f(x,y) &= 0\\
    \partial^2y f(x,y) &= 2x\\
    \partial x \partial y f(x,y) &= \partial x 2xy = 2y\\
    \partial y \partial x f(x,y) &= \partial y y^2 = 2y
\end{align*}
\end{example}
  
\begin{definition}{Hesse Matrix}{}
Die Matrix aller zweiten Ableitungen $\bk{\partial_i\partial_j f(x)}_{i,j = 1}^n$ einer 2-mal stetig differenzierbaren Funktion $f$ bei $x$ ist die \textbf{Hesse-Matrix}. Nach Schwarz \ref{satz:satz_von_schwarz} ist diese symmetrisch.
$$H(x,y) = \bk{\partial_i\partial_j f(x)}_{i,j = 1}^n = \mat{
\diffp[2]{f}{x_1}(x) & \cdots & \diffp{f}{{x_1}{x_n}}(x)\\
\vdots & \ddots & \vdots\\
\diffp{f}{{x_n}{x_1}}(x) & \cdots & \diffp[2]{f}{x_n}(x)
} = H^T(x)$$
\end{definition}
\begin{example}
Für $f(x,y) = xy^2$ gilt 
$$H(x) = \mat{0 & 2y\\2y & 2x} $$
\end{example}

\begin{definition}{$k$-mal stetig differenzierbar}{}
$f: U \to \R$ heisst \textbf{$k$-mal stetig differenzierbar}, wenn alle partiellen Ableitungen
$$\partial_{i_1} ... \partial_{i_l} f(x), 1 \leq i_1, ..., i_l\leq n$$
der Ordnung $l \leq k$ existieren und stetige Funktionen von $x \in U$ sind.
\end{definition}

\begin{korollar}{Reihenfolge der partiellen Ableitungen}{}
Durch induktives Anwenden vom Satz von Schwarz hängt die partielle Ableitung nicht von der Reihenfolge $(i_1, ..., i_l)$ ab.
\end{korollar}

\begin{example}
$$\partial_1 \partial_3 \partial_2 \partial_1 \partial_2 \partial_2 f = \partial_1^2 \partial_2^3 \partial_3 f$$
\end{example}

\begin{definition}{Multiindexnotation}{}
Jede höhere partielle Ableitung kann als
$$ \partial^\alpha f = \partial_1^{\alpha_1}...\partial_n^{\alpha_n} f$$
mit $\alpha (\alpha_1, ..., \alpha_n) \in \N_0^n$ geschrieben werden
\end{definition}

\begin{example}
Mit $\alpha = (2,0,2,1)$ gilt
$$\partial^\alpha f= \partial_1\partial_1\partial_3\partial_3\partial_4 f$$
\end{example}

\section{Taylorapproximation}
Wir wollen nun die Differenzierbarkeit verwenden, um höherdimensionale Approximationen machen zu können. Zur Erinnerung: In $\R$ hatten wir für eine $k+1$-mal stetig differenzierbare Funktion $g: \R \to \R$
$$g(x)=g(0) + g'(0)x + \frac{1}{2}g''(0)x^2 + ... + \frac{g^{(k)}(0)}{k!}x^k + R_k(x)$$
wobei das Restglied definiert ist als
$$R_k(x) = \frac{1}{k!}\int_0^x(x-t)^kg^{(k+1)}(t)dt$$
mit
$$R_k(x) = \mathcal{O}(\abs{x}^{k+1}) \quad x \to 0$$
also $\exists c: \abs{R_k(x)} \leq c\abs{x}^{k+1}$ für genügend kleine $\abs{x}$. Um diese Formel auf $n$-Variablen zu verallgemeinern, betrachten wir für die Funktion $f: \R^n \supseteq U \to \R$ die Parametrisierung
$$t \mapsto f(x+th) =: g(t) \quad h \in \R^n$$
und machen eine Taylorapproximation der Ordnung $k$:
\begin{align*}
    g(0) &= f(x)\\
    g'(t) &= \sum_{i=1}^n \partial_i f(x+th) h_i = D_{x+th}f\\
    g''(t) &= \sum_{i,j=1}^n \partial_i  \partial_j f(x+th)h_i h_j\\
    \vdots\\
    g^{(k)}(t)&=\sum_{i_1,...,i_k=1}^n  \partial_{i_1} ... \partial_{i_k}  f(x+th) h_{i_1}\cdot...\cdot h_{i_k}
\end{align*}
\begin{definition}{$k$-te Ableitung}{}
Die $k$-te Ableitung von $f: U \to \R$ bei $x \in U$ ist das homogene Polynom
$$h \mapsto \bk{D^n_xf}(h) = \sum_{i_1,...,i_k=1}^n  \partial_{i_1} ... \partial_{i_k}  f(x) h_{i_1}\cdot...\cdot h_{i_k} \in \R$$
und es gilt $D^1_x f = D_xf$
\end{definition}
\begin{satz}{Taylorapproximation}{}
Sei $f: \R^n \supseteq U \to \R$ eine $(k+1)$-mal stetig differenzierbare Funktion und $x \in U, x+h \in U$
\begin{align*}
    f(x+h) =\underbrace{f(x) + \underbrace{\sum_{i=1}^n \partial_i f(x) h_i}_{D_{x}f (h)} + \frac{1}{2}\underbrace{\sum_{i,j=1}^n \partial_i  \partial_j f(x)h_i h_j}_{h^T \cdot H(x)\cdot h = D^2_xf(h)}+\cdots+ \frac{1}{k!}\bk{D^k_xf}(h)}_{\text{Taylorapproximation $k$-ter Ordnung}} + R_k(h)
\end{align*}
wobei
$$R_k(h) = \frac{1}{k!}\int_0^1(1-s)^k D^{k+1}_{x+sh}(h) ds = \mathcal{O}(\norm{h}^{k+1})$$
also $\abs{R_k(h)} \leq c \norm{h}^{k+1}$ für kleine $h$.
\end{satz}
Für den Spezialfall $k = 1$ und $f$ 2-mal stetig differenzierbar gilt
$$f(x+h) = f(x) + \sum_{j=1}^n \partial_j f(x) h_j + \int_0^1 (1-s) \sum_{i,j = 1}^n \underbrace{\partial_i \partial_j f(x+sh) h_i h_j}_{\partial_i \partial_j f(x) + o(1)} ds$$
wobei $\lim_{h \to 0} o (1) = 0$ gilt. Durch Ausführen des Integrals erhalten wir den nächsten Term der Taylor-Approximation:
\begin{align*}
    f(x+h) &= f(x) + \sum_{j=1}^n \partial_j f(x) h_j  + \frac{1}{2}\sum_{i,j=1}^n \partial_i \partial_j f(x)h_i h_j + o(\norm{h}^2)\\
    &= f(x) + D_xf h + \frac{1}{2} h^TH(x)h + o(\norm{h}^2)
\end{align*}
(Mit 3-mal stetig differenzierbarem $f$ hätten wir eine bessere gross-$\mathcal{O}(\norm{h}^3)$ Konvergenz erhalten)

\begin{remark}
In der Formel für $D^k_xf$ tragen viele Terme nach dem Satz von Schwarz gleich bei, e.g. $\partial_1 \partial_2 \partial_1 f =  \partial_2 \partial_1^2 f = \partial_1^2\partial_2f$. Mit der Multiindexnotation kann man $D_x^kf$ schreiben als 
$$\frac{1}{k!}D_x^kf (h) = \sum_{\substack{\alpha \in \N_0^n\\\norm{\alpha}_1 = k}} \frac{1}{\alpha!} \partial^\alpha f(x) h^\alpha $$
wobei $\alpha! = \alpha_1! \cdots \alpha_n!$ und $h^\alpha = h_1^{\alpha_1}\cdots h_n^{\alpha_n}$ gilt. Also können wir für eine Taylorapproximation von Grad $k$ schreiben als
$$f(x+h) = f(x) + \sum_{\substack{\alpha \in \N_0^n\\1 \leq \norm{\alpha}_1 \leq k}} \frac{1}{\alpha!} \partial^\alpha f(x) h^\alpha + \mathcal{O}(\norm{h}^{k+1})$$
\end{remark}

\section{Extrema}
\begin{definition}{Extremalwerte}{}
$f: X \to \R$ \textbf{nimmt ein Maximum} in $x_0 \in X$ an, falls $\forall x \in X: f(x) \leq f(x_0)$ mit dem \textbf{Maximum} $f(x_0)$ und der \textbf{Maximalstelle} $x_0$.

$x_0$ heisst \textbf{striktes Maximum}, falls $f(x) < f(x_0)$ gilt für alle $x \in X \setminus \{x_0\}$.

Falls $X$ ein metrischer Raum ist, heisst $x_0$ \textbf{lokales Maximum}, falls es ein $\varepsilon > 0$ gibt, sodass $f(x) \leq f(x_0)$ für alle $x \in B_\varepsilon(x_0)$ gilt.

Analog gelten diese Definitionen für \textbf{Minima}. $f(x_0)$ heisst \textbf{(lokales) Extremum} resp. $x_0$ \textbf{(lokale) Extremalstelle}, falls $x_0$ ein (lokales) Maximum oder Miminum ist.
\end{definition}

\begin{satz}{Steigung von lokalen Extemalstellen}{}
Sei $U \subseteq \R^n$ offen und $f: U \to \R^m$ eine Abbildung mit einer lokalen Extremalstelle $x_0$. Falls $f$ bei $x_0$ differnezierbar ist, dann gilt $D_{x_0}f=0$:
$$x_0 \text{ lokale Extremalstelle und }D_{x_0} \text{ existiert} \implies D_{x_0}f=0$$
\end{satz}
\begin{proof}[Beweisskizze]
Wir wählen einen Vektor $v$ und erhalten mit $f(x_0 + tv)$ das eindimensionale Äquivalent. Nun können wir das selbe Argument aus $\R$ (Prop. 7.17 im Skript) verwenden.
\end{proof}
Die Annahme für die Umkehrung ''$D_{x_0}f=0 \implies x_0$ lokale Extremalstelle'' ist jedoch nicht hinreichend.
\begin{definition}{kritischer Punkt}{}
$x_0 \in U$ heisst \textbf{kritischer Punkt} einer differenzierbaren Funktion $f: U \to \R$, falls $D_{x_0}f=0$ 
\end{definition}
Betrachten wir zuerst $n = 1$:
\begin{satz}{Kriterium für Extremum in $\R$}{}
Sei $f:(a,b) \to \R$ 2-mal stetig differenzierbar und $f'(x_0)= 0$. Wenn $f''(x_0) < 0$ gilt, dann nimmt $f$ in $x_0$ ein lokales Maximum an ($f''(x_0) > 0$ bei lokalem Minimum).
\end{satz}
\begin{proof}[Beweisskizze] Wir verwenden die Taylorapproximation zweiten Grades von $f(x_0 + h)$ und teilen druch $h^2$:
$$\frac{f(x_0+h)-f(x_0)}{h^2} = \frac{1}{2}f''(x_0) + \frac{o(\abs{h}^2)}{\abs{h}^2}$$
Den klein-$o$ Teil kann im Betrag beliebig klein gemacht werden, also bleibt mit $f''(x_0) > 0$ rechts etwas positives (resp. mit $f''(x_0) < 0$ etwas negatives) übrig; wir erhalten $f(x_0+h) < f(x_0)$, also eine strikte Minimalstelle.
\end{proof}
\begin{remark}
Wir können im Allgemeinen keine Aussage machen, falls $f''(x_0) = 0$ gilt, e.g. für $f(x) = x^3$ bei $x_0 = 0$.
\end{remark}
Für allgemeine $n$ verwenden wir die symmetrische Hesse-Matrix:
\begin{definition}{Positiv-Definitheit}{}
Eine symmetrische Matrix $A = A^T$ heisst \textbf{positiv definit}, falls die quadratische Form
$$Q_A(v) = v^TAv = \sum_{i,j = 1}^n A_{ij}v_iv_j$$
$Q_A(v)>0$ für alle $v\in \R^n\setminus\{0\}$ erfüllt.
$$ A \text{ positiv definit} \iff \forall v\in \R^n\setminus\{0\} v^TAv > 0 \iff \text{ alle Eigenwerte positiv}$$
$A$ heisst \textbf{negativ definit}, falls $-A$ positiv definit ist.
\end{definition}

\begin{example} Für $n=2$ gilt im Allgemeinen:
$$v^T \mat{a & b\\b&d}v = a v_1^2+dv_2^2+2bv_1v_2 = a\bk{v_1 + \frac{b}{a}}^2 + \frac{ad-b^2}{a}v_2^2$$
also ist $\smat{a & b\\b & d}$ positiv definit genau dann, wenn $a > 0$ und $ad-b^2 = \abs{\smat{a & b\\b & d}} > 0$ gilt.
\end{example}

\begin{satz}{Kriterium für Extremum in $\R^n$}{}
Sei $f: U \to \R$ 2-mal stetig differenzierbar und $f'(x_0)= 0$, dann gilt
\begin{itemize}
    \item $H(x_0)$ positiv definit $\implies x_0$ ist lokale Minimalstelle
    \item $H(x_0)$ negativ definit $\implies x_0$ ist lokale Maximalstelle
\end{itemize}
\end{satz}
\begin{proof}[Beweisskizze] Analog zu oben verwenden wir auch hier die Taylorapproximation und erhalten:
$$\frac{f(x_0++h)-f(x_0)}{\norm{h}^2} = \frac{1}{2}\frac{h^T}{\norm{h}} H(x_0) \frac{h}{\norm{h}} + \frac{o(\norm{h}^2)}{\norm{h}^2}$$
Wir bemerken, dass die quadratische Form der zweiten Ableitung nur normierte Vektoren $\in \set{v \in \R^n}{\norm{v}=1}$ aufnimmt. Da diese Bildmenge abgeschlossen ist, nimmt $v^TH(x_0)v$ ein Minimum $m$ an. Falls $H(x_0)$ positiv definit ist, gilt $m > 0$, also können wir die rechte Seite wie zuvor mit kleinem $h$ positiv abschätzen und erhalten eine striktes lokales Minimum bei $x_0$. Analog für negativ definite $H(x_0)$.
\end{proof}
\begin{example}
Sei $f(x,y) = x^2 + y^2$, dann ist $H(x,y) = \smat{2 & 0\\0 & 2}$ im Allgemeinen aber insbesondere bei $(x,y) = 0$ positiv definit.

Hingegen ist für $f(x,y) = x^2 - y^2$ die Hesse-Matrix $H(x,y) = \smat{2 & 0\\0 & -2}$ weder positiv noch negativ definit, wir erhalten einen Sattelpunkt.
\end{example}
\begin{remark}
Wir nennen eine symmetrische Matrix $A$ \textbf{indefinit}, falls es Vekoren $v_{+}, v_-$ gibt, sodass
$$v_+^TAv_+ > 0, v_-^TAv_- <0$$
gilt. Dadurch ist $x_0$ weder eine Minimal- noch Maximalstelle.

$A$ heisst \textbf{positiv semidefinit}, falls $v^T A v \geq 0$ für alle $v$ gilt. In diesem Fall gibt es keine allgemeine Aussage:
\end{remark}
\begin{example} Sei $f(x,y) = ax^2 + by^2 + x \sin y$ mit $D_{x,y}f = \smat{2ax+\sin y\\2by + x \cos y}$, also ist $(x,y) = 0$ ein kritischer Punkt. Es gilt $H(0,0) = \smat{2a & 1\\1 & 2b}$. Wir erhalten
\begin{itemize}
    \item $H(0,0)$ positiv definit $\iff a > 0, 4ab - 1 > 0 \iff 0$ ist Minimalstelle
    \item $H(0,0)$ negativ definit $\iff a < 0, 4ab - 1 > 0 \iff 0$ ist Maximalstelle 
    \item $H(0,0)$ indefinit $\iff 4ab-1 < 0 \iff 0$ ist Sattelpunkt
    \item $H(0,0)$ semidefinit $\iff 4ab=1$ (unklar)
\end{itemize}
\end{example}

\section{Konvexität}
Wir haben die Konvexität von Mengen bereits in \ref{def:konvexe_menge} gesehen, nun wollen wir einen solchen Begriff für Abbildungen definieren:
\begin{definition}{konvexe Funktion}{}
Sei $U \subseteq \R^n$ offen und konvex, dann ist $f: U \to \R$ eine \textbf{konvexe Funktion}, falls für alle $x\ne y \in U$ gilt:
$$f(x+t(y-x)) \leq f(x) + t(f(y)- f(x))$$
für alle $t \in [0,1]$.

$f$ heisst \textbf{streng konvex}, wenn für alle $x\ne y \in U$ und $t \in (0,1)$ gilt
$$f(x+t(y-x)) < f(x) + t(f(y)- f(x))$$
\end{definition}
Graphisch bedeutet dies für den Fall $n=1$, dass die Sekante zwischen den Punkten $(x, f(x))$ und $(y,f(y))$ stets über dem Graphen von $f(x)$ liegt.

Für $n=1$ können wir folgendes Lemma formulieren:
\begin{lemma}{Steigung von konvexen Funktionen}{}
Sei $f:(a,b) \to \R$ eine Funktion, dann ist $f$ genau dann (streng) konvex, wenn für alle $a < x_1 < x_2 < x_3 < b$ gilt
$$\frac{f(x_2)-f(x_1)}{x_2-x_1} \stackrel{(<)}{\leq} \frac{f(x_3)-f(x_2)}{x_3-x_2}$$
\end{lemma}
In Worten also wachsen Differenzenquotienten, wenn man von links nach rechts geht. Es folgt das Lemma:
\begin{lemma}{}{}
Sei $f:(a,b) \to \R$ eine (streng) konvexe Funktion und $x_0 \in (a,b)$ eine lokale Minimalstelle, dann gilt $f(x_0) = \min_{x \in (a,b)} f(x)$ (und $x_0$ ist die einzige Minimalstelle).
\end{lemma}

\begin{satz}{Minimalstelle einer konvexen Funktion}{}
Sei $U = (a,b)$ und $f: U \to \R$ eine konvexe Funktion mit einem lokalen Minimum bei $x_0$, dann ist dann gilt $f(x_0) = \min_{x \in (a,b)} f(x)$ ein globales Minimum. Falls $f$ streng konvex ist, ist $x_0$ die einzige Minimalstelle.
$$f \text{ (streng) konvex und } x_0 \text{ ein lokales Minimum} \implies x_0 \text{ (eindeutige) globale Minimalstelle}$$

Dies gilt auch für konvexe $U \subseteq \R^n$.
\end{satz}
\begin{proof}[Beweisskizze]
Wir wollen zeigen, dass für beliebige $x > x_0 \implies f(x) \stackrel{(>)}{\geq} f(x_0)$ gilt. Wir verwenden hierfür obiges Lemma indem wir ein $x_2$ zwischen $x_0$ und $x$ finden, für welches laut der (strikten) Konvexität
$$\frac{f(x)-f(x_2)}{x-x_2}\stackrel{(<)}{\leq}\frac{f(x_2)-f(x_0)}{x_2-x_0}$$
gilt. Wir wollen $x_2$ so gewählt haben, dass mit dem lokalen Minimum bei $x_0$ gilt: $\frac{f(x_2)-f(x_0)}{x_2-x_0} \stackrel{(>)}{\geq} 0$ gilt, wodurch wir mit Einsetzen $f(x) \stackrel{(>)}{\geq} f(x_0)$  erhalten. Analog für $x < x_0$.

Für allgemeine $n \in \N $, also mit konvexem $U \subseteq \R^n$ können wir das eindimensionale Argument verwenden, indem wir $f$ mit einer Geraden parametrisieren: $f\vert_{\text{Gerade in }U} \text{ konvex} \iff f $ konvex. Wir sehen, dass $x_0$ ein globales Minimum ist.
\end{proof}

\begin{satz}{Kriterium der zweiten Ableitung}{}
Sei $f:U = (a,b) \to \R$ 2-mal stetig differenzierbar
$$\forall x\in (a,b) : f''(x) \geq 0 \implies f \text{ konvex}$$
$$\forall x\in (a,b) : f''(x) > 0 \implies f \text{ streng konvex}$$

Sei $f:U \subseteq \R^n \to \R$ 2-mal stetig differenzierbar mit der Hesse-Matrix $H(x) = \partial_i \partial_j f(x)_{i,j}, x \in U$:

 \centering$H(x)$ positiv-definit $\implies f$  streng konvex

\centering$H(x)$ positiv semidefinit $\implies f $ konvex
\end{satz}
\begin{proof}[Beweisskizze]
Da mit $f''(x) \stackrel{(>)}{\geq} 0$ die Ableitung $f'(x)$ (streng) monoton wachsend ist und da $f$ stetig ist, können wir mit dem Mittelwertsatz für beliebige $x_1 < x_2 < x_3$ entsprechende $\xi_1 < \xi_2$ finden, sodass gilt $$\frac{f(x_2)-f(x_1)}{x_2-x_1} = f'(\xi_1) \stackrel{(<)}{\leq} f'(\xi_2) = \frac{f(x_3)-f(x_2)}{x_3-x_2}$$

Für $U \subseteq \R^n$ parametrisieren wir $f$ wieder entlang eines Vektors $v$: $g(t) = f(x_0 + tv)$. Wir erhalten $\diffp[2]{g}{t} = v^TH(x+tv)v \stackrel{pos. def.}{>} 0$, also können wir aus dem eindimensionalen die Konvexität folgern.
\end{proof}
\section{Parameterintegrale}
Parameterintegrale sind von der Form:
$$\int_a^b f(x,t) dt$$
wobei $x \in U \subseteq R^n$ ein ''Parameter'' ist.
\begin{example} Elliptische Integrale sind von der Form:
$$K(x) = \int_0^{\frac{\pi}{2}}\frac{1}{\sqrt{1-x^2\sin^2t}}dt$$
oder
$$E(x) = \int_0^{\frac{\pi}{2}}{\sqrt{1-x^2\sin^2t}}dt$$
mit $\abs{x} < 1$. Zum Beispiel ist die Schwingungsperiode eines Pendels gegeben durch
$$T = 4 \sqrt{\frac{l}{g}}K\bk{\sin\frac{\theta_0}{2}}$$
wobei $\theta_0$ die Amplitude ist. Auch der Umfang einer Ellipse mit Halbachsen $a \geq b$ und Exzentrizität $e = \sqrt{1-\frac{b^2}{a^2}}$ ist gegeben druch
$$b = 4 a E(e)$$
\end{example}

\begin{satz}{Parameterintegral}{feynman_int}
Sei $f: U \times [a,b] \to \R, U \subseteq \R^n, a<b$ mit $(x,t) \mapsto f(x,t)$ stetig und $\partial_k f(x,t)$ existieren für alle $k = 1,...,n$ und sind stetig auf $U \times [a,b]$, dann ist die Funktion 
$$F: U \to \R, \int_a^bf(x,t) dt$$
stetig differenzierbar und es gilt für alle $k = 1,...,n$
$$\partial_k F(x) = \int_a^b\partial_kf(x,t) dt$$
\end{satz} 
\begin{proof}[Beweisskizze] Um zu zeigen, dass $F$ stetig ist, verwenden wir die Kompaktheit von $B = \quer{B_\eta(x_0)}\times [a,b] \subseteq  \R^{n+1}$, wobei $\eta$ so klein gewählt wird, sodass der Ball noch in $U$ enthalten ist. Da $f$ stetig ist, ist $f$ auf $B$ gleichmässig stetig. Wir erhalten $\forall \Tilde{\varepsilon} \exists \delta:\abs{x-x_0} < \delta \implies \abs{f(x,t) - f(x_0,t)}<\Tilde{\varepsilon}$. Es folgt somit
$$\abs{F(x)-F(x_0)} = \abs{ \int_a^b f(x,t) dt -  \int_a^b f(x_0,t) dt} \leq  \int_a^b \abs{f(x,t) - f(x_0,t)}dt < \Tilde{\varepsilon} (b-a)$$
Mit $\Tilde{\varepsilon} = \frac{\varepsilon}{b-a}$ für beliebiges $\varepsilon$ erhalten wir die Stetigkeit von $F$.

Für die Existenz der Formel für $\partial_k F(x)$ bilden wir den Differenzenquotienten in die Richtung $e_k$. Bilden wir die Differenz zum behaupteten Wert, so erhalten wir unter Verwendung des Mittelwertsatzes:
\begin{align*}
    \abs{\frac{F(x_0+se_k) - F(x_0)}{s}- \int_a^b \partial_kf(x_0,t) dt} &= \abs{\int_a^b\partial_kf(x_0+\xi_k(t,s)e_k,t) - \partial_kf(x_0,t)dt} \\ &\leq \int_a^b\abs{\underbrace{\partial_kf(x_0+\xi_k(t,s)e_k,t) - \partial_kf(x_0,t)}_{< \text{beliebiges } \varepsilon > 0 \text{ wenn $s$ klein genug}}dt}
\end{align*}
wobei die unterklammerte Behauptung wegen $0 < \xi_k(t,s) < s$ und der Stetigkeit von $\partial_kf$ gilt. Wir können also die Differenz beliebig klein machen und die Behauptung gilt.

Um die Stetigkeit von $\partial_kF$ zu zeigen, kann man wie bei der Stetigkeit von $F$ selber vorgehen.
\end{proof}

\begin{example}[Umfang der Ellipse]
$E(x) = \int_0^1 \sqrt{1-x^2\sin^2t} dt, -1 < x < 1$
    \begin{align*}
        E'(x) &= \int_0^1\frac{-x \sin^2 t}{\sqrt{1-x^2\sin^2t}}\\
        x E'(x) &= \int_0^1\frac{-1+1-x^2 \sin^2 t}{\sqrt{1-x^2\sin^2t}}dt \\ &= -K(x) + E(x)\\
        xE'(x) + K(x) &= E(x)
    \end{align*}
\end{example}

\begin{example} Berechne $\int_0^1 \frac{\log(t+1)}{t^2 + 1} dt$:
\begin{align*}
    F(x) &= \int_0^1 \frac{\log(xt+1)}{t^2+1}dt\\
    F'(x) &= \int_0^1\frac{t}{(xt+1)(t^2+1)}dt\\
    &=\frac{1}{x^2+1}\int_0^1 \frac{x+t}{t^2+1}-\frac{t}{tx+1}dt\\
    &= \frac{\pi x + 2\log 2 - 4 \log(x+1)}{4(x^2+1)}\\
    F(1) &= F(0) + \int_0^1F'(x) dx \\
    &= \int_0^1 \frac{\pi x + 2\log 2 - 4 \log(x+1)}{4(x^2+1)}\\
    &=\int_0^1\frac{\pi x + 2\log 2}{x^2+1} - F(1)\\
    F(1) &= \frac{1}{2} \int_0^1 \frac{\pi x + 2\log 2}{x^2+1} dx \\
    &= \frac{\pi}{8}\log 2
\end{align*}
\end{example}

\section{konservative Vektorfelder}
Wir wollen nun die uns aus der Physik bekannten Vektorfelder mathematisch genauer betrachten. Wir haben für Wegintegrale in Vektorfeldern bereits gesehen, dass diese invariant unter Reparametrisierung sind (Abschnitt $\ref{cha_reparametrisierun}$), oder im Fall von konservativen Vektorfeldern komplett wegunabhängig sind.
\begin{definition}{Vektorfeld}{}
$f: \R^n \supseteq U \to \R^n$ mit $U$ offen heisst \textbf{Vektorfeld}.
\end{definition}
\begin{definition}{Gradientenfeld}{}
$v$ heisst \textbf{Gradientenfeld} oder Potentialfeld, falls es eine differenzierbare Funktion $f: U \to \R$ (das Potenzial) gibt, sodass $v(x) = \nabla f(x)$, also 
$$\mat{v_1(x)\\\vdots\\v_n(x)} = \mat{\partial_1f(x)\\\vdots\\\partial_nf(x)}$$
\end{definition}
\begin{example}[Gravitationsfeld] Für $v=\R^3\setminus \{0\} \to \R^3$ mit $x \mapsto -\frac{x}{\abs{x}^3}$ gilt mit $V(x) = -\frac{1}{\abs{x}}$
$$v(x) = -\nabla V(x) = \nabla f(x)$$
\end{example}
\begin{lemma}{Verschiebung des Potenzials}{}
Seien $f_1, f_2: U \to \R^n$ zwei Potenziale und $v = \nabla f_1 = \nabla f_2$ ein Gradientenfeld alle auf einem Gebiet $U$, dann ist die Differenz $f_1-f_2$ stets eine Konstante.
\end{lemma}
\begin{proof}
$v = \nabla f_1 = \nabla f_2 \implies \nabla(f_1-f_2) = 0 \implies f_1 - f_2$ ist konstant.
\end{proof}
\begin{lemma}{Wegunabhängigkeit}{wegunabhaengigkeit}
Sei $v = \nabla f$ ein stetiges Gradientenfeld auf einem Gebiet $U$ und $\gamma:[a,b] \to U$ ein stetig differenzierbarer Weg, dann gilt
$$\int_\gamma \nabla f \cdot dx = f(\gamma(b))- f(\gamma(a))$$

Das Lemma gilt auch für stückweise stetig differenzierbare Wege.
\end{lemma}
\begin{proof}
\begin{align*}
    \int_\gamma \nabla \cdot dx &= \int_a^b (\nabla f)(\gamma(t))\cdot\Dot{\gamma}(t)dt\\
    &= \int_a^b\diff{}{t}f(\gamma(t))dt\\
    &= f(\gamma(b)) - f(\gamma(a))
\end{align*}
Bei stückweise stetigen Wegen können wir die Zerlegung stetig integrieren, wobei sich die Zwischenterme alle wegkürzen lassen.
\end{proof}

\begin{example}
Es sind nicht alle Vektorfelder Gradientenfelder: $v(x,y) = \smat{-y\\x}$ auf $\R^2$ ist kein Gradientenfeld, da z.B. kreisförmige Wegintegrale $\neq 0$ mit selben Start- und Endpunkt möglich sind.
\end{example}
\begin{example}[\textsc{Biot, Savant, Ampère}] Magnetfeld eines stromdurchflossenen Leiters ist gegeben durch
$$B(x) = \frac{\mu I}{2 \pi}\mat{-\frac{y}{x^2+y^2}\\\frac{x}{x^2+y^2}}$$
mit $(x,y,z) \in \R^3\setminus\{z-\text{Achse}\}$. Wir erhalten für den Weg $\gamma:[0, 2\pi]\to \R^3, \theta \mapsto \smat{R\cos \theta\\R\sin\theta\\a}$ mit $\Dot{\gamma}(\theta) = \smat{-R\sin\theta\\R \cos \theta\\0}$:
$$\int_\gamma B\cdot dx = \frac{\mu I }{2 \pi} \int_0^{2\pi} \mat{R\cos \theta\\R\sin\theta\\a} \cdot \mat{-R\sin\theta\\R \cos \theta\\0} d\theta =  \frac{\mu I }{2 \pi} \int_0^{2\pi} 1 d\theta = \mu I \ne 0$$
\end{example}


\begin{definition}{konservatives Vektorfeld}{}
Ein stetiges Vektorfeld $v: U \to \R^n$ heisst \textbf{konservativ}, falls die Wegintegrale von $v$ nur von den Endpunkten abhängen. D.h. für stückweise stetig differenzierbare Wege $\gamma_1:[a_1, b_1] \to U, \gamma_2:[a_2,b_2] \to U$ mit $\gamma_1(a_1) = \gamma_2(a_2)$ und $\gamma_1(b_1) = \gamma_2(b_2)$ gilt
$$\int_{\gamma_1}v \cdot dx = \int_{\gamma_2}v\cdot dx$$
\end{definition}

Aus der Wegunabhängigkeit von Gradientenfeldern (Lemma \ref{lem:wegunabhaengigkeit}) folgt direkt, dass diese konservativ sind.

\begin{satz}{konservativ $\iff$ Gradientenfeld}{}
Sei $v$ ein stetiges Vektorfeld auf einem Gebiet $U \subseteq \R^n$, dann ist $v$ genau dann konservativ, wenn $v$ ein Gradientenfeld ist.
\end{satz}

Um diesen Satz zu beweisen verwenden wir folgendes Lemma:
\begin{lemma}{}{}
Sei $U$ ein Gebiet und $p,q \in U$, dann gibt es einen stückweise stetig differenzierbaren Weg $\gamma:[a,b]\to U$, sodass $p=\gamma(a)$ und $q = \gamma(b)$
\end{lemma}
\begin{proof}[Beweisskizze Lemma] Da $U$ offen und wegzusammenhängend ist, können wir einen stetigen Weg $\Tilde{\gamma}:[a,b] \to U$ zwischen $p$ und $q$ finden, jedoch ist dieser nicht unbedingt stückweise stetig differenzierbar. Wir verwenden die Kompaktheit vom Bild $\Tilde{\gamma}([a,b])$ um eine endliche Teilüberdeckung aus Bällen zu finden, die $\Tilde{\gamma}([a,b])$ wie eine Kette aus überlappenden Bällen überdeckt. Nun ziehen wir (stetig differenzierbare) Geraden zwischen Punkten der überlappenden Regionen und konstruieren daraus ein stetig differenzierbares $\gamma$.
\end{proof}
\begin{proof}[Beweisskizze Satz] Noch z.z.: $v$ konservativ $\implies$ Gradientenfeld:
Für ein $x_0$ setzen wir das Potential $f(x) = \int_{\gamma_x} v \cdot dx$ mit einem stückweise differenzierbaren $\gamma$ zwischen $x$ und $x_0$. Man verifiziert, dass $\nabla f = v$ gilt, indem wir die partiellen Ableitungen $\partial_if$ mit $v_i$ vergleichen:
$$\partial_if(x) = \lim_{s \to 0} \frac{f(x+se_i)-f(x)}{s} \stackrel{...}{=} \int_0^1v_i(x+tse_i) dt \stackrel{s \to 0}{\longrightarrow} \int_0^1v_i(x)dt = v_i(x)$$
\end{proof}

\subsection{Integrabilitätsbedingung}
Die Forderung ''$v$ ist konservativ'' ist nur sehr mühsam zu überprüfen, da für beliebige zwei Punkte alle Wegintegrale dazwischen auf Wegunabhängigkeit überprüft werden müssten. Falls wir stattdessen die stetige Differenzierbarkeit von $v$ fordern, können wir $v$ mit schnell auf die Existenz eines Potenzials schliessen.

Sei also $v:U \to \R^n$ ein stetig differenzierbares Vektorfeld. Wenn $v=\nabla f$ ein Potenzialfeld ist, dann gilt nach dem Satz von Schwarz
$$\partial_iv_j(x) = \partial_i\partial_jf(x)=\partial_j\partial_i f(x) = \partial_jv_i(x)$$
Dies ist die \textbf{Integrabilitätsbedingung}.

\begin{example}Beispiele von vorher:

$v = \smat{-y\\x}$, also $\partial_xv_y(x,y) = \partial_xx=1 \ne -1 = \partial_yv_x(x,y)$, also ist dieses Feld nicht integrabel.

\textsc{Biot-Savant}-Feld: $\smat{-\frac{y}{x^2+y^2}\\\frac{x}{x^2+y^2}}$, also
$$\partial_xv_y = \partial_x \frac{x}{x^2+y^2} = \frac{1}{x^2+y^2}-\frac{2x^2}{\bk{x^2+y^2}^2}=\frac{y^2-x^2}{\bk{x^2+y^2}^2}=\partial_yv_x$$
erfüllt die Integrabilitätsbedingung.
\end{example}
\begin{remark}[Rotation]
Für $n=3$: Die Rotation eines stetig differenzierbaren Vektorfeldes $v:\R^3\supseteq U \to \R^3$ ist das Vektorfeld
$$\rot v (x) = \nabla \times v(x) = \mat{\partial_2v_3(x)-\partial_3v_2(x)\\\partial_3v_1(x)-\partial_1v_3(x)\\\partial_1v_2(x)-\partial_2v_1(x)} = \mat{\partial_1\\\partial_2\\\partial_3}\times\mat{v_1\\v_2\\v_3}$$
Also ist die Integrabilitätsbedingung äquivalent zu $\rot v = 0$.
\end{remark}

\begin{definition}{sternförmige Gebiete}{}
Eine offene Teilmenge $U \subseteq \R^n$ heisst \textbf{sternförmig} [star-shaped], falls es ein $x_0 \in U$ gibt, sodass alle Strecken $\quer{x_0x} = \set{x_0+t(x-x_0)}{t\in [0,1]}$ für alle $x\in U$ enthalten sind.
\end{definition}
Also sind sternförmige Mengen wegzusammenhängend uns somit Gebiete.

\begin{satz}{Integrabilitätsbedingung}{}
Sei $v : U \to \R^n$ ein stetig differenzierbares Vektorfeld auf einem sternförmigen Gebiet $U$, das die Integrabilitätsbedingung $\partial_iv_j = \partial_jv_i$ erfüllt. Dann gibt es eine Funktion $f:U\to \R$, sodass $v = \nabla f$ gilt.
\end{satz}
\begin{proof}
Wir setzen wieder $f(x) = \int_{\gamma_x}v\cdot dt$ mit einem geraden Weg $\gamma_x:[0,1]$ zwischen $x_0$ und $x$, also $\gamma_x(t) = x_0 + t(x-x_0)$. Wir wollen zeigen, dass mit der Integrabilitätsbeldingung (IB) $\partial_jf(x) = v_j(x)$ gilt:
\begin{align*}
    \partial_jf(x) &= \diffp{}{x_j}\int_0^1v(\gamma(t))\cdot\Dot{\gamma}(t) dt\\
    &=\int_0^1\diffp{}{x_j}\sum_{i=1}^nv_i(x_0+t(x-x_0))(x_i-x_{0,i}) \qquad\text{(Parameterintegral \ref{satz:feynman_int})}\\
    &=\int_0^1 \sum_{i=1}^n\partial_jv_i(x_0+t(x-x_0))t(x_i-x_{0,i})dt + \int_0^1v_j(x_0+t(x-x_0))dt\\
    &\stackrel{IB}{=} \int_0^1 \sum_{i=1}^n\partial_iv_j(\gamma(t))t(x_i-x_{0,i})dt + \int_0^1v_j(\gamma(t))dt\\
    &=\int_0^1\nabla v_j(\gamma(t))t\cdot\Dot{\gamma}(t)dt + \int_0^1v_j(\gamma(t))dt\\
    &=\int_0^1\bk{\diff{}{t} v_j(\gamma(t))}tdt + \int_0^1v_j(\gamma(t))dt\\
    &\stackrel{pI}{=}\sbk{v_j(\gamma(t))t}_0^1 - \cancel{\int_0^1v_j(\gamma(t))\underbrace{\diff{}{t}t}_{=1}dt}+\cancel{\int_0^1v_j(\gamma(t))dt}\\
    &= v_j(\gamma(1))\cdot 1-v_j(\gamma(0))\cdot 0 =v_j(x)
\end{align*}
\end{proof}

Die Sternförmigkeit ist notwendig, da ansonsten verschiedene Potenziale für einen Punkt gefunden werden können. Bei \textsc{Biot-Savant} kann man zwar ein Potential für Bälle ohne $\{0\}$ finden und auch mehrere von diesen Bällen um den Nullpunkt herum unter der Wahl einer entsprechenden Konstanten zusammenfügen, jedoch wird man bei einer vollen Umdrehung um $0$ eine Diskontinuität feststellen müssen.

\begin{example}[Veranschaulichung von \textsc{Biot-Savant}] Spiralförmige Rutschbahn: an jeder Stelle führt die Neigung zu einer Hangabtriebskraft (Vektorfeld), jedoch verändert sich das Gravitationspotential nach einer Runde um die Höhe einer Umdrehung.
\end{example}

\section{implizite Funktionen}
\begin{example}
$F(T,m) = m - \tanh\bk{m\frac{T_0}{T}}$ ist eine implizite Definition der spontanen Magnetisierung $m$ eines Magnets als Funktion der Temperatur $T$ (Im Ising-Modell in der Molekularfeldapproximation)
\end{example}

explainer 8.41

\begin{satz}{implizierte Funktionen}{}
Sei $U \subseteq \R^n\times \R^m$ offen, $F: U \to \R^m$ eine stetige Abbildung und $(x_0, y_0) \in U$ erfüllen $F(x_0,y_0)$ (ist eine Lösung). Weiter existieren alle $\diffp[]{F_i}{y_j}$ stetig und die Matrix $\bk{\diffp[]{F_i}{y_j}(x_0, y_0)}_{i,j=1,...,m}$, dann gibt es $\alpha, \beta > 0$ und eine eindeutige Abbildung $f:B_\alpha(x_0) \to \R^m$, sodass
$$(x,y) \in B_\alpha(x_0) \times B_\beta(y_0), F(x,y)=0$$
Falls des Weiteren $F$ $d$-mal stetig differenzierbar ist ($d \geq 1$), so ist auch $f$ $d$-mal stetig differenzierbar und es gilt
$$D_xf = -(\partial_yF(x,f(x)))^{-1} \partial_x F(x,f(x))$$
mit $x \in B_\alpha(x_0)$
\end{satz}
9:00

09.51
    \printbibliography
\end{document}

