\chapter*{Documentation}

\paragraph{Was ist die Documentation?}
Die Documentation soll einzelne Verwendungshinweise für die im Dokument verwendeten Packages bieten. Sie hat keinen Anspruch auf Vollständigkeit und wird im finalen Dokument nicht mehr sichtbar sein. 

\section*{Environments}
\begin{remark}[zu den Klammern in \LaTeX] Es ist einbissl blöd, ich weiss... Weil wir aber zwei verschiedene Packages verwenden, müssen Titel und Referenzen anders deklariert werden:

Für die ''farbigen'' Environments Definition (def), Lemma (lem), Satz (satz), Tipp (tipp) brauchts:

\begin{verbatim}
\begin{definition}{My Def}{mydef_kuerzel}
    <stuff...>
\end{definition}
\end{verbatim}

Die erste \{\}-Klammer definiert den angezeigten Titel, die zweite das Kürzel für die Referenz. Man kann sie auch leer lassen, jedoch darf man sie nicht weglassen. Referenzieren geht mit \verb|\ref{def:mydef_kuerzel}|. Das \verb|def:| bezieht sich auf die jeweilige Abkürzung des Environments abgetrennt mit einem :. (Das musst du nicht in die Definition des Kürzels schreiben!)

Für die normalen Envs. proof (prf), example (ex), remark (rem), exercise brauchts: 

\begin{verbatim}
\begin{example}[Example Title] \label{ex_myexample}
    <stuff...>
\end{example}
\end{verbatim}

Der Titel gehört in eine \textit{eckige Klammer}, die Referenz wird mit \verb|\label{}| deklariert. Hierbei sind die Kürzel mit der Abkürzung der Envs nur Konventionen. Referenzieren geht ganz normal mit \verb|\ref{ex_myexample}|. Einzige Tücke: Falls du keinen Titel willst, dann mach kein \verb|[]| hin oder man muss einen Abstand \verb|[ ]| dazwischensetzen, ansonsten spackt's.
\end{remark}

\subsection*{Log}
Vollständig dokumentierte Vorlesungen:
\begin{itemize}[label=--]
    \item (19.2.21) alle VL bis 5.10.
    \item todo: (28 VL)
    \item (10.12.20) VL 10.12.
    \item (14.12.20) VL 14.12. bis ''Reparametrisierung''
    \item[] (Ende HS2020)
\end{itemize}

